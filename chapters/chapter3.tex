\chapter{The straw tracker}\label{chaptertrk}

\textit{This Chapter provides a brief introduction on the working principles 
of a drift tube and then a  detailed description of the Mu2e straw tracker. 
This includes a description of the tracker mechanical structures 
(straw, panels, planes and stations) and the front-end and DAQ electronics. 
Most of the discussion is based on \cite{kola} and \cite{bobbb}.}

\section{Drift Tubes}
Gas detectors are capable of measuring charged particle coordinates. 
They provide great spatial resolution and detection efficiency at a low cost, Ref. \cite{kola}. 
There are many different gas ionizazion detectors, one of these is the drift tube.
The basic configuration of a drift tube is shown in Figure \ref{fig:drifttube}.
The cylindrical conducting tube, the cathode, is grounded.
A hollow cylindrical conducting tube is grounded and serves as the cathode.
The tube is filled with a combination of noble gas, often Argon, and quench gas. 
A thin sensing wire is suspended along the cylindrical cathode axis. 
The wire, called anode, receives a high voltage. Long and thin drift tubes, known as $straw$ $tubes$, have a similar form.
\begin{figure}[!h]
    \centering
    \includegraphics[width =0.8\textwidth]{figures/png/Screenshot_20240324_232621.png}
    \caption{Schematics of a drift tube, Ref. \cite{kola}.}
    \label{fig:drifttube}
    \end{figure}
Assuming an anode radius $a$, a cathode radius $b$ and using Gauss theorem, the electric field is:
\begin{equation}\label{avalanche}
    E(r)=\frac{1}{r}\frac{\lambda}{2\pi \epsilon}=\frac{1}{r}\frac{V}{ \text{ln}(b/a)} \qquad (a<r<b)
\end{equation}
Where $\lambda$ is linear charge density on the wire and $\epsilon$ is the dielectric constant of the gas.
Thinner wires are typically preferred in drift tubes. A greater electric field near the wire increases the amplification factor 
of the drift tube at the same voltage. Additionally, smaller wires improve spatial resolution, Ref. \cite{kola}. 
\subsection{Gas Ionization}
Two are the interactions that can deposit energy when particles traverse the gas volume: ionization and excitation.
Collisions between a charged particle $C$ and an atom $A$ can result in the ejection of one or more electrons: 
$A \ C \rightarrow A^+ e^- C$, or $A \ C \rightarrow A^{++} e^- e^- C$, if more than one electron is released.
This process is called primary ionisation. The mean energy loss per path length can be determined using the Bethe-Bloch formula.
A noble gas atom $A$ can turn also into an excited state $A^*$ through the interaction $A \ C \rightarrow A^* C$. 
If the excitation energy of $A^*$ is higher than the ionization potential of another species $B$ in the gas, the quencher, the
Penning Effect can produce ionization through $A^*B \rightarrow A B^+ e^-$. In addition, the noble gases can
also form molecular ions through processes such as $A^* A \rightarrow A_{2}^{*} \rightarrow A^+_2 e^-$.
A secondary ionisation can occurr through these processes or through electrons that have sufficient energy for generating more ions.
To compute the average number of electron-ion pairs produced by an initial particle, divide its total energy loss 
by the average energy required to make an electron-ion pair. Due to the energy lost during excitation, this average 
does not match the gas ionisation potential. Measurements showed an average of one electron-ion pair every 
30 eV, with variations depending on gas composition and starting particle. Except for very slow particles, 
this value remains constant regardless of their initial energy.
Without an electric field, electrons and ions created during ionisations spread uniformly. 
Collisions cause them to lose energy and eventually reach thermal equilibrium with the surrounding gas. 
They eventually recombine. An electron maximal range during ionisation is correlated with its initial kinetic 
energy. In a normal temperature and pressure gas, a 10 keV electron may be stopped in approximately 1 mm. 
Ionised electrons often have reduced kinetic energy, leading to a shorter range.
\subsection{Drift of Ions and Electrons}
The electric field accelerates free electrons and ions towards the anode and cathode 
along the field lines. As charges accelerate, they scatter on other particles in the gas, 
losing energy. The directions of motion are randomised, and maximum speeds are set. As a result, 
charges move uniformly along the electric field. This is referred to as the drift velocity of the 
charge. It is superimposed with the thermal motions.
Drift velocities for ions and electrons varies based on several parameters. Ions have a bigger mass 
than electrons and their masses are comparable to those of gas molecules. During collisions, gas 
molecules absorb a significant portion of the energy gained from ions during acceleration. In a drift 
tube detector just a little amount of electric field energy enters the energy associated with ion motion, 
making it comparable to the initial thermal energy before acceleration. 
The ion drift velocity $v_i$ is proportional to the reduced electric field $E/N$ where N is the number density of the gas 
and it is typically $\mathcal{O}(10^3)$ cm/ns, except in the region near to the anode wire where a stronger 
electric field is present. The ion thermal velocity is typically $\mathcal{O}(10^4)$ cm/ns at room temperature.
On the opposite, only a small fraction of the energy is released during elastic collision from electrons, 
so they acquire more energy from the electric field than their thermal energy.
Many different factors impact electron drift velocity. Some gas molecules, such as H$_2$O or CO$_2$, 
can interact with electrons to produce negative ions due to their great electron affinity. In rare situations, 
electrons gather enough energy to exceed gas molecule excitation threshold, resulting in inelastic collisions. 
Electron drift velocity is a complex function of electric field intensity due to several variables influencing 
electron collisions across a large energy range. Figure \ref{fig:drift} shows the electron drift
velocity at different electric field strengths in argon-carbon dioxide mixtures (Ar-CO$_2$) of different
proportions, Ref. \cite{ZHAO1994485}.
\begin{figure}[!h]
    \centering
    \includegraphics[width =0.7\textwidth]{figures/png/Screenshot_20240330_102206.png}
    \caption{Electron drift velocity versus electric field in Ar:CO$_2$ mixtures of different proportions, Ref. \cite{ZHAO1994485}. 
    80\%:20\% Ar:CO$_2$ mixture is the gas used in the Mu2e tracker.}
    \label{fig:drift}
\end{figure}
Electron drift velocities in drift tubes are typically $\mathcal{O}(10^6)$ cm/ns.
This is significantly higher than ion drift velocities and comparable to electron thermal velocity under the same conditions. 
The radial coordinate of an ionisation can be computed using the electron drift velocity and time.
Since drifting electrons and ions are scattered on gas molecules, they also diffuse along their trajectory. 
Electrons diffuse significantly quicker than ions because of their high velocity. Electron diffusion limits 
the intrinsic resolution of drift tubes used to measure incoming particle coordinates. CO$_2$ has internal 
degrees of freedom at low collision energies, preventing electron energies from exceeding thermal energy until 
field strengths above about 2 kV/cm. This improves intrinsic spatial resolution.
\subsection{Avalanche Multiplication}
Electrons can ionise when they face a high electric field near the anode wire. 
The released secondary electrons form tertiary electrons, and so on. Free electrons rapidly multiply, 
resulting in an avalanche. In a drift tubes, where the electron mean free path is about the order of $\mu$m, 
an avalanche develops when the electric field approaches $\mathcal{O}(10)$ kV/cm. 
According to Equation \ref{avalanche}, using $a \sim \mathcal{O}(10^{-3})$ cm, $b \sim \mathcal{O}(1)$ cm and
a normal voltage of 1-2 kV, the avalanches can occur within $\mathcal{O}(100) \ \mu$m from the anode wire. 
Electrons from the avalanche are collected on the anode wire within 1 ns, while positively charged ions move towards the cathode.
Drifting ions mostly generate signals in electrodes via induction. Figure \ref{fig:avalanche} shows the model of an ionisation avalanche.
\begin{figure}[!h]
    \centering
    \includegraphics[width =0.7\textwidth]{figures/png/Screenshot_20240330_182509.png}
    \caption{The model of an ionisation avalanche forming at the anode wire of a proportional tube or chamber, Ref. \cite{kola}. 
    (a) In the drift volume, electrons and ions are generated and drift to their corresponding electrodes. 
    (b) Near the wire, the electron achieves a high enough field to induce secondary ionisation, resulting in an avalanche. 
    (c) The electric field separates charges created during an avalanche. 
    (d) Electrons have higher lateral diffusion than ions, causing the avalanche to expand 
    around the wire and produce a positive charge cloud in the shape of a drop. 
    (e) Electrons from the avalanche reach the anode within nanoseconds, but ions take longer, up to ms, to reach the cathode.}
    \label{fig:avalanche}
\end{figure}
\subsubsection{Avalanche Gain}
When an avalanche develops, the amplification factor is around $10^4-10^6$. The number of electrons freed per unit path length is given
by the first Townsend ionization coefficient $\alpha=\sigma_{ion}n=1/\lambda_{ion}$ and this depends on the electric field $E$, 
as a higher electric field corresponds to a higher kinetic energy of the electron, that increases the ionization cross-section.
The increase $dN$ of the number of electron-ion pairs over a path length $ds$ is, Ref. \cite{kola}:
\begin{equation}
    dN=\alpha(E)Nds
\end{equation}
Solving this equation, we can easily obtain the gas amplification $G$:
\begin{equation}\label{av}
    G=\frac{N(s_a)}{N_0}=\text{exp} \left( \int_{s_0}^{s_a} \alpha(E(s)) ds \right)=\text{exp} \left( \int_{E_{min}}^{E(a)} \frac{\alpha(E(s))}{dE/ds} dE \right)
\end{equation}
where $N_0$ corresponds to unamplified electrons in $s=s_0$ and $E_{min}$ corresponds to the minimum energy 
for ionisation to occurr. The energy distribution depends on the electric field which is position dependent. 
Since the free path is inversely proportional to the particle density in the gas, $E_{min}(\rho)=E_{min}(\rho_0)\rho/\rho_0$.
It is reasonable to say that the coefficient is proportional to the field strength, $\alpha= \beta E$, in the low field region. 
Adding this relation with Equation \ref{avalanche} and \ref{av}:
\begin{equation}
     \text{ln}(G)=\beta \ a \ E(a) \ \text{ln}\left( \frac{E(a)}{E_{min}}\right)
\end{equation}
where $\beta$ can be related to $w_i$, that is the energy spent for one ionisation and its value is equal to $e \Delta V$.
As the voltage drop per unit path length is $dV = E(s)ds = (\alpha/\beta)ds$, we obtain $dN=N \beta dV$. Integrating, we can see that
$\beta= \text{ln}(2)/\Delta V$, so the gain in a drift tube is:
\begin{equation}\label{XXX}
     \text{ln}(G)=\frac{ \text{ln}(2)}{\Delta V} \ a \ E(a)  \ \text{ln}\left( \frac{E(a)\rho_0}{E_{min}(\rho_0)\rho}\right) \qquad E(a)=\frac{V}{a ln(b/a)}
\end{equation}
which is the Diethorn's formula. 
Gain measurements with variable $\rho$/$\rho_0$, $a$, and $E(a)$ can provide the parameters $E_{min} (\rho_0)$ and $\Delta V$. 
The gas temperature $T$, pressure $P$ and operating voltage $V$ significantly impact the gain of a drift tube with a particular shape and gas mixture. 
\subsubsection{Quench Gas}
To avoid subsequent avalanches, the drift tube gas combination may contain a quench gas, such as CO$_2$, 
methane, or other hydrocarbons. During an avalanche, photons are created by gas deexcitation and electron 
attachment to electronegative species, resulting in negative ions. Photons can generate ionisations 
outside the primary avalanche zone or create free electrons on the cathode surface, resulting in secondary avalanches.
The difficulty arises when the signal created is not proportional to the deposited energy by the original particle 
and is no longer localised to the energy deposition point. Enough intense photons can induce a chain reaction of 
secondary avalanches, leading to a continuous discharge. The use of quench gas prevents subsequent 
avalanches by absorbing ionising photons before they travel far. A tiny quantity of quench gas during normal operation 
can significantly decrease secondary avalanches and breakdowns.
\subsubsection{Operation Modes of Gaseous Ionization Detectors}
In drift tube detectors, the number of electron-ion pairs formed during an avalanche is 
proportional to the starting number of electrons, as shown in the gain computation. To 
operate in a proportional mode, an appropriate voltage is needed to reduce the effects 
of avalanche charges on the electric field. Figure \ref{fig:gaseous} illustrates how a 
gaseous ionisation detector may work in multiple modes based on the operating voltage. Higher operating 
voltage leads to higher charges on the electrodes. Low voltage causes ionisation charges to recombine 
before reaching electrodes, leading to no signal collection. At higher voltages, in ionisation chamber 
region, charges can drift to electrodes, but the electric field is insufficient for avalanches to occur. 
Increasing the operational voltage leads to drift tubes and proportional counters. When the voltage 
becomes high enough, proportionality is lost. When electrons from an avalanche are collected, 
the high density of positive ions near the anode might affect the electric field. 
Electrons in future avalanches that enter the area between the positive ion cloud and the wire face a 
decreased electric field, resulting in lower amplification. The electric field becomes greater in the 
tail of the avalanche, which is far from the wire than the ion cloud. This range of voltage is called region of 
limited proportionality. When the operating voltage reaches high values, 
avalanches create sufficiently energy photons to cause secondary avalanches 
that propagate across the detector, independently from the quench gas. 
This results in detector saturating the output. This way of operating is 
called breakdown mode, commonly known as the Geiger-Muller mode.
\begin{figure}[!h]
    \centering
    \includegraphics[width =0.5\textwidth]{figures/png/Screenshot_20240330_203416.png}
    \caption{The dependence of particle gain on applied voltage in gaseous ionisation detectors. 
    The numbers on the axes are for orders of magnitude only and they depend on the device geometry and gas concentration. 
    Drift tubes operate in the proportional mode, Ref. \cite{kola}.}
    \label{fig:gaseous}
    \end{figure}
\subsection{Signal Creation and Propagation}
Drift tube signals do not originate from avalanche charges. In this case, the anode wire would 
receive the complete signal within a few ns. Signal pulses are created by charges on electrodes 
caused by electron and ion mobility. The Shockley-Ramo theorem, Ref. \cite{kola}, can be 
used to determine the induced charge and current. 
The Shockley-Ramo theorem yields various important results. The total induced charge of a moving charge 
$q$ is determined by the initial and final positions only.  A charge pair induces the same amount of 
charge on an electrode as the charge collected on it. Furthermore, if all electrodes are treated as 
an unity, their weighted potential will be one. 
If one electrode completely encloses the others, the weighted field in the contained region is always zero. 
This means that the total induced current across all electrodes is always zero. The Shockley-Ramo theorem can 
be applied to the drift tube. In an avalanche with $N$ electron-ion pairs, we evaluate the induced current 
signal on the anode wire. The current on the cathode has an additional negative sign. The weighted 
potential and field in the straw depend on the radial coordinate $r$:
\begin{equation}
    \psi_w(r)=\frac{\text{ln}(b/r)}{\text{ln}(b/a)}
\end{equation}
\begin{equation}
    \textbf{E}_w(r)=\frac{1}{\text{ln}(b/a)}\hat{\textbf{r}}
\end{equation}
As previously mentioned, the drift velocity of ions, $u$, is proportional to the electric field intensity $E$. Therefore, $u$ = $\mu \cdot E$. 
Then the ion trajectory fulfils:
\begin{equation}
    u=\frac{dr(t)}{dt}=\frac{\mu V_0}{r(t)\text{ln}(b/a)}
\end{equation}
When ionisation occurs at the anode, the starting condition can be approximated as $r(0) = a$. So, the solution to the given differential equation is:
 \begin{equation}
    r(t)=a \sqrt{1+\frac{t}{t_0}} \qquad t_0=\frac{a^2 \ln (b / a)}{2 \mu V_0}
    \end{equation}
Here, $t_0$ represents the characteristic time, which is generally on the order of 1 ns. 
The time $t$ falls within the range $0 < t < t_{max}$, where $t_{max} = t_0 [(b/a)^2: 1]$. 
The induced current and charge from the wire are expressed as:
 \begin{equation}
    I_{i o n}^{i n d}(t)=-(e N) E_w(r) \frac{\mathrm{d} r(t)}{\mathrm{d} t}=-\frac{e N}{2 \ln (b / a)} \frac{1}{t+t_0}
    \end{equation}
    \begin{equation}
        Q_{\text {ion }}^{\text {ind }}(t)=\int_0^t I^{\text {ind }}\left(t^{\prime}\right) \mathrm{d} t^{\prime}=-\frac{e N}{2 \ln (b / a)} \ln \left(1+\frac{t}{t_0}\right)
        \end{equation}
The total charge created on the wire by the travelling electrons is:
\begin{equation}
    Q_e^{i n d}=-e N\left[\psi_w(a)-\psi_w\left(r_{a v g}\right)\right]=-e N \frac{\ln \left(r_{a v g} / a\right)}{\ln (b / a)}
    \end{equation}
    where $r_{avg}$ is the average position of ionisations in the avalanche. When compared to the total charge 
    induced $Q^{ind}_{tot}(t_{max}) = -eN$, electron mobility in the avalanche accounts for just 1-2\%. 
    Positive ion drift from the avalanche accounts for the majority of the signal. A signal propagates 
    to both ends of the drift tube from the avalanche location.
    Signals with distinct frequency components can propagate at varying velocities. This causes the signal to disperse.
    When determining the longitudinal coordinate of an avalanche using the signal arrival time difference between straw tube ends, various complications emerge.
%For low-frequency components of the signal (for a meter-long drift tube, this translates to a frequency much less than 300 MHz), 
%a quasi-electrostatic approach is sufficient. The signals seen at tubes ends are influenced by impedances between and on the electrodes. 
%On the other hand, for the high-frequency components of the signal, the tube needs to be treated as a transmission line. 
%The propagation speed of a signal wave with frequency $\omega$ is then $c/\sqrt{\omega}$ $\epsilon$($\omega$), 
%where $c$ is the speed of light and $\epsilon$ is the dielectric constant of the gas. As $\epsilon$ is a function of $\omega$, 
%components of different frequencies propagate at slightly different velocities. This leads to a dispersion (widening) of the signal. It brings some subtleties when using signal arrival time difference between straw tube ends to determine the longitudinal coordinate of an avalanche.
\section{The Mu2e tracker}
\subsection{The sensitive unit: the straw}

The Mu2e tracker straw tube arrays use the same detection principles as the gaseous 
ionisation detectors, Ref. \cite{kola}, 
but it adopts significant different design and manufacturing improvements to 
meet the experiment precise requirements.
Figure \ref{fig:trkpencil} shows the edge of a straw tube, compared to a pencil.
\begin{figure}[!h]
    \centering
    \includegraphics[width =0.5\textwidth]{figures/png/Screenshot_20240327_000000.png}
    \caption{Picture of one Mu2e straw tube (compared to a pencil), Ref. \cite{trk}.}
    \label{fig:trkpencil}
    \end{figure}


All the straws have the same diameter of 5 mm, but their length 
is between a minimum of 0.33 m and a maximum of 1.17 m, 
depending on their position on the panel, as shown in Figure \ref{fig:gonzalo}.

The straw is wound with two layers of 6 $\mu$m-thick metallized 
Mylar separated by one 3 $\mu$m layer of glue. The straw wall is thus 15 $\mu$m thick: 
this minimizes the amount of detector material and thus the total energy 
loss of electrons in the detector. Moreover, this minimises the probability 
of significant deflections of the electron trajectory which makes pattern 
recognition and track reconstruction much simpler both at trigger 
and offline level, and allows to achieve the required excellent momentum 
resolution. The straw tube anodes are made of gold-plated tungsten wires with 
a diameter of 25 $\mu$m. The straw and the anode wires are tensioned 
and work-hardened to minimise sagging effects.
Figure \ref{fig:trktubessmon} shows the straw termination machanical 
structure that allows to hold the anode wire in place.
To increase the straw mechanical strength, two brass tubes 
are joined to both ends of the straw suing silver epoxied.
    \begin{figure}[!h]
        \centering
        \includegraphics[width =\textwidth]{figures/png/Screenshot_20240706_153158.png}
        \caption{The straw termination, depicted both exploded and assembled, 
        features a brass tube connected to the straw using silver epoxy. 
        An insulator (green) is inserted into a brass tube 
        (red) to prevent breakdown near the tube's end. The sense 
        wire is soldered into the brass pin and epoxied to the 
        injection-molded plastic. Post-assembly, the brass tube 
        facilitates connection to the cathode, while the brass pin 
        enables connection to the anode.}
        \label{fig:trktubessmon}
        \end{figure}
        To ensure electrical insulation of the anode wire, 
        a kapton sleeve is inserted inside the brass tube. 
        The kapton sleeve holds an injection-molded plastic 
        insert which cointains a semicylindrical duct that 
        allows gas flow into and out of the tube. The insert 
        has a groove along its axis and a U-shaped brass anode 
        pin inserted at the end. To avoid slippage, the anode 
        wire is epoxied into the groove and soldered to the 
        anode pin. A T-shaped pin protects the anode pin from 
        breaking by covering the groove in the plastic insert. 
        The pin protector is epoxied to the insert with an extra 
        brass ring connecting them. A ground clip is silver-epoxied 
        to two adjacent straws on the brass tubes and rings to 
        provide a shared ground connection.
        To optimise the shape of the electric field within 
        the straw, assembly procedures have been developed 
        to align the anode wires to the panels with a 
        precision of at least 25 $\mu$m in the radial direction and 50 $\mu$m in the perpendicular direction. 
From the straws, signals are sent to a common preamplifier board 
via the anode pin pair and grounding clip. The tracker front 
end eletronic will be described in Section \ref{tfee}.
\subsection{The building blocks: from the panels to the station}
Groups of 96 straws are assembled in two staggered layers of 48 straws and make a 
panel (Figure \ref{fig:trktubes}). 
\begin{figure}[!h]
\begin{subfigure}[t]{0.5\textwidth}
    \centering
    \includegraphics[width =\textwidth]{figures/png/Screenshot_20240326_234405.png}
    \caption{}
    \label{fig:trktubes}
    \end{subfigure}
    ~
    \begin{subfigure}[t]{0.5\textwidth}
        \centering
        \includegraphics[width =\textwidth]{figures/png/Screenshot_20240327_000131.png}
        \caption{}
        \label{fig:strawtubes}
        \end{subfigure}
        \caption{(a): Arrangement of the straws within a panel, Ref. \cite{trk}. (b): Expanded view of the panel edge, Ref. \cite{trk}.}
    \end{figure}

Each panel spans a 120° arc (Figure \ref{fig:gonzalo}).
The dual-layer geometry was chosen since it allows to determine 
easily which side of the sense wire the particle passes through 
and thus solve the left-right ambiguity. This simplifies pattern 
recognition and increases tracking robustness and efficiency. 
A 1.25 mm separation between two consecutive straws accommodates 
manufacturing tolerances and allows for straws expansion due to gas 
pressure. This requires each straw to be self-supporting across its 
length. Channels within a panel are numbered from 0, 
corresponding to the radially most internal and longest straw, 
to 95, corresponding to the most external and shortest straw. 
   
All the tracker panels have been assembled at Fermilab Lab3 
Facility, as shown in Figure \ref{fig:gonzalo}. An expanded view 
of the panel edge is also shown in Figure \ref{fig:strawtubes}.
            \begin{figure}[!h]
                \centering
                \includegraphics[width =\textwidth]{figures/png/image.png}
                \caption{Fully assembled panel (Lab3 Fermilab Facility).}
                \label{fig:gonzalo}
                \end{figure}

            The straws are filled with a gas mixture of 80\%:20\% Ar:CO$_2$ at 1 atm.
            Since the DS internal volume is evacuated to 10$^{-4}$ Torr, 
            the straws must endure such pressure difference. Under normal temperature and 
            pressure conditions, the panel must have an average leak rate below 0.014 cm$^3$/min.  
            The nominal operating voltage of the straw is 1450 V. 
            Preliminary tests showed that the straw 
            tube gain is 1.25 $\times$10$^4$ at 1250 V and 7 $\times$ 10$^4$ at 
            1425 V. According to Diethorn's calculation, Equation \ref{XXX}, 
            the gas gain at 1450 V is around 1 $\times$10$^5$.
 

All panels are x-ray scanned 
to accurately measure and document the wire 
locations of straw tube channels. 

Three panels rotated of 120° make a face and two faces rotated of 
30° make a plane (Figure \ref{fig:trkpanel} (Upper right-left) and \ref{fig:trueplane}).
Although the rotation of 30° between the two faces introduces more mechanical complexity if 
compared to a rototation of 60°, this configuration provides the better stereo performance.

Once a plane is assembled, a cooling ring is fitted around its outer circumference. 
The front-end electronics is hosted in a ring around the outer 
circumference of the panel.

Two identical planes rotated of 180° around a vertical axis make one station 
(Figure \ref{fig:trkpanel} (Upper right-right)). 
During assembly, the second plane is rotated of 180° around the vertical axis.

At the moment of writing this Thesis, almost all the tracker planes have 
already been assembled at Fermilab Lab3 Facility and station assembly is now work in progress, 
as shown in Figure \ref{fig:trueplane}.

\begin{figure}[!h]
    \centering
    \includegraphics[width =0.65\textwidth]{figures/png/Screenshot_20240706_163056.png}
    \caption{One tracker plane fully assembled at Fermilab Lab3 Facility.}
    \label{fig:trueplane}
\end{figure}
\subsection{The entire straw-tracker detector}

The entire detector is made of 18 stations (Figure \ref{fig:trkpanel} 
(bottom)) assembled together with a complex mechanical structure. 
Horizontal beams maintain longitudinal alignment of the rings. 
A thicker ring and two thinner rings placed at the downstream end 
of the detector stiffen the structure. The beams and the stiffening rings 
are made of stainless steel.

The detector rests on four bearing blocks attached to the 
stiffening rings and placed near the horizontal beams.
The connection between the detector and the bearing blocks 
is kinetic, to avoid over-constraining and distorting the frame. 
The only constraint of the four points is along the vertical 
direction, and vertical adjustment screws allow to level and 
center the frame. Once fully assembled, a thorough mechanical 
survey of the detector will be performed before moving to the Mu2e experimental hall. 

One feature of the station that will be relevant for this Thesis 
and will be discussed in more detail in \ref{planning}, is the layout of the gas system. 
Figure \ref{fig:gassystem} shows the layout of the gas tubes 
placed on the external surface of the station. 
This specific layout makes orienting the assembled 
station horizontally particularly challenging, since the gas tubes could be damaged.

\begin{figure}[!h]
    \centering
    \includegraphics[width =0.3\textwidth]{figures/png/Screenshot_20240706_163631.png}
    \caption{Gas system layout in the assembled station.}
    \label{fig:gassystem}
\end{figure}

\subsection{The tracker Front-End Electronics}\label{tfee}
The front-end electronics performs amplification, digitization and data packaging 
for transmission to the Data Acquisition System (DAQ) of the straw-tube signals. 
In particular, each single straw provides two hit times and one waveform, 
which are necessary to reconstruct the hit position along the straw, 
thereby allowing the use of costant fraction discrimination and mitigating time walk effects. 
A schematic representation of the entire system logic is reported in Figure \ref{fig:flowfee}.
\begin{figure}[!h]
    \centering
    \includegraphics[width =0.9\textwidth]{figures/png/Screenshot_20240529_133230.png}
    \caption{Signal flow through front end electronics, Ref. \cite{bartoszek2015mu2e}.}
    \label{fig:flowfee}
    \end{figure}
All the front-end electronics is installed on multiple Printed Circuit Boards 
(PCBs) mounted on the outer section of the panel. 
The PCBs are shown in Figure \ref{fig:trackerfee}, Ref. \cite{vadimmu2e}.
All PCBs are situated in the outer section of the panel. 
The panel has two sides, named high-voltage side (HV-side), 
which handles the high-voltage supply distribution to the straws, and 
calibration-side (CAL-side), that contains a circuitry that can inject calibration
pulses to the channel.

Signals from the straw tube channels are first read 
out from both ends by the pre-amplifiers (preamps).

Both sides of the panel have one Analog Mother Board (AMB) and one Jumper board, 
which task consist of directing signals from the preamps towards the Digital 
Motherboard (DMB) positioned at the center, then to the Digitizer Readout and 
Assembler Controller (DRAC) (mounted on top of the DMB) to be processed and 
temporarily stored. 

Moreover, the AMBs and the DMB handle 
low voltage distribution and the AMB on the HV side is responsible for 
distributing the high voltage to the straw anode wires
The low voltage power supply is connected to the panel through the KEY board which 
contains also an optical fiber link and a JTAG inferface for communication.
The two boards are also equipped with sensor to monitor environmental parameters, 
including temperature, pressure and humidity.

In addition, the frontends components were chosen to sustain high level of radiation.

The first step of signal processing is the signal pre-amplification performed 
by preamps mounted on two AMBs placed on the two lateral sides of the panel. 
The analog signals are transmitted from the preamps via 
microstrip transmissione lines and through a Jumper board to the DRAC.
Most of the data processing functions are performed by the DRAC board. 
The analog signals are processed by the Time-to-Digital Converters 
(TDCs) and to Analog-to-Digital Converters 
(ADCs) installed on the DRAC. Digitized data are then transferred to the Readout 
Controller (ROC) for data packaging and transmission to the Mu2e DAQ 
(Figure \ref{fig:flowfee}). 

To summarise, each straw-tube is an independent detector channel and it is equipped with:
\begin{itemize}
    \item two preamp channels, one for each end;
    \item two TDC channels, one for each end (192 in total);
    \item one ADC channel, measuring sum of both ends (96 in total);
    \item one High Voltage feed.
\end{itemize}
\begin{figure}[!h]
\centering
\includegraphics[width =0.8\textwidth]{figures/png/Screenshot_20240131_111836.png}
\caption{Overview of the straw-tracker front-end electronics, Ref.  
\cite{vadimmu2e}. The CAL-Side of the panel is not shown.}
\label{fig:trackerfee}
\end{figure}


\subsubsection{The pre-amplifiers}\label{preampss}
All the 96 straw-tubes are read out from both ends. 
Two adjacent tubes are connected to the same preamplifier. 
Each panel is thus equipped with 48 preamplifiers on the HV 
side and 48 preamplifiers on the CAL side. The preamplifiers 
are mounted veritcally on the AMBs. To minimise signal reflections, 
the preamplifiers have a matching 300 $\Omega$ input impedance. 
The function of the preamplifiers is to convert the straw tubes 
current signals into voltage signals which are amplfied and shaped. 
There are some differences between the preamplifiers on the CAL and 
HV sides. The CAL preamplifier can inject a calibration pulse into 
the channel, which allows to test the readout electronics without a 
high voltage source. The HV preamplifier performs the high voltage 
distribution to the straw tube.
%The voltage gains of individual channels, different from the gas gain, are set by control signal from the DRAC. A bias voltage, adjustable for each side of a channel, is applied to the signals. 
\subsubsection{The Digitizer Readout and Assembler Controller board}\label{DRAC}
The $brain$ of the panel is called Digitizer Readout and Assembler Controller (DRAC) board. 
The DRAC performs digitization, packaging, temporary storage and data transfer to 
the Mu2e DAQ system. It also controls all the panel operations The schematics of the 
Figure \ref{fig:drac} shows a picture of the board: a battery of comparators 
and ADCs, three DDR3 memories and three Field-Programmable Gate Arrays (FPGAs) 
are clearly visible. 
The two FPGAs on the left and on the right are named digi-FPGAs: 
each of them receives the data from 48 straw channels, performs data 
monitoring, buffering and assembles data packets which are then 
transferred to the central FPGA named Readout Controller (ROC) \ref{ROC}. 
The ROC handles communication, monitors slow control variables and 
controls all panel operations.

\begin{figure}[!h]
\centering
\includegraphics[width =\textwidth]{figures/png/Screenshot_20240204_115052.png}
\caption{DRAC board schematics, Ref. \cite{drac}. 
The DRAC board is the brain of the tracker panel. ADCs, FPGAs, DDR3 
memories and compators are also shown.}
\label{fig:drac}
\end{figure}
Figure \ref{fig:flowfee} shows the data flow through the panel front-end and the DRAC:
\begin{itemize}
    \item The two signals from the two sides of a straw tube are transmitted 
    from the preamps to the DRAC;
    \item In the DRAC, the two biased signals are fed to zero-crossing comparators, 
    which generate square pulses if the signals are above their respective thresholds;
    \item The squared signals are transmitted to 16-bit TDCs implemented 
    in firmware in the digi-FPGAs. Timing digitization is performed at the rate 
    of 20 ps per count, including the determination of the arrival time and the 
    time over threshold. In addition to the drift time, the TDCs 
    measure the time difference between the two signals received from the 
    same straw to dermine the hit position along the straw. With an 
    intrinsic TDC resolution of 25 ps, taking into account also the comparator jitter, 
    the noise and other external effects, the resulting time resolution 
    is of the order of 70 ps. At run time, a hit on a straw 
    is considered only if both ends of the same straw simultaneously have a 
    signal above threshold;
    \item The two signals from the two ends of the same straw are also 
    added by an integrator. The total sum is digitized by a 12-bit 
    (10-bit ENOB) ADC at 40 MHz and then transmitted to the digi-FPGA;
    \item The digi-FPGA creates one data packet for each hit based on the 
    TDC and ADC information. This allows to suppress false triggers caused by 
    random electrical noise;
    \item The data packets are transferred to the ROC \ref{ROC}, and 
    temporarily stored in the DDR3 memories for later access by the DAQ system. 
    There is a total of 8 Gb of memory space available on each DRAC.
\end{itemize}
\paragraph{The $Read \ Out \ Controller$}\label{ROC} 
The main function of the ROC is to collect data from the digitizer boards 
digi-FPGAs, buffer data and transfer them to DAQ. They continuously stream out 
the zero-suppressed data collected between two proton pulses to the DTCs, 
Ref. \cite{GIOIOSA2023167732}. The buffer stage is fundamental, 
since during the beam inter-spill time (836 ms out of each 1333 ms), 
we want us to be able to take data from cosmic rays, even if the rate 
will be very low. For this purpose, the ROCs include external DRAM. 
The communication is flexible, thanks to the 
programmable nature of digitizer, ROC and DAQ.
%It is important to save also the voltage signal, since the pulse height 
%information can help us to reject proton background, a significant source of noise 
%or to distinguish muons from electrons. The proton signal will appear as a saturated flag, 
%since the proton $dE/dx$ is $\times$50 the electron one.

\subsubsection{The Data Acquisition System}\label{tdaqtra}

The Mu2e Data Acquisition system (DAQ) is based on a $streaming$ readout architecture: 
all detector data are digitised, zero-suppressed in the front-end electronics 
and then transferred from the detectors for further processing and storage. 
This architecture results in a large data throughput in the DAQ but offers a 
significant flexibility for data selection and analysis. Moreover, it makes 
the DAQ architecture simpler.

Figure \ref{fig:linktodaq}, Ref.\cite{GIOIOSA2023167732}, shows the global 
architecture of the Mu2e DAQ. 
The Readout Controllers of all the detectors are shown 
on the left side. The central box includes the main DAQ 
components: the Run Control Host, 40 DAQ servers, the 
Detector Control System (DCS), and the Event Building 
Switch. The Mu2e data rates will be substantial: for 
efficient data handling this segment of the DAQ heavily 
relies on firmware implementation. The right box includes 
the offline components, which perform data storage and 
offline processing. 

DAQ operation during data taking will be coordinated by the Run Control Host, 
which will manage a predefined Run Plan. During an active spill (approximately the first 
43 ms in the 48 ms bunch extraction cycle described in Chapter \ref{mu2echapter}), 
the experiment receives RF Zero-Crossing Markers from the Accelerator 
synchronised to the 1695 ns proton pulse cycles. This defines the 
event window\footnote{Event windows are also assigned during off-spill periods 
when no beam batches are received. 
These event windows are longer due to the reduced data rate.}. 
Based on these markers, the Command Fan-Out (CFO) module within the 
Run Control Host generates a 40 MHz system clock and embeds Event 
Window Markers (EWMs) into this clock to denote the start of event 
windows. The CFO distributes this encoded system clock and run control 
packets to the Data Transfer Controllers (DTCs) in the DAQ servers. 
DTCs then pass the encoded clock to the ROCs within the detectors, 
where the EWMs are recovered with a fixed latency relative to the 
initial RF Zero-Crossing Markers. The local ROCs use the EWMs to distinguish 
data collected during consecutive event windows. For the tracker, 
this involves generating a DDR3 memory address at the start of each 
event window to store tracker hits recorded during that period. 

Data readouts from the ROCs to the DAQ system are triggered by Data Requests. 
The Data Requests can be configured via the CFO as per the Run Plan and are 
typically issued to the tracker and calorimeter ROCs following each event 
window through the DTCs. Conversely, the CRV Data Requests are issued via 
software. The ROCs respond to these Data Requests by transmitting the 
corresponding data to the DTCs. Data from multiple DTC sources are routed 
through the Event Building Switch to a designated DTC, pre-processed, and 
then transmitted to the Online Processing module within the same DAQ 
server\footnote{The DTCs are implemented as commercial PCIe cards located 
within the DAQ Servers.}. If a trigger decision is made, the corresponding 
CRV data are also read out from the ROCs through software-generated Data 
Requests. Event data from all detectors are collected at the Data Logger 
and subsequently transferred to long-term storage for future offline 
analysis. Calculations estimate a data rate of 35 GBps (with a 30\% overhead) 
from the ROCs and an annual storage requirement for the entire experiment of 
approximately 7 PB. 

In addition to detector data, the DTCs also handle the acquisition of 
slow control variables from the detectors. For the tracker, these variables 
include panels temperature, pressure, humidity, voltages, and currents at 
specified PCB nodes, high voltages, and channel threshold settings across 
all panels. These data are managed by the DCS Host and recorded in long-term 
storage. The data are also accessible in real- time in the Control Room for 
monitoring purposes. Other detector control tasks and operations are managed 
through DCS commands via the DTCs. For the tracker, this includes panel 
configuration, calibration using preamp-generated pulses, and disconnection 
of channel high voltage when necessary. 

\begin{figure}[!h]
    \centering
    \includegraphics[width =0.8\textwidth]{figures/png/Screenshot_20240206_144803.png}
    \caption{Mu2e Data Acquisition system architecture, Ref. \cite{GIOIOSA2023167732}.}
    \label{fig:linktodaq}
    \end{figure}


\subsection{Requirements on tracker performance}
Here a resume of the requirements that the tracker must satisfy to ensure the 
success of the experiment is presented, Ref. \cite{trkreq}.
A momentum resolution less than 180 keV/c for a 105 MeV/c electron is needed in the nominal
1 T solenoidal field, as measured at the front face of the tracker volume (before
passing through any tracker related material). Non-Gaussian tails, particularly any
high-side tail, must be controlled such that the DIO background results in much less
than one event at design sensitivity. To reach this, simulation results indicate that 
a single straw requires around 4 cm of longitudinal and 200 $\mu$m of transverse 
resolution for drift path lengths. 
It must have an acceptance of approximately 20\% for conversion electrons.
The tracker must operate in an ambient vacuum (< $10^{-4}$ Torr).
It should be able to withstand a rate of 5 MHz per straw (highest rate straw) 
500 ns after the spill. This is
for background studies. The nominal experiment live time starts 700 ns after the spill.
\subsubsection{Momentum scale calibration}
One of the most challenging aspects of the Mu2e experiment is achieving the 
required momentum resolution to distinguish the CE electron from the background.

Mu2e requires at least one source of high-energy calibration electrons or 
positrons to independently measure the absolute momentum scale of the tracker. 
The momentum scale calibration relies on accurately reconstructing the 
$\pi^+ \rightarrow e^+ \nu_e$ peak, with its success critically dependent on the 
suppression of $\mu^+ \rightarrow e^+ \nu_e \bar{\nu}_\mu$ decays in flight (DIF). 
Several approaches have been explored, with the most promising ones being:

\begin{enumerate}
    \item Special run with adjusted TS collimators: operating with 
    reversed TS collimators and a reduced DS field 
    (70\% of the nominal value), this setup is designed to capture the 
    leptonic decay ($\pi^+ \rightarrow e^+ \nu_e$) of stopping $\pi^+$ in the nominal stopping target;
    
    \item Low intensity run with reduced DS magnetic field: 
    running at low intensity and a further reduced DS field (50\% of the nominal value), 
    this approach aims to gather high-statistics data to accurately fit 
    the Michel edge of the DIO momentum distribution.
\end{enumerate}

Several important considerations include:

\begin{itemize}
    \item Early time measurement: to effectively 
    reconstruct the peak in the momentum distribution 
    of positrons from $\pi^+ \rightarrow e^+ \nu_e$ decays, 
    measurements must be performed at early times, approximately 
    $T \sim 300$ ns. This necessitates operating at a reduced proton 
    beam intensity to mitigate pileup concerns;
    
    \item Special instrumentation requirement: to minimize 
    background from muon decays in flight, the calibration demands 
    the installation of a beam degrader in the DS.
\end{itemize}

The momentum scale calibration described involves two key measurements: 
\begin{enumerate}
    \item The edge of the positron momentum spectrum from $\mu^+ \rightarrow e^+ \nu_e \bar{\nu}_\mu$ decays at $B = 0.5$ T;
    \item The $\pi^+ \rightarrow e^+ \nu_e$ peak at $B = 0.7$ T.
\end{enumerate}

Each of these measurements determines the momentum scale at their respective 
magnetic field values. By combining these measurements, we can extrapolate the 
momentum scale to the nominal magnetic field, $B = 1.0$ T. This concept is illustrated in Figure \ref{fig:momscale}.

\begin{figure}[h]
\centering
\includegraphics[width=0.7\textwidth]{figures/png/Screenshot_20240707_180742.png}
\caption{Illustration of the momentum scheme calibration.}
\label{fig:momscale}
\end{figure}

The calibration scheme assumes that the magnetic field is initially set based 
on readings from an NMR probe, and any necessary scale corrections can be 
effectively managed by rescaling the value of the nominal magnetic field.