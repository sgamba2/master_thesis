\chapter{First steps towards the station calibration}\label{planning}
\textit{Following the comprehensive testing of the DAQ and FEE, the subsequent crucial 
step towards achieving a fully operational tracker is the calibration phase. This 
calibration will be carried out exploiting cosmic muons. The primary objective of 
this calibration process is to accurately determine signal propagation times and 
channel-to-channel delays within each straw. To accomplish this, it is essential 
to perform an unbiased reconstruction of the longitudinal position of the hits 
within the straws. This would be easily achieved using the station orientated horizontally. 
However, certain technical and mechanical contingencies prevent horizontal calibration, 
necessitating the exclusive focus on vertical orientation.
In this Chapter, I am focusing on 
reconstructing the trajectories of cosmic muons within a vertically oriented 
station, as well as analyzing potential biases and systematic errors that could 
arise from this particular orientation.}
\section{Overview of the timing calibration}
As mentioned in the introduction to this chapter, the main 
goal of this calibration is to determine signal propagation 
and channel-to-channel delays for each straw.
When a particle creates a signal in a straw, it propagates 
on the central wire towards both ends. 
The arrival times at the ends will be called $t_1$ and $t_2$, 
which are given by the following equations:
\begin{equation}
\begin{aligned}
    t_1 &= t_0 + \frac{x_{\text{track}}}{v} + t_d + d_1 \\
    t_2 &= t_0 + \frac{L - x_{\text{track}}}{v} + t_d + d_2
\end{aligned}
\end{equation}
where $t_0$ is the muon arrival time, $t_d$ is the drift time, 
$L$ is the length of the straw, $x_{\text{track}}$ is the 
reconstructed track position on the wire, $v$ 
is the propagation velocity of the electric signal along the central 
wire and $d_i$ are the delays introduced 
introduced by the front end electronics. 
$t_1$ and $t_2$ are measured  
by the TDC timing on the HV side and on the CAL side.
The measurement of $\Delta t_{12}$ allows 
determining the position $x_{\text{track}}$ that 
the particle has passed, 
while $(t_1 + t_2) / 2$ allows measuring, 
up to an offset common to all channels, the drift time. 
If we subtract and add the two equations, we obtain:
\begin{equation}\label{ffffff}
    \begin{aligned}
        \Delta t_{12} &= \frac{2x_{\text{track}}-L}{v} +(d_1-d_2)  \\
        \frac{(t_1 + t_2)}{2} &= t_d+t_0+\frac{d_1+d_2}{2}-\cfrac{L}{2v} 
    \end{aligned}
    \end{equation}
The first equation contains the time 
difference $\Delta t_{12}$ between the 
signals at the two straw ends as measured 
by the TDCs, the reconstructed track coordinate 
$x_{track}$ along the wire, the propagation 
velocity $v$ of the signal along the wire and 
the difference between the delays introduced by 
the front-end electronics.
This type of calibration will be 
achieved using cosmic rays. 
To determine $v$ in the first equation of \ref{ffffff}, 
an unbiased reconstruction of the 
track projected on the wires is needed.
Since the station is not yet calibrated, 
it is only possible 
to use the information whether a straw 
has been hit or not. 
The station's horizontal orientation 
facilitates unbiased 
reconstruction, since cosmic muons 
are distributed according to 
cos$^2\theta$ and they are predominantly vertical 
(Section \ref{distcos}), 
thus crossing the straws mostly perpendicularly.
However, vertical orientation is preferred 
for some gas system contingencies 
\ref{gassystem} and because the station will 
be vertical during the experiment. 
In the following sections, I will 
introduce the simulation performed to 
reconstruct cosmic tracks with 
a vertically oriented station, aiming 
to understand possible biases in 
determining longitudinal position 
caused by the non-uniform illumination of a panel.
\subsection{Challenges with the horizontal station orientation: gas distribution system and other constraints}\label{gassystem}
The natural choice for calibrating the station would 
be to place it horizontally, as previously explained. 
However, there are several challenges with this orientation, 
one of which is the gas distribution. The gas distribution 
system of the panel is equipped 
with two check valves: the first one is located on 
the supply side, the second one on the exhaust side. 
These valves operate using a simple mechanism where 
a ball is pressed onto an O-ring to form a seal. The 
only operational mode for these valves is to be oriented 
vertically to ensure proper sealing. In our specific 
case, this is not an issue for the check valves on the 
supply side, since they should be kept open during 
calibration with the gas flowing. On the other hand, 
the check valves on the exhaust side of the panel must 
be closed at all times to prevent gas from leaking before 
reaching the panel. 
Figure \ref{fig:gassystem} shows a schematic 
representation of the operational mode of the supply 
and exhaust valves. The red arrows represent the 
direction of the gas flow. On the supply side (Figure 
\ref{fig:gassystem} (Left)), the ball has to rest on the 
O-ring, or the gas flows downwards towards the atmosphere, 
rather than flowing upwards towards the panel. According 
to these specifications, it seems that the only operational 
mode of the station is in the vertical position, which is 
the actual orientation of the stations in the fully assembled 
tracker. 
Additionally, other reasons include space constraints, as placing 
a station horizontally takes up significantly more room space than 
vertically. Moreover, the station is quite fragile, and 
excessive movement could lead to potential issues.



\begin{figure}[!h]
    \centering
    \includegraphics[width =0.4\textwidth]{figures/png/gassystem.png}
    \caption[Schematic view of the valves located in a tracker panel.]{Schematic view of the valves located in a tracker panel.}
    \label{fig:gassystem}
\end{figure}
To ensure optimal performance during Mu2e data-taking, it's essential to 
conduct the initial calibration with the system in a vertical orientation. 
This helps us achieve conditions that closely 
match those of the actual detector. However, if the vertical 
orientation doesn't prove useful for calibration, 
we will need to explore new mechanical solutions.

For this reason, in this Chapter I will report on a simulation study 
which shows that the vertical orientation introduces a significant bias on 
the $x_{track}$ which does not allow to perform the station 
calibration. 



\section{Cosmic muons as a calibration source}
Cosmic muons will play a fundamental role in the 
calibration of the entire Mu2e detector system and, 
in particular, the tracker. They have unique 
characteristics that make calibration with cosmic rays 
complementary to other techniques:

 

\begin{itemize}
    \item calibration samples with cosmic muons 
    can be taken during standard detector operations, 
    with the same detector conditions as the physics samples;
    \item the cosmic muons flux 
    ($\sim 1 \ \text{cm}^{-2} \text{min}^{-1}$ 
    for horizontal detectors with a mean
    energy of $\sim$4 GeV, Ref. \cite{muonflux}) 
    is sufficient large to allow gathering a 
    substantial amount of 
    physics data in a relatively short time and 
    thus monitoring continuously the detector response;
    \item a cosmic muon is a minimum ionizing 
    particle (MIP) and 
    its energy loss is almost 
    independent of its energy;
    \item the speed of the cosmic muons is equal to the 
    speed of light $c$, and the time they take to traverse 
    a detector, in our case one tracker station, can be 
    used to align the time offsets of all the channesl 
    without any external time reference.
\end{itemize}
\subsection{Cosmic muons energy and angle distributions}\label{distcos}
The muon flux at sea level is usually described by the standard Gaisser's formula, Ref. \cite{guan2015parametrization}:
\begin{equation}
    \frac{d I}{d E_\mu d \Omega d t d S}=\frac{0.14}{\mathrm{~cm}^2 \mathrm{~s} \ \mathrm{sr}}\left(\frac{E_\mu}{\mathrm{GeV}}\right)^{-2.7} \quad\left[\frac{1}{1+\frac{1.1 E_\mu \cos \theta}{115 \mathrm{GeV}}}+\frac{0.054}{1+\frac{1.1 E_\mu \cos \theta}{850 \mathrm{GeV}}}\right]
    \end{equation}
where $E_\mu$ is the muon energy and $\theta$ is the polar angle of the muon. 
The two terms in brackets correspond to the contribution of the charged pions and kaons, while 
the small contribution from charm and heavier flavors is neglected. 
This simplified formula doesn't take into account muon decays and the curvature of the Earth, 
thus it is only valid for zenith angles $\theta < 70^\circ$ and for energies $E > \frac{100}{\cos \theta}$ GeV.
A modified version of the standard Gaisser formula, called Gaisser Tang model, is used to account for low energy and large zenith angle effects, Ref. \cite{guan2015parametrization}:
\begin{equation}\label{cosmicmuonflux}
    \frac{d I}{d E_\mu d \Omega d t d S}=\frac{0.14}{\mathrm{cm}^2 \mathrm{~s} \ \mathrm{sr} }\left( \frac{E_\mu}{\mathrm{GeV}} \left(1+\frac{3.64 \mathrm{GeV}}{E_\mu\left(\cos \theta^*\right)^{1.29}}\right)\right)^{-2.7}\left[\frac{1}{1+\frac{1.1 E_\mu \cos \theta^*}{115 \mathrm{GeV}}}+\frac{0.054}{1+\frac{1.1 E_\mu \cos \theta^*}{850 \mathrm{GeV}}}\right]
\end{equation}
where the second term in the bracket is the same as in the standard formula, except that the zenith angle 
$\theta$ is substituted by the angle $\theta^*$. The relation between cos$\theta$
and cos$\theta^*$ is given by:
\begin{equation}
    \cos \theta^*=\sqrt{\frac{(\cos \theta)^2+P_1^2+P_2(\cos \theta)^{P_3}+P_4(\cos \theta)^{P_5}}{1+P_1^2+P_2+P_4}}
    \end{equation}
The parameters $P_1$ = 0.102573, $P_2$ = -0.068287, $P_3$ = 0.958633, $P_4$ = 0.0407253, and $P_5$ = 0.817285 were calculated using 
a dedicated simulation of muon production in the atmosphere. A representation of the angles is shown in Figure \ref{fig:anglesinmuon}.
Another factor is included in Equation \ref{cosmicmuonflux} to account for the possibility of muon decays, which are more 
significant at low energies. The numerical constants are derived by fitting experimental data from various cosmic muon studies.
\begin{figure}[!h]
    \centering
    \includegraphics[width =0.6\textwidth]{figures/png/Screenshot_20240526_140716.png}
    \caption[The relation of $\theta^*$ to $\theta$.]{The relation of the observed zenith angle of muons, $\theta^*$, to 
    the zenith angle at the muon production point in the atmosphere, $\theta$. 
    $R$ is the radius of the Earth, Ref.\cite{guan2015parametrization}.}
    \label{fig:anglesinmuon}
\end{figure}

\subsection{Monte Carlo cosmic generation with CRY}
Several Monte Carlo programs allow to simulate sea-level 
cosmic ray mouns: the most widely used is CRY, developed 
by LANL. CRY functions as a generator for air showers 
induced by primary cosmic rays. The CRY package uses 
precomputed input tables derived from comprehensive 
MCNPX 2.5.0 simulation of protons in the energy range 
between 1 GeV and 100 TeV, at the top of the atmosphere.
The generation of muons and other secondary particles 
is governed by the pion and kaon decays. The package 
generates the cosmic muon flux within a zenith angle range 
of 0-90°, following a cos$^2 \theta$ distribution, and an 
energy range between 1-100 GeV, following the Gaisser 
Tang parameterization.The CRY package accounts for the 
dependence of cosmic muon flux on various parameters, 
including altitude, latitude, and solar activity. CRY 
provides muon flux data at three different altitudes: 
sea level, 2100 m, and 11300 m, with sea level being 
selected for this study. The latitude is set at 41.8° N, 
corresponding to Fermilab. To avoid variations of the 
primary cosmic muon flux due to solar activity, I chose 
a common day (6-21-2021) outside the maximum and minimum 
sunspot cycle for the simulation. While there are other 
packages designed to produce cosmic muons with specific 
energy and zenith angles, such as CORSIKA, I chose CRY 
for its straightforward implementation and compatibility with GEANT4. 

\subsection{Monte Carlo sample and coordinate system }\label{genplane}
Cosmic muons are uniformly generated within a horizontal plane approximately 11 m 
above the beam axis. The GEANT4 simulation also incorporates the effects of the external neutron shield surrounding the DS 
and the concrete ceiling of the Mu2e experimental hall. Starting from the generation plane, 
cosmic muons pass through the concrete ceiling and walls situated outside the experimental hall. 
As they propagate downwards to the detector, they interact with the materials within the building. 
The external neutron shield surrounding the DS consists of concrete ($\rho$=2.3 g/cm$^3$) and has a thickness of 
roughly 0.9 m, while the ceiling above the Mu2e experimental area comprises 1.8 m of concrete. 
The entire structure is enclosed within a volume of air at standard temperature and pressure.

The station was simulated in an extracted position, 
which means it was placed outside the solenoid. 
Given that the Earth's magnetic field is approximately 
$4 \div 5 \times 10^{-5}$ T, 
and considering that the dimensions of the station are 
on the order of 1-2 m, 
along with the fact that the majority of particles 
under consideration have energies around $\sim$ 10 GeV, 
the curvature radius of the particles is approximately 
$R\sim p[\text{GeV}]/(B[\text{T}]\cdot 0.3) \sim 30$ m. 
This radius is significantly larger than the station's 
dimensions, allowing us to reconstruct particle trajectories 
as straight lines.
For this reason, in the simulation we used a 
magnetic field intensity equal to zero. 

Before starting explaining the event selection and the 
reconstruction, let me define the coordinate system, 
as shown in Figure \ref{fig:coordinate}.
\begin{figure}[!h]
    \centering
    \includegraphics[width =0.5\textwidth]{figures/png/Screenshot_20240526_164527.png}
    \caption[Schematic view of a tracker station and the coordinate system.]{Schematic view of a tracker station and definition of the coordinate system used for the analysis.}
    \label{fig:coordinate}
\end{figure}
\section{The events selection}\label{eventselection}

I adapted the Monte Carlo to focus solely on 
events where a cosmic muon hits only the first 
tracker station.

Each straw can be parametrized as a straight line 
lying on a plane $z = z_i$, where $z_i$ is the $z$ 
coordinate of the panel which contains the straw:
\begin{equation}\label{equaretta}
    (D_{x,i}t+M_{x,i},D_{y,i}t+M_{y,i},z_i)
\end{equation}
where $D_{x,i}$, $D_{y,i}$ are the $x$ and $y$ 
$i$-th straws' directions, while $M_{x,i}$, $M_{y,i}$ 
are the $x$, $y$ straws' midpoints. $z_i$ is the 
$z$ coordinate of the $i$-th straw.
%One dimension is correlated with the other two, so we have just two informationful dimensions.
In this parametrization, the straws are all parallel 
on the $y-z$ plane.

To conduct this study, it is necessary to 
reconstruct the 3D trajectory of the cosmic muon, 
which requires at least two 3D points along the muon's 
trajectory. These two 3D points can be 
determined by the intersection of the 
$x$ and $y$ coordinates from two pairs 
of non-parallel straws. Since all straws 
within a single panel are parallel to each other, 
more than one hit per face is needed. The $z$ 
coordinate is defined as the average of the 
straws' $z$ coordinates. To avoid selecting parallel straws, 
the first selection cut 
was to require at least one hit per face.

Since multiple hits can occur within a 
single panel due to signal spread, another 
selection criterion was to choose events 
with fewer than three hit straws per panel. 
This criterion was selected to minimize errors 
in track reconstruction.
\subsection{Panels illumination pattern}
To perform accurate timing calibration, 
the distribution of the hits should be 
as uniform as possible across the panel. 
Figure \ref{fig:illumination} 
shows the hit distribution after applying 
the selection cuts outlined in Section 
\ref{eventselection}, for panel 0 and plane 0, 
which corresponds to the panel in the 
upper right part of Figure \ref{fig:coordinate}. 
The hit distributions for all the panels in 
both planes are similar, as shown in Appendix \ref{appendix2}.
\begin{figure}[!h]
    \centering
    \includegraphics[width =0.8\textwidth]{figures/pdf/xp_vs_yp_panel0.pdf}
    \caption[Monte Carlo illumination across 
    the panel (panel 0, plane 0).]{
        The Monte Carlo illumination 
        across the panel (panel 0, plane 0).
        The selection requires cosmic muons which 
        traverse four faces within one station, with 
        fewer than three hits per panel.}
    \label{fig:illumination}
\end{figure}
Figure \ref{fig:illumination} shows a spotty and 
non-uniform panel illumination. This is due to the 
fact that the only 4/4 overlap regions are 
limited to the edges of the panels (darkest 
areas in Figure \ref{fig:coordinate}). 

Moreover, the 4/4 request selects specific 
muon directions in the $y$-$z$ plane. 
Figure \ref{fig:myz} shows the 
$m_{yz}=\Delta y /\Delta z$ distribution for the selected muons. 
As expected, there are no tracks corresponding 
to the $m_{yz}\sim 0$, which are horizontal tracks. 
The vertical tracks, which correspond to 
$m_{yz} \rightarrow \infty$, are also 
not selected by our algorithm, since the 4/4 
request is not satisfied by vertical muons. 
The consequence of the 4/4 request is also a 
significant reduction of the total rate, 
which will be further investigated in 
Section \ref{ratetracker}. 
The most preferred muon  
directions are those with 
$|m_{yz}| \sim 1$, corresponding to 
cosmics with 
an angle of approximately 45° 
from the vertical direction.
\begin{figure}[!h]
    \centering
    \includegraphics[width =0.8\textwidth]{figures/pdf/myz.pdf}
    \caption[The $y-z$ direction 
    distribution of muons satisfying 4/4 request.]{
        $m_{yz}=\Delta y /\Delta z$  
    distribution of muons satisfying 4/4 request.}
    \label{fig:myz}
\end{figure}

Additionally, Figure \ref{fig:illumination} 
shows that there are almost no hits in the 
panel central region, which could result in 
waveform non-linearities.
In particular, the signals generated by 
particles which pass near the end of 
the straw are wider, and wider signals 
have greater charge and height compared 
to the others. Consequently, the leading 
edge could be very sharp, potentially 
causing timing systematics. 

\subsection{Expected muon rate}\label{ratetracker}
The rate of cosmics generated on the horizontal 
plane above the beam axis can be computed by 
integrating the flux on the solid angle, on 
the muon energy described in Section \ref{genplane}:
\begin{equation}
   R= \int \frac{\text{d} I}{d E_\mu \text{d} \Omega \text{d} t \text{d} S} \text{d} S_{\perp} \text{d} \Omega \text{d} E_\mu
    \end{equation}
where $\text{d}S_\perp= \text{d}S \text{cos}\theta$ 
is the projection of the surface element on the 
plane perpendicular to the cosmics flux. 
The energy and angle ranges selected for this 
simulation are respectively 0.5 GeV < $E_\mu$ < 500 GeV, 
which selects the muons not absorbed in the 
concrete layer posed above the tracker, 
and 0° < $\theta^*$ < 90°. 
If we consider a generation plane of area  
$\sim 5$ m$^2$, the resulting rate is $R \sim 380$ Hz. 
%The rate of cosmic muons detected by each straw, $R_{straw}$, can be easily obtained as:
%\begin{equation}
%    R_{straw}=\frac{R}{(N_{faces}\cdot N_{panels}\cdot N_{straws} )}
%\end{equation}
%where $N_{faces}=4$ is the number of hit faces, $N_{panels}=3$ is the number of panels in one face and $N_{straws}=96$ is the number of straws in one panel.
%The resulting rate is about 0.5 Hz. This simple computation does not take into account the complicated tracker geometry and the station orientation.
%Let us normalize by a factor that is the ratio between the number of selected events and the total number of produced events, which is about 
%$\sim10^{-3}$. The rate then changes to $\sim$ mHz. 
%To perform a proper calibration, at least $\sim$1000 hits per straw are needed, which requires at least 8 days of data taking.
%This rate calculation is quite straightforward, but the duration required for such a simple calibration is notably long.
The number of generated events 
was chosen to be $N_{gen}\sim 2\times 10^8$, since the 
number of cosmics that 
hit a station is about 8$\times 10^5$ and 
this is a good number of events to perform 
a reasonable calibration.
Simulating 2$\times 10^8$ cosmic 
muons on the horizontal plane allows for an 
approximate estimate of the time necessary to 
collect a data sample sufficient to perform the 
calibration, that is
$\Delta t = N_{gen}/R\sim 6$ days, which is 
excessively long 
for such a simple calibration. 

\section{Reconstruction of a straight cosmic track}\label{reconstruction}
To determine $x_{track}$ in Equation \label{ffffff}, 
it is necessary to reconstruct the cosmic track 
starting from the stereo hit coordinates.
The only information about a straw are, as 
mentioned in equation \ref{equaretta}:
\begin{itemize}
    \item the straw direction $(D_{x,i},D_{y,i})$;
    \item the straw midpoints $(M_{x,i},M_{y,i})$;
    \item the straw $z_i$ coordinate.
\end{itemize} 
The reconstruction process begins 
with the sole information of whether 
a straw has been hit or not.
To improve the hit 
spatial resolution adjacent straws hit 
in a panel, which are
due to the same particle, are combined 
to form the object called $ClusterHit$. 
A maximum of three straws are used to reconstruct a 
$ClusterHit$.
This object effectively represents a 
straw with its direction aligned to the 
straw directions and its midpoint 
being the average of the hit straw 
midpoints in that panel. Since four 
panels in four different faces have been selected,  
four $ClusterHit$s are reconstructed 
within a station. The new $ClusterHit$ 
midpoints will be referred to as 
$(x_{m,i}, y_{m,i}, z_{m,i})$.
Two $ClusterHit$s from different 
faces within the same plane are 
combined to form a $StereoHit$ on 
the $x-y$ plane. 
The $x$ and $y$ coordinates of the 
$StereoHit$ are determined by the 
intersection point of the two 
$ClusterHit$s. The $z$ coordinate of 
the $StereoHit$ is the average of the 
$z$ coordinates of the $ClusterHit$s.
The situation is reproduced in Figure \ref{fig:stco}.
\begin{figure}[!h]
    \centering
    \includegraphics[width =0.6\textwidth]{figures/png/Screenshot_20240810_210144.png}
    \caption[Schematic view of a cosmic muon hitting the vertical oriented station.]{Schematic view of a cosmic muon (light blue) hitting the vertical oriented station. The red lines are the $ClusterHit$s, the dark blue dots 
    are the $StrawHit$s and the green dots are the $StereoHit$s.}
    \label{fig:stco}
\end{figure}
Considering only one plane and the following equations as the $x-y$ $Combo Hit$s equations:
\begin{equation}
    \begin{aligned}
        y&=\frac{v_1}{u_1}(x-x_{m,1})+y_{m,1} \\
        y&=\frac{v_2}{u_2}(x-x_{m,2})+y_{m,2} 
    \end{aligned}
    \end{equation}
where $u_i$ is the $ClusterHit$ $x$ direction and $v_i$ is the $ClusterHit$ $y$ direction, the $StereoHit$ coordinates will be:
\begin{equation}\label{x}
    \begin{aligned}
x&=\frac{u_1 u_2(y_{m,1}-y_{m,2}+\frac{v_2}{u_2}x_{m,2}-\frac{v_1}{u_1}x_{m,1})}{v_2 u_1 - v_1 u_2}\\
y&=\frac{v_2}{u_2}\left(\frac{u_1 u_2(y_{m,1}-y_{m,2}+\frac{v_2}{u_2}x_{m,2}-\frac{v_1}{u_1}x_{m,1})}{v_2 u_1 - v_1 u_2}-x_{m,2}\right)+y_{m,2}
\end{aligned}
\end{equation}
In two different planes, there are two $StereoHit$s available, allowing for the reconstruction of the track on $x-y$ and $y-z$ planes. 
Referring to these $StereoHit$s as $(x_1,y_1,z_1)$ and $(x_2,y_2,z_2)$ respectively, the final track slope on $x-y$ plane will be 
denoted as $m_{xy}$, while on the $y-z$ plane it will be denoted as $m_{yz}$. The corresponding $y$-intercepts will be $q_{xy}$ and $q_{yz}$.
The formulas for computing these parameters are as follows:
\begin{equation}
    \begin{aligned}
m_{xy}&=\frac{y_2-y_1}{x_2-x_1}\\
q_{xy}&=-m_{xy} \cdot x_1+y_1
\end{aligned}
\end{equation}
\begin{equation}
    \begin{aligned}
m_{yz}&=\frac{y_2-y_1}{z_2-z_1}\\
q_{yz}&=-m_{yz} \cdot z_1+y_1
\end{aligned}
\end{equation}
To determine the four hit positions of the reconstructed track on the panels, we need to intersect the track with the panels' $z_i$ coordinates. 
The reconstructed hits on the panels will be denoted as $R_{x,i}$ and $R_{y,i}$.
\begin{equation}
    \begin{aligned}
 R_{y,i}&=m_{yz}\cdot z_i+q_{yz}\\
 R_{x,i}&=\frac{(m_{yz}\cdot z_i+q_{yz}-q_{xy})}{m_{xy}}
\end{aligned}
\end{equation}
The bias on the longitudinal position is the difference between the reconstructed coordinate and the true Monte Carlo position in the panel frame.

\section{Results}

Our target is to achieve a longitudinal resolution better than 4 cm, thus ensuring that every bias on the reconstructed coordinate 
remains below this threshold. For each panel, I plotted the reconstructed longitudinal coordinate, as shown for panel 0 in Figure \ref{fig:recx}. 
Similar patterns were observed for all panels. In this distribution, the bumps are a consequence of the four-hit face requirement and the panels' overlap areas being located at the extreme edges. 
It is important to observe is that different bumps correspond to different straws, as can also be seen in Figure \ref{fig:illumination}.
\begin{figure}[!h]
    \centering
    \includegraphics[width=0.8\textwidth]{figures/png/x_panel0.png}
    \caption[The reconstructed longitudinal coordinate in the 0th panel frame.]{The reconstructed longitudinal coordinate in the 0th panel frame.}
    \label{fig:recx}
\end{figure}
The positional bias between the reconstructed and true longitudinal coordinates is plotted in 
Figure \ref{fig:bias}. The true coordinate is the mean coordinate of the Monte Carlo hits. The bias range is approximately [-6, 6] cm, 
indicating a significant systematic factor affecting the reconstructed hit. The distribution is nearly symmetric, similar for all panels. 
There are four peaks corresponding to the four different spots shown in Figure \ref{fig:illumination}.

\begin{figure}[!h]
    \centering
    \includegraphics[width=0.8\textwidth]{figures/png/panel_00_x_bias.png}
    \caption[The bias between the reconstructed and the true hit coordinate.]{The bias between the reconstructed 
    and the true longitudinal hit coordinate in the 0th panel frame. 
    The true coordinate is the mean coordinates of Monte Carlo hits.}
    \label{fig:bias}
\end{figure}
The 2D distribution of the longitudinal bias versus the true position, shown in 
Figure \ref{fig:rec2D}, shows four different spots on the $x$ axis corresponding to 
the overlap regions. Each spot refers to different straws. The two spots on the $y$ axis correspond to cosmics with different orientations. 
The first spot (CAL side) correspond to 90° panels overlap.
\begin{figure}[!h]
    \centering
    \includegraphics[width=0.8\textwidth]{figures/png/panel_00_x_bias_vs_x.png}
    \caption[The 2D histogram of the longitudinal bias versus the true coordinate.]{The 2D histogram of the longitudinal bias versus the true longitudinal coordinate.}
    \label{fig:rec2D}
\end{figure}
To assess the systematic impact on the reconstructed coordinates, it is essential to visualize the TProfile of the bias versus the true coordinates. 
Profile histograms are used to represent the mean value of $y$ and its error for each bin in $x$. 
The default displayed error is the standard error on the mean. 
The profile is depicted in Figure \ref{fig:profile}. It reveals a systematic effect in determining the 
longitudinal position within a range exceeding [-4,4] cm. The initial portion of the histogram (CAL side) 
corresponds to the overlap of 90° panels. This plot indicates that the mean may not serve as an accurate estimator of the bias. 
Under these conditions, the calibration is expected to become challenging.
\begin{figure}[!h]
    \centering
    \includegraphics[width=0.8\textwidth]{figures/png/panel_00_x_bias_vs_x_prof.png}
    \caption[The profile of the longitudinal bias versus the true longitudinal coordinate.]{The profile of the longitudinal bias versus the true longitudinal coordinate.}
    \label{fig:profile}
\end{figure}
The main issue with this type of reconstruction lies in the fact that $m_{yz}=\frac{\Delta y}{\Delta z}$ 
is not accurately reconstructed. Consequently, the reconstructed coordinates depend on this value, as demonstrated in \ref{reconstruction}. 
This discrepancy arises when the true hit position, far from the midpoint of the straws, results in incorrectly reconstructed tracks'
direction on the $y-z$ plane, leading to horizontal cosmic rays being reconstructed. 
Figure \ref{fig:myzrec} shows the reconstructed $m_{yz}$ distribution, highlighting a significant peak at the value of 0, 
indicating horizontal cosmic rays, which are not possible in reality. This plot needs to be compared with the true $m_{yz}$ distribution, Figure \ref{fig:myz}.


\begin{figure}[!h]
    \centering
    \includegraphics[width=0.8\textwidth]{figures/png/myz_rec.png}
    \caption[The reconstructed $y-z$ direction distribution of cosmic rays.]{The reconstructed $y-z$ direction distribution of cosmic rays ($m_{yz}=\Delta y /\Delta z$).}
    \label{fig:myzrec}
\end{figure}
\section{Conclusions}
This Chapter demonstrated that the vertical orientation for calibrating a station is not optimal. This is due to several reasons:
\begin{itemize}
    \item The selection criteria in Section \ref{eventselection} select cosmic muons with orientations that affect panel illumination. 
    The illumination is non-uniform, and moreover, there are almost no hits in the central region of the panel, which could result in waveform non-linearities;
    \item The rate of cosmic events is $R \sim 380$ Hz, and the expected duration of the calibration is about 6 days without interruptions, 
    which is excessive for the simple calibration we intend to perform;
    \item The bias range is approximately [-6, 6] cm. The 2D distribution of the longitudinal bias versus the true position shows 
    four distinct spots on the $x$ axis corresponding to the overlap regions. Each spot refers to different straws.
\end{itemize}
Therefore, an alternative orientation or method should be considered to optimize the calibration process and achieve more accurate results.

%yongi cap 3.1.5.2
%con il tracker in orizzontale , dato che le tracks sono verticali colpirebbero per panel sicuro almeno 2 straws che sono sovrappposte e quindi in totale 8, se hanno una certa angolazione anche 12






%yongi:
%As discussed in Chapter 3, the Mu2e Tracker panels need to satisfy a series of performance
%requirements for successful operations and the designed resolution. While various individual
%component and single panel tests in the past confirmed their respective effectiveness, it was of great
%interest to extend the tests to a system of multiple panels over a longer period of time under a more
%realistic setup (i.e., similar to that of the actual experiment setup), and to get better quantitative
%understandings of the detector performances. Hence, a VST was conducted.
%The overall goal of the VST was to “exercise the full tracker operation and readout chain,
%from the amplified wire signal to its storage in digitized format” [1]. The test used a whole plane of
%six panels from panel pre-production,1 which was 1/36 of the full Tracker.
%The VST contained two phases. In the first phase, the plane was laid horizontally on a
%test bench for diagnostics and horizontal position runs, as shown in Figure 5.1(a). In the figure
%the supporting systems, including gas lines, DC-to-DC converters for low voltage supplies, high
%voltage supplies, and the Raspberry Pi (marked as RPi) for panel controls, are all labeled.2 In this
%phase, the plane operations and performances were checked. The plane data output to the DAQ
%server through the optical fiber links (see Chapter 4 and Appendix B) were finalized and tested.
%Small cosmic-ray datasets werereconstruction of the track longitudinal position taken to verify the ability to synchronize data taking among the
%panels. And 55 Fe source scan studies were performed for calibrations. The second phase of the



%reconstruction of the track longitudinal position
%tracker/vstplan
%We also plan to take significant amounts of cosmic ray data in at least two configurations. First, cosmic ray data will be taken with the plane in the current horizontal
%position. This will be taken using a modified readout scheme involving the serial
%connection as described above. Software exists to take the output raw data and manually convert it into an art format that can be processed by the official Mu2e software
%packages. This first cosmic data will be used to confirm the functionality of the system - including a cross-check of the panel to panel time synchronization, and the
%overall timing measurement performance. 
%Additionally, enough statistics could allow
%for important preliminary calibrations of delta-t resolution, longitudinal propagation
%velocities, efficiencies, and channel to channel variations. The fairly uniform distribution and wide phase-space of tracks in the horizontal configuration can make these
%analyses more straightforward than in the vertical configuration. Additionally a better measure of the expected showering will allow an improved analysis of vertical data.
%Finally, it may be possible to make some measure of the difference between vertical
%and horizontal straw alignment compared to expectations from Duke measurements
%and gravitational sag
%Later, the plane will be positioned vertically and another month of cosmic data
%will be taken. At this point readout may be through the fiber and DTC for improved
%livetime. As described above we will move towards automatic processing of this
%data and have it copied and backed up to tape. This data will be most useful for
%testing the track reconstruction. There exists software for reconstructing cosmic
%tracks without a magnetic field, and it has been tested in a limited fashion with
%data from a single vertical panel (docdb-33914). The implementation for a single
%plane in either horizontal or vertical configuration is being developed.
%Finally, a more thorough calibration of the plane can be done using a systematic
%scan across each panel with an Fe55 source. This will allow further measurements of
%the response as a function of straw and position, and a determination of the variability
%between panels.