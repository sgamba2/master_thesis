\chapter{Pre-pattern recognition studies}\label{delta}
\textit{
The estimated data volume for Mu2e data-taking 
is at least 7 PByte per year: it will thus be crucial to 
exploit all the possible handles to optimize 
the CPU and memory usage. For example, the simulation 
shows that the primary source of hits in the 
Mu2e tracker will be low energy electrons and positrons, called $\delta$-electrons. 
Therefore, the Mu2e Collaboration has made 
a huge effort to develop solutions to identify 
and flag these hits as soon as possible in the 
data-taking and not use them in pattern 
recognition and track reconstruction. 
The most important constraint is 
making sure that hits generated by CE are 
not erroneously flagged as $\delta$-electrons, 
since this would compromise CE track reconstruction. 
There are currently two main algorithms being 
developed for this purpose. This Chapter reports 
on the first systematic study performed to compare 
the performance of the two algorithms and determine 
the best solution for data-taking. }



\section{$\delta$-electrons as source of background}

The performance of the detectors and the Mu2e 
physics reach have been thoroughly studied with 
the Monte Carlo simulation. In terms of 
occupancy, we know that the dominant 
source of hits in the tracker are low energy electrons and positrons,
in the following referred to as  
$\delta$-electrons. To be more precise, in 
decreasing order of importance, the primary sources of hits are 
electrons from Compton scattering, electron-positron pairs, 
and delta rays. Compton-scattered electrons are produced 
when photons, generated by various processes, 
interact with the detector material. These photons primarily 
originate from neutron captures, which excite the 
nuclei and lead  
to subsequent photon emission. Typically, 
these photons have 
energies of a few MeV. Neutrons are 
produced in the process of nuclear muon capture. Pair-production 
electrons and positrons are 
generated during nuclear  
processes, where pairs of electrons 
and positrons are created.
Delta rays, 
or secondary ionization electrons, 
are generated when high-energy 
charged particles interact with the detector material.


\subsection{Compton scattering}\label{compton}
The Compton effect (Figure \ref{fig:compt}) is the 
scattering of a photon by a free or quasi-free electron. 
An electron is considered "quasi-free" when the energy of 
the incoming photon is significantly higher than the 
electron's binding energy ($E_\gamma \gg E_B$). The 
process is termed Compton scattering if the 
electron is ejected from the atom, carrying away the 
recoil momentum. This effect is most prominent in an 
extended energy region around 1 MeV, with the region 
being much larger for low $Z$ materials compared to high $Z$ materials.

\begin{figure}[!h]
    \centering
    \includegraphics[width =0.4\textwidth]{figures/png/Screenshot_20240812_204345.png}
    \caption[The Compton effect.]{
    The Compton effect \cite{kola}.}
    \label{fig:compt}
\end{figure}

Since the photon scatters quasi-elastically off the electron, 
the energy and angle of the scattered photon are correlated. 
To describe this relationship, we use the 4-momenta defined as 
follows: $k = (E_\gamma, \mathbf{k}c)$ and $p_e = (m_e c^2, 0)$ 
represent the 4-momenta of the photon and the electron (at rest) 
before scattering, and $k' = (E'_\gamma, \mathbf{k}'c)$ and 
$p'_e = (E'_e, \mathbf{p}'_e c)$ represent the 4-momenta 
after scattering. The angle between the scattered photon and 
the incident photon is denoted as $\theta_\gamma$, while the 
angle of the electron is denoted as $\theta_e$. By applying 
energy-momentum conservation:

\begin{equation}\label{compcons}
k + p_e = k' + p'_e
\end{equation}
\begin{equation}\label{compcons2}
(k - k')^2 = (p'_e - p_e)^2 \Rightarrow -k \cdot k' = m_e^2 c^4 - p'_e \cdot p_e
\end{equation}
\begin{equation}
\Rightarrow E_\gamma E'_\gamma (1 - \cos \theta_\gamma) = m_e c^2 \left(E'_e - m_e c^2\right) = m_e c^2 \left(E_\gamma - E'_\gamma \right)
\end{equation}

The right-hand side of the last equation uses the kinetic energy of the electron:

\begin{equation}
T = E'_e - m_e c^2 = E_\gamma - E'_\gamma
\end{equation}

which follows from the energy part of equation \ref{compcons}. 
The energy of the scattered electron as a function of the photon scattering 
angle is derived from equation \ref{compcons2}:

\begin{equation}\label{diffeq}
E'_e = \frac{E_\gamma \cdot \epsilon \cdot (1 - \cos \theta_\gamma)}{1 + \epsilon (1 - \cos \theta_\gamma)}+m_e
\end{equation}

where $\epsilon = \frac{E_\gamma}{m_e c^2}$.

The differential cross section per (free) electron, known as the 
Klein-Nishina formula, is calculated using methods from quantum electrodynamics:

\begin{equation}\label{kleinnishina}
\frac{d\sigma}{d\Omega} = \frac{r_e^2}{2} \frac{1 + \epsilon (1 - \cos \theta_\gamma)}{[1 + \epsilon (1 - \cos \theta_\gamma)]^2} \left(1 + \cos^2 \theta_\gamma + \frac{\epsilon^2 (1 - \cos \theta_\gamma)^2}{1 + \epsilon (1 - \cos \theta_\gamma)} \right)
\end{equation}

An electron bound in an atom can only be considered quasi-free 
if the photon's energy is significantly higher than the electron's 
binding energy. As the photon energy increases, more shell electrons 
become quasi-free, leading to the Compton cross section per atom 
approaching proportionality to $Z$, with individual electrons 
contributing incoherently:

\begin{equation}
\sigma_C^{\text{atom}} = Z\sigma_C
\end{equation}

where $\sigma_C$ is the Klein-Nishina cross section for a 
single free electron. The Compton cross section decreases at 
lower energies, where coherent scattering (Rayleigh scattering) 
off the entire atom (without ionizing the electron shell) becomes dominant.

By reformulating the Klein-Nishina formula, one can obtain the 
differential dependence of the Compton cross section on the 
kinetic energy of the recoil electron $T = E_\gamma - E'_\gamma$:

\begin{equation}
\frac{d\sigma}{dT} = \frac{\pi r_e^2}{m_e c^2 \epsilon^2} \left[2 + \frac{t^2}{\epsilon^2 (1 - t)^2} + \frac{t}{1 - t}\left(t - \frac{2}{\epsilon}\right)\right]
\end{equation}

where $t = T/E_\gamma$. Because the scattering process is 
elastic, there is a one-to-one relationship between the 
energy and angle $\theta_e$ of the electron:

\begin{equation}
\cos \theta_e = \frac{T(E_\gamma + m_e c^2)}{E_\gamma \sqrt{T^2 + 2m_ec^2 T}} = \frac{1 + \epsilon}{\sqrt{\epsilon^2 + 2\epsilon/t}}
\end{equation}

The maximum energy transfer to the electron is obtained 
from equation \ref{diffeq} for backward scattering of the 
photon ($\theta_\gamma = 180^\circ$), corresponding to 
forward scattering of the electron ($\theta_e = 0^\circ$). 
The electron's kinetic energy reaches its maximum value in 
this case, $T \rightarrow T_{\text{max}}$. In the measured 
energy spectrum, this leads to the so-called "Compton edge" at:

\begin{equation}
T_{\text{max}} = \frac{E_\gamma \cdot 2\epsilon}{1 + 2\epsilon}
\end{equation}

which lies slightly below the photopeak. The energy difference 
between the photopeak and the Compton edge $E'_\gamma(\theta = \pi)$ 
decreases with increasing $E_\gamma$ and approaches:

\begin{equation}
E'_\gamma(\theta = \pi) \approx \frac{m_e c^2}{2} \text{ for } E_\gamma \gg m_e c^2
\end{equation}

\subsection{Pair production}
In the Coulomb field of a charge, a photon can 
convert into an electron-positron pair (Figure 
\ref{fig:pprod})\footnote{Photon emission by an 
electron (bremsstrahlung) and pair production are closely 
related processes. By modifying the bremsstrahlung diagram-changing 
the outgoing photon to an incoming one and the incoming electron to 
an outgoing positron-one obtains the pair production 
diagram. The matrix elements of these processes are 
related, at least in the lowest order. Consequently, both 
processes are treated together in the foundational work by 
Bethe and Heitler, often referred to as the 'Bethe-Heitler processes'.}.

\begin{figure}[!h]
    \centering
    \includegraphics[width=0.4\textwidth]{figures/png/Screenshot_20240812_204755.png}
    \caption[The pair production.]{The pair production \cite{kola}.}
    \label{fig:pprod}
\end{figure}

The energy of the photon must exceed twice the electron 
mass plus the recoil energy transferred to the field-producing 
charge. For most elements, pair production predominantly 
occurs in the Coulomb field of the nucleus. For nuclei, 
the recoil energy is usually negligible, leading to a 
threshold energy for pair production of:
\begin{equation}
    E_{\gamma} \geq 2m_e c^2 + 2 \frac{m_e^2}{m_{\text{nucleus}}} c^2
\end{equation}

If the nuclear charge is not screened by atomic electrons 
(for low energies, the photon must come relatively close to 
the nucleus to make pair production probable, meaning it 
interacts with the "bare" nucleus),
\begin{equation}
    1 \ll \epsilon \ll \frac{1}{\alpha Z^{1/3}}
\end{equation}
the pair-production cross section is given by:

\begin{equation}
    \sigma_{\text{pair}} = 4 \alpha r_e^2 Z^2 \left(\frac{7}{9} \ln 2 \epsilon - \frac{109}{54}\right) \text{ cm}^2/\text{atom}
\end{equation}

However, for complete screening of the nuclear charge ($\epsilon \gg 1/\alpha Z^{1/3}$):
\begin{equation}\label{sigmapair}
    \sigma_{\text{pair}} = 4 \alpha r_e^2 Z^2 \left(\frac{7}{9} \ln \frac{183}{Z^{1/3}} - \frac{1}{54}\right) \text{ cm}^2/\text{atom}
\end{equation}

At high energies, pair production can occur even 
at relatively large impact parameters between the 
photon and the nucleus. In this case, the screening 
effect of atomic electrons must be considered. For 
large photon energies, the pair-production cross 
section approaches an energy-independent value as given by
 Equation \ref{sigmapair}. Ignoring the small term in the equation, 
 the asymptotic value of $1/54$ is expressed as:
\begin{equation}
    \sigma_{\text{pair}} \approx \frac{7}{9} \cdot 4 \alpha r_e^2 Z^2 \ln\left(\frac{183}{Z^{1/3}}\right) \approx \frac{7}{9} \cdot \frac{1}{X_0} \cdot \frac{A}{N_A \cdot \rho}
    \label{eq:paircross_radiationlength}
\end{equation}

The energy is uniformly distributed between the produced 
electrons and positrons at low and medium energies, but becomes 
slightly asymmetric at high energies.

The field of the nucleus is formed by the coherent sum of $Z$ 
nucleon charges, leading to the $Z^2$ dependence of the pair 
production cross section.

Even with large momentum transfers $\Delta p$ to the nucleus, 
the energy transfer $(\Delta p)^2/2M$ remains small due to the 
large nuclear mass $M$. After pair creation, the remaining 
energy is equally divided between the $e^+$ and the $e^-$.

\subsection{Delta rays}
$\delta$-rays, or $knock-on$ electrons, 
are produced when a projectile particle collides 
centrally with shell electrons, resulting in 
significant energy transfers.
These electrons 
gain high kinetic energy and can be described 
through elastic collisions with quasi-free electrons. 
By considering the energy-momentum conservation relation 
and using the Lorentz factors $\gamma$ and $\beta$, the 
relationship between the kinetic energy $T$ of the 
$\delta$-ray and the emission angle $\theta$ can be derived as:

\begin{equation}
\cos \theta = \frac{T(\gamma + m_e / M)}{\gamma \beta \sqrt{T^2 + 2T m_e c^2}}
\end{equation}

\begin{equation}
T(\theta) = \frac{2 m_e c^2 \beta^2 \gamma^2 \cos^2 \theta}{\gamma^2(1 - \beta^2 \cos^2 \theta) + 2 \gamma m_e / M + m_e^2 / M^2}
\end{equation}

The maximum energy transfer $T_{\text{max}}$ occurs at $\theta = 0^\circ$, 
while the minimum energy, $T_{\text{min}}$, occurs at $\theta = 90^\circ$. 
At highly relativistic energies ($\gamma \gg 1$ and $\theta \gg 1/\gamma$), 
the energy-angle relationship becomes independent of the incoming particle's properties.

The rate of $\delta$-rays per energy interval $dT$ and path length $dx$ is given by:

\begin{equation}
\frac{d^2 N}{dx \, dT} = n_e \frac{d\sigma}{dT}
\end{equation}

which, when combined with the electron density and the differential cross section, becomes:

\begin{equation}
\frac{d^2 N}{dx \, dT} = \frac{1}{2} z^2 \frac{Z}{A} K \rho \frac{1}{\beta^2} \frac{F(T)}{T^2}
\end{equation}

Here, $K$ is the constant from the Bethe-Bloch formula, 
and $F(T)$ is a function accounting for spin dependence. 
Integration over $T$ and $x$ provides the number of $\delta$-rays in a medium of thickness $\Delta x$:

\begin{equation}
N = \frac{1}{2} z^2 \frac{Z}{A} K \rho \Delta x \frac{1}{\beta^2} \left(\frac{1}{T_{\text{min}}} - \frac{1}{T_{\text{max}}}\right) \approx 0.077 \frac{\text{MeV cm}^2}{\text{g}} z^2 \rho \Delta x \frac{1}{T_{\text{min}}}
\end{equation}

The emission angle dependence is given by:

\begin{equation}
\frac{dT}{d \cos \theta} = 4 m_e c^2 \frac{\cos \theta}{\sin^4 \theta}
\end{equation}

Substituting this into the rate equation yields:

\begin{equation}
\frac{d^2 N}{dx \, d \cos \theta} = \frac{1}{2} z^2 \frac{Z}{A} K \rho \frac{1}{\cos^3 \theta} \frac{1}{m_e c^2} \approx 0.15 \frac{\text{cm}^2}{\text{g}} z^2 \rho \frac{1}{\cos^3 \theta}
\end{equation}

This expression diverges as $\theta$ approaches $90^\circ$, 
where $T$ approaches zero, indicating a limitation in the 
assumption of a free electron. The resulting distributions 
suggest that $\delta$-rays emitted at small angles can 
significantly affect the spatial resolution in detectors, 
particularly through ionization clusters that broaden the 
track of the mother particle.



\section{Monte Carlo samples}\label{datasample}
For our studies, we used three different  
Monte Carlo samples: two samples of 
CE signal generated at two different 
proton pulse intensities, and one sample of 
antiproton annihilation events. 
The production of particles within 
the PT and their tracking from the PS to the 
DS is handled by the Mu2e Offline software 
(Appendix \ref{mu2eana}). The simulation of 
particle interactions and the event processing are based on 
GEANT4, and they are handled 
by the art framework and data management is governed by the SAM system (Appendix \ref{mu2eana}).

Mu2e uses a multi-stage simulation to generate and simulate events efficiently. 
The method involves generating events, partially simulating 
them, and saving the intermediate results. Later stages resume the simulation from the saved 
state. This approach optimizes both the time required for the event generation 
and the disk space usage. 

At the first stage, the interactions of protons at the PT are simulated.
Produced secondary particles are traced
up to the DS and the information about particles which reached the DS is stored.
At the second stage, the surviving particles are propagated through the upstream portion of
the DS and muons stopped in the ST are recorded.
The third and the following stages deal with the simulation of the physics processes in the Mu2e detector and the hit generation and digitization.
For the CE dataset, a fraction of muons are assumed to decay into CEs. For the $\bar{p}$ dataset, 
$\bar{p}$ annihilation at rest in the ST is simulated based on the positions 
and times of the stopped antiprotons. Background electrons from annihilation 
result from decays such as $\pi^0 \to \gamma \gamma$, followed by photon conversions, and $\pi^- \to \mu^- \nu$, 
followed by $\mu^-$ decays. During this stage, the raw simulated data are digitized 
into simple C++ classes or structs, using the detector's raw data.

A typical Mu2e event includes multiple pileup hits from particles produced 
by muon captures in the ST, as well as particles entering the DS from 
the TS. The pileup hits make 
the majority of the detector's hits. The pileup level 
depends on the proton pulse intensity. The Mu2e pileup 
simulation assumes that the pulse intensity varies on a 
time scale much longer than 2 $\mu$s, meaning all proton 
pulses around the simulated one have the same intensity. 
Under this assumption, a transformation is applied to hits 
with time $T_i > 1695$ ns outside the microbunch limits, 
assigning them a residual time of $t_i = T_i \div 1695$ ns, 
effectively accounting for hits from previous proton pulses 
that would otherwise be attributed to later microbunches.

For the low-intensity mode, with a mean intensity of 
$1.6 \times 10^7$ protons/pulse, which in Mu2e 
jargon is named "1BB", approximately 25,000 
muons stop in the ST per pulse. In the high-intensity 
mode, named "2BB", this number is about 2.5 times higher (Section \ref{pulsedprotonbeam}).

The datasets used for CE signal plus pileup 
analysis will be referred to as $CE-1BB$ and $CE-2BB$ 
for 1BB and 2BB pileup, respectively. The dataset for 
antiproton analysis without pileup will be 
referred to as $PBAR-0BB$.
For datasets with pileup, 
the pileup hits are explicitly added to the hits from 
the signal process. The antiproton sample, however, 
has been simulated w/o the pileup. 
\section{$\delta$-electrons in the Mu2e tracker}\label{trackerdeltas}

In Mu2e, low-energy electrons 
and positrons, with a momentum below 20 MeV/c and 
referred to as $\delta$-electrons, 
generate the majority of hits in the tracker. 
Identifying the $\delta$-electrons hits as early 
as possible is important for improving the track reconstruction efficiency and 
the optimization of memory and the CPU usage. There are also several physics 
reasons why it would be important to identify 
those hits correctly:
\begin{itemize}
    \item the main Mu2e goal is the search for the 
    CE signal: flagging erroneously even a small 
    fraction of the hits generated by the 
    CE as $\delta$-electrons reduces the CE track reconstruction efficiency. 
    Figure \ref{fig:momhits} shows that in the 
    Monte Carlo sample $CE-1BB$, the simulated CE 
    hits are just 1\% of the total hits in the tracker;
    \item counting the number of protons 
    will be a complementary procedure to the 
    STM to determine the muon stopping rate: 
    the simulation shows that is possible to 
    estimate the number of muons captured in 
    the stopping target by counting the number 
    of protons produced in the nuclear muon 
    capture;
    \item misidentification of muons 
      and pions as $\delta$-electrons 
    may result in erroneous background estimate. 
    This is particularly significant for the 
    antiproton background.
    $p\bar{p}$ annihilation at rest in the ST 
    can produce signal-like electrons, which 
    constitute a background to the CE search. An estimate of this background is 
    presently affected by a large systematic 
    uncertainty. Mu2e has developed a data-driven 
    procedure to improve the estimate.
    $p\bar{p}$ annihilation at rest in the ST 
    can produce events with two or more tracks, 
    each with a momentum around 100-200 MeV/c. 
    For $p\bar{p}$ annihilation, 
    the rate of multi-track events 
    is about 500 times higher 
    than the rate of events with a single 
    signal-like electron. 
    For $10^4$ $p\bar{p}$ annihilation events 
    generated, about 3.7\% of 
    the events contained two reconstructable 
    particle tracks. Therefore, 
    the identification and reconstruction of 
    multi-track events could be 
    used to constrain the $p\bar{p}$ background. 
    Thus, it is crucial not to flag hits 
    generated by muons or pions as $\delta$-electron 
    hits, since this would compromise the 
    reconstruction efficiency. 
\end{itemize}
Figure \ref{fig:momhits} shows the 
momentum distribution (Monte Carlo truth) of 
particles that make at least one hit in the 
tracker for a simulated CE sample ($CE-1BB$, 
Section \ref{datasample}). 
For each Monte Carlo particle, the 
histogram is filled a number of times 
equal to the number of hits generated 
by that particle.
The distribution shows that the 
majority of hits originate from 
low-energy electrons and positrons 
(orange), which constitute approximately 
75\% of the total number of hits. 
There is an asymmetry between the 
number of hits below 20 MeV/c 
produced by electrons and positrons: 
electrons account for 71\% of all hits in 
the tracker, while positrons contribute only 4\%. The difference  
arises because electrons are produced also by Compton scattering, which is 
the primary source of hits in the energy 
range of 1 MeV. This difference 
will be crucial in 
the following sections when discussing 
the $\delta$ flagging efficiency.

The distribution also shows that 14\% of 
the total hits are due to protons, 
which are produced by nuclear processes. 
According to the simulation, the bump at low energy in the Monte Carlo momentum 
distribution of protons and deuterons producing 
hits in the tracker (green) results from inelastic neutron scattering.
Larger momentum values correspond to protons produced in muon 
captures at rest. Their kinetic energy ranges from about 
5 to 20 MeV, resulting in low $\beta \gamma$ values, 
which makes them heavily ionizing particles. 
The deposited energy will be one of the variables used to discriminate protons. 
The distribution has a maximum 
and then decreases at 250 MeV/c, which is 
primarily due to the finite length of the tracker and the increase in 
longitudinal momentum of protons within the tracker acceptance.

It is important to note that the bump 
around 50 MeV/c in the positron distribution should not be 
present. The source is so far unexplained. 
We expect $N(\mu^+ \rightarrow e^+ )/N(\mu^- \rightarrow e^- )$ 
to be about $10^{-3}$ for muons entering the DS. The DIO on the IPA 
(Section \ref{detectorsolenoid}) should also be around $10^{-3}$ compared to the DIF of 
negative muons. The simulation of $\mu^+$ may 
contain some errors and we are still 
investigating this discrepancy. However, this issue is not problematic for the analysis 
of low-momentum electrons and positrons, 
as the momentum ranges are different.

\begin{figure}[!h]
        \centering
        \includegraphics[width =0.95\textwidth]{figures/png/Screenshot_20240812_152905.png}
    \caption[Monte Carlo momentum distribution 
    of particles producing hits in the Mu2e 
    tracker ($CE-1BB$ data sample).]{
        Momentum distribution (Monte Carlo truth)  
       of particles producing at 
       least one hit in the tracker 
       ($CE-1BB$ data sample).  
       The momentum distribution 
       of all particles making hits is 
       depicted in dark blue, with electrons 
       shown in pink, positrons in light 
       blue, $\delta$-electrons in orange, protons 
       and deuterons in 
       light green, and CEs in dark green. }
       \label{fig:momhits}
\end{figure}


Figure \ref{fig:pbar} shows the momentum distribution of 
particles produced in the $p\bar{p}$ annihilation at the ST. 
The data sample ($PBAR-0BB$) is described in Section \ref{datasample}.
For each Monte Carlo particle, the histogram 
has an entry per hit generated by that particle. 
Particles produced in the $p\bar{p}$ annihilation 
are mostly pions and muons.

The momentum distribution has a 
maximum in the 100-200 MeV/c range. Photons are also 
produced, and they can undergo Compton scattering and 
pair production, which explains the presence of a 
$\delta$-electron peak that is about $\sim$100 
times lower than the one in Figure \ref{fig:momhits}.

\begin{figure}[!h]
    \centering
    \includegraphics[width =0.9\textwidth]{figures/png/Screenshot_20240815_124710.png}
\caption[Monte Carlo momentum distribution 
of particles producing hits in the Mu2e 
tracker ($PBAR-0BB$ data sample).]{
    Momentum distribution (Monte Carlo truth) 
    of particles producing at 
   least one hit in the tracker 
   ($PBAR-0BB$ data sample). 
   The momentum distribution 
   of all particles making hits is 
   depicted in dark blue, with electrons 
   shown in pink, positrons in light 
   blue, $\delta$-electrons in orange, protons 
   and deuterons in green, pions in 
   light brown and muons 
   in black. }
   \label{fig:pbar}
 \end{figure}


Figure \ref{fig:momhits} shows that 
the majority of the hits originate from $\delta$-electrons, 
with protons and deuterons being the second most 
common source. Figure \ref{fig:afbef} 
presents an example of comparison of an 
event display before (Left) and after (Right) the background 
hits have been flagged and removed.  
Flagging those hits is crucial for 
several reasons: it prevents unnecessary 
data from being sent to the pattern 
recognition algorithms, improving their efficiency and 
conserving the CPU resources.

\begin{figure}[!h]
    \begin{subfigure}[b]{0.4\linewidth}
        \centering
        \includegraphics[scale = 0.3]{figures/png/Screenshot_20240811_123612.png}
        \subcaption{Before.}
        \label{fig:bef}
    \end{subfigure}
    \begin{subfigure}[b]{0.7\linewidth}
        \centering
        \includegraphics[scale = 0.3]{figures/png/Screenshot_20240811_124245.png}
        \subcaption{After.}
        \label{fig:af}
    \end{subfigure}
    \caption[Before and After background hits flagging.]{
        Before and After background hits flagging. 
        The $x-y$ plane views of a CE 
    event with 2BB pile-up in the tracker (Section \ref{pulsedprotonbeam}). 
    The segments are the $hit$ tracker straws. 
    The hits marked in
    red are from electrons and the ones in 
    blue are from positrons. Left (Right): before (after) $\delta$-electron hit flagging.}
    \label{fig:afbef} 
\end{figure}






\section{$\delta$-electrons flagging algorithms}
A brief description of the Mu2e Offline software tools  
is reported in Appendix \ref{mu2eana}, and the reconstruction 
process is described in Appendix \ref{eventreco}. 
Flagging $\delta$-electrons is done before the time clustering 
and pattern recognition.

Mu2e has developed two types of hit flagging algorithms:
\begin{itemize}
    \item $FlagBkgHits$, described in Section \ref{flagbkghits};
    \item $DeltaFinder$, described in Section \ref{deltafinder}.
\end{itemize}

\subsection{The $FlagBkgHits$ algorithm}\label{flagbkghits}
The detailed description of multivariate 
analysis (MVA) and the 
process of MVA training is beyond the 
scope of this work. 
Nevertheless, since this technique is 
one of the options 
developed for the background hit flagging, I 
will briefly outline 
the fundamental principles involved.

When searching for patterns in a multivariable space, 
a common procedure involves 
defining a set of statistical 
models that analyze the measured 
variables and estimate 
the probability that these are consistent with the 
sought pattern. Once the variables are selected, 
the MVA is trained to recognize patterns by 
evaluating examples known to the trainer, 
allowing for the feedback to refine the 
pattern identification procedure.

The first selection concerns the 
mean energy deposited in  
the $ComboHit$s used to create a 
$StereoHit$ and excludes 
those with a deposited energy above 5 keV.
%
The algorithm then classifies as 
the proton hits all hits with a deposited 
energy above 4.5 keV.
Since $\delta$-electrons tend to form small, dense 
clusters of hits on $x$-$y$ plane, an algorithm based on a clustering 
approach was developed: concentrated clusters in the 
$x$-$y$ plane are sought using a clustering algorithm, 
as $\delta$-electrons are highly likely to create 
such hit clusters.
The $FlagBkgHits$ algorithm uses  
an artificial neural network (ANN)-based
technique to classify those clusters.

The ANN was trained in a supervised mode, 
using CE and $\delta$-electrons hits from 
Monte Carlo data sample.

The resolution on the measurement of the 
hit position in the wire direction is limited to a few cm. 
However, an improved position measurement can be 
achieved by using various layers of straws that have large overlaps in the transverse plane. 
By obtaining two hit measurements from a pair of 
intersecting straws and ensuring they fall within 
a time window on the order of the maximum drift time, 
one can deduce that these hits were produced by the 
same particle and occurred at the intersection of the 
two straws in the $x-y$ plane ($StereoHit$).

The clustering process uses the $x$ and $y$ coordinates 
of the selected hits. A random hit is chosen to define 
the initial cluster, after which the following iterative 
steps are applied. The centroid of each cluster is 
computed, and hits whose distances from a cluster 
centroid fall within a specified inner threshold 
and time window are added to that cluster. Hits 
with distances from all existing clusters greater 
than an outer threshold are used to seed new 
clusters, while hits that fall between the two 
thresholds remain unassigned. This process 
generates a new set of clusters, preparing input 
for the next iteration. During each 
iteration, every hit is reconsidered as a 
potential new point in a cluster, including 
those already assigned. The clustering process 
continues until convergence is achieved, i.e., 
when an iteration no longer results in any changes.


\subsection{The $DeltaFinder$ algorithm}\label{deltafinder}

The $DeltaFinder$ algorithm has been designed 
to identify $\delta$-electron hit patterns. This 
algorithm relies on the 
fact that $\delta$-electrons typically 
form a straight line in 
the $r$-$z$ plane (cyan lines in Figure 
\ref{fig:yzviewdelta}) 
and appear as a spot with a diameter 
of less than $\sim$3 cm in the 
$x$-$y$ plane within the 1 T  
magnetic field. On the other hand, 
CEs create entirely different 
patterns of hits, which appear as oblique segments in 
the $r$-$z$ plane due to their 
helical trajectories (red lines 
in Figure \ref{fig:yzviewdelta}). 
\begin{figure}[!h]
    \centering
    \includegraphics[width =0.6\textwidth]{figures/png/Screenshot_20240811_123048.png}
    \caption[$\delta$-electrons and CE patterns in $r-z$ plane.]{
        $\delta$-electrons and CE patterns in the $r-z$ plane. 
        The white rectangles represent four of the eighteen tracker stations. 
        The cyan straight lines represent a possible $\delta$-electron 
        patterns, while the red lines a possible CE pattern.
        }
    \label{fig:yzviewdelta}
\end{figure}
\subsubsection{Step 1: identifying $\delta$-electron segments}
$DeltaFinder$ first tries to 
identify $\delta$-electron track 
segments within each station 
individually. These segments, 
parallel to the beam axis, are 
called $seed$s. Since hits 
from the same electron should be 
close in both time and space, 
and $\delta$-electrons may hit 
multiple straws within the 
same panel, the algorithm clusters 
these straw hits in space and time, applying various 
cleanup cuts to ensure the 
selected patterns are consistent with the 
$\delta$-electron hit patterns. 
The maximum allowed time 
difference between two hits within 
a station to form a $seed$ is set to 40 ns.

This cleanup based on the $x$-$y$ coordinates is performed by computing a $\chi^2$. The $seed$ 
is reconstructed using three, or four $ComboHit$s. In each case, the intersections between pair of straws 
are determined, and the $x$-$y$ coordinates of the $seed$ are given 
by the center of gravity of these intersections. 

The algorithm extends the $seed$s by requiring 
hits to be sufficiently close to the intersection 
point in time and space, performing multiple checks 
to avoid over-efficiency in 
hit flagging. For each $seed$, 
the mean deposited energy is calculated from the energy 
deposited in all $ComboHit$s. 
This selection optimizes data processing by 
reducing the total number of hits that need to be analyzed.

\begin{figure}[!h]
    \centering
    \includegraphics[width =0.6\textwidth]{figures/png/Screenshot_20240811_115854.png}
    \caption[A $\delta$ candidate $seed$.]{A 
    $\delta$ candidate $seed$. The four 
    coloured segments are the tracker straws that were
    hit in the same station.}
    \label{fig:deltaseeds}
\end{figure}

It is necessary to clarify one detail. 
Figure \ref{fig:energydeposited} shows the 
distribution of the simulated deposited energy 
in the tracker for $\delta$ electrons, CEs, and 
protons in the case of a 1BB pileup. 

To optimize data processing, a hit energy 
threshold could be applied to the $DeltaFinder$ 
to reduce the total number of hits that 
need to be analyzed, thereby speeding up the process. Moreover, only about 
4\% of CE hits have energies above 3.5 keV 
(and 1\% above 5 keV), so the loss of CE 
hits would be minimal, resulting in a 
faster overall algorithm. 

However, such an energy cutoff would have a 
significant impact for the algorithm's performance. 
Starting with fewer hits, especially in 
stereo intersections, reduces the efficiency of identifying the correct $seed$. 

\begin{figure}[!h]
    \centering
    \includegraphics[width =0.8\textwidth]{figures/png/Screenshot_20240729_151910.png}
\caption[Monte Carlo deposited energy 
distribution in the tracker.]{
   The Monte Carlo deposited energy 
   distribution in the tracker ($CE-1BB$ data sample).  
   The red distribution refers to 
   CEs, while the 
   green and the blue one to $\delta$-electrons 
   and protons respectively. 
   The peaks and tails correspond to 
   particles with such high energy 
   deposition that they result in a 
   saturated waveform.
}
   \label{fig:energydeposited}
\end{figure}
\subsubsection{Step 2: connecting the $seed$s}
After the selection based on mean deposited energy, 
$DeltaFinder$ attempts to connect segments that 
are close in both the $x$-$y$ plane and time 
across different stations to form $\delta$-electron 
candidates. A valid candidate must have at 
least two segments and a minimum of five $ComboHit$s. 
Accidentally reconstructed segments of 100 MeV electrons  
remain unconnected due 
to their separation in the $x$-$y$ plane. 

$DeltaFinder$ links $\delta$-electron $seed$s 
across stations, attempting to associate new $seed$s 
with already found $\delta$-electron candidates.
If no match is 
found, a new candidate is created. Good $\delta$-electron 
candidates are marked, and their hits are 
flagged to prevent their inclusion in proton candidate searches.

\subsubsection{Step 3: identifying proton candidates}
Finally, $DeltaFinder$ identifies proton candidates 
using the $seed$s with the mean deposited energy above 3 keV. First, it checks 
if a $seed$ is consistent in time with any existing 
proton candidate. If no consistency is found, 
the hits of the $seed$ are added to a new proton candidate.


\section{Performance analysis and comparison}
The comparison between $FlagBkgHits$ and $DeltaFinder$ 
is performed at two levels, with each testing 
different aspects of the algorithms: 
\begin{itemize}
    \item \textbf{Hit-level comparison}: this 
    phase focuses 
    on estimating how accurately individual hits are 
    flagged, providing the most direct method for 
    assessing and comparing the performance of the two 
    algorithms. This stage allows for an unbiased 
    comparison without the influence of subsequent 
    reconstruction stages. It also includes a 
    direct evaluation of proton counting;
    
    \item \textbf{High-level comparison}: this phase focuses on the 
    comparison of the tracks reconstruction efficiencies. The main figure 
    of merit at this stage is the 
    reconstruction of CE tracks.

\end{itemize}

\subsection{Hit-level comparison}
Before starting with the hit-level 
comparison of the two algorithms, during the analysis, 
we first observed an over-efficiency 
in the proton hit flagging by $DeltaFinder$. 
This primarily impacted the flagging of 
protons, muons, and pions, as shown in 
Table \ref{tab:1bbcelebefore} and Table 
\ref{tab:0bbpbarbefore}. In Tables 
\ref{tab:1bbcelebefore}, \ref{tab:0bbpbarbefore}, 
\ref{tab:2bbcele}, \ref{tab:1bbcele} and 
\ref{tab:0bbpbar}, $f_p$ and 
$f_e$ represent 
the number of $ComboHit$s flagged as 
protons and electrons, respectively, 
divided by the total number of 
$ComboHit$s. DF and FBH denote the $DeltaFinder$ 
and $FlagBkgHits$ algorithms, 
respectively. Each row corresponds to the particle 
under study.

\begin{center}
    \begin{table}[h!]
    \centering
    \renewcommand{\arraystretch}{1.}
    \begin{tabular}{| c | c | c |} 
    \hline
    & $f_{p}$ & $f_{e}$ \\
    \hline
    $p$     & 96.0\% & 1.0\% \\
    \hline
    \end{tabular}
    \caption{$DeltaFinder$ proton 
    flagging results before the 
    adjustment -selecting a proton 
    candidate with more than four hits 
    and a deposited energy above 3 keV- ($CE-1BB$ data sample). 
    $f_p$ and $f_e$ represent 
    the number of proton $ComboHit$s 
    flagged as proton and electron hits, respectively, 
    divided by the total number of the proton/electron $ComboHit$s.}
    \label{tab:1bbcelebefore}
    \end{table}
\end{center}
    
\begin{center}
    \begin{table}[h!]
        \centering
        \renewcommand{\arraystretch}{1.}
        \begin{tabular}{| c | c | c | c | c|} 
        \hline
        &   $f_{p}$ &   $f_{e}$\\
        \hline
        $\mu$ &  5.8\%  & 5.0\%\\
        \hline
        $\pi$ & 2.5\% &  11.2\%\\
        \hline
        \end{tabular}
        \caption{
            $DeltaFinder$ muon 
        and pion flagging results before the 
        adjustment -selecting a proton 
        candidate with more than four hits 
        and a deposited energy above 3 keV- 
        ($PBAR-0BB$ data sample). $f_p$ and 
        $f_e$ represent 
        the number of particle $ComboHit$s 
        flagged as proton and electron hits, respectively, 
        divided by the total number of $ComboHit$s.}
        \label{tab:0bbpbarbefore}
    \end{table}
\end{center}

The number of muon (5.8\%) and pion (2.5\%) hits flagged as protons is quite high,
since as already mentioned in Section \ref{trackerdeltas} the fraction 
of protons in this type of events is really low. 
Figure \ref{fig:0pbarbefore} shows the  
distribution of the total 
number of muon $ComboHit$s (red) 
and those flagged as $\delta$-electrons (blue) 
and protons (green) as a function 
of the particle momentum. 
The distribution shows that a significant number of 
muons hits are misidentified 
as protons hits at low momentum. 
According to the Bethe-Bloch formula, 
such hits have 
higher energy loss and can thus be 
most likely flagged as 
protons. We thus implemented a 
number of corrections to the algorithm. 
Since $seed$s may have 
accidentally attached hits, we imposed a 
condition requiring a $good$ proton 
candidate to have more than four 
hits with a deposited energy greater than 3 keV.

This adjustment reduces the proton hit flagging  
efficiency by a $\sim$10\% 
(Tables \ref{tab:1bbcele} and \ref{tab:2bbcele}). 
However, it significantly reduces the 
number of muons and pions flagged as 
protons, by approximately 
a factor of 2 and a factor of 6 
(Table \ref{tab:0bbpbar}), respectively.

 \begin{figure}[!h]
            \centering
            \includegraphics[width =0.7\textwidth]{figures/png/Screenshot_20240805_222923.png}
        \caption[The  
        distribution of the total 
        and flagged number of muon 
        $ComboHit$s as a function of the particle momentum.]{The  
        distribution of the total number of 
        muon $ComboHit$s 
        (red) and of the muon $ComboHit$s 
        flagged as $\delta$-electrons (blue) 
        and protons (green) as a function of the particle momentum ($PBAR-0BB$ data sample). }
           \label{fig:0pbarbefore}
\end{figure}

Moving to the performance and comparison 
between the two algorithms, Tables \ref{tab:2bbcele} and \ref{tab:1bbcele} 
show the hit-level comparison between the two algorithms 
for the $CE-1BB$ and $CE-2BB$ data samples.
\begin{center}
    \begin{table}[h!]
    \centering
    \renewcommand{\arraystretch}{1.}
    \begin{tabular}{| c | c | c | c |} 
    \hline
    &    $f_{p}$ DF & $f_{e}$ FBH  & $f_{e}$ DF \\
    \hline
    e$^-$ ($p<$20 MeV/c)      & 3.7\%   & 75.1\% & 72.7\%\\
    \hline
    e$^-$ (20$<p<$80 MeV/c)  & 2.2\%   & 49.0\%& 29.9\%\\
    \hline
    e$^-$ (80$<p<$110 MeV/c)  & 0.8\%  &  7.4\%& 4.3\%\\
    \hline
    $p$       &  83.6\%  &  & 2.2\%\\
    \hline
    e$^+$ ($p<$20 MeV/c) & 0.4\%    &   84.2\%& 87.9\%\\
    \hline
    \end{tabular}
    \caption{Electrons, 
    positrons and protons hit-level 
    comparison ($CE-2BB$ data sample). 
    FBH and DF denote  
    the $FlagBkgHits$ and the 
    $DeltaFinder$ algorithms, 
    respectively. $f_p$ and $f_e$ represent 
    the number of particle $ComboHit$s 
    flagged as proton and electron hits, respectively, 
    divided by the total number of $ComboHit$s.
    }\label{tab:2bbcele}
    \end{table}
    \end{center}

    \begin{center}
    \begin{table}[h!]
    \centering
    \renewcommand{\arraystretch}{1.}
    \begin{tabular}{| c | c | c | c |} 
    \hline
   &  $f_{p}$ DF & $f_{e}$ FBH  & $f_{e}$ DF \\
    \hline
    e$^-$ ($p<$20 MeV/c)     & 2.5\%   & 75.9\% & 72.5\%\\
    \hline
    e$^-$ (20$<p<$80 MeV/c)  & 1.0\%   & 50.0\%& 27.4\%\\
    \hline
    e$^-$ (80$<p<$110 MeV/c)   & 0.3\%  &  5.7\%& 3.4\%\\
    \hline
    $p$                 &         83.7\%   &  & 1.0\%\\
    \hline
    e$^+$ ($p<$20 MeV/c)    & 0.2\%    &   85.5\%& 88.5\%\\
    \hline
    \end{tabular}
    \caption{Electrons, 
    positrons and protons hit-level 
    comparison ($CE-1BB$ data sample). 
    FBH and DF denote the $FlagBkgHits$
    and the $DeltaFinder$ 
    algorithms, respectively. $f_p$ and 
    $f_e$ represent 
    the number of particle $ComboHit$s 
    flagged as proton and electron hits, respectively, 
    divided by the total number of $ComboHit$s.}
    \label{tab:1bbcele}
    \end{table}
  \end{center}

  
The fraction of electron and positron hits 
flagged as $\delta$-electron hits differs due to the different momentum distributions  
of these particles at low energies. Figure \ref{fig:detail} shows a zoomed-in view of the 
$\delta$-electron energy distribution (Monte Carlo truth) for particles with  
at least one hit in the tracker for energies below 2 MeV. 
Electrons are shown in pink and positrons in black. 
Each bin represents the number of hits corresponding to a 
specific Monte Carlo particle. The pink distribution peaks 
near the electron mass, that corresponds to a value of $\theta_\gamma \sim 0$ (Section \ref{compton}), 
while the black distribution falls down to zero. 

\begin{figure}[!h]
    \centering
    \includegraphics[width =0.7\textwidth]{figures/png/Screenshot_20240820_154854.png}
    \caption[The electron and positron energy distribution for $E<2$ MeV.]{The electron (pink) and positron (black) 
    energy distribution for $E<2$ MeV.}
    \label{fig:detail}
\end{figure}

Figures \ref{fig:eff1} and \ref{fig:eff2} show the efficiency (i.e., the number of $\delta$-electron 
$ComboHit$s flagged as $\delta$-electrons over the total number of $ComboHit$s 
for each particle type) as a function of particle momentum for positrons (red) 
and electrons (blue) using the $FlagBkgHits$ (left) or $DeltaFinder$ (right) algorithms.

This plot shows a dependence on the particle momentum.
The efficiency plot is, in fact, convoluted with the momentum 
distribution. 

At low momentum values (below 1-2 MeV/c), the efficiency tends to be higher for positrons. 
This is because the total number of positron hits with 
such low momentum is extremely small, as positrons are not produced by  
Compton scattering. A common case of failure occurs when particles produce 
only a single hit in the tracker, making both stereo reconstruction and 
$seed$ reconstruction impossible. Compton electrons typically produce  
one or two hits per station, which does not allow to properly flag them. 
There are lots of electron $ComboHit$s in this energy range which remain 
unflagged. This explains the lower electron hit flagging efficiency.

However, $FlagBkgHits$ performs better with electrons having $E < 2$ MeV. 
This is because the minimum number of hits required to generate a 
seed with $DeltaFinder$ is three, and in this energy range, many 
particles produce only one or two hits in the tracker. Since positrons 
are not present in this energy range (as they are not produced by 
Compton scattering), the efficiency of positron hit flagging is 
higher for $DeltaFinder$ (Tables \ref{tab:2bbcele} and \ref{tab:1bbcele}).

At higher momentum, the two algorithms have a similar performance and the 
differences between positrons and electrons become negligible. 
Figure \ref{fig:efficiency} shows that as the momentum 
increases, the efficiency decreases. This occurs because 
higher momenta correspond to larger radii and a 
greater spread of hits on the $x$-$y$ plane. This trend is also observed for particles in the 
next selected momentum range ($20 \ \text{MeV/c} <p<80 \ \text{MeV/c}$).

A common case of $\delta$ flagging failure is when hits 
occur on straws that are parallel to each other, especially those near 
the center of the tracker, where stereo and $seed$ reconstruction is not possible. 

\begin{figure}[!h]
    \centering
    \begin{subfigure}[t]{0.5\textwidth}
        \centering
        \includegraphics[width=1.\textwidth]{figures/png/Screenshot_20240818_155835.png}
        \caption{}
        \label{fig:eff1}
    \end{subfigure}%
    ~ 
    \begin{subfigure}[t]{0.5\textwidth}
        \centering
        \includegraphics[width=0.95\textwidth]{figures/png/Screenshot_20240813_203916.png}
        \caption{}
        \label{fig:eff2}
    \end{subfigure}
    \caption[The electrons (blue) and positrons (red) hit flagging efficiency versus 
    particle momentum in the low-momentum range.]{The electrons 
    (blue) and positrons (red) hit flagging efficiency versus 
    particle momentum in the low-momentum range. The left plot shows the 
    results for $FlagBkgHits$, while the right plot shows those for $DeltaFinder$.
    }
    \label{fig:efficiency}
  \end{figure} 

$FlagBkgHits$ flags 70\% more CE hits 
as $\delta$-electrons compared to $DeltaFinder$ (Tables \ref{tab:2bbcele} 
and \ref{tab:1bbcele}). The main difference between the two algorithms lies in 
the primitives: $StereoHit$s and $seed$s. If the same 
$\delta$-electron hits three different straws, the $StereoHit$ 
information results in multiple intersection points being 
created. On the other end, intersecting hit wires and reconstructing a $seed$ helps 
to determine the delta electron segment position in 3D.

Concerning proton flagging, we could not compare the two 
algorithms since the $FlagBkgHits$, before creating $StereoHit$s, 
applies a preliminary selection cut on the energy deposited in 
the $StrawHit$s of 5 keV. $FlagBkgHits$ then identifies 
protons as all particles with a deposited energy greater than 
4.5 keV. It was not possible to perform a direct comparison, 
as $DeltaFinder$ does not apply any cuts on the energy deposited in 
the $StrawHit$s, making the comparison unreliable and biased. 
Therefore, we reported the fraction of protons flagged as protons only by $DeltaFinder$.

The results for the 1BB and 2BB data samples are comparable.



      
      Concerning $\mu$ and $\pi$ (Table \ref{tab:0bbpbar}), 
      we observe an approximate factor of 
      4 difference between the probabilities of hit mis-ID by the two algorithms 
      for muons, and about a factor of 3.3 for pions. This occurs because 
      muons and pions often produce more than one hit in a single station. However, the 
      $DeltaFinder$ algorithm can distinguish these hits as it 
      is based on identifying $\delta$-electron hit patterns. 
      The main problem of $FlagBkgHits$ is that it is based on a supervised 
      training. It can distinguish one type of particle (CEs) from 
      another ($\delta$-electrons) but could get confused when 
      the data include particles not used in the training, such as 
      those from $p\bar{p}$ annihilation. 
      
      On average, pions produced from $p\bar{p}$ annihilations 
      have higher momenta than muons, resulting in a higher false 
      positive rate. This occurs because the curvature of the pion tracks 
      is smaller than that of the muon tracks,
      making them more likely to form close in $\phi$ segments, thus appearing more similar to $\delta$-electron.


    \begin{center}
        \begin{table}[h!]
        \centering
        \renewcommand{\arraystretch}{1.}
        \begin{tabular}{| c | c | c | c|} 
        \hline
         &  $f_{p}$ DF &  $f_{e}$ FBH & $f_{e}$ DF\\
        \hline
        $\mu$  &  2.7\%  & 13.0\% & 3.2\%\\
        \hline
        $\pi$ & 0.4\% & 23.8\%& 7.3\%\\
        \hline
        \end{tabular}
        \caption{Muon and Pion 
        hit-level comparison ($PBAR-0BB$ data sample). FBH and DF denote 
        the $FlagBkgHits$ and $DeltaFinder$ algorithms respectively. $f_p$ and $f_e$ represent 
        the number of particle $ComboHit$s flagged as proton and electron hits, respectively, 
        divided by the total number of $ComboHit$s.}
        \label{tab:0bbpbar}
        \end{table}
        \end{center}

\subsection{High-level comparison}
The unflagged hits are then passed to the time clustering 
algorithm followed by the pattern recognition algorithm, 
after which the fit of the found track candidates 
is performed. 
Figure \ref{fig:highlevel1} shows the momentum 
distribution of reconstructed tracks for the $CE-1BB$ 
data sample using both algorithms. 
In the case of $FlagBkgHits$ (blue), less proton hits are  
correctly flagged, and non-flagged hits are
used in the pattern recognition. 
On the other end, $DeltaFinder$ (red) successfully flags 
proton hits, preventing them from 
being used by the pattern recognition. 
Figure \ref{fig:highlevel2} provides a zoomed-in view 
of the reconstructed track momentum distribution within 
the CE momentum range, where the results of track reconstruction 
for the two flagging algorithms are almost identical.

\begin{figure}[!h]
    \begin{subfigure}[b]{0.5\textwidth}
        \centering
        \includegraphics[width = 1.1\textwidth]{figures/png/Screenshot_20240820_162125.png}
        \subcaption{}
        \label{fig:highlevel1}
    \end{subfigure}
    \begin{subfigure}[b]{0.5\textwidth}
        \centering
        \includegraphics[width = 1.1\textwidth]{figures/png/Screenshot_20240820_160904.png}
        \subcaption{}
        \label{fig:highlevel2}
    \end{subfigure}
    \caption[Momentum distribution of reconstructed tracks.]{The momentum distribution of reconstructed tracks. 
    Tracks reconstructed using $FlagBkgHits$ ($DeltaFinder$) 
    as the $\delta$-flagger are shown in blue (red). 
    The distribution in (a) covers a higher momentum range, while (b) 
    provides a zoomed-in view of the momentum range [100, 107] MeV/c.}
    \label{fig:highlevel} 
\end{figure}

There is an excess of reconstructed tracks in the region of momentum above 120 MeV/c 
for $FlagBkgHits$. 
The excess is due to unflagged proton and deuteron hits used by the pattern recognition.

Table \ref{tab:recoeffcele} reports 
the fraction of the simulated CE events that have at least one 
reconstructed track (for both 
$CE-2BB$ and $CE-1BB$ data samples). 
This fraction of CEs is nearly identical 
between the two algorithms, 
as expected from Figure \ref{fig:highlevel}.
Table \ref{tab:1bbcele} shows 
that the difference in the number of 
hits between the two algorithms 
is less than one hit per track, which 
justifies the observed similarities in 
Table \ref{tab:recoeffcele}.
For both 1BB and 2BB datasets, the relative difference between the 
CE reconstruction efficiencies corresponding to two algorithms is less than 1\%.



\begin{center}
    \begin{table}[h!]
    \centering
    \renewcommand{\arraystretch}{1.}
    \begin{tabular}{| c | c | c | c | c |} 
    \hline
    & FBH 2BB & DF 2BB & FBH 1BB & DF 1BB  \\
    \hline
    fraction of CE events with $N_{tracks}>0$ & 36.7\% & 36.5\% & 37.9\% & 37.9\%\\
    \hline
    \end{tabular}
    \caption{Fraction of reconstructed CE events ($CE-2BB$ and $CE-1BB$ data sample). FBH and DF denote  
    $FlagBkgHits$ and $DeltaFinder$ algorithms, respectively.}
    \label{tab:recoeffcele}
\end{table}
\end{center}

To better understand the performance of the 
two algorithms, we analyzed events where at least one 
track was reconstructed using one hit flagger while no track  
was reconstructed with the other hit flagger. A well defined class 
of events was identified in which the effects of hit flaggers are 
mitigated by the time clustering algorithm during the reconstruction.

Figures \ref{fig:TZCluster1} ($FlagBkgHits$) and 
\ref{fig:TZCluster2} ($DeltaFinder$) show the 
$time$ versus $z$ coordinate 
(the one aligned with the tracker axis) for 
the same event processed by both algorithms. 
CE hits are represented by large red dots, 
$\delta$-electrons 
by small brown dots, and protons by large 
blue dots. Violet rectangles denote particles associated with 
the same $TimeCluster$, 
which is formed by grouping hits that 
align along a linear 
path in time versus $z$ space.
\begin{figure}[!h]
    \centering
    \includegraphics[width =0.7\textwidth]{figures/png/Screenshot_20240819_153229.png}
    \caption[$TimeCluster$ on $time-z$ plane.]{$TimeCluster$ (green) on $time-z$ plane. 
    The $TimeCluster$ is reconstructed using the $FlagBkgHits$ output.
    CE hits are represented by large red dots, $\delta$-electrons 
    by small brown dots, and protons by large blue dots. Violet 
    squares denote particles associated with the same $TimeCluster$.}
    \label{fig:TZCluster1}
\end{figure}
\begin{figure}[!h]
    \centering
    \includegraphics[width =0.7\textwidth]{figures/png/Screenshot_20240819_153730.png}
    \caption[$TimeCluster$ on $time-z$ plane.]{$TimeCluster$ (green and orange) on $time-z$ 
    plane. The $TimeCluster$ is reconstructed using the $DeltaFinder$ output.
    CE hits are the large red dots, $\delta$-electrons 
    the small brown ones, and protons the large blue ones. Violet 
    squares denote particles associated with the same $TimeCluster$.}
    \label{fig:TZCluster2}
\end{figure}
The time clustering process begins 
by combining hits within a specific $time-z$ 
window to create $chunks$ (with a 
20 ns time window and a 5-plane z-window). 
A window qualifies as a $chunk$ if 
it contains more than three $ComboHit$s. 
Every potential pair of chunks 
within a certain time 
proximity is tested together, 
and the pair, that minimizes 
the $\chi^2/ndof$ when the hits 
are fit to a linear line, 
is combined. This procedure is 
repeated until no further combinations 
yield a $\chi^2/ndof$ below a set 
threshold. If a chunk 
exceeds a minimum number of straw 
hits, it is saved as a cluster.

Since particles grouped together 
may be at different $z$ values, 
$TimeCluster$s can exhibit varying 
$z$ coordinates. In Figure \ref{fig:TZCluster1}, 
all hits are grouped into a single cluster (green), 
while in Figure \ref{fig:TZCluster2} 
(green and orange), CE hits are divided into 
two different $TimeCluster$s.  
Specifically, when the $DeltaFinder$ algorithm is used,
hits from the same 
particle are split into separate time clusters and no 
track are reconstructed. In contrast, 
with $FlagBkgHits$, unflagged hits are utilized by the 
time clusterer to $connect$ particle hits, 
leading to successful track reconstruction. 

This example highlights the importance of refining 
the time cluster finder to improve the track reconstruction.


Table \ref{tab:recoeffpbar} shows the fraction of reconstructed $p\bar{p}$ events, 
with at least two reconstructed tracks with the reconstructed momenta above 80 MeV/c or 90 MeV/c. The 80 
MeV/c cut approximately corresponds to the minimum particle momentum for which 
a track of a particle coming from the ST can be reconstructed 
in the Mu2e tracker. The 90 MeV/c cut is applied to ensure the 
suppression of DIO events to a negligible level. 
For Run I, requiring a DIO electron momentum above 90 MeV/c 
yields an estimate of approximately $10^{-2}$ events with 
two DIO electrons. With a track reconstruction efficiency of 
approximately 0.1, this corresponds to roughly $10^{-4}$ events with two 
reconstructed DIO electron tracks.
Assuming a uniform distribution in time, the number of such events
within a 100 ns window is approximately $10^{-5}$.

The $DeltaFinder$ algorithm shows an advantage of 
approximately 22\% in reconstructing 
two tracks, attributed to its ability to distinguish between 
muon and pion hits compared to $\delta$-electrons (Table \ref{tab:0bbpbar}). 
The difference in hit-level efficiencies results in more 
than 12 hits per track difference between the two algorithms. After 
applying momentum cuts, the difference between the algorithms 
appears to be similar to that observed without a momentum cut.


\begin{center}
        \begin{table}[h!]
        \centering
        \renewcommand{\arraystretch}{1.}
        \begin{tabular}{| c | c | c |} 
            \hline
            &  FBH & DF\\
            \hline
            fraction of events with $N_{tracks} \geq 2$ &  1.8\% & 2.2\%\\
            \hline
            fraction of events with $N_{tracks} \geq 2$ \& $p>$80 MeV/c & 1.7\% & 2.1\%\\
            \hline
            fraction of events with $N_{tracks} \geq 2$ \& $p>$90 MeV/c & 1.6\% & 2.0\%\\
            \hline
            \end{tabular}
        \caption{Fraction of reconstructed $p\bar{p}$ events 
        ($PBAR-0BB$ data sample). FBH and DF denote the $FlagBkgHits$ and the $DeltaFinder$ algorithm, respectively.}
        \label{tab:recoeffpbar}
        \end{table}
\end{center}

\section{Conclusions}
The simulation shows that the majority 
of hits in the tracker is 
generated by $\delta$-electrons, i.e. electrons 
and positrons with momenta below 20 MeV/c. 
Two distinct algorithms have been developed in the Mu2e Offline to 
identify those hits and exclude them from the pattern recognition: 
$FlagBkgHits$ and $DeltaFinder$. 
The $FlagBkgHits$ algorithm initially clusters the hits and then uses an ANN 
to classify them, while $DeltaFinder$ reconstructs only clusters of hits  
consistent with those produced by low-momentum particles or protons and deuterons.

I performed a systematic two-level comparison of the performance of the two algorithms:

\begin{itemize}
    \item hit-level comparison: the focus is on evaluating the 
    accuracy with which individual hits are flagged, providing a 
    direct method for comparing the algorithms' performance. In terms 
    of $\delta$-electron flagging, both algorithms perform similarly in the 
    considered energy range. Electrons (below 1-2 MeV/c) often produce only one 
    or two hits, making them harder to flag, which reduces the electron hit flagging 
    efficiency. The $FlagBkgHits$ algorithm works better for electrons below 
    2 MeV, since the minimum number of hits to form a 
    proper $seed$ for $DeltaFinder$ is three. 
    Since positrons are not present in this energy range (as they 
    are not produced by Compton scattering), the efficiency of positron hit flagging is higher for $DeltaFinder$.
    At higher momentum, both positron and 
    electron flagging efficiency become similar, but overall efficiency 
    decreases as momentum increases due to a wider spread of hits.
    $FlagBkgHits$ flags approximately 70\% more CE hits than $DeltaFinder$.
    The main difference between the two algorithms results from  
    the primitives: $StereoHit$s and $seed$s. When a $\delta$-electron hits three 
    straws, the $StereoHit$ creates multiple intersection points, while the 
    intersection of hit wires and the consequent reconstruction of 
    a $seed$ allows to better determine the delta electron segment position in 3D. 
    Furthermore, $DeltaFinder$ can flag proton hits (approximately 84\%), while $FlagBkgHits$ 
    can only flag hits based on large energy deposits, but it does not specifically identify proton candidates.
    To correctly estimate the background, it is crucial to not misidentify 
    product hits of $p\bar{p}$ annihilation as $\delta$-electrons and protons.
    This necessitates examining muon and pion hit flagging. $FlagBkgHits$ has been trained 
    on datasets lacking muons and pions and  
    flags these particles at rates roughly 4 and 3.3 times higher, 
    respectively, than $DeltaFinder$;
    
    \item high-level comparison: this is a study of the algorithms' 
    impact on subsequent stages of event reconstruction 
    and the track reconstruction efficiency. 
    The main observed difference between the two algorithms 
    is an excess of tracks in the momentum range above 120 MeV/c for  
    $FlagBkgHits$. 
    This increase is due to $FlagBkgHits$ failing to 
    properly flag proton and deuteron hits, allowing these particles to be sent to the 
    reconstruction stage. 
    For both 1BB and 2BB datasets, the relative difference between the 
    CE reconstruction efficiencies corresponding to two algorithms is less than 1\%.
    Moreover, the $DeltaFinder$ algorithm demonstrates a significant advantage, 
    showing approximately a 22\% higher fraction of $\bar{p}$ events with at least two reconstructed tracks compared to the alternative method.
\end{itemize}

A comparison of the timing performance of the 
two algorithms has also been performed.
For $FlagBkgHits$, it was necessary to account for the time 
spent creating the $StereoHit$s, while $DeltaFinder$ independently 
reconstructed the $seed$s. Processing $CE-1BB$ sample required about 
0.14 ms/event for $FlagBkgHits$ plus an additional 0.15 ms/event, 
compared to 0.42 ms/event for $DeltaFinder$. For $CE-2BB$ events, 
$FlagBkgHits$ took approximately 0.37 ms/event plus 0.34 ms/event, 
while $DeltaFinder$ required 1.1 ms/event. The Mu2e specification states 
that the processing time per event should not exceed 5 ms. The 
difference in processing time is approximately 0.13 ms/event with 
1BB pileup and 0.39 ms/event with 2BB pileup. While this difference 
does not appear critical at the moment, $DeltaFinder$ needs to improve its timing performance.

The primary drawback of $FlagBkgHits$ is its reliance 
on the supervised ANN training using CE and $\delta$-electron samples. 
The problem with this method is when other particles, such as cosmic muons 
and those from $p\bar{p}$ annihilation, are introduced into the algorithm. 
In principle, additional samples could be incorporated into the training 
process, but these would increase the technical complexity. 
Additionally, the training was performed using Monte Carlo data 
rather than real data, posing a potential risk when transitioning 
to actual data taking. 

