\chapter{Charged particle in a magnetic field}\label{appendix1}
The motion of a charged particle in a magnetic field can be described by the Lorentz force:
\begin{equation}
    \mathbf{F}=q \ \mathbf{v}\times\mathbf{B}
\end{equation}
where $q$ is the particle charge, $\mathbf{v}$ is the particle speed and $\mathbf{B}$ is 
the magnetic field.
A charged particle moving in a uniform solenoidal field describes the 
combination of a free trajectory and a circular motion, namely a helix, with the property:
\begin{equation}\label{partincamp}
    |\mathbf{B}|\rho=\frac{p_\perp}{|q|}
\end{equation}
where $\rho$ is the helix radius, $\p_\perp$ is the transverse momentum component.
This is the simplest example, although more complex magnetic fields can 
produce a more complex trajectories. In this appendix, I will go over how to 
use a gradient to accelerate particles and also how particles perpendicularly 
drift in a curved magnetic field. The intensity of a magnetic field with a 
non-null gradient changes with the position. The force derived from the gradient is 
proportional to the particle magnetic momentum $\mu$, where $E$ represents 
energy. It can be written as:
\begin{equation}
   \mathbf{F}=-\mu \nabla \mathbf{B} \qquad \qquad \mu=\frac{c^2 p_\perp^2}{2 E B}
\end{equation}
The force alters the direction of momentum, not the particle's energy. If the 
gradient is strong enough, it can reverse the direction of motion and reflect 
the particle like a magnetic mirror. Complex gradients can trap particles in 
certain regions or prevent them from staying there for too long.
The other important feature is the use of curved magnetic fields. In a curved 
solenoid, the particle orbit drifts perpendicular to the bending plane. 
The drift velocity $v_D$ and total displacement $D$ can be determined from 
the path along the curved solenoid $S$:
\begin{equation}
    v_D=\frac{m \gamma c}{e B R}(v_\parallel ^2+\frac{1}{2}v_\perp ^2)
\end{equation}
\begin{equation}
    D \propto p S (\frac{1}{cos \theta} + cos \theta)
\end{equation}
In the equations above, parallel and perpendicular refer to the magnetic field direction 
and $R$ represents the solenoid bending radius. $\theta$ is 
the angle of the helix axis and the magnetic field and the sign of the drift 
relies on the sign of the charge. These characteristics show that a curved 
solenoid can be used to separate particle beams of opposite charge.
