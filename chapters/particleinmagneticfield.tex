\chapter{Charged particle in magnetic field}\label{appendix1}
The motion of a charged particle in a magnetic field can be described by the Lorentz force:
\begin{equation}
    \mathbf{F}=q \ \mathbf{v}\times\mathbf{B}
\end{equation}
A charge particle moving in a uniform solenoidal field describes the combination of free trajectory and a circular motion, namely a helix, with the property:
\begin{equation}\label{partincamp}
    |\mathbf{B}|\rho=\frac{p_\perp}{|q|}
\end{equation}
This is the simplest example, although more complex magnetic fields can produce alternative trajectories. In this section, I will go over how to employ a gradient to accelerate particles and also how particles perpendicularly drift in a curved magnetic field. A magnetic field with a non-null gradient changes in intensity with position. The force derived from the gradient is proportional to the particle magnetic momentum $\mu$, where $E$ represents energy. It can be written as:
\begin{equation}
   \mathbf{F}=-\mu \nabla \mathbf{B} \qquad \qquad \mu=\frac{c^2 p_\perp^2}{2 E B}
\end{equation}
The force alters the direction of momentum, not the particle's energy. If the gradient is strong enough, it can reverse the direction of motion and reflect the particle like a magnetic mirror. Complex gradients can trap particles in certain regions or prevent them from staying there for too long.
The other important feature is the use of curved magnetic fields. In a curved solenoid, the particle orbit point drifts perpendicular to the bending plane. The drift velocity $v_D$ and total displacement $D$ can be evaluated based on the path along the curved solenoid $S$:
\begin{equation}
    v_D=\frac{m \gamma c}{e B R}(v_\parallel ^2+\frac{1}{2}v_\perp ^2)
\end{equation}
\begin{equation}
    D \propto p S (\frac{1}{cos \theta} + cos \theta)
\end{equation}
In the equations above, parallel and perpendicular refer to the magnetic field and $R$ represents the solenoid bending radius. On the other hand, $\theta$ is the angle of the helix from the magnetic field axis and the sign of the drift relies on the sign of the charge. These features can be used to separate particles of various charges using a curved solenoid.
