\chapter{Commissioning of the tracker DAQ and FEE}\label{commissioning}
\textit{In this Chapter, I present the initial results of the tracker DAQ commissioning. 
Understanding the readout process is essential before reading out the detector. 
The first part focuses on the validation of the ROC readout through Monte Carlo simulation. 
The second one illustrates the initial data quality monitoring of the tracker preamps. 
Finally, the last section provides a brief overview of the DAQ performance studies.
}

  \section{The test stand setup}\label{des}
  Figure \ref{fig:blockdiagram} and \ref{fig:TS1} respectively show a 
  schematic representation and a picture of the tracker DAQ test-stand 
  installed at the IERC facility at Fermilab.
  \begin{figure}[!h]
    \centering
    \includegraphics[width =0.5\textwidth]{figures/png/Screenshot_20240712_102528.png}
    \caption[Block diagram representation of the tracker DAQ test-stand]{Block diagram representation of the tracker DAQ test-stand. The two purple 
    blocks represent the two channel batteries (48 CAL channels on the left, 
    and 48 HV channels on the right), connected to their respective digi-FPGA 
    (light green boxes), located in the DRAC board (dark green box). The 
    digi-FPGAs are connected to the ROC (yellow box) that manages the communication with the
    DTC (orange box). The DTC is connected to the DAQ 
    computer (pink box - mu2edaq09).}
    \label{fig:blockdiagram}
    \end{figure}







  The test-stand includes the entire readout 
  chain of an entire 96-channels tracker panel:  
  48 channels connected to the digi-FPGA-1 and 
  48 channels to the second one. 
  Each channel represents a tracker straw 
  which was absent from our test stand.

  The DRAC board was connected via optical 
  fiber to one DTC installed in DAQ computer (mu2edaq09).
  The DTC was programmed, for most of our 
  tests, to emulate the CFO 
  functionality to send the request to the 
  ROC to perform the readout of one event.
  The DTC emulated the CFO by sending a request 
  for one event, which was immediately 
  followed by reading the event.


    Depending on the test to perform, it was possible to 
    select one between two ROC operational modes:
    \begin{itemize}
    \item  \textbf{ROC internal mode}: the ROC was emulating the data 
    itself (user-defined) without reading digi-FPGAs;
    \item  \textbf{ROC external mode}: the ROC receives data 
    from the digi-FPGAs. This will be the mode used during 
    Mu2e data-taking.
    \end{itemize}
    For most of the tests performed, the ROC was operated 
    in the external mode: in some cases we simply used the 
    digi-FPGAs internal pulser, in some other cases we injected 
    calibration pulses directly in the CAL-side of the preamps at 
    a frequency we could customise.
    \begin{figure}[!h]
        \centering
        \includegraphics[width =0.6\textwidth]{figures/jpg/IMG_20240219_090538.jpg}
        \caption[The tracker DAQ test stand.]{The tracker DAQ test stand (TS1) installed at IERC facility 
        at Fermilab. The test-stand includes the full readout chain 
        of one entire tracker panel: 48 channels on the HV-side 
        and 48 channels on the CAL-side, 1 DRAC board connected 
        via optical fiber to one DTC installed in the DAQ computer 
        (not shown).}
        \label{fig:TS1}
        \end{figure}
    Figure \ref{fig:timing} shows the timing diagram 
    for the readout of one detector channel.
    There are two important parameters: 
    the Event Window width ($T_{EW}$) shown in red, 
    that represents the distance between two consecutive 
    proton pulses (called heartbeats - HB's - separated by 
    $\sim$1.7 $\mu$s during Mu2e data taking), and 
    the separation $T_{gen}=1/f_{gen}$ between 
    two consecutive hits (represented by grey triangles), 
    where $f_{gen}$ is the generator frequency. 
    Each hit consists of two 16-bytes packets.
    For the purpose of these tests and DAQ commissioning, 
    the system allows for the flexibility to vary 
    the $T_{EW}$ between 700 ns to 50 
    $\mu$s. 

    The ROC firmware has an internal hit buffer 
    which stores up to 255 hits, which is adequate 
    given the expected hit multiplicity 
    in the detectors at run time. 
    Depending on $T_{gen}$ and $T_{EW}$, the data 
    taking can proceed in two different modes:
    \begin{itemize}
      \item \textbf{The ROC regular mode}: the total number of hits 
      within the event window is less than 255.
      In this case the ROC hit buffer doesn't get filled up 
      and the total number of 
      hits may vary from one event to another one. It is 
      expected that the number 
      of hits will not exceed the threshold of 255, 
      with an average of approximately 12 hits;
    \item \textbf{The ROC overflow mode}: the event window 
    is large enough, 
    so the total number of generated hits is 
      greater than 255. In this case
      the ROC hit buffer always gets filled up, and only 
      the first 255 hits are read out. 
      The hits readout after the first 255 are lost.
    \end{itemize}
   
    Since the timing of the readout (i.e. the Event Window) 
    is indipendent from the timing sequence 
    of the pulse generator, the number of pulses contained in 
    the Event Window can be different between two different 
    windows. 
    For example, Figure \ref{fig:timing} shows 
    that for two different relative timing offsets between 
    the Event Window and the sequence of generated pulses, 
    the number of pulses can be 
    either three (Figure \ref{fig:timing} Left) 
    or four (Figure \ref{fig:timing} Right).

    \begin{figure}[!h]
    \centering
    \includegraphics[width =1\textwidth]{figures/png/finalimg.png}
    \caption[Graphic illustration of pulses in an event window.]{Graphic illustration of pulses in an event window.}
    \label{fig:timing}
    \end{figure}

    The relative timing offsets among different channels of 
    the same digi-FPGA are of the order of few ns and can be 
    measured. The pulse sequences from the two digi-FPGAs are offset 
    relative to each other by a time interval $\Delta t$, which is 
    constant for as long as the DRAC board is 
    powered up and varies randomly between 0 and $T_{gen}$ 
    when DRAC is powercycled.


    Depending on the event window and the start of the pulse 
    sequence, the first channel in the read out sequence 
    reads out as many hits as allowed. Once all hits are read out, 
    the read out process for this channel 
    stops, and the process moves to the second one, 
    which then begins its own read out phase. This procedure 
    is repeated for all channels, following a fixed order.
\section{Validation of ROC read out and buffering}
The first test I performed has been to verify the correct performance of 
ROC buffering.
During these test, a single ROC was connected to the DTC. 
Data were collected with digi-FPGAs pulsed by their internal pulsers, 
with the ROC set in the external mode. 
Concering the digi-FPGAs pulse generators, they can operate at two specific frequencies, 
31.29 MHz/(2$^7$+1), or approximately 250 kHz, 
and 31.29 MHz/(2$^9$+1), or approximately 60 kHz.
        
\subsection{Development of the ROC bit-level simulation}\label{MonteCarlo}
 
ROC's digital readout logic allows to be 
emulated with a bit-level C++ simulation, 
which I contributed to develop. 
For each event, the simulated parameters 
are the number of hits in each channel and the total 
number of readout hits per event, which cannot exceed 255. 


The main steps of the simulation are the following ones:
\begin{itemize}
\item
  The start of the Event Window is set at $t=0$;
\item
  In each digi-FPGA, the timing of the first generated pulse 
  is randomly sampled from a uniform distribution between 0 and $T_{gen}$;
\item
Once the first pulse has been generated, the following pulses are 
added at the relative distance of $T_{gen}$,
  until the absolute time of the next pulse is above $T_{EW}$;
\item
  In the readout part of the simulation, 
  the pulses are readout following a predefined channel readout ordering;; 
\item
  the readout \textit{continues} until all the simulated hits 
  have been included, or the maximum threshold of 255 hits has been reached.
\end{itemize}
The simulation allows to introduce the offset between digi-FPGA-1 
and digi-FPGA-2 timing sequences and, 
internally to each digi-FPGA, the channel-to-channel 
offsets, as user-defined parameters.
To fine-tune the simulation, we measured the channel-to-channel 
offset from the data. For each digi-FPGA, we selected 
a reference channel
and measured the offset of the 
remaining 47 channels relative to it.
The reference channel is the first channel to 
be readout in each digi-FPGA: channel 91 for 
digi-FPGA-1 and 94 for digi-FPGA-2. 
Figure \ref{fig:delay1}(\ref{fig:delay2}) shows an example of 
distribution of the offset between the reference 
channel 91(94) and a randomly selected channel, 0(44) for 
digi-FPGA-1(digi-FPGA-2). 
The mean offset was determined with a simple 
Gaussian fit of the distribution. 
This procedure was repeated for all channels 
relative to each digi-FPGA.
All the measured offsets were of the order of a 
few ns and were inserted in the simulation because 
it was important to take into account all possible 
effects in the emulation of the ROC hit buffering 
to perform an accurate comparison between data and 
simulation, although we were aware of the fact that 
given the order of magnitude of these offsets, the 
impact was negligible.

  \begin{figure}[!h]
          \centering
      \includegraphics[width=0.75\textwidth]{figures/png/Screenshot from 2023-12-03 11-50-50.png}
      \caption[Delay between channel 0 and the reference channel in digi-FPGA-1.]{digi-FPGA-1: histogram of the delay between 
      channel 0 and the reference channel 91, the first 
      to be readout.}
      \label{fig:delay1}
    \end{figure}
    \begin{figure}[!h]
          \centering
      \includegraphics[width=0.75\textwidth]{figures/png/Screenshot from 2023-12-03 11-50-33.png}
      \caption[Delay between channel 44 and the reference channel in digi-FPGA-2.]{digi-FPGA-2: histogram of the delay between 
      channel 44 and the reference channel 94, the first 
      to be readout.}
      \label{fig:delay2}
    \end{figure}
   
    The comparison between the data and the simulation 
    will be presented in the following sections, where 
    we will refer to $occupancy$ as the total number of hits 
    recorded in a given channel during the test run.
\subsection{The \textit{ROC buffer overflow} mode}
In the \textit{ROC buffer overflow} configuration (referred to 
as RUN281 in the following), 
the $T_{EW}$ was set to 50 $\mu$s and $f_{gen}$ to $\sim$60 kHz, 
which corresponds to 
a time distance between two consecutive pulses of 
approximately 16 $\mu$s.
This implies that the number of hits in a 
readout window could be 
either 3 or 4, depending on the offset 
between the readout window and the generated pulses sequence.
\subsubsection{Hit timing distribution and channels occupancy}\label{over}
We first checked the distribution of the hit time for 
each readout channel, and the distribution of the 
total number of hits (occupancy) in one specific 
readout channel as a function of the channel number. 
Figure \ref{fig:1} shows the hit time 
distributions for two selected channels, 
channel 0 of digi-FPGA-1 and channel 2 of digi-FPGA-2.
The top distribution is, as expected, uniform, 
however the bottom one shows some unexpected features.
\begin{figure}[!h]
  \begin{subfigure}[b]{\textwidth}
      \centering
      \includegraphics[width=0.7\textwidth]{figures/pdf/figure_00007_timedistr_roc_simulation_ch0_281.pdf}
      \label{fig:t1}
  \end{subfigure}
\\
  \begin{subfigure}[b]{\textwidth}
      \centering
      \includegraphics[width=0.7\textwidth]{figures/pdf/figure_00003_timedistr_roc_simulation_ch2_281.pdf}
      \label{fig:t2}
  \end{subfigure}
     \caption[The time distribution of hits.]{(Top): the time distribution of hits in the channel 0 in the digi-FPGA-1.
     (Bottom): the time distribution of hits in the channel 2 in the digi-FPGA-2.}
     \label{fig:1}
\end{figure}
The explanation for the two different distributions 
can be found from Figure \ref{fig:2} (Top), 
which shows the occupancy as a function of channel number 
(red is for data, blue for simulation).

The ordering of the bins in the histogram corresponds 
to the channel readout ordering: this means that 
the first bin on the left corresponds to the first 
readout channel and the last one on the right 
corresponds to the last readout channel.
In other words, channels on the left 
are at the beginning of the readout sequence 
and are, in these conditions, always 
completely readout. On the other hand, channels on 
the right are in the 
later part of the readout sequence and may 
not have all the hits readout, 
since the ROC buffer may be in overflow.
  \begin{figure}[!h]
    \begin{subfigure}[b]{\textwidth}
        \centering
        \includegraphics[width=0.7\textwidth]{figures/pdf/figure_00004_nhitsvschannel_roc_simulation_281.pdf}
        \label{fig:tt1}
    \end{subfigure}
  \\
    \begin{subfigure}[b]{\textwidth}
        \centering
        \includegraphics[width=0.7\textwidth]{figures/pdf/figure_00014_nhitsvschannel_roc_simulation_281.pdf}
        \label{fig:tt2}
    \end{subfigure}
       \caption[The occupancy histogram (RUN281).]{(Top): number of hits versus the channel number 
       (data in red, Monte Carlo in blue). The two distributions 
       are normalized to the same number of events.
       The channels are numbered in the readout order. Not all 96 channels 
       are present in the histogram 
       because the maximum threshold of 255 hits was reached with fewer channels.
       (Bottom): zoom on the last channels in the readout sequence. The data 
       and MC distributions differ from each other by $\sim$ 10$^{-3}$.}
       \label{fig:2}
  \end{figure}


Figure \ref{fig:66} shows the distribution of the number of 
hits of channel 0 of digi-FPGA-1. As expected, 
the majority of events 
has 3 hits, with a tail of 4 hits. 
This provides an explanation for Figure \ref{fig:2} (Top).
\begin{figure}[!h]
\centering
\includegraphics[width =0.7\textwidth]{figures/pdf/figure_00066_nhits_ch00_run281.pdf}
\caption[The distribution of the number of hits in channel 0.]{
  The distribution of the number of hits in channel 
  0 of the digi-FPGA-1 (RUN281).
}
\label{fig:66}
\end{figure}
The first 69 bins of the histogram (Figure \ref{fig:2} (Top)) 
show a flat distribution with the maximum observed occupancy. 
Additionally, there are two more plateaus with slightly lower 
occupancy (less than 10\% lower): one over the following 
7 bins (with a small $dent$ at the end), and the other 
spanning the last 9 bins.
This pattern can be generated by adding together the 
following three types of events.
\begin{itemize}
  \item \textbf{0-68th read out channels}: 
  the 48 channels of the digi-FPGA-1, the first one to be readout, 
  are those with 4 hits per channel, 
  which makes a total of 192 hits stored in the ROC buffer. 
  In these conditions, 63 hits can still be stored.
  Because of the delay between the first and second digi-FPGA, 
  only three hits can be read out by each channel 
  of the digi-FPGA-2. The first 21 channels of the 
  digi-FPGA-2 will be those with 3 hits per channel, resulting in a 
  total of 255 hits. All the hits sent to the following 
  channels are lost.
  \item \textbf{0-75th read out channels}: 
  the 48 channels of the digi-FPGA-1 are 
  those with 3 hits per channel, 
  which makes a total of 144 hits stored in the ROC buffer.
  The \textit{dent} at the end of the second 
  plateau is due to the 
  fact that the digi-FPGA-2 contributes 111 hits and this number 
  is not an integer of 4. 
  The first 27 channels of the digi-FPGA-2 contribute
  4 hits per channel each, but 
  the three hits from channel 28 in the 
  readout sequence
  fill up the total ROC buffer of 255 hits. 
  \item \textbf{0-85th read out channels}: 
  the 48 channels of the digi-FPGA-1 and the 
  first 37 channels to be read out in the digi-FPGA-2 
  are those with 3 hits, 255 hits in total.

\end{itemize}
Not all 96 channels are present in the 
histogram because the maximum 
threshold of 255 hits was reached with 
fewer channels, in particular the number of 
85 read out channels corresponds to the 
sum of 48 read out channels in the first 
digi-FPGA and 37 read out channel in the digi-FPGA-2.


Figure \ref{fig:2} (Bottom) shows a zoom on the 
rightmost channels of the distribution.
The relative difference between the data and 
the MC distributions 
is at a level of $10^{-3}$, which is a 
very good agreement.


Coming back to Figure \ref{fig:1}, the first 
channels in the readout sequence
always have all their hits read out,
while the channels in the end of the readout sequence do not,
as the ROC hit buffer gets filled up after
the first 255 hits are read out.
This results in a uniform time distribution 
for the first channels readout and in a non-uniform
time distribution for the last readout channels, 
depending on $T_{gen}$ and $T_{EW}$.
The dips in the hit timing distribution for 
channel 2 are defined by the timing offset
between the two digi-FPGA pulsers. 


%%%%%%%%%%%%%%%%%%%%%%%%%%%%%%%%%%%%%%%%%%%%%%%%%%%%%%%%%%%%%%%%%%%%%%%%%%%%%%
\subsubsection{Number of hits}
Figure \ref{fig:3} shows that, in the \textit{ROC buffer overflow} 
mode, the number of readout hits is exactly 255 for all events.

\begin{figure}[!h]
\centering
\includegraphics[width =0.7\textwidth]{figures/pdf/figure_00008_nhits_281.pdf}
\caption[The distribution of the total number of hits read out per event.]{
  The distribution of the total number of hits read out per event. 
  (data in red, Monte Carlo in blue). The two distributions 
  are normalized to the same number of events.
}
\label{fig:3}
\end{figure}
\subsection{The \textit{ROC buffer regular} mode }
In the \textit{regular} configuration, which will be referred 
to as RUN105038, the event window $T_{EW}$ was chosen to be 25 $\mu$s
and the pulser rate $f_{gen}$ 60 kHz, 
which corresponds to $T_{gen}=16 \ \mu$s.

\subsubsection{Time distribution and occupancy}
Figure \ref{fig:4} shows the 
distributions
of the number of hits in two channels, one from the 
digi-FPGA-1 and another one, from the digi-FPGA-2 and Figure 
\ref{fig:5} shows the occupancy as 
a function of the channel number. 
In this readout configuration, the expected number of 
hits in a given channel
within the event window is one or two, and the total 
number of pulses is always below 255.
There is no overflow of the hit buffer and no pulses are lost.
\begin{figure}[!h]
  \begin{subfigure}[b]{\textwidth}
      \centering
      \includegraphics[width=0.7\textwidth]{figures/pdf/figure_00001_timedistr_roc_simulation_10538.pdf}
      \label{fig:ttt1}
  \end{subfigure}
\\
  \begin{subfigure}[b]{\textwidth}
      \centering
      \includegraphics[width=0.7\textwidth]{figures/pdf/figure_00012_timedistr_roc_simulation_ch2_105038.pdf}
      \label{fig:ttt2}
  \end{subfigure}
     \caption[The hit time distribution.]{(Top): the hit time distribution for this in channel 2, the digi-FPGA-2. 
     (Bottom): the hit time distribution for hits in channel 0, the digi-FPGA-1.}
     \label{fig:4}
\end{figure}

In this mode, the readout of a given channel is not affected 
by the readout of previous
channels and the \textit{occupancy} distribution shown 
in Figure \ref{fig:5} is, as expected, uniform.
\begin{figure}[!h]
\centering
\includegraphics[width =0.7\textwidth]{figures/pdf/figure_00002_nhitsvschannel_roc_simulation_2.pdf}
\caption[The occupancy histogram (RUN105038).]{The number of hits versus the channel number for RUN105038 
(data in red, Monte Carlo in blue). The two distributions 
are normalized to the same number of events. All 96 channels are read out.}
\label{fig:5}
\end{figure}

Given this choice of $f_{gen}$ and $T_{EW}$, 
the maximum number of hits in one Event Window is 2.
Figure \ref{fig:67} shows the distribution of the 
number of hits in the channel 0 (FPGA-1).
\begin{figure}[!h]
\centering
\includegraphics[width =0.7\textwidth]{figures/pdf/figure_00067_nhits_ch00_run105038.pdf}
\caption[The distribution of the number of hits in channel 0.]{
  The distribution of the number of hits in channel 0, digi-FPGA-1, for RUN105038.
  Entries in the $n$(hits)=0 bin are due to the readout errors.
}
\label{fig:67}
\end{figure}

%%%%%%%%%%%%%%%%%%%%%%%%%%%%%%%%%%%%%%%%%%%%%%%%%%%%%%%%%%%%%%%%%%%%%%%%%%%%%%
\subsubsection{Number of hits}
Compared to RUN281, the event window in RUN105038 
was twice shorter
and the ROC readout buffer wasn't getting filled up.

Figure \ref{fig:6} shows the distribution of the total 
number of hits per event in the $regular$ mode. 
There are two higher peaks, one at 144 and one at 192. 
The peak at 192 occurs when all 48 channels of digi-FPGA-1 
and all 48 channels of digi-FPGA-2 have 2 hits each. The 
peak at 144 occurs when all channels of digi-FPGA-1 have 
2 hits, and all channels of digi-FPGA-2 have 1 hit, 
or vice versa. The total number of hits within the event 
window depends on the relative offset of the event window 
with respect to the digi-FPGA pulsers, and it varies from 
144 to 192. As before, the relative difference between the 
data and the MC distributions is at the level of $10^{-3}$, 
which indicates very good agreement.
\begin{figure}[!h]
\centering
\includegraphics[width =0.7\textwidth]{figures/pdf/figure_00009_nhits_105038.pdf}
\caption[The distribution of the total number of hits per event in \textit{regular} mode.]{
  The distribution of the total number of hits per event in \textit{regular} mode 
  (data in red, Monte Carlo in blue). The two distributions 
  are normalized to the same number of events.
}
\label{fig:6}
\end{figure}


\section{Study of preamplifiers performance}\label{dqm}
After having validated the correct ROC buffering, 
a set of different tests was accomplished to check 
the performance of the preamplifiers along with the readout chain.
During the test described in this section, the 
same test stand as in Section \ref{des} was 
used and one or two ROCs 
were connected to the same DTC, so one more 
test stand (TS2) was used. Preamps on the CAL 
side were added to perform the test.
As mentioned in Section \ref{preampss}, the 
preamps on the CAL side 
include circuitry that can inject calibration 
pulses into the channel. 
These pulses can be sent at an arbitrary rate and a frequency 
of 50 kHz was chosen. 
The data acquisition event window was set 
to 50 $\mu$s. 
In this case, the number of pulses within 
one Event Window could be 2 or 3.
To perform this test, I developed the initial step to the  
real-time monitoring and diagnostic tools: 
a set of real-time histograms to allow an 
easy check of the signal uniformity among 
channels within the same ROC or across 
multiple ROCs, and among different events.
The diagnostics tools allowed the identification 
of errors, such as the presence of an 
invalid channel ID, more hits than the maximum 
allowed in a given channel 
and an undefined link ID between ROCs.

\subsection{Test 1: channel occupancy versus channel ID}\label{nhitvschid}
The first test I conducted was to examine the channel occupancy. 
This was a crucial test, as it was important to determine if a 
channel was inactive, which would result in zero readout hits. 
It was also essential to detect any cross-talks (unwanted 
coupling between channels) which could cause the erroneous 
read out of hits in the incorrect channels.
It was also possible to check if the involved 
channels showed more hits than expected. 


For the previous reasons, we setup our system 
to pulse one every eight channels.
To scan all the 96 channels of the panel, 
we performed multiple runs choosing each time 
a different sequence of channels.
Figure \ref{fig:normalhits} shows the 
distribution of the occupancy for one channel 
sequence, which shows no anomalies.
\begin{figure}[!h]
      \centering
      \includegraphics[width=0.65\textwidth]{figures/pdf/run105421_nh_vs_ch.pdf}
      \caption[The regular occupancy plot.]{Regular distribution of number of hits versus channel ID. 
      The 0th channel is the first to be pulsed.
      As a consequence, the active channels are the 8th, 16th, 24th, 
      32nd, 40th, 48th, 56th, 64th, 72nd, 80th and 88th. 
      The number of hits is 3 or 4 for all channels. 
      No cross talks were observed in any neighbour channel.}
     \label{fig:normalhits}
\end{figure}
Figure \ref{fig:dead} shows a run with one dead channel (ID=94) 
and a channel (ID=70) with more than 
three hits, which was not expected. 
In the case of the dead channel, the problematic preamp 
was replaced. In the second case, to 
investigate this more thoroughly, 
I made the distribution of the time difference 
$\Delta t$ between consecutive pulses 
(Figure \ref{fig:deltatnhits}).
Since the pulser provides pulses at 
the frequency of 50 kHz, the $\Delta t$ 
distribution should show one single peak at 20 $\mu$s. 
On the other hand, there are also two 
pronounced peaks at 16 $\mu$s and 4 $\mu$s. 
In particular, the analysis of the 
waveforms of the hits in the 4 $\mu$s peak 
showed an inversion of the waveforms: 
this has been further investigated 
and will be discusse in more detail in Section \ref{inv}.
\begin{figure}[!h]
  \centering
  \includegraphics[width=0.65\textwidth]{figures/pdf/run105346_nh_vs_ch.pdf}
  \caption[The occupancy plot (dead channel and more hits than expected).]{Distribution of number of hits versus channel ID. 
  The 6th channel is the first to be pulsed.
  As a consequence, the channels that should be active are the 
  14th, 22nd, 30th, 38th, 46th, 54th, 62nd, 70th, 78th, 86th and 94th. 
  The 94th channel is not responding. In this case, the preamp 
  was substituted. The number of hits is not the same 
  for all channels. The 14th, 22nd, 38th, 46th, 54th, 62nd, 
  70th channels have more hit than expected.
  To address this issue, the distribution of $\Delta t$ between 
  hits of one of this channels is reported in Figure \ref{fig:deltatnhits}.
  No cross talks were observed in any neighbour channel.}
 \label{fig:dead}
\end{figure}


\begin{figure}[!h]
  \centering
  \includegraphics[width=0.65\textwidth]{figures/png/deltathits.png}
  \caption[$\Delta t$ distribution between hits in the 70th channel.]{$\Delta t$ distribution between hits in the 70th channel. The $\Delta t$ distribution between hits 
  should peak at 20 $\mu$s, as the pulser operates at a frequency of 50 kHz. However, peaks at approximately 
  16 $\mu$s and 4 $\mu$s are observed. All waveforms at around 4 $\mu$s are inverted compared to the regular ones. 
  The waveforms are shown in Section \ref{wf}, and the reason for this behaviour will be explained in the same section.}
 \label{fig:deltatnhits}
\end{figure}

Figure \ref{fig:cross} shows a clear example of 
cross-talk among channels. 
We tried to characterize this phenomenon and it was  
noticed that it occurred 
only when the ID of the first channel to be 
pulsed is odd.
The cross talk was also asymmetric 
(the cross-talk from 3rd channel to 5th was observed, 
but no 3rd$\rightarrow$1st).
As described in Chapter \ref{chaptertrk} 
and shown in Figure \ref{fig:spacepreamps}, 
the preamps PBC boards are mounted on vertical 
boards. Each board contains two preamps, each of one 
connected to one of the two straw-tubes ends: odd numbers 
are associated to the preamplifier closer to the horizontal 
board, while even numbers are associated to 
the preamplifier farther from the horizontal board. For this 
reason, it is not surprising that the cross talk 
appears to be between channels placed closer to 
the horizontal board. 
Moreover, we observed the cross-talk 
only in the first 20 channels IDs.
This is due to the fact that the 
distance between consecutive preamp 
boards is slightly narrower for the 
first channels (Figure \ref{fig:spacepreamps}).
At the moment of writing this Thesis, the 
possible solutions are still being investigated. 
\begin{figure}[!h]
  \centering
  \includegraphics[width=0.65\textwidth]{figures/pdf/run105420_nh_vs_ch.pdf}
  \caption[The occupancy plot and cross talks.]{Distribution of the number of hits versus channel ID. The 7th channel is the first to be pulsed. 
  As a consequence, the only channels that should be active are the 15th, 23rd, 31st, 39th, 47th, 55th, 63rd, 71st, 79th, 87th and 95th. 
  Channels 9th, 17th and 25th were also observed to be active, indicating cross talks.}
 \label{fig:cross}
\end{figure}

\begin{figure}[!h]
  \centering
  \includegraphics[angle=90,width=0.8\textwidth]{figures/jpg/photo_6028424923279639562_y.jpg}
  \caption[The zoomed view of the Analog Mother Board with the 
  preamplifier boards installed.]{The zoomed view of the Analog Mother Board with the 
  preamplifier PCB boards installed. The distance between 
  two consecutive boards decreases from left to right. 
  The first preamps in the occupancy plot are the ones on the right.}
 \label{fig:spacepreamps}
\end{figure}
The ouput of the test was reported to the tracker panel 
experts, and they are still working on it.

\subsection{Test 2: Analysis of the readout pulses waveforms}\label{wf}
The test described in this section involved the study 
of the reconstructed waveforms from charge injection. 
The sampling frequency of the ADC is 40 MHz, which resulted 
in a waveform bin width of 25 ns (Section \ref{DRAC}). 
The maximum number of samples for one waveform is 30. 


Figure \ref{fig:normalwf} shows an example 
of reconstructed waveform (RUN105421). 
The reconstructed waveforms shown below 
have already had the baseline subtracted. 
In Section \ref{basel}, I will provide a deeper 
analysis of the estimation of the waveform baseline. 
In first approximation, a good waveform is 
characterized by having a flat 
distribution in the first 10 samples, indicating the 
absence of noise at the start. It should feature a high 
positive charge peak with a sharp leading edge, which is 
the rising slope of the waveform. Additionally, the 
waveform should have a negative tail, contributing to 
its overall shape.
The negative tail is generated by a 
differentiating circuit that shapes the signal 
using a high-pass filter. This circuit was adopted 
because at high count rates two pulses can overlap: a 
new pulse can arrive before the previous one 
has returned to zero, thus leading to an overlap 
with the initial pulse's undershoot. This overlap 
may reduce the apparent amplitude of the subsequent 
pulse and generated the undesired broadening of peaks 
in the energy spectrum. 
The filter does not affect the rapid leading edge 
of the pulse because the time constant of the differentiator 
is not small compared to the rise time.

\begin{figure}[!h]
  \centering
  \includegraphics[width=0.65\textwidth]{figures/pdf/wf_ch50_0.pdf}
  \caption[A regular waveform.]{Regular waveform (50th channel).}
 \label{fig:normalwf}
\end{figure}


\subsubsection{Estimate of the waveform pedestal}\label{basel}
A straightforward procedure was adopted to estimate the pedestal: 
we simply take the mean value of the first 
10 samples of the waveform. 
In each channel, the stability of the 
pedestal is an indicator of the 
level of noise within the electronic chain. 
The pedestal of each channel varies 
according to the tolerances of 
the electrical components on the board.
Figures \ref{fig:baseline1} and \ref{fig:baseline2} 
show the distributions of 
the pedestals for two different channels (RUN105421).
In both cases, the pedestal is close to 
210 ADC counts, with a 
FWHM$=2 \sqrt{2 \text{ln}2}\sigma\sim$4.5 
ADC counts ($\sigma \sim$1.9 ADC counts).
  \begin{figure}[!h]
      \centering
      \includegraphics[width=0.65\textwidth]{figures/png/baseline_ch00.png}
      \caption[The fitted baseline distribution of channel 0.]{The fitted baseline distribution of channel 0.}
      \label{fig:baseline1}
  \end{figure}
  \begin{figure}[!h]
      \centering
      \includegraphics[width=0.65\textwidth]{figures/png/baseline_ch08.png}
      \caption[The fitted baseline distribution of channel 8.]{The fitted baseline distribution of channel 8.}
      \label{fig:baseline2}
\end{figure}

In some cases, a lower baseline value is observed. 
We observed dips of specific depths, 
for example 64, 128, or 192, which may suggest the 
possibility of malfunctioning 6th 
and 7th bit of the ADC (Figure \ref{fig:dips}). 
To address this issue and to reconstruct 
correctly the waveform, we simply did not used the bins with the 
negative peaks and estimated the pedestals with 
the remaning samples.
\begin{figure}[!h]
  \centering
  \includegraphics[width=0.65\textwidth]{figures/pdf/wf_ch58_1.pdf}
  \caption[An example of waveform with dips.]{Waveform of the 58th channel. Before subtracting 
  the baseline from the ADC counts, the dips were identified. The depth of the dips is 128.}
 \label{fig:dips}
\end{figure}
\subsubsection{Inverted waveforms}\label{inv}
As discussed in Section \ref{nhitvschid}, 
a larger number of hits than 4 
in channel 70 was observed, 
leading to a different $\Delta t$ distribution 
between hits than expected. 
While the $\Delta t$ peak was expected at 
20 $\mu$s, in some channels two 
additional peaks were identified: 
one at 16 $\mu$s and another at 4 $\mu$s. 
The 4 $\mu$s peak is 
characterized by inverted waveforms 
(Figure \ref{fig:inverted}). 
The 16 $\mu$s peak is characterized 
by regular waveforms, as the 20 $\mu$s peak.
It is reasonable to think that this 
phenomenon arises from the erroneous 
generation of 4 $\mu$s long pulses, 
triggering them on both the leading 
and trailing edges. In fact, 
the sum of 4 $\mu$s and 16 $\mu$s 
is the inverse of the pulser frequency. 
One potential solution to address 
this issue could involve varying the pulse length. 
However, this approach was not pursued 
in this Thesis due to time constraints.
\begin{figure}[!h]
  \centering
  \includegraphics[width=0.65\textwidth]{figures/pdf/wf_ch50_1.pdf}
  \caption[An inverted waveform.]{Inverted waveform of the 50th channel.}
 \label{fig:inverted}
\end{figure}
\subsubsection{The waveform charge and pulse height}\label{threshold}
To determine the the integral of the positive and negative 
parts of the waveform, the pulse height and the first 
sample, a constant threshold discriminator, which we 
set at 5 ADC counts, was used. 
The first sample is defined as the first 
one to be above the threshold, 
and the positive charge 
is the integral of the region above the 
threshold. The negative charge 
is the integral of the waveform's 
negative tail. The pulse height is the 
highest ADC count reached 
by the waveform. The distributions of 
charge, pulse height, first sample and negative charge 
are shown in Figures \ref{fig:ch1}, \ref{fig:ph2}, \ref{fig:fs2}, 
and \ref{fig:nch2}.

\begin{figure}[!h]
      \centering
      \includegraphics[width=0.65\textwidth]{figures/pdf/charge.pdf}
      \caption[The (positive) charge distribution of the waveforms.]{The (positive) charge distribution of the waveforms (channel 66).}
      \label{fig:ch1}
  \end{figure}

\begin{figure}[!h]
  \centering
  \includegraphics[width=0.65\textwidth]{figures/pdf/pulseheight.pdf}
  \caption[The pulse height distribution of the waveforms.]{The pulse height distribution of the waveforms (channel 66).}
  \label{fig:ph2}
\end{figure}

\begin{figure}[!h]
  \centering
  \includegraphics[width=0.65\textwidth]{figures/pdf/fs.pdf}
  \caption[The first sample distribution of the waveforms.]{The first sample distribution of the waveforms (channel 66).}
  \label{fig:fs2}
\end{figure}

\begin{figure}[!h]
    \centering
    \includegraphics[width=0.65\textwidth]{figures/pdf/negcharge.pdf}
    \caption[The negative charge distribution of the waveforms.]{The negative charge distribution of the waveforms (channel 66).}
    \label{fig:nch2}
\end{figure}
We first analysed the distributions 
of the positive charge and pulse height   
The charge and pulse height distribution 
exhibit a non-trivial shape. 
Three and two peaks were observed in both distributions; 
it was verified that 
these peaks were not correlated with 
the number of hits. Moreover, a clear
correlation between the positive charge and pulse 
height peaks is observable (Figure \ref{fig:chvsph}). 
\begin{figure}[!h]
  \centering
  \includegraphics[width=0.65\textwidth]{figures/pdf/phch1.pdf}
  \caption[The 2D distribution of pulse height versus (positive) charge.]{2D distribution of pulse height versus (positive) charge.}
  \label{fig:chvsph}
\end{figure}
To understand the origin of this pattern, we computed $s_{mean}$, 
that is the 
waveform mean sample weighted with the charge: 
\begin{equation} 
  s_{mean} = \frac{\sum_i \text{sample}_i \cdot q_i }{\sum_i q_i} 
\end{equation}
Figure \ref{fig:smean} shows the $s_{mean}$ distribution 
for channel 66. 
This distribution shows that the time difference $\Delta s$ 
between consecutive pulses (approximately 2 ns), is well below the 
bin width, which is 25 ns. 
The next step was to check if $s_{mean}$ could be 
correlated with positive charge and pulse height peaks, 
as demonstrated in Figure \ref{fig:chtm} and Figure \ref{fig:phtm}.
\begin{figure}[!h]
  \centering
  \includegraphics[width=0.65\textwidth]{figures/pdf/tmean1.pdf}
  \caption[The distribution of the waveform $s_{mean}$.]{The distribution of the waveform mean sample weighted with the (positive) charge (channel 66).}
  \label{fig:smean}
\end{figure}
\begin{figure}[!h]
  \centering
  \includegraphics[width=0.65\textwidth]{figures/pdf/chtmean.pdf}
  \caption[2D distribution of (positive) charge versus $s_{mean}$.]{2D distribution of (positive) charge versus $s_{mean}$.}
  \label{fig:chtm}
\end{figure}
\begin{figure}[!h]
  \centering
  \includegraphics[width=0.65\textwidth]{figures/pdf/phtmean1.pdf}
  \caption[2D distribution of pulse height versus $s_{mean}$.]{2D distribution of pulse height versus $s_{mean}$.}
  \label{fig:phtm}
\end{figure}
The pulse height and charge peaks are perfectly 
correlated with the $s_{mean}$ peaks. 
In particular, few ns of delay between waveforms, 
correlated with the ADC clock, 
result in three or two peaks in the positive charge and 
pulse height distributions. 
These peaks are merely artifacts of the pulser 
timing shifted with respect to the ADC 
clock and are not problematic. This behaviour 
was also confirmed with a simple simulation.
The simulation generates two series of triangular 
pulses, each delayed with respect to the other 
and models the behaviour 
of the corresponding ADC outputs. The ADC is 
simulated by partitioning signals into bins and 
calculating the signal mean within each bin. 
The delay between the two series is lower 
than the width of the ADC bins. The output of 
the simulated ADC consists of square waves 
for each series of triangular pulses. The 
integrals and maximum values of the square 
waves over defined intervals are then computed.
The result is shown in Figure \ref{fig:simsim}. 
The simulation is not intended to replicate 
reality in its entirety, 
but rather serves as a tool for understanding 
the correlation between the ADC clock and pulser timing. 
While the simulated positive charge and pulse height 
distributions effectively exhibit peaks, they 
do not precisely 
match the number of peaks seen in Figure 
\ref{fig:ch1} and \ref{fig:ph2}. Since the 
ADC begins integration 
at the same time for both pulse series, any 
temporal shift between waves results in 
lower or greater waveform values on the 
$y$-axis at identical times. This discrepancy 
translates into distinct ADC outputs and
consequently affects the observed pulse height. 
The same principle applies to the charge. 
Moreover, the pulse height peaks exhibit 
clear distinctions with respect to the charge ones, 
a characteristic also evident in Figures 
\ref{fig:ch1} and \ref{fig:ph2}.
\begin{figure}[!h]
  \centering
  \includegraphics[width=0.85\textwidth]{figures/png/pres.png}
  \caption[A simulation of charge and pulse height distribution behaviour.]{Simulation of charge and pulse height distribution behaviour. 
  In the upper plot, triangular pulses representing charge injection waveforms are 
  plotted alongside the square waves from the simulated ADC. The red wave corresponds to the dark red pulses, 
  while the light blue wave corresponds to the dark blue pulses, shifted from the red one. In the central plot, 
  the distribution of charge is displayed, and in the bottom plot, the distribution of pulse height is shown. 
  The light blue distribution corresponds to the light blue ADC, and the same goes for the red one.}
  \label{fig:simsim}
\end{figure}
Lower values of pulse height or positive charge correspond to 
inverted waveforms, as mentioned in Sections \ref{nhitvschid} 
and \ref{wf}, 
or to glitches. In the case of inverted waveforms, 
the positive charge value was $\sim$100 and the pulse height 
$\sim$20, while for glitches, $\lesssim$70 and $\sim$35. 
An example of a glitch is shown in Figure 
\ref{fig:glitches}. The origin of the glitches 
is not yet understood.
\begin{figure}[!h]
  \centering
  \includegraphics[width=0.65\textwidth]{figures/pdf/glitch.pdf}
  \caption[Glitch waveform of channel 0.]{Glitch waveform of channel 0.}
  \label{fig:glitches}
\end{figure}
Special channels with excessive noise were identified 
through the charge distribution. 
Figure \ref{fig:noisywf} shows an example of a noisy waveform. 
Since the threshold mentioned in Section \ref{threshold} 
is 5 ADC counts, charge values for these channels were 
consistently lower than expected (approximately 10). 
This is due to the fact that only the first sample above 
5 was considered as the entire waveform.
To address the issue, the channel voltage threshold was 
either adjusted or the preamp was replaced.
\begin{figure}[!h]
  \centering
  \includegraphics[width=0.65\textwidth]{figures/pdf/noise.pdf}
  \caption[Noisy waveform of the 36th channel.]{Noisy 
  waveform of the 36th channel.}
  \label{fig:noisywf}
\end{figure}

\subsubsection{Channels response uniformity}
From the perspective of data taking, it is important to 
find an easy way to identify as soon as possible 
problematic channels and check the uniformity of channel responses.
For this reason, plots of positive charge, first sample, 
pulse height and baseline means versus channel ID were produced. 
These plots are shown in Figures \ref{fig:means}. 
The outliers in the plots are due to crosstalk, 
noisy channels, inverted waveforms and glitches. 
These histograms are the first example that will be used during the 
online Data Quality Monitoring.
\begin{figure}[!h]
  \centering
  \begin{subfigure}[t]{0.5\textwidth}
      \centering
      \includegraphics[width=\textwidth]{figures/pdf/bl_vs_ch1.pdf}
      \caption{}
  \end{subfigure}%
  ~ 
  \begin{subfigure}[t]{0.5\textwidth}
      \centering
      \includegraphics[width=\textwidth]{figures/pdf/q_vs_ch1.pdf}
      \caption{}
  \end{subfigure}
  ~ 
  \begin{subfigure}[t]{0.5\textwidth}
    \centering
    \includegraphics[width=\textwidth]{figures/pdf/ph_vs_ch1.pdf}
    \caption{}
\end{subfigure}%
~ 
\begin{subfigure}[t]{0.5\textwidth}
    \centering
    \includegraphics[width=\textwidth]{figures/pdf/fs_vs_ch1.pdf}
    \caption{}
\end{subfigure}
  \caption[A first example of Data Quality Monitoring.]{Baseline (a), charge (b), pulse height (c), first sample (d) mean value versus channel ID.}
  \label{fig:means}
\end{figure}

\iffalse
\section{Towards a fully functional DAQ: studies with external CFO}
The brief and straightforward test described in 
this section uses the CFO in a mode expected to be 
implemented during data acquisition. 
In the test, sequences of HBs \ref{des} separated 
by a time constant  
\( T_0 \) were generated and sent to the DTC. The 
test procedure was fully programmable. 
The \textit{run plan} was written using a simple, 
basic-like language, 
compiled, loaded into the CFO, and executed. This 
approach covers nearly 
all conceivable use cases. We operated in two distinct modes:
\begin{itemize}
  \item \textit{Emulated CFO} mode: the DTC generates HBs and 
  EWMs \ref{tdaqtra} and reads one ROC;
  \item \textit{External CFO} mode: the CFO generates HBs and 
  EWMs, sends them to the DTC, and the DTC reads the ROC.
\end{itemize}
Both DTC and CFO were installed in the DAQ computer mu2edaq09. 
The block diagram scheme is similar to the one 
shown in Figure \ref{fig:blockdiagram}, 
with the exception of the CFO and DTC blocks 
(Figure \ref{fig:secondtest}).
In both modes, the CFO (or the DTC, in the 
\textit{emulated CFO} mode) 
sends a sequence of \( (N+1) \) EWMs. Afterward, the DTC 
asynchronously reads \( N \) events. 
Time synchronization between the CFO, DTCs, and 
ROCs was managed either 
internally (using FPGA) or with an external clock. 
The test setup involved two FPGAs - one acting as the CFO and 
one as the DTC - and one tracker test stand (one ROC). 
We conducted short runs with \( N \) events, 
each EW lasting  1.7 $\mu$s. 

\begin{figure}[!h]
  \centering
  \includegraphics[width=0.65\textwidth]{figures/png/Screenshot_20240714_131546.png}
  \caption[The DTC and CFO block diagram with an external CFO.]{The DTC and CFO block diagram 
  used for the DAQ performance studies with an external CFO.}
  \label{fig:secondtest}
\end{figure}



The output was analyzed, and the test was repeated multiple times to ensure reproducibility. 
My contribution to this test involved reading the data, whose format is shown in Figure \ref{fig:dataformat}, 
and comparing the number of events and status codes between runs with the same \textit{run plan}, 
both for \textit{emulated} and \textit{external CFO} mode. 
\begin{figure}[!h]
  \centering
  \includegraphics[width=\textwidth]{figures/png/Screenshot_20240628_104216.png}
  \caption[The data format.]{The data format: each line (pink) in the data sequence is called \textit{data packet}. 
  The first three packets (grey) denote one event. The green and cyan squares represent respectively
  the number of packets in one event and the event window tag, which identifies a single event. 
  The following packets (black) represent the ROC headers, six in total, but just one ROC was connected. 
  The orange and red squares indicate the payload length (the number of packets after 
  the ROC header, including the ROC header) and the ROC ID respectively. The blue square represents 
  the ROC status. There are different status codes: 0x0101 indicates a full payload, 0x0100 an empty one, 
  and 0x0108, 0x0180, and 0x0508 are the codes for timeouts.}
  \label{fig:dataformat}
\end{figure}
Different \textit{run plans} were created, with different number of events generated in both modes:
\begin{itemize}
  \item \textit{External CFO} mode: the chosen number of generated 
  events per run was 66, 130, 258, 514, and 1026. These numbers 
  were chosen because we were reading out the events emulated by 
  the ROC, and the payload is fully defined by the EW tag. 
  It was possible to generate 64 different patterns, and the 
  first two patterns were easy to remember, seen at the end of each run. 
  Therefore, each time the number of events was doubled and 2 events were added at the end.
  \item \textit{Emulated CFO} mode: since it was the most stable mode, 
  it was decided to start with a greater number of events per run (10k and 100k).
\end{itemize}


The results for the \textit{emulated CFO} mode are shown in Figure 
\ref{fig:corruptionemulated}. This figure displays the total number of events, 
the number of empty (0x100) and full events (0x101), and the number of timeouts 
(0x108, 0x180, and 0x508) versus the run number per \textit{run plan}. We expect 
to see all events read out correctly, with just the presence of empty and full events 
as scheduled in the \textit{run plan}. This mode was much more stable compared to the 
\textit{external CFO} mode. As shown in the 10k events generated \textit{run plan} (Figure \ref{fig:corruptionemulated}(a)), 
the events are read out correctly, except for the 13th and the 19th runs. 
For the 100k \textit{run plan}, the majority of the events seem to be read out correctly, 
except for the 5th and 10th runs. It is possible to see a number of 0x0508 timeouts that is 
$10^{-3}$ the total number of events in most of the runs. This suggests the presence of a 
jitter with the external clock, indicating that the results depend on the source of timing synchronization.


\begin{figure}[!h]
  \centering
  \begin{subfigure}[t]{0.5\textwidth}
      \centering
      \includegraphics[width=1.1\textwidth]{figures/pdf/10k.pdf}
      \caption{}
  \end{subfigure}%
  ~ 
  \begin{subfigure}[t]{0.5\textwidth}
      \centering
      \includegraphics[width=1.1\textwidth]{figures/pdf/100k.pdf}
      \caption{}
  \end{subfigure}
  \caption[The number of events versus run number (emulated CFO).]{The \textit{emulated CFO} mode. Subfigures (a) and (b) show 10k and 100k events generated respectively. 
  The total number of events (black), the number of empty events (0x100 - green -) and full events (0x101 - blue -), 
  and the number of timeouts (0x108 - red -, 0x180 - brown -, and 0x508 - pink -) versus the run number in a \textit{run plan} are shown.
  }
  \label{fig:corruptionemulated}
\end{figure} 

The results for the \textit{external CFO} mode are shown in Figure \ref{fig:corruption}. 
These graphics show the total number of events, the number of empty (0x100) and full events (0x101), and 
the number of timeouts (0x108, 0x180, and 0x508) versus the run number per \textit{run plan}.
As can be seen in Figure \ref{fig:corruption}(a), all events are read out correctly, without any timeouts.
However, as the number of events generated increases, issues begin to arise.
In the cases of 130 and 258 events generated (Figures \ref{fig:corruption}(b) and (c)), only the first run 
was not read out correctly, resulting in only timeouts being read out.
Before starting the first run of the series, a new \textit{run plan} was uploaded, 
which suggests there may be an issue with the upload.
The case of 514 events (Figure \ref{fig:corruption}(d)) is different from the previous two: 
only the first run is read out correctly, ruling out previous assumptions. The main problem is 
that in the subsequent runs, approximately 60\% of the events are lost. Only the last events 
show up, and the first Event Window tag of the run is different from zero. The sum of the number 
of events read out and the first EW tag equals 514.
All events in the following runs appear to be timeouts. The same issue occurs with the 
1026 events generated case, where 30\% of events are lost.


\begin{figure}[!h]
  \centering
  \begin{subfigure}[t]{0.5\textwidth}
      \centering
      \includegraphics[width=1.1\textwidth]{figures/pdf/66.pdf}
      \caption{}
  \end{subfigure}%
  ~ 
  \begin{subfigure}[t]{0.5\textwidth}
      \centering
      \includegraphics[width=1.1\textwidth]{figures/pdf/130.pdf}
      \caption{}
  \end{subfigure}
  ~ 
  \begin{subfigure}[t]{0.5\textwidth}
    \centering
    \includegraphics[width=1.1\textwidth]{figures/pdf/258.pdf}
    \caption{}
\end{subfigure}%
~ 
\begin{subfigure}[t]{0.5\textwidth}
    \centering
    \includegraphics[width=1.1\textwidth]{figures/pdf/514.pdf}
    \caption{}
\end{subfigure}
~ 
\begin{subfigure}[t]{0.5\textwidth}
    \centering
    \includegraphics[width=1.1\textwidth]{figures/pdf/1026.pdf}
    \caption{}
\end{subfigure}
  \caption[The number of events versus run number (emulated CFO).]{The  $external \ CFO$ mode. Subfigures (a), (b), (c), (d), (e) show 66, 130, 258, 514 and 1026 events generated respectively. 
  The total number of events (black), the number of empty events (0x100 - green -) and full events (0x101 - blue -), 
  and the number of timeouts (0x108 - red -, 0x180 - brown -, and 0x508 - pink -) versus the run number in a \textit{run plan} are shown.}
  \label{fig:corruption}
\end{figure}



These results have been sent to the firmware developers for further checks.



















\section{High rate software testing}
\subsection{Firmware-Based Data Acquisition}
The Mu2e Trigger and Data Acquisition (TDAQ) system collects digitized data from the Tracker, 
Calorimeter, Cosmic Ray Veto and Beam Monitoring components (Stopping Target Monitor and Extinction Monitor) 
and delivers that data to online and offline processing for analysis. It must merge data from $\sim$450 subsystems 
and apply filters to reduce data volume by a factor of 100 before storing it offline. It is also responsible for 
detector synchronization, control, monitoring and operator interfaces. The Mu2e DAQ system uses a $streaming$ readout technique, 
which means that all detector data from the experiment is digitized and zero-suppressed in their respective Front End Electronics 
(FEEs) before being transferred. This strategy results in a high data flow in the DAQ system while providing greater flexibility in data selection and analysis.
\subsubsection{Expected rate}
The data taking periods will be divided into two modes: on-spill and off-spill. The on-spill mode includes periods when 8 GeV 
proton bunches are colliding with the production target. The off-spill mode includes all other periods, such as between bunches, calibration periods and commissioning.
According to section \ref{accel}, it is possible to estimate the on-spill event contribution as follows:
\begin{itemize}
    \item $\frac{43.1 \text{ ms (time of one spill)}}{1695 \text{ ns (digitization time)}}$ = 25,000 pulses per spill;
    \item 8 (number of spills) $\times$ 25,000 (pulses per spill) = 200,000 on-spill events per cycle;
    \item $\frac{200,000 \text{ (on-spill events per cycle)}}{1.4 \text{ s (cycle time)}}$ = 145,000 on-spill events per second.
\end{itemize}
%di questi 1.4s, 1.055s si riferisce all'offspill e 0.4s all'onspill.
Meanwhile, the off-spill event contribution is:
\begin{itemize}
    \item $145,000 \text{ (on-spill events per second)} \times 0.4 \text{ s (off-spill time)}$ = 58,000 off-spill events per cycle;
    \item $\frac{58,000 \text{ (off-spill events per cycle)}}{1.4 \text{ s (cycle time)}}$ = 41,000 off-spill events per second.
\end{itemize}
The total input rate will be 186,000 events/s. %Hz
There is a mean factor 100 of trigger, so 1800 events/s.
The detectors will generate $\sim$150,000 Byte per event of zero-suppressed data, for an average data rate of $\sim$90 GB/s when beam is present. 
To reduce DAQ bandwidth requirements, this data is buffered in Readout Controller (ROC) memory during the spill period and transmitted to the 
DAQ over the full supercycle for an average data rate of $\sim$28 GB/s.
The total input is 150,000 Byte per event $\times$ 1800 events/s that is 270 MB/s.
\subsection{ARTDAQ}
\section{Data flow testing}
\section{ROC firmware testing}


2214 straw hits per event (DocDB-10898)
- Event = micro bunch (1.6usecs)
 Each hit = 32 bytes (ignores special characters, CRCs…)
 So total 44GB/s
 Assume uniform in z (which is definitely a bad assumption, 6:1
front back asymmetry) this 200MB/s per ROC
\fi

 %(1695 ns), that is 90 GB/s and there is a buffering factor of 0.4/1.4, that makes 28 GB/s.


%In Figure \ref{fig:linktodaq}, the general design of the Mu2e DAQ system is shown.

%The left blocks represent the Readout Controllers (ROCs) in different detectors. The center 
%lock houses the DAQ system's online components, which include the Run Control Host, 40 DAQ servers, 
%the Detector Control System (DCS) and the Event Building Switch. The Run Control Host receives beam status 
%and timing information from the Accelerator Controls network and operator commands from the remote control room. 
%The Detector Control System (DCS) is the window onto the status and health of the Mu2e detector. The Event Building 
%(EVB) function combines these subsets to form a complete detector data set for analysis by an online processor, Ref. 
%\cite{bartoszek2015mu2e}. Event building is typically done in a switching network to sustain high rates. The right block 
%houses the DAQ system's offline components, which are $\mu$sed for data storage and processing. During an active spill 
%(the first approximately 43 ms of the 48 ms bunch extraction cycle outlined in Chapter \ref{accel}), the experiment receives
% RF Zero-Crossing Markers from the Accelerator that are synchronized to the 1695 ns proton pulse cycles (the event windows). 
% Based on these markers, the Command Fan-Out (CFO) module within the Run Control Host generates a 40 MHz system clock and encodes 
% Event Window Markers (EWMs) in the system clock to indicate the start of the event windows. The CFO then sends the encoded system 
% clock, along with run control packets, to Data Transfer Controllers (DTCs) in the DAQ servers. The DTCs\footnote{The Mu2e Data 
% Transfer Controller (DTC), Ref. \cite{ryan} takes data from various Read-Out Controllers and may conduct event construction and 
% data preprocessing. The DTC module connects a maximum of six ROCs to the Trigger and Data Acquisition (TDAQ) servers, 
% which execute the TDAQ online software framework.} then transfer the encoded clock to the detectors' ROCs, where the EWMs 
% are recovered with fixed delay relative to the original RF Zero-Crossing Markers and $\mu$sed in the local ROCs to discriminate 
% data acquired during consecutive event windows. The Tracker generates a DDR3 memory address at the beginning of each event window. 
% The relevant memory area is designated to hold Tracker hits received during that event window. Data requests trigger data readouts 
% from the ROCs to the DAQ system. The Data Requests are modifiable through the CFO as described by the Run Plan, although they are 
% initially given to the Tracker and Calorimeter ROCs via the DTCs following each event window.
% Each tracker hit is composed of a data packet having a fixed length of 128 bits (16 bytes):
 %\begin{itemize}
  %   \item 16 bit header - it contains information as a packet header, a channel identifier to 
   %  specify the channel so the ROC can assign the hit to a wire number and a packet checksum;
    % \item 16 bit - TDC left straw end;
    % \item 16 bit - TDC right straw end;
    % \item 8$\times$10 bit ADC.
% \end{itemize}






%A packet of 128 bits can be transferred every 640 ns (200 Mbps).An additional 32
%bits must be added as an end-of-file marker after the data $\mu$spill hit data is buffered.
%The ROC-to-DAQ connection is made via fiber optic links arranged in rings, with multiple ROC per ring, as shown in Figure \ref{fig:linktodaq}. 
%This is possible since a single optical link can handle 2.6 Gbps, while the ROC output is around 230 Mbps. This value comes from the fact that the 
%highest rate for any 4 straws group (corresponding to one digitizer data line to the ROC) is 240 kHz or 30 Mbps (at 128 bits/hit) and the Main Injector 
%supplies Mu2e beam only 32\% of the time. The ROC monitors slow control variables and controls panel operations too.


