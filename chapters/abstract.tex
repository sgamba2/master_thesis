\begin{abstract}
\noindent
The primary objective of the Mu2e experiment is to search for neutrino-less coherent $\mu \rightarrow e$ conversion in the 
field of an aluminum nucleus, $\mu^- \text{Al} \rightarrow e^-\text{Al}$. This process's signal is a monoenergetic electron 
with an energy of approximately 104.97 MeV, Ref. \cite{bartoszek2015mu2e}. 
In the Standard Model (SM), the branching ratio for this process, or for the similarly 
intriguing $\mu \rightarrow e \gamma$ decay, is expected to be of the order of $\mathcal{O}(10^{-54})$. 
These values are well below the sensitivity of any current experimental capabilities, but several models of physics beyond the 
SM predicts a much higher relative rate that could reach an observable level in some super-symmetry scenarios. 
The SINDRUM II experiment established an upper limit on muon conversion 
at $7 \times 10^{-13}$ (90\% C.L.) \cite{SINDRUMII:2006dvw}, and the Mu2e collaboration aims to improve this limit by four orders of magnitude. 
Observing this process would provide unambiguous evidence of physics beyond the Standard Model and its search is complementary to 
the direct searches for new physics currently carried out at the CERN Large Hadron Collider. 

Mu2e will make use of a sophisticated setup to achieve its goals. An 8.9 GeV proton beam from Fermilab will strike a tungsten target, producing $\mu$s through $\pi$s decay. 
These particles are transported via a series of superconducting solenoids to an aluminum stopping target, where the conversion may occur. 
The system includes multiple components to precisely identify the conversion electrons and distinguish them from various backgrounds, including a straw tracker, 
an electromagnetic calorimeter, a Cosmic Ray Veto and a Stopping Target Monitor.

One of the most important Mu2e subsystems is the tracker, which must provide very good momentum resolution to distinguish the monochromatic 
conversion electron signal from the background. Energy loss in the tracker needs to be minimal; thus, the experiment will use a straw 
tube tracker, Ref. \cite{bobbb}. Its annular shape follows the helical trajectories of conversion electrons in the magnetic field. 
The tracker consists of panels containing arrays of straw tubes arranged like hap chords. It is required to measure the conversion electrons with a momentum 
resolution better than 180 keV/c. 

My activity at Fermilab has focused on the commissioning of the Mu2e DAQ system and the Vertical Slice Test (VST) of the tracker. 
The VST encompasses the entire testing chain from the straws to the readout, to processed data on disk. Our ultimate goal was to conduct 
a time calibration of the first assembled station of the tracker, aiming for a longitudinal position resolution better than 4 cm, and to determine the channel-to-channel delay and the drift velocity.

Before proceeding, validating data from the tracker panels is necessary. The tracker's readout system has undergone several 
tests to validate its performance, ensuring accurate and reliable data collection. This includes testing readout logic and firmware 
with simulated and real data to confirm functionality, and developing methods to monitor and validate data accuracy, including identifying and addressing cross-talk and noise issues.

Our objective is to gather data from cosmic rays to develop the time calibration of the tracker. First, it was important to assess the feasibility of 
performing this calibration with a vertical orientation of the first station. For this purpose, straight cosmic rays were simulated in one station to 
understand how the varying luminosity of the straws can impact timing calibration and hit segment reconstruction. This calibration is crucial for achieving 
precise momentum resolution, relying on accurate reconstruction of straw hit positions. Our goal is to determine the drift velocity in the straws and the time 
offset from channel to channel. This involves reconstructing straight cosmic segments within each station by assessing whether a straw has been hit and the 
relative orientations of the straws. From there, we can locate intersection points of straws on a 2D plane, assuming these to be the particle hit coordinates. 
This process is repeated for all pairs of straws, followed by fitting all the points together. The actual position of a hit within a straw can then be reconstructed, 
identifying biases by comparing the MC hits position with the reconstructed ones. Our target is to achieve a longitudinal resolution better than 4 cm, ensuring that every 
bias remains below this threshold. During data collection, each straw will undergo calibration by correlating the time difference between ends with the reconstructed position.
\end{abstract}
