\begin{abstract}
\noindent


The primary objective of the Mu2e experiment at Fermilab is to search 
for the neutrino-less coherent $\mu \rightarrow e$ conversion in the field of an 
aluminum nucleus ($\mu^- \text{Al} \rightarrow e^- \text{Al}$). The signature 
of this process is a monochromatic conversion electron with an energy of 
approximately 104.97 MeV \cite{bartoszek2015mu2e}. Within the Standard 
Model (SM), the branching ratio for this process, including neutrino masses 
and oscillation, is expected to be less than $\mathcal{O}(10^{-50})$.
This value is far beyond current experimental capabilities.
However, models of physics beyond the SM predict much higher relative rates,
approaching an observable level.
The SINDRUM II experiment set an upper limit on muon conversion at $7 \times 10^{-13}$ (90\% C.L.) on Au target 
\cite{SINDRUMII:2006dvw}, and the Mu2e collaboration aims to improve this limit by 
four orders of magnitude. Observing this process would provide a clear evidence of physics 
beyond the Standard Model. A brief discussion on the theoretical and 
experimental components essential to understand the objectives 
of the Mu2e experiment is provided in Chapter \ref{intr}.


Mu2e adopts a sophisticated experimental setup to achieve its goals, further 
described in Chapter \ref{mu2echapter}. An 8.9 GeV proton 
beam from Fermilab strikes a tungsten target, producing muons through pion decays.
These particles are transported via a series of superconducting solenoids to an aluminum 
stopping target, where the conversion may occur. The system incorporates several 
components to accurately identify the conversion electrons and differentiate 
them from various backgrounds, including a straw tracker, an electromagnetic 
calorimeter, a cosmic ray veto. 


The tracker must provide excellent momentum resolution, approximately 1 MeV/c, 
to distinguish the monochromatic conversion electron signal from the background. 
To minimize energy loss, a straw tube tracker will be used \cite{bobbb}.
The detector has a modular design, consisting 
of basic elements referred to as $panel$s, $face$s, $plane$s, and $station$s. The 
tracker comprises panels containing arrays of straw tubes, which are the 
sensitive units, arranged like the harp chords. Chapter \ref{chaptertrk} 
provides an overview of the working principles of the straw tracker and its modular design.


This Thesis presents a comprehensive study of the Mu2e tracker, 
covering complementary aspects from initial commissioning to optimization and first 
steps of the calibration processes. My work at Fermilab has focused on the 
complete Data Acquisition (DAQ) testing from both hardware and software 
perspectives. I was involved in the commissioning of the Mu2e DAQ system and 
the Vertical Slice Test (VST) of the tracker. The VST encompasses the entire 
testing chain, from the straws to the readout, to processed data on disk.


Chapter \ref{commissioning} details the commissioning of the tracker 
DAQ system, emphasizing the importance of understanding the readout 
process before data acquisition. This includes validating the readout 
logic and firmware through Monte Carlo simulations to confirm functionality 
and buffering, monitoring data quality from the tracker preamplifiers and 
front-end electronics, and assessing overall DAQ performance to ensure 
reliability during future calibration and data-taking.

Chapter \ref{planning} discusses the initial steps towards the tracker 
calibration. Our ultimate goal is to perform a time calibration of 
the first assembled station of the tracker using cosmic muons, aiming for a longitudinal hit 
position resolution better than 4 cm. This involves determining the signal 
propagation times and channel-to-channel delays. Due to technical constraints, the 
first option is to perform the with a vertically oriented station. I performed a 
Monte Carlo study to determine the impact of this orientation on the quality of the calibration, 
in particular on the cosmic track reconstruction, focusing on the 
potential biases that could arise. Together, 
these studies provide essential insights into the operation, optimization, 
and calibration of the Mu2e tracker system, contributing to its 
overall performance and reliability.

Given the high data volume expected during Mu2e operations, estimated 
at approximately 7 PBytes per year, optimizing memory usage and minimizing 
CPU consumption are critical. A significant challenge lies in effectively 
flagging $\delta$-electron hits, which are the primary source of hits in 
the tracker, without compromising the efficiency of conversion electron 
hit detection and track reconstruction. $\delta$-electrons originate from 
Compton scattered electrons, pair production electrons and positrons, and 
delta rays. Compton scattered electrons are produced when photons, from 
neutron capture, interact with the detector material. Typically, these 
photons have energies of a few MeV. Pair production electrons and positrons 
are generated during nuclear recoil processes. Delta rays 
are produced when high-energy charged particles collide 
with the detector material. A detailed study of pre-pattern recognition 
and a thorough comparison of two algorithms for $\delta$-electron flagging is 
provided in Chapter \ref{delta} to address this issue.





\end{abstract}