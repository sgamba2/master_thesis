\begin{abstract}
\noindent


The primary objective of the Mu2e experiment at Fermilab is to search 
for the neutrino-less coherent $\mu \rightarrow e$ conversion in the field of an 
aluminum nucleus ($\mu^- \text{Al} \rightarrow e^- \text{Al}$). The signature 
of this process is a monochromatic Conversion Electron (CE) with an energy of 
approximately 104.97 MeV \cite{bartoszek2015mu2e}. Within the Standard 
Model (SM), the branching ratio for this process, including neutrino masses 
and oscillation, is expected to be less than $\mathcal{O}(10^{-50})$.
This value is far beyond current experimental capabilities.
However, models of physics beyond the SM predict much higher relative rates,
approaching an observable level.
The SINDRUM II experiment set an upper limit on muon conversion at $7 \times 10^{-13}$ (90\% CL) on Au target 
\cite{SINDRUMII:2006dvw}, and the Mu2e collaboration aims to improve this limit by 
four orders of magnitude. Observing this process would provide a clear evidence of physics 
beyond the Standard Model. A brief discussion of the theoretical and 
experimental aspects is provided in Chapter \ref{intr}.


Mu2e adopts a sophisticated experimental setup to achieve its goals, further 
described in Chapter \ref{mu2echapter}. 
The central part of the Mu2e detector is the tracker, that consists of 18 tracking stations. 
The tracker must provide excellent momentum resolution, approximately 1 MeV/c, 
to distinguish the monochromatic CE signal from the background. 
To minimize the energy losses, a straw tube tracker will be used \cite{bobbb}.
Chapter \ref{chaptertrk} provides an overview of the straw tracker design and its 
working principles.


This Thesis presents a comprehensive study of the Mu2e tracker, 
covering complementary aspects from initial commissioning to optimization and first 
steps of the calibration processes. My work at Fermilab has been focused on the 
complete Data Acquisition (DAQ) testing from both hardware and software 
perspectives. I was involved in the commissioning of the Mu2e DAQ system and 
the Vertical Slice Test (VST) of the tracker. The VST encompasses the entire 
testing chain, from the straws to the readout, and to processed data on disk. 
I was also focused on the offline analysis, especially on pre-pattern 
recognition studies, to explore the best methods for identifying 
$\delta$-electrons during the data taking.


Chapter \ref{commissioning} details the commissioning of the tracker 
DAQ system, emphasizing the importance of understanding of the readout 
process before the data acquisition. This includes validating the readout 
logic and firmware through Monte Carlo simulations to confirm functionality 
and buffering, monitoring the quality of the data from the tracker preamplifiers and 
front-end electronics, and assessing overall DAQ performance to ensure 
reliability during future calibration and data-taking.

Chapter \ref{planning} discusses the initial steps towards the tracker 
calibration. The ultimate goal is to perform a time calibration of 
the first assembled station of the tracker using cosmic muons, aiming for a longitudinal hit 
position resolution better than 4 cm. This involves determining the signal 
propagation times and channel-to-channel delays. I performed a 
Monte Carlo study to determine the impact of the station orientation
on the quality of the calibration, in particular on the cosmic track reconstruction,
focusing on potential biases that could arise. These studies provide essential insights into the operation, optimization, 
and calibration of the Mu2e tracker system.

Given the high data volume expected during Mu2e operations, estimated 
at approximately 7 PBytes per year, optimizing memory usage and minimizing 
CPU consumption are critical. A significant challenge lies in effectively 
flagging $\delta$-electron hits, which are the primary source of hits in 
the tracker, without compromising the efficiency of CE  
hit detection and track reconstruction. A detailed study of pre-pattern recognition 
and a thorough comparison of two $\delta$-electron flagging algorithms is 
provided in Chapter \ref{delta}.

In Chapter \ref{conclusions}, the findings are concisely summarized, 
offering a comprehensive synthesis of the research and emphasizing the 
key insights derived from this study.



\end{abstract}