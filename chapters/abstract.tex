\begin{abstract}
\noindent


The primary objective of the Mu2e experiment at Fermilab is to search 
for neutrino-less coherent $\mu \rightarrow e$ conversion in the field of an 
aluminum nucleus ($\mu^- \text{Al} \rightarrow e^- \text{Al}$). The signature 
of this process is a monochromatic conversion electron with an energy of 
approximately 104.97 MeV \cite{bartoszek2015mu2e}. Within the Standard 
Model, the branching ratio for this process, as well as for the $\mu \rightarrow e \gamma$ 
decay, {\red is expected to be of the order of $\mathcal{O}(10^{-54})$}.

\todo[inline,color=green!20,linecolor=gray,tickmarkheight=10pt] {not correct, double check}

These values 
are far \del{below the sensitivity of}\add{beyond} current experimental capabilities.
However, \del{several}
models of physics beyond the Standard Model predict \del{a} much higher relative rate\add{s},
\del{potentially reaching}\add{approaching} an observable level\del{ in certain supersymmetry scenarios}.
The SINDRUM II experiment 
set an upper limit on muon conversion at $7 \times 10^{-13}$ (90\% C.L.) \add{on Au target} 
\cite{SINDRUMII:2006dvw}, and the Mu2e collaboration aims to improve this limit by 
four orders of magnitude. {\red Observing this process would provide a clear evidence of physics 
beyond the Standard Model, complementing direct searches for new physics currently 
conducted at the CERN Large Hadron Collider.}

\todo[inline,color=green!20,linecolor=gray,tickmarkheight=10pt] {why would an observation complement searches?}

Mu2e adopts a sophisticated experimental setup to achieve its goals. An 8.9 GeV proton 
beam from Fermilab strikes a tungsten target, producing muons through pion decays.
These particles are transported via a series of superconducting solenoids to an aluminum 
stopping target, where the conversion may occur. The system incorporates several 
components to accurately identify the conversion electrons and differentiate 
them from various backgrounds, including a straw tracker, an electromagnetic 
calorimeter, a Cosmic Ray Veto, {\red and a Stopping Target Monitor.}

\todo[inline,color=green!20,linecolor=gray,tickmarkheight=10pt]
{what is the purpose of mentioning the  STM in the abstract ? }

The tracker must provide excellent momentum resolution, approximately 1 MeV/c, 
to distinguish the monochromatic conversion electron signal from the background. 
To minimize energy loss, a straw tube tracker will be used \cite{bobbb}.

{\red Its annular shape is designed to follow the helical trajectories of conversion 
  electrons in the magnetic field.}

\todo[inline,color=green!20,linecolor=gray,tickmarkheight=10pt]
{ really ? }

The detector has a modular design, consisting 
of basic elements referred to as $Panel$s, $Face$s, $Plane$s, and $Station$s. The 
tracker comprises panels containing arrays of straw tubes, which are the 
sensitive units, arranged like the harp chords.

This Thesis presents a comprehensive study of the Mu2e tracker \del{system}, 
covering various aspects from initial commissioning to optimization and first 
steps of the calibration processes. My work at Fermilab has focused on the 
complete Data Acquisition (DAQ) testing from both hardware and software 
perspectives. I was involved in the commissioning of the Mu2e DAQ system and 
the Vertical Slice Test (VST) of the tracker. The VST encompasses the entire 
testing chain, from the straws to the readout, to processed data on disk.


Chapter \ref{commissioning} details the commissioning of the tracker 
DAQ system, emphasizing the importance of understanding the readout 
process before data acquisition. This includes validating the readout 
logic and firmware through Monte Carlo simulations to confirm functionality 
and buffering, monitoring data quality from the tracker preamplifiers and 
front-end electronics, and assessing overall DAQ performance to ensure 
reliability during future calibration and data-taking.


Given the high data volume expected during Mu2e operations, estimated 
at approximately 7 PBytes per year, optimizing memory usage and minimizing 
CPU consumption are critical. A significant challenge lies in effectively 
flagging $\delta$-electron hits, which are the primary source of hits in 
the tracker, without compromising the efficiency of conversion electron 
hit detection and track reconstruction. $\delta$-electrons originate from 
Compton scattered electrons, pair production electrons and positrons, and 
delta rays. Compton scattered electrons are produced when photons, from 
neutron capture, interact with the detector material. Typically, these 
photons have energies of a few MeV. Pair production electrons and positrons 
are generated during nuclear recoil processes. Delta rays, {\red or secondary 
  ionized electrons},

\todo[inline,color=green!20,linecolor=gray,tickmarkheight=10pt]{ what is a secondary ionized electron' ? }

are produced when high-energy charged particles collide 
with the detector material. A detailed study of pre-pattern recognition 
and a comparison of two algorithms for $\delta$-electron flagging is 
provided in Chapter \ref{delta} to address this issue.


Chapter \ref{planning} discusses the initial steps towards the calibration 
phase of the tracker. Our ultimate goal is to conduct a time calibration of 
the first assembled station of the tracker, aiming for a longitudinal 
position resolution better than 4 cm. This involves determining \add{the} signal 
propagation times and channel-to-channel delays using cosmic muons. The 
calibration will be conducted with a vertically oriented station due 
to technical constraints, and the impact of this orientation on trajectory 
reconstruction and potential biases is thoroughly analyzed. Together, 
these studies provide essential insights into the operation, optimization, 
and calibration of the Mu2e tracker system, contributing to its 
overall performance and reliability.


\end{abstract}