\begin{abstract}
\noindent


The primary objective of the Mu2e experiment is to search for neutrino-less 
coherent $\mu \rightarrow e$ conversion in the field of an aluminum nucleus, 
$\mu^- \text{Al} \rightarrow e^-\text{Al}$. The signal for this process is a 
monoenergetic electron with an energy of approximately 104.97 MeV \cite{bartoszek2015mu2e}. 
In the Standard Model (SM), the branching ratio for this process, as 
well as for the similarly intriguing $\mu \rightarrow e \gamma$ 
decay, is expected to be of the order of $\mathcal{O}(10^{-54})$. 
These values are well below the sensitivity of current experimental 
capabilities. However, several models of physics beyond the SM predict 
a much higher relative rate, potentially reaching an observable level 
in some supersymmetry scenarios. The SINDRUM II experiment established 
an upper limit on muon conversion at $7 \times 10^{-13}$ (90\% C.L.) 
\cite{SINDRUMII:2006dvw}, and the Mu2e collaboration aims to improve 
this limit by four orders of magnitude. Observing this process would 
provide unambiguous evidence of physics beyond the Standard Model, 
complementing direct searches for new physics currently conducted at 
the CERN Large Hadron Collider.

Mu2e employs a sophisticated setup to achieve its goals. An 8.9 GeV 
proton beam from Fermilab strikes a tungsten target, producing $\mu$s 
through $\pi$s decay. These particles are transported via a series of 
superconducting solenoids to an aluminum stopping target, where the 
conversion may occur. The system includes multiple components to precisely 
identify the conversion electrons and distinguish them from various backgrounds, 
including a straw tracker, an electromagnetic calorimeter, a Cosmic Ray Veto, and a Stopping Target Monitor.

The tracker must provide excellent 
momentum resolution to distinguish the monochromatic 
conversion electron signal from the background. To 
minimize energy loss, a straw tube tracker will be used 
\cite{bobbb}. Its annular shape follows the helical 
trajectories of conversion electrons in the magnetic field. 
The detector has a modular design, made of basic elements named $Panel$, $Face$, 
$Plane$ and $Station$. The tracker consists of panels containing 
arrays of straw tubes, which are the sensitive units, arranged like harp chords. 

My work at Fermilab has focused on the complete DAQ testing from both 
hardware and software perspectives. I was involved in the commissioning 
of the Mu2e DAQ system and the Vertical Slice Test (VST) of the tracker. 
The VST encompasses the entire testing chain, from the straws to the readout, to processed data on disk.

The tracker's DAQ and front-end electronics (FEE) have undergone several tests 
to validate their performance, ensuring reliability during future calibrationa and data-taking. 
This includes testing readout logic and firmware with emulated data to confirm functionality and the buffering 
of the DAQ components. The FEE monitoring and validation was performed. In particular, all preamps of a tracker panel 
were tested and identifying and addressing cross-talk and noise issues, to garantee the correct functionality.

The Mu2e experiment is expected to generate 7 PBytes of data annually, primarily 
from $\delta$-electrons (Compton scattered electrons, pair production electrons and positrons, and delta rays). 
Compton scattered electrons are generated when photons, from neutron capture, interact with the detector material. 
Typically, these photons have energies of a few MeV. Pair production electrons and positrons are 
generated during nuclear recoil processes. Delta rays, or secondary ionized electrons, 
are generated when high-energy charged particles collide with the detector material. 
It is crucial to effectively flag these hits 
without compromising the efficiency of CE and other background hit detection and track reconstruction. 
From the software point of view, I performed a 
comparison between the two algorithms developed to flag these hits, improving their performance. This analysis is 
important as optimizing memory usage and minimizing CPU consumption is crucial.

Our ultimate goal was to conduct a time calibration of the first assembled station 
of the tracker, aiming for a longitudinal position resolution better than 4 cm, 
and to determine the channel-to-channel delay and the propagation velocity on the 
wire. Our objective is to gather data from cosmic rays to develop the time calibration 
of the tracker, ensuring that every bias remains below this threshold. First, it was 
important to assess the feasibility of performing this calibration with a vertical orientation 
of the first station, considering mechanical contingencies. To this end, straight cosmic rays 
were simulated in one station to understand how the varying luminosity of the straws can impact 
timing calibration and hit segment reconstruction. This calibration is crucial for achieving 
precise momentum resolution, relying on accurate reconstruction of straw hit positions. This 
involves reconstructing straight cosmic segments within each station by assessing whether a 
straw has been hit and the relative orientations of the straws. From there, we can locate 
intersection points of straws on a 2D plane, assuming these to be the particle hit coordinates. 
This process is repeated for all pairs of straws, followed by fitting all the points together. 
The actual position of a hit within a straw can then be reconstructed, identifying biases by 
comparing the Monte Carlo hit positions with the reconstructed ones.
\end{abstract}
