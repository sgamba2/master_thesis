\chapter{Comparing $\delta$-electrons flagging algorithms}\label{delta}
\textit{In the context of DAQ testing and performance analysis during Mu2e data-taking, 
where the data volume is expected to reach approximately 7 PBytes per year, optimizing 
memory usage and minimizing CPU consumption is crucial. The primary source of hits in the 
Mu2e tracker will be $\delta$-electrons. Therefore, it is essential to effectively flag these 
hits without compromising the efficiency of CE hit detection and track reconstruction. This 
Chapter presents a comparison between two algorithms designed for $\delta$-electron flagging.}

\section{$\delta$-electrons as source of background}

The primary source of hits in Mu2e is due to $\delta$-electrons, 
which mainly consist of Compton-scattered electrons, pair-production 
electrons and positrons, and delta rays, listed in decreasing 
order of prevalence. Compton-scattered electrons are produced 
when photons, generated by various processes, interact with the 
detector material. These photons primarily originate from 
neutron capture by atoms, which excites the nucleus and leads 
to photon emission during decay. Typically, these photons have 
energies of a few MeV. The neutrons are produced when muons are 
captured by atoms, resulting in unstable isotopes that decay by 
emitting neutrons. Pair-production electrons and positrons are 
generated during nuclear recoil processes, where pairs of electrons 
and positrons are created to conserve energy and momentum. Delta rays, 
or secondary ionization electrons, are generated when high-energy 
charged particles collide with the detector material.


\subsection{Compton scattering}
The Compton effect (Figure \ref{fig:compt}) refers to the 
scattering of a photon by a free or quasi-free electron. 
An electron is considered "quasi-free" when the energy of 
the incoming photon is significantly higher than the 
electron's binding energy ($E_\gamma \gg E_B$). The 
scattering process is termed Compton scattering if the 
electron is ejected from the atom, carrying away the 
recoil momentum. This effect is most prominent in an 
extended energy region around 1 MeV, with the region 
being much larger for low $Z$ materials compared to high $Z$ materials.

\begin{figure}[!h]
    \centering
    \includegraphics[width =0.4\textwidth]{figures/png/Screenshot_20240812_204345.png}
    \caption[The Compton effect.]{
    The Compton effect \cite{kola}.}
    \label{fig:compt}
\end{figure}

Since the photon scatters quasi-elastically off the electron, 
the energy and angle of the scattered photon are interdependent. 
To describe this relationship, we use the 4-momenta defined as 
follows: $k = (E_\gamma, \mathbf{k}c)$ and $p_e = (m_e c^2, 0)$ 
represent the 4-momenta of the photon and the electron (at rest) 
before scattering, and $k' = (E'_\gamma, \mathbf{k}'c)$ and 
$p'_e = (E'_e, \mathbf{p}'_e c)$ represent the 4-momenta 
after scattering. The angle between the scattered photon and 
the incident photon is denoted as $\theta_\gamma$, while the 
angle of the electron is denoted as $\theta_e$. By applying 
energy-momentum conservation:

\begin{equation}\label{compcons}
k + p_e = k' + p'_e
\end{equation}
\begin{equation}\label{compcons2}
(k - k')^2 = (p'_e - p_e)^2 \Rightarrow -k \cdot k' = m_e^2 c^4 - p'_e \cdot p_e
\end{equation}
\begin{equation}
\Rightarrow E_\gamma E'_\gamma (1 - \cos \theta_\gamma) = m_e c^2 \left(E'_e - m_e c^2\right) = m_e c^2 \left(E_\gamma - E'_\gamma \right)
\end{equation}

The right-hand side of the last equation uses the kinetic energy of the electron:

\begin{equation}
T = E'_e - m_e c^2 = E_\gamma - E'_\gamma
\end{equation}

which follows from the energy part of equation \ref{compcons}. 
The energy of the scattered photon as a function of the scattering 
angle is derived from equation \ref{compcons2}:

\begin{equation}\label{diffeq}
E'_\gamma = \frac{E_\gamma}{1 + \epsilon (1 - \cos \theta_\gamma)}
\end{equation}

where $\epsilon = \frac{E_\gamma}{m_e c^2}$.

The differential cross section per (free) electron, known as the 
Klein-Nishina formula, is calculated using methods from quantum electrodynamics:

\begin{equation}\label{kleinnishina}
\frac{d\sigma}{d\Omega} = \frac{r_e^2}{2} \frac{1 + \epsilon (1 - \cos \theta_\gamma)}{[1 + \epsilon (1 - \cos \theta_\gamma)]^2} \left(1 + \cos^2 \theta_\gamma + \frac{\epsilon^2 (1 - \cos \theta_\gamma)^2}{1 + \epsilon (1 - \cos \theta_\gamma)} \right)
\end{equation}

An electron bound in an atom can only be considered quasi-free 
if the photon's energy is significantly higher than the electron's 
binding energy. As the photon energy increases, more shell electrons 
become quasi-free, leading to the Compton cross section per atom 
approaching proportionality to $Z$, with individual electrons 
contributing incoherently:

\begin{equation}
\sigma_C^{\text{atom}} = Z\sigma_C
\end{equation}

where $\sigma_C$ is the Klein-Nishina cross section for a 
single free electron. The Compton cross section decreases at 
lower energies, where coherent scattering (Rayleigh scattering) 
off the entire atom (without ionizing the electron shell) becomes dominant.

By reformulating the Klein-Nishina formula, one can obtain the 
differential dependence of the Compton cross section on the 
kinetic energy of the recoil electron $T = E_\gamma - E'_\gamma$:

\begin{equation}
\frac{d\sigma}{dT} = \frac{\pi r_e^2}{m_e c^2 \epsilon^2} \left[2 + \frac{t^2}{\epsilon^2 (1 - t)^2} + \frac{t}{1 - t}\left(t - \frac{2}{\epsilon}\right)\right]
\end{equation}

where $t = T/E_\gamma$. Because the scattering process is 
elastic, there is a one-to-one relationship between the 
energy and angle $\theta_e$ of the electron:

\begin{equation}
\cos \theta_e = \frac{T(E_\gamma + m_e c^2)}{E_\gamma \sqrt{T^2 + 2m_ec^2 T}} = \frac{1 + \epsilon}{\sqrt{\epsilon^2 + 2\epsilon/t}}
\end{equation}

The maximum energy transfer to the electron is obtained 
from equation \ref{diffeq} for backward scattering of the 
photon ($\theta_\gamma = 180^\circ$), corresponding to 
forward scattering of the electron ($\theta_e = 0^\circ$). 
The electron's kinetic energy reaches its maximum value in 
this case, $T \rightarrow T_{\text{max}}$. In the measured 
energy spectrum, this leads to the so-called "Compton edge" at:

\begin{equation}
T_{\text{max}} = \frac{E_\gamma \cdot 2\epsilon}{1 + 2\epsilon}
\end{equation}

which lies slightly below the photopeak. The energy difference 
between the photopeak and the Compton edge $E'_\gamma(\theta = \pi)$ 
decreases with increasing $E_\gamma$ and approaches:

\begin{equation}
E'_\gamma(\theta = \pi) \approx \frac{m_e c^2}{2} \text{ for } E_\gamma \gg m_e c^2
\end{equation}

\subsection{Pair production}
In the Coulomb field of a charge, a photon can 
convert into an electron-positron pair (Figure 
\ref{fig:pprod})\footnote{Photon emission by an 
electron (bremsstrahlung) and pair production are closely 
related processes. By modifying the bremsstrahlung diagram-changing 
the outgoing photon to an incoming one and the incoming electron to 
an outgoing positron-one obtains the pair production 
diagram. The matrix elements of these processes are 
related, at least in the lowest order. Consequently, both 
processes are treated together in the foundational work by 
Bethe and Heitler, often referred to as the 'Bethe-Heitler processes'.}.

\begin{figure}[!h]
    \centering
    \includegraphics[width=0.4\textwidth]{figures/png/Screenshot_20240812_204755.png}
    \caption[The pair production.]{The pair production \cite{kola}.}
    \label{fig:pprod}
\end{figure}

The energy of the photon must exceed twice the electron 
mass plus the recoil energy transferred to the field-producing 
charge. For most elements, pair production predominantly 
occurs in the Coulomb field of the nucleus. For nuclei, 
the recoil energy is usually negligible, leading to a 
threshold energy for pair production of:
\begin{equation}
    E_{\gamma} \geq 2m_e c^2 + 2 \frac{m_e^2}{m_{\text{nucleus}}} c^2
\end{equation}

If the nuclear charge is not screened by atomic electrons 
(for low energies, the photon must come relatively close to 
the nucleus to make pair production probable, meaning it 
interacts with the "bare" nucleus),
\begin{equation}
    1 \ll \epsilon \ll \frac{1}{\alpha Z^{1/3}}
\end{equation}
the pair-production cross section is given by:

\begin{equation}
    \sigma_{\text{pair}} = 4 \alpha r_e^2 Z^2 \left(\frac{7}{9} \ln 2 \epsilon - \frac{109}{54}\right) \text{ cm}^2/\text{atom}
\end{equation}

However, for complete screening of the nuclear charge ($\epsilon \gg 1/\alpha Z^{1/3}$):
\begin{equation}\label{sigmapair}
    \sigma_{\text{pair}} = 4 \alpha r_e^2 Z^2 \left(\frac{7}{9} \ln \frac{183}{Z^{1/3}} - \frac{1}{54}\right) \text{ cm}^2/\text{atom}
\end{equation}

At high energies, pair production can occur even 
at relatively large impact parameters between the 
photon and the nucleus. In this case, the screening 
effect of atomic electrons must be considered. For 
large photon energies, the pair-production cross 
section approaches an energy-independent value as given by
 Equation \ref{sigmapair}. Ignoring the small term in the equation, 
 the asymptotic value of $1/54$ is expressed as:
\begin{equation}
    \sigma_{\text{pair}} \approx \frac{7}{9} \cdot 4 \alpha r_e^2 Z^2 \ln\left(\frac{183}{Z^{1/3}}\right) \approx \frac{7}{9} \cdot \frac{1}{X_0} \cdot \frac{A}{N_A \rho}
    \label{eq:paircross_radiationlength}
\end{equation}

The energy is uniformly distributed between the produced 
electrons and positrons at low and medium energies, but becomes 
slightly asymmetric at high energies.

The field of the nucleus is formed by the coherent sum of $Z$ 
nucleon charges, leading to the $Z^2$ dependence of the pair 
production cross section.

Even with large momentum transfers $\Delta p$ to the nucleus, 
the energy transfer $(\Delta p)^2/2M$ remains small due to the 
large nuclear mass $M$. After pair creation, the remaining 
energy is equally divided between the $e^+$ and the $e^-$.

\subsection{Delta rays}
High-energy $\delta$-rays, or $knock-on$ electrons, 
are produced when a projectile particle collides 
centrally with shell electrons, resulting in 
significant energy transfers. These electrons 
gain high kinetic energy and can be described 
through elastic collisions with quasi-free electrons. 
By considering the energy-momentum conservation relation 
and using the Lorentz factors $\gamma$ and $\beta$, the 
relationship between the kinetic energy $T$ of the 
$\delta$-ray and the emission angle $\theta$ can be derived as:

\begin{equation}
\cos \theta = \frac{T(\gamma + m_e / M)}{\gamma \beta \sqrt{T^2 + 2T m_e c^2}}
\end{equation}

\begin{equation}
T(\theta) = \frac{2 m_e c^2 \beta^2 \gamma^2 \cos^2 \theta}{\gamma^2(1 - \beta^2 \cos^2 \theta) + 2 \gamma m_e / M + m_e^2 / M^2}
\end{equation}

The maximum energy transfer $T_{\text{max}}$ occurs at $\theta = 0^\circ$, 
while the minimum energy, $T_{\text{min}}$, occurs at $\theta = 90^\circ$. 
At highly relativistic energies ($\gamma \gg 1$ and $\theta \gg 1/\gamma$), 
the energy-angle relationship becomes independent of the incoming particle's properties.

The rate of $\delta$-rays per energy interval $dT$ and path length $dx$ is given by:

\begin{equation}
\frac{d^2 N}{dx \, dT} = n_e \frac{d\sigma}{dT}
\end{equation}

which, when combined with the electron density and the differential cross section, becomes:

\begin{equation}
\frac{d^2 N}{dx \, dT} = \frac{1}{2} z^2 \frac{Z}{A} K \rho \frac{1}{\beta^2} \frac{F(T)}{T^2}
\end{equation}

Here, $K$ is the constant from the Bethe-Bloch formula, 
and $F(T)$ is a function accounting for spin dependence. 
Integration over $T$ and $x$ provides the number of $\delta$-rays in a medium of thickness $\Delta x$:

\begin{equation}
N = \frac{1}{2} z^2 \frac{Z}{A} K \rho \Delta x \frac{1}{\beta^2} \left(\frac{1}{T_{\text{min}}} - \frac{1}{T_{\text{max}}}\right) \approx 0.077 \frac{\text{MeV cm}^2}{\text{g}} z^2 \rho \Delta x \frac{1}{T_{\text{min}}}
\end{equation}

The emission angle dependence is given by:

\begin{equation}
\frac{dT}{d \cos \theta} = 4 m_e c^2 \frac{\cos \theta}{\sin^4 \theta}
\end{equation}

Substituting this into the rate equation yields:

\begin{equation}
\frac{d^2 N}{dx \, d \cos \theta} = \frac{1}{2} z^2 \frac{Z}{A} K \rho \frac{1}{\cos^3 \theta} \frac{1}{m_e c^2} \approx 0.15 \frac{\text{cm}^2}{\text{g}} z^2 \rho \frac{1}{\cos^3 \theta}
\end{equation}

This expression diverges as $\theta$ approaches $90^\circ$, 
where $T$ approaches zero, indicating a limitation in the 
assumption of a free electron. The resulting distributions 
suggest that $\delta$-rays emitted at small angles can 
significantly affect the spatial resolution in detectors, 
particularly through ionization clusters that broaden the 
track of the mother particle.


\section{$\delta$-electrons producing hits in the Mu2e tracker}

In the Mu2e experiment, low-energy electrons and positrons, referred to as $\delta$-electrons, 
with a momentum below 20 MeV/c, constitute the majority of the tracker hits. Managing these 
hits is crucial for optimizing memory efficiency and CPU consumption. From a physics point 
of view, there are different reasons why it is important to identify those hits:
\begin{itemize}
    \item the signal we want to observe is the CE. Flagging even a small fraction of 
    potential CE hits is extremely dangerous, as it could result in the failure to 
    reconstruct the corresponding tracks.
    As shown in the histogram in Figure \ref{fig:momhits}, the simulated CE hits make up 
    just 1\% of the total hits in the tracker;
    \item flagging protons would assist the STM in counting the muon stopping rate. 
    It is possible to estimate the number of muons captured in the stopping target 
    by counting the number of protons produced during nuclear muon capture, 
    which is one of the possible processes a stopped muon can undergo (61\%);
    \item misidentifying muons and pions as protons or $\delta$-electrons could lead to a 
    incorrect estimate of the background events. This is particularly significant for the antiproton background.
    $p\bar{p}$ annihilation at rest in the ST can produce events with more than one track, 
    each with a momentum around 100 MeV/c. For $p\bar{p}$ annihilation events 
    in the ST, the rate of multi-track events is about 500 times higher 
    than the rate of events with a single signal-like electron. 
    For $10^4$ $p\bar{p}$ annihilation events generated, about 3.7\% of 
    the events contained two reconstructable particle tracks. Therefore, 
    the identification and reconstruction of multi-track events could be 
    used to constrain the $\bar{p}$ background. Thus, it is crucial
     not to flag muons or pions, as the fraction of multi-track events is very low.
\end{itemize}
Figure \ref{fig:momhits} shows the true momentum distribution of 
hits that make at least one hit in the tracker in the case of a 
dataset containing at least one CE per event. Each momentum bin is filled with 
the number of hits corresponding to a specific Monte Carlo particle. 
The distribution reveals that the majority of hits originate from 
low-energy electrons and positrons (orange), constituting approximately 
75\% of the total number of hits. This histogram corresponds to 
the 1BB pileup scenario (Section \ref{pulsedprotonbeam}). 
There is an asymmetry between the number of hits below 20 MeV/c 
produced by electrons and positrons: electrons account for 
71\% of all $\delta$-electron hits in the tracker, while positrons 
contribute only 4\%. This discrepancy arises because only electrons 
undergo Compton scattering, which is the primary source of hits at 
energies around 1 MeV. This difference will be crucial in 
the following sections when discussing the $\delta$ flagging efficiency.

As evident from the histogram, 14\% of the total hits are due to protons, 
which are produced by nuclear processes. Their kinetic energy ranges 
from about 5 to 20 MeV, resulting in low $\beta \gamma$ values, 
making them heavily ionizing particles. The deposited energy will be 
one of the variables used to discriminate protons.

It is important to note that the bump around 50 MeV/c in 
the positron distribution should not be present. According 
to the Monte Carlo truth, this is due to muon Decay-In-Flight, 
and we expect $N(\mu^+ \rightarrow e^+ )/N(\mu^- \rightarrow e^- )$ to be about $10^{-3}$ 
for muons entering the DS. The Decay-In-Orbit on the IPA 
(Section \ref{detectorsolenoid}) should also be around 
$10^{-3}$ compared to the DIF of negative muons. 
The simulation of $\mu^+$ may contain some errors, 
which we have reported to the simulation specialists. 
However, this issue is not problematic for the analysis 
of low-momentum electrons and positrons, as the momentum ranges are different.

\begin{figure}[!h]
        \centering
        \includegraphics[width =0.95\textwidth]{figures/png/Screenshot_20240812_152905.png}
    \caption[Monte Carlo momentum distribution of particles producing hits in the Mu2e tracker (conversion electron dataset and pileup).]{
       The Monte Carlo momentum distribution of particles producing at 
       least one hit in the Mu2e tracker (conversion electron dataset and pileup). The distribution
       corresponds to 1BB pile-up (Section \ref{pulsedprotonbeam}). The momentum distribution 
       of all particles making hits is depicted in dark blue, with electrons 
       shown in pink, positrons in light blue, $\delta$s in orange, protons and deuterons in 
       light green, and CEs in dark green. }
       \label{fig:momhits}
\end{figure}


Figure \ref{fig:pbar} shows the true momentum distribution of 
particles produced in the $p\bar{p}$ annihilation at the ST. 
Each momentum bin contains the number of hits corresponding 
to a specific Monte Carlo particle. The dataset used to generate 
this histogram has no pileup (0BB). For datasets with pileup, 
the pileup hits are explicitly added to the hits from 
the signal process. In this case, however, no pileup hits 
are added, meaning the data represents pure signal. 
The particles produced in the $p\bar{p}$ annihilation 
are mostly pions, muons, and a few electrons. 
It can be observed that the momentum distribution 
peaks in the 100-200 MeV/c range. Photons are also 
produced, and they can undergo Compton scattering and 
pair production, which explains the presence of a 
$\delta$-electron peak that is about $\sim$100 
times lower than the peak in Figure \ref{fig:momhits}.

\begin{figure}[!h]
    \centering
    \includegraphics[width =0.9\textwidth]{figures/png/Screenshot_20240815_124710.png}
\caption[Monte Carlo momentum distribution of particles producing hits in the Mu2e tracker ($\bar{p}$ dataset and no pileup).]{
   The Monte Carlo momentum distribution of particles producing at 
   least one hit in the Mu2e tracker ($\bar{p}$ dataset and no pileup). 
   The momentum distribution 
   of all particles making hits is depicted in dark blue, with electrons 
   shown in pink, positrons in light blue, $\delta$s in orange, protons and deuterons in green, pions in light brown and muons 
   in black. }
   \label{fig:pbar}
\end{figure}

As shown in Figure \ref{fig:momhits}, the majority of the recorded hits 
originate from $\delta$-electrons, with protons being the second most 
common source. Figure \ref{fig:afbef} presents an example of comparison of an 
event before and after the background hits have been flagged. 
Flagging these hits is crucial for several reasons: it prevents unnecessary 
data from being sent to the pattern recognition algorithms, thereby conserving 
CPU resources and reducing processing time. Moreover, it is important to avoid 
storing these hits on tape, as doing so would lead to inefficiencies in data storage.

\begin{figure}[!h]
    \begin{subfigure}[b]{0.4\linewidth}
        \centering
        \includegraphics[scale = 0.3]{figures/png/Screenshot_20240811_123612.png}
        \subcaption{Before.}
        \label{fig:bef}
    \end{subfigure}
    \begin{subfigure}[b]{0.7\linewidth}
        \centering
        \includegraphics[scale = 0.3]{figures/png/Screenshot_20240811_124245.png}
        \subcaption{After.}
        \label{fig:af}
    \end{subfigure}
    \caption{Before and After background hits flagging. The transverse $x-y$ views of a CE 
    event with 2BB pile-up (Section \ref{pulsedprotonbeam}). The segments are the $hit$ tracker straws. The hits marked in
    red are from electrons and the ones in blue are from positrons.}
    \label{fig:afbef} 
\end{figure}






\section{Mu2e $\delta$-electrons rejection algorithms}
The Mu2e Offline software tool is illustrated in 
Appendix \ref{mu2eana}, and the reconstruction 
process is described in Appendix \ref{eventreco}. 

The flagging of $\delta$-electrons is a step in the 
Mu2e reconstruction process that occurs before time clustering and pattern recognition.

In Mu2e Offline, there are two types of flagging algorithms:
\begin{itemize}
    \item $FlagBkgHits$, described in Section \ref{flagbkghits};
    \item $DeltaFinder$, described in Section \ref{deltafinder}.
\end{itemize}

\subsection{The $FlagBkgHits$ Algorithm}\label{flagbkghits}
The detailed description of multivariate analysis (MVA) and the 
process of MVA training are beyond the scope of this work. 
Nevertheless, since this technique is one of the options 
developed for background flagging, I will briefly outline 
the fundamental principles involved.

When searching for patterns in a multivariable space, 
a common procedure involves defining a set of statistical 
models that analyze the measured variables and estimate 
the probability that these are consistent with the 
sought pattern. Once the variables are selected, 
the MVA is trained to recognize patterns by 
evaluating examples known to the trainer, 
allowing for feedback to refine the 
pattern identification process.

The first selection concerns the energy deposited by 
the $ComboHit$ and excludes strongly ionizing particles, 
which have a deposited energy above 4.5 keV. Other selections 
concern the timing, position, and spread of the $ComboHit$ 
in the $x$-$y$ plane. The maximum cluster timing 
and diameter are 20 ns and 5 mm, respectively.

Since $\delta$-electrons tend to form small, dense 
clusters of hits on $x$-$y$ plane, an algorithm based on a clustering 
approach was developed: concentrated clusters in the 
$x$-$y$ plane are sought using a clustering algorithm, 
as $\delta$-electrons are highly likely to reside 
in such clusters. The $FlagBkgHits$ algorithm employs 
a hit-level MVA to ensure that hits within a 
$\delta$-dominant cluster truly belong there, 
and a cluster-level MVA to differentiate 
$\delta$-dominant clusters from those 
predominantly containing conversion electrons.

The current resolution of straw drift tube 
position measurements is limited to a few cm. 
However, an improved position measurement can be 
achieved by leveraging the multiple straw layers that 
significantly overlap in the transverse plane. 
By obtaining two hit measurements from a pair of 
intersecting straws and ensuring they fall within 
a time window on the order of the maximum drift time, 
one can deduce that these hits were produced by the 
same particle and occurred at the intersection of the 
two straws in the projected plane. This method, 
involving two-dimensional information from the two 
straws, is referred to as the stereo information method.

The clustering process uses the $x$ and $y$ coordinates 
of the selected hits. A random hit is chosen to define 
the initial cluster, after which the following iterative 
steps are applied. The centroid of each cluster is 
computed, and hits whose distances from a cluster 
centroid fall within a specified inner threshold 
and time window are added to that cluster. Hits 
with distances from all existing clusters greater 
than an outer threshold are used to seed new 
clusters, while hits that fall between the two 
thresholds remain unassigned. This process 
generates a new set of clusters, preparing the 
system for the next iteration. During each 
iteration, every hit is reconsidered as a 
potential new point in a cluster, including 
those already assigned. The clustering process 
continues until convergence is achieved, i.e., 
when an iteration no longer results in any changes.


\subsection{$DeltaFinder$ Algorithm}\label{deltafinder}

$DeltaFinder$ is an algorithm designed to identify $\delta$-electron 
hit patterns rather than CE hits. This algorithm relies on the 
fact that $\delta$-electrons typically form a straight line in 
the $r$-$z$ plane (cyan lines in Figure \ref{fig:yzviewdelta}) 
and appear as a spot with a diameter of less than 3 cm in the 
$x$-$y$ plane within the Mu2e magnetic field. In contrast, 
CE hits create entirely different patterns, appearing as oblique lines in 
the $r$-$z$ plane due to their helical trajectories (red lines in Figure \ref{fig:yzviewdelta}). 
In the process, all hits associated with detected objects are treated as $\delta$-electron hits.
\begin{figure}[!h]
    \centering
    \includegraphics[width =0.6\textwidth]{figures/png/Screenshot_20240811_123048.png}
    \caption[$\delta$-electrons and CE patterns in $r-z$ plane.]{
        $\delta$-electrons and CE patterns in the $r-z$ plane. 
        The white squares represent some of the tracker stations. 
        The cyan straight lines represent the $\delta$-electron 
        patterns in the $r-z$ plane, while the red lines a CE signal in the same plane.
        }
    \label{fig:yzviewdelta}
\end{figure}
\subsubsection{Step 1: Identifying $\delta$-Electron Segments}
$DeltaFinder$ first seeks to identify $\delta$-electron track 
segments within each station individually. These segments, 
parallel to the beam axis, are called $seeds$. Since hits 
from the same electron should be close in both time and space, 
and $\delta$-electrons may hit multiple straws within the 
same panel, the algorithm clusters these straw hits in 
space-time, applying various cleanup cuts to ensure the 
selected patterns resemble $\delta$-electron hits. 
The maximum allowed time difference between two hits within 
a station to form a $seed$ is set to 40 ns.

These cleanups based on the $x$-$y$ coordinates are performed 
by computing a $\chi^2$. The $seed$ is reconstructed using 
two, three, or four $ComboHit$s. In each case, the 
intersections between two straws are determined, 
and the $x$-$y$ coordinates of the $seed$ are given 
by the center of gravity of these intersections. 
In the case of only two hits, the center of gravity c
orresponds to the stereo hit. Explicit stereo hit 
reconstruction is not used for all hits.

The algorithm extends the $seed$s by requiring 
hits to be sufficiently close to the intersection 
point in time and space, performing multiple checks 
to avoid over-efficiency in hit flagging. For each $seed$, 
the mean deposited energy is calculated from the energy 
deposited in all $ComboHit$s. Based on the mean 
deposited energy of a $seed$ in a station, hits with 
energy above 5 keV are considered proton hits. 
This selection optimizes data processing by r
educing the total number of hits that need to be analyzed.

\begin{figure}[!h]
    \centering
    \includegraphics[width =0.6\textwidth]{figures/png/Screenshot_20240811_115854.png}
    \caption[A $\delta$ candidate $seed$.]{A $\delta$ candidate $seed$. The four coloured segments are the tracker straws that were
    hit in one station.}
    \label{fig:deltaseeds}
\end{figure}

It is necessary to do a clarification. 
Figure \ref{fig:energydeposited} shows the 
distribution of the simulated deposited energy 
in the tracker for $\delta$ electrons, CEs, and 
protons in the case of a 1BB pileup. 

To optimize data processing, a hit energy 
threshold could be applied to the $DeltaFinder$ 
to reduce the total number of hits that need to be analyzed, 
thereby speeding up the process. Moreover, only about 
4\% of CE hits have energies above 3.5 keV 
(and 1\% above 5 keV), so the loss of CE 
hits would be minimal, resulting in a faster overall algorithm. 

However, implementing such an energy cutoff has 
significant implications for the algorithm's performance. 
Starting with fewer hits, especially in stereo intersections, 
decreases the likelihood of identifying the correct $seed$. 
Additionally, a higher energy cutoff increases 
the probability of false positives, as the algorithm 
aims to count protons too.

\begin{figure}[!h]
    \centering
    \includegraphics[width =0.8\textwidth]{figures/png/Screenshot_20240729_151910.png}
\caption[Monte Carlo deposited energy distribution in the Mu2e tracker.]{
   The Monte Carlo deposited energy distribution in the Mu2e tracker. The distribution
   corresponds to 1BB pile-up (Section \ref{pulsedprotonbeam}). The red distribution refers to 
   conversion electrons, while the green and the blue one to $\delta$-electrons and protons respectevely.
}
   \label{fig:energydeposited}
\end{figure}
\subsubsection{Step 2: Connecting $seed$s}
After the selection based on mean deposited energy, 
$DeltaFinder$ attempts to connect segments that 
are close in both the $x$-$y$ plane and time 
across different stations to form $\delta$-electron 
candidates. A valid candidate must have at 
least two segments and a minimum of five straw hits. 
Reconstructed segments from a 100 MeV electron 
typically remain unconnected due to their separation in the $x$-$y$ plane. 

$DeltaFinder$ links $\delta$-electron $seed$s 
across stations, attempting to associate new $seed$s 
with existing $\delta$ candidates. If no match is 
found, a new candidate is created. Good $\delta$ 
candidates are marked, and their hits are 
flagged to prevent their inclusion in proton candidate searches.

\subsubsection{Step 3: Identifying Proton Candidates}
Finally, $DeltaFinder$ identifies proton candidates 
using the remaining $seed$s, which are more likely 
to have deposited energy above 3 keV. First, it checks 
if a $seed$ is consistent in time with any existing 
proton candidates. If no consistency is found, 
the hits of the $seed$ are added to a new proton candidate.


\section{Analysis of the performance and comparison}
\subsection{Dataset}
\subsection{Low level benchmark}
\subsection{High level benchmark}

\section{Conclusions}


\iffalse

 The detailed background rates for these processes at this energy scale are not very
well studied. In Mu2e, background rates calculated according to figures given in Nuclear Physics of
Muon Capture (D. F. Measday) are used to create simulations. However, due to the lack of
documentation at the energy scales involved in this experiment, there is a large uncertainty in
background rates which requires background removal to be insensitive to the level of backgrounds
This primary purpose of study is to address one prevalent source of background hits: the
secondary ionized electrons, which are referred to as "deltas." Deltas consist primarily of Compton
scattered electrons, pair production electrons, and delta rays, in order of decreasing prevalence.
Compton scattered electrons are produced when photons from various processes strike material
inside the detector. This causes the ejection of electrons that create the background events. The
photons primarily originate from neutron capture by atoms, which leads to an excited nuclear state
that decays via photon emission. These photons generally have energies on the order of a few MeV.
The neutrons, in turn, are created when muons are captured by atoms, creating unstable isotopes that
decay by ejecting neutrons. Pair production electrons, on the other hand, are produced when
processes involving nuclear recoil also produce pairs of electrons and positrons in order to conserve
energy and momentum. Delta rays, or secondary ionized electrons, are produced when charged
energetic particles collide with detector material.






DELTAFINDER:
step 1: Find $\delta$-electron track segments separately within each station
connect segments, allow 1 station wide "gaps"
dominant sources of failures:
I stations with MC particle producing hits in only one face;
I hits with wrong coordinates along the wire - long tails of the $\Delta$T distribution;
I recover hits in "empty" stations to improve efficiency;
I optimization of the algorithm timing performance;
electron hit energy dependence on momentum: path length within the straw depends on momentum
only about 4\% of CE hits have energies above 3.5keV (1\% above 5 keV)
consider several hit energy cutoffs: 3.5 keV, 5keV, 7keV, use all hits




\subsection{Delta electron features}








Figure 2.1 shows the momentum distribution of deltas for 10,000 events.
Most deltas have much lower momenta than conversion electrons. This produces a significant
difference in their trajectories through the tracker, as shown in Figures 2.2 to 2.4. With the naked eye
one can already see dramatic differences.
The most striking quality of deltas is that they tend to create small, dense clusters of hits, as
shown in Figures 2.3 and 2.4. This motivates a "clustering approach"*: we search for such
concentrated clusters in the x-y plane using a clustering algorithm since deltas are very likely to reside
in such a cluster. However, the reverse containment does not hold - clusters can also contain many
conversion electron hits. Removing conversion electron hits can severely harm subsequent
reconstruction, so a filter is necessary. 
low energy (delta, compton, photon conversion) electrons are the largest source of the hits
in the tracker - about 2/3 of the total
language: $\delta$-electron - an electron with P<20 MeV/c
radius of a 10 MeV/c electron in the nominal Mu2e field < 3 cm, close to the resolution
along the wire
very specific topology - multiple hits with the same (X,Y) , within the resolution
\fi