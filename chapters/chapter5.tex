\chapter{Comparing $\delta$-electrons flagging algorithms}\label{delta}
\textit{In the context of DAQ testing and performance analysis during Mu2e data-taking, 
where the data volume is expected to reach approximately 7 PBytes per year, optimizing 
memory usage and minimizing CPU consumption is crucial. The primary source of hits in the 
Mu2e tracker will be $\delta$-electrons. Therefore, it is essential to effectively flag these 
hits without compromising the efficiency of CE hit detection and track reconstruction. This 
Chapter presents a comparison between two algorithms designed for $\delta$-electron flagging.}

\section{$\delta$-electrons as source of background}

The primary hit source in Mu2e is given by $\delta$-electrons, that  
primarily consist of Compton scattered electrons, pair production 
electrons and positrons, and delta rays, listed in decreasing order of prevalence. Compton 
scattered electrons are generated when photons, produced by various processes, 
interact with the detector material. These photons primarily originate from 
neutron capture by atoms, which excites the nucleus, leading to photon emission 
during decay. Typically, these photons have energies of a few MeV. The neutrons 
are produced when muons are captured by atoms, resulting in unstable isotopes that 
decay by emitting neutrons. Pair production electrons and positrons are generated during nuclear 
recoil processes, where pairs of electrons and positrons are produced to conserve 
energy and momentum. Delta rays, or secondary ionized electrons, are generated when 
high-energy charged particles collide with the detector material.

\subsection{Compton scattering}
The Compton effect (Figure \ref{fig:compt}) refers to the 
scattering of a photon by a free or quasi-free electron. 
An electron is considered "quasi-free" when the energy of 
the incoming photon is significantly higher than the 
electron's binding energy ($E_\gamma \gg E_B$). The 
scattering process is termed Compton scattering if the 
electron is ejected from the atom, carrying away the 
recoil momentum. This effect is most prominent in an 
extended energy region around 1 MeV, with the region 
being much larger for low $Z$ materials compared to high $Z$ materials.

\begin{figure}[!h]
    \centering
    \includegraphics[width =0.4\textwidth]{figures/png/Screenshot_20240812_204345.png}
    \caption[The Compton effect.]{
    The Compton effect \cite{kola}.}
    \label{fig:compt}
\end{figure}

Since the photon scatters quasi-elastically off the electron, 
the energy and angle of the scattered photon are interdependent. 
To describe this relationship, we use the 4-momenta defined as 
follows: $k = (E_\gamma, \mathbf{k}c)$ and $p_e = (m_e c^2, 0)$ 
represent the 4-momenta of the photon and the electron (at rest) 
before scattering, and $k' = (E'_\gamma, \mathbf{k}'c)$ and 
$p'_e = (E'_e, \mathbf{p}'_e c)$ represent the 4-momenta 
after scattering. The angle between the scattered photon and 
the incident photon is denoted as $\theta_\gamma$, while the 
angle of the electron is denoted as $\theta_e$. By applying 
energy-momentum conservation:

\begin{equation}\label{compcons}
k + p_e = k' + p'_e
\end{equation}
\begin{equation}\label{compcons2}
(k - k')^2 = (p'_e - p_e)^2 \Rightarrow -k \cdot k' = m_e^2 c^4 - p'_e \cdot p_e
\end{equation}
\begin{equation}
\Rightarrow E_\gamma E'_\gamma (1 - \cos \theta_\gamma) = m_e c^2 \left(E'_e - m_e c^2\right) = m_e c^2 \left(E_\gamma - E'_\gamma \right)
\end{equation}

The right-hand side of the last equation uses the kinetic energy of the electron:

\begin{equation}
T = E'_e - m_e c^2 = E_\gamma - E'_\gamma
\end{equation}

which follows from the energy part of equation \ref{compcons}. 
The energy of the scattered photon as a function of the scattering 
angle is derived from equation \ref{compcons2}:

\begin{equation}\label{diffeq}
E'_\gamma = \frac{E_\gamma}{1 + \epsilon (1 - \cos \theta_\gamma)}
\end{equation}

where $\epsilon = \frac{E_\gamma}{m_e c^2}$.

The differential cross section per (free) electron, known as the 
Klein-Nishina formula, is calculated using methods from quantum electrodynamics:

\begin{equation}\label{kleinnishina}
\frac{d\sigma}{d\Omega} = \frac{r_e^2}{2} \frac{1 + \epsilon (1 - \cos \theta_\gamma)}{[1 + \epsilon (1 - \cos \theta_\gamma)]^2} \left(1 + \cos^2 \theta_\gamma + \frac{\epsilon^2 (1 - \cos \theta_\gamma)^2}{1 + \epsilon (1 - \cos \theta_\gamma)} \right)
\end{equation}

An electron bound in an atom can only be considered quasi-free 
if the photon's energy is significantly higher than the electron's 
binding energy. As the photon energy increases, more shell electrons 
become quasi-free, leading to the Compton cross section per atom 
approaching proportionality to $Z$, with individual electrons 
contributing incoherently:

\begin{equation}
\sigma_C^{\text{atom}} = Z\sigma_C
\end{equation}

where $\sigma_C$ is the Klein-Nishina cross section for a 
single free electron. The Compton cross section decreases at 
lower energies, where coherent scattering (Rayleigh scattering) 
off the entire atom (without ionizing the electron shell) becomes dominant.

By reformulating the Klein-Nishina formula, one can obtain the 
differential dependence of the Compton cross section on the 
kinetic energy of the recoil electron $T = E_\gamma - E'_\gamma$:

\begin{equation}
\frac{d\sigma}{dT} = \frac{\pi r_e^2}{m_e c^2 \epsilon^2} \left[2 + \frac{t^2}{\epsilon^2 (1 - t)^2} + \frac{t}{1 - t}\left(t - \frac{2}{\epsilon}\right)\right]
\end{equation}

where $t = T/E_\gamma$. Because the scattering process is 
elastic, there is a one-to-one relationship between the 
energy and angle $\theta_e$ of the electron:

\begin{equation}
\cos \theta_e = \frac{T(E_\gamma + m_e c^2)}{E_\gamma \sqrt{T^2 + 2m_ec^2 T}} = \frac{1 + \epsilon}{\sqrt{\epsilon^2 + 2\epsilon/t}}
\end{equation}

The maximum energy transfer to the electron is obtained 
from equation \ref{diffeq} for backward scattering of the 
photon ($\theta_\gamma = 180^\circ$), corresponding to 
forward scattering of the electron ($\theta_e = 0^\circ$). 
The electron's kinetic energy reaches its maximum value in 
this case, $T \rightarrow T_{\text{max}}$. In the measured 
energy spectrum, this leads to the so-called "Compton edge" at:

\begin{equation}
T_{\text{max}} = \frac{E_\gamma \cdot 2\epsilon}{1 + 2\epsilon}
\end{equation}

which lies slightly below the photopeak. The energy difference 
between the photopeak and the Compton edge $E'_\gamma(\theta = \pi)$ 
decreases with increasing $E_\gamma$ and approaches:

\begin{equation}
E'_\gamma(\theta = \pi) \approx \frac{m_e c^2}{2} \text{ for } E_\gamma \gg m_e c^2
\end{equation}

\subsection{Pair production}
In the Coulomb field of a charge, a photon can 
convert into an electron-positron pair (Figure 
\ref{fig:pprod})\footnote{Photon emission by an 
electron (bremsstrahlung) and pair production are closely 
related processes. By modifying the bremsstrahlung diagram-changing 
the outgoing photon to an incoming one and the incoming electron to 
an outgoing positron-one obtains the pair production 
diagram. The matrix elements of these processes are 
related, at least in the lowest order. Consequently, both 
processes are treated together in the foundational work by 
Bethe and Heitler, often referred to as the 'Bethe-Heitler processes'.}.

\begin{figure}[!h]
    \centering
    \includegraphics[width=0.4\textwidth]{figures/png/Screenshot_20240812_204755.png}
    \caption[The pair production.]{The pair production \cite{kola}.}
    \label{fig:pprod}
\end{figure}

The energy of the photon must exceed twice the electron 
mass plus the recoil energy transferred to the field-producing 
charge. For most elements, pair production predominantly 
occurs in the Coulomb field of the nucleus. For nuclei, 
the recoil energy is usually negligible, leading to a 
threshold energy for pair production of:
\begin{equation}
    E_{\gamma} \geq 2m_e c^2 + 2 \frac{m_e^2}{m_{\text{nucleus}}} c^2
\end{equation}

If the nuclear charge is not screened by atomic electrons 
(for low energies, the photon must come relatively close to 
the nucleus to make pair production probable, meaning it 
interacts with the "bare" nucleus),
\begin{equation}
    1 \ll \epsilon \ll \frac{1}{\alpha Z^{1/3}}
\end{equation}
the pair-production cross section is given by:

\begin{equation}
    \sigma_{\text{pair}} = 4 \alpha r_e^2 Z^2 \left(\frac{7}{9} \ln 2 \epsilon - \frac{109}{54}\right) \text{ cm}^2/\text{atom}
\end{equation}

However, for complete screening of the nuclear charge ($\epsilon \gg 1/\alpha Z^{1/3}$):
\begin{equation}\label{sigmapair}
    \sigma_{\text{pair}} = 4 \alpha r_e^2 Z^2 \left(\frac{7}{9} \ln \frac{183}{Z^{1/3}} - \frac{1}{54}\right) \text{ cm}^2/\text{atom}
\end{equation}

At high energies, pair production can occur even 
at relatively large impact parameters between the 
photon and the nucleus. In this case, the screening 
effect of atomic electrons must be considered. For 
large photon energies, the pair-production cross 
section approaches an energy-independent value as given by
 Equation \ref{sigmapair}. Ignoring the small term in the equation, 
 the asymptotic value of $1/54$ is expressed as:
\begin{equation}
    \sigma_{\text{pair}} \approx \frac{7}{9} \cdot 4 \alpha r_e^2 Z^2 \ln\left(\frac{183}{Z^{1/3}}\right) \approx \frac{7}{9} \cdot \frac{1}{X_0} \cdot \frac{A}{N_A \rho}
    \label{eq:paircross_radiationlength}
\end{equation}

The energy is uniformly distributed between the produced 
electrons and positrons at low and medium energies, but becomes 
slightly asymmetric at high energies.

The field of the nucleus is formed by the coherent sum of $Z$ 
nucleon charges, leading to the $Z^2$ dependence of the pair 
production cross section.

Even with large momentum transfers $\Delta p$ to the nucleus, 
the energy transfer $(\Delta p)^2/2M$ remains small due to the 
large nuclear mass $M$. After pair creation, the remaining 
energy is equally divided between the $e^+$ and the $e^-$.

\subsection{Delta rays}
High-energy $\delta$-rays, or $knock-on$ electrons, 
are produced when a projectile particle collides 
centrally with shell electrons, resulting in 
significant energy transfers. These electrons 
gain high kinetic energy and can be described 
through elastic collisions with quasi-free electrons. 
By considering the energy-momentum conservation relation 
and using the Lorentz factors $\gamma$ and $\beta$, the 
relationship between the kinetic energy $T$ of the 
$\delta$-ray and the emission angle $\theta$ can be derived as:

\begin{equation}
\cos \theta = \frac{T(\gamma + m_e / M)}{\gamma \beta \sqrt{T^2 + 2T m_e c^2}}
\end{equation}

\begin{equation}
T(\theta) = \frac{2 m_e c^2 \beta^2 \gamma^2 \cos^2 \theta}{\gamma^2(1 - \beta^2 \cos^2 \theta) + 2 \gamma m_e / M + m_e^2 / M^2}
\end{equation}

The maximum energy transfer $T_{\text{max}}$ occurs at $\theta = 0^\circ$, 
while the minimum energy, $T_{\text{min}}$, occurs at $\theta = 90^\circ$. 
At highly relativistic energies ($\gamma \gg 1$ and $\theta \gg 1/\gamma$), 
the energy-angle relationship becomes independent of the incoming particle's properties.

The rate of $\delta$-rays per energy interval $dT$ and path length $dx$ is given by:

\begin{equation}
\frac{d^2 N}{dx \, dT} = n_e \frac{d\sigma}{dT}
\end{equation}

which, when combined with the electron density and the differential cross section, becomes:

\begin{equation}
\frac{d^2 N}{dx \, dT} = \frac{1}{2} z^2 \frac{Z}{A} K \rho \frac{1}{\beta^2} \frac{F(T)}{T^2}
\end{equation}

Here, $K$ is the constant from the Bethe-Bloch formula, 
and $F(T)$ is a function accounting for spin dependence. 
Integration over $T$ and $x$ provides the number of $\delta$-rays in a medium of thickness $\Delta x$:

\begin{equation}
N = \frac{1}{2} z^2 \frac{Z}{A} K \rho \Delta x \frac{1}{\beta^2} \left(\frac{1}{T_{\text{min}}} - \frac{1}{T_{\text{max}}}\right) \approx 0.077 \frac{\text{MeV cm}^2}{\text{g}} z^2 \rho \Delta x \frac{1}{T_{\text{min}}}
\end{equation}

The emission angle dependence is given by:

\begin{equation}
\frac{dT}{d \cos \theta} = 4 m_e c^2 \frac{\cos \theta}{\sin^4 \theta}
\end{equation}

Substituting this into the rate equation yields:

\begin{equation}
\frac{d^2 N}{dx \, d \cos \theta} = \frac{1}{2} z^2 \frac{Z}{A} K \rho \frac{1}{\cos^3 \theta} \frac{1}{m_e c^2} \approx 0.15 \frac{\text{cm}^2}{\text{g}} z^2 \rho \frac{1}{\cos^3 \theta}
\end{equation}

This expression diverges as $\theta$ approaches $90^\circ$, 
where $T$ approaches zero, indicating a limitation in the 
assumption of a free electron. The resulting distributions 
suggest that $\delta$-rays emitted at small angles can 
significantly affect the spatial resolution in detectors, 
particularly through ionization clusters that broaden the 
track of the mother particle.


\section{$\delta$-electrons producing hits in the Mu2e tracker}

In the Mu2e experiment, low-energy electrons and positrons, defined as $\delta$-electrons, having a momentum below 20 MeV/c, 
constitute the majority of the tracker hits, 
making their management crucial for memory efficiency and CPU consumption optimization. 
From a physics point of view, there are different reasons why it is important to identify those hits:
\begin{itemize}
    \item the signal we want to observe is the CE. Flagging also a little fraction of the 
    potential CE hits is extremely dangerous, 
    leading to the possibility of not reconstructing at all one of those tracks;
    \item flagging protons would help the STM in the counting of muon stopping rate.
    It is possible to count the number
    of muons captured in the stopping target by counting the number of 
    protons produced in the process of the nuclear muon capture. This is one of
    the possible processes a stopped muon can undergo and it can lead to the
    ejection of charged particles.
    \item flagging muons and pions as protons or $\delta$-electrons could lead to a misidentification of 
    the background. In particular, the most affected background is the antiproton one. 
    The $\bar{p}$ background in Mu2e, at the low muon beam energy of about 50 MeV/c has a
    unique feature: $p\bar{p}$ annihilation at rest in the ST can produce events with more than one track
    with momentum around 100 MeV/c. For the $p\bar{p}$ annihilation in the ST events, the rate of multi-track 
    events is about 500 times higher than the rate of events
    with 1 signal like electron. So, a way to identify and reconstruct the two-track events
    and estimate the $\bar{p}$ background by comparison. For $10^4$ pp annihilation 
    events generated, about 3.7\% of the events contained 2 particle reconstructable tracks.
    Thus, the identification and reconstruction of multi-track events could be used to constrain the
    p background. So, it is extremely important not to flag muons or pions, since this fraction is really low.
\end{itemize}


Figure \ref{fig:momhits} shows the Monte Carlo momentum distribution of hits that make at least 
one hit in the tracker. The distribution shows that the majority of hits come from the low energy 
electrons and positrons (orange) $\sim$75\% (Section \ref{pulsedprotonbeam}). 
It is interesting to see the asimmetry between the number of hits below 20 MeV/c produced by electrons and positrons: 
electrons constitute the 71\% of all $\delta$-electrons hits in the tracker, while positrons are just the 4\%.
This difference will be crucial in the next sections while showing the flagging efficiency.

To be precise, the bump around 50 MeV/c in the positrons distribution should not be present.
From the Monte Carlo truth we know that is muon Decay-In-Flight and 
I would expect $N(\mu^+ \rightarrow e^+ )/N(\mu^- \rightarrow e^- ) \sim 10^{-3}$ for muons entering the DS.
The Decay-In-Orbit on IPA (Secion \ref{detectorsolenoid}) should be also $10^{-3}$ with respect to the DIF of negative muons.
The simulation of $\mu^+$ should be affected by some errors, and we reported this to the specialists.
This is not problematic for the analysis of low momentum electrons and positrons, since the momentum range is 
different.

\begin{figure}[!h]
        \centering
        \includegraphics[width =0.95\textwidth]{figures/png/Screenshot_20240812_152905.png}
    \caption[Monte Carlo momentum distribution of particles producing hits in the Mu2e tracker (conversion electron dataset and pileup).]{
       The Monte Carlo momentum distribution of particles producing at 
       least one hit in the Mu2e tracker (conversion electron dataset and pileup). The distribution
       corresponds to 1BB pile-up (Section \ref{pulsedprotonbeam}). The momentum distribution 
       of all particles making hits is depicted in dark blue, with electrons 
       shown in pink, positrons in light blue, $\delta$s in orange, protons in 
       light green, and CEs in dark green. }
       \label{fig:momhits}
\end{figure}





For completeness, Figure \ref{fig:pbar} shows 

\begin{figure}[!h]
    \centering
    \includegraphics[width =0.95\textwidth]{figures/png/Screenshot_20240813_114832.png}
\caption[Monte Carlo momentum distribution of particles producing hits in the Mu2e tracker ($\bar{p}$ dataset and no pileup).]{
   The Monte Carlo momentum distribution of particles producing at 
   least one hit in the Mu2e tracker ($\bar{p}$ dataset and no pileup). 
   
   
   The momentum distribution 
   of all particles making hits is depicted in dark blue, with electrons 
   shown in pink, positrons in black, $\delta$s in cyan, protons and deuterons in green, pions in light green and muons 
   in orange. }
   \label{fig:momhits}
\end{figure}





Figure \ref{fig:afbef} shows the before and after the background hits flagging stages of
an event.  
\begin{figure}[!h]
    \begin{subfigure}[b]{0.4\linewidth}
        \centering
        \includegraphics[scale = 0.3]{figures/png/Screenshot_20240811_123612.png}
        \subcaption{Before.}
        \label{fig:bef}
    \end{subfigure}
    \begin{subfigure}[b]{0.7\linewidth}
        \centering
        \includegraphics[scale = 0.3]{figures/png/Screenshot_20240811_124245.png}
        \subcaption{After.}
        \label{fig:af}
    \end{subfigure}
    \label{fig:afbef} 
    \caption{Before and After background hits flagging. The transverse $x-y$ views of a CE 
    event with 2BB pile-up (Section \ref{pulsedprotonbeam}). The segments are the $hit$ tracker straws. The hits marked in
    red are from electrons and the ones in blue are from positrons.}
\end{figure}



An electron or a positron with a momentum of 10 MeV/c would have a radius of less than 3 cm in the 
Mu2e magnetic field. These particles typically produce a distinctive pattern in the 
Mu2e tracker, often resulting in multiple hits with nearly identical $(x, y)$ coordinates.

\section{Mu2e $\delta$-electrons rejection algorithms}
The flagging of $\delta$-electrons is a stage in the Mu2e reconstruction 
tool that occurs before time clustering and pattern recognition. 
It is executed after the trigger. The Mu2e Offline software tool is 
illustrated in Appendix \ref{mu2eana}, and the reconstruction process is described in Appendix \ref{eventreco}. 

In Mu2e Offline, there are two different types of rejection algorithms:
\begin{itemize}
    \item $DeltaFinder$;
    \item $FlagBkgHits$.
\end{itemize}
These algorithms will be described in the following sections.
\subsection{$FlagBkgHits$ Algorithm}
The detailed description of multivariate analysis (MVA) and the process of MVA 
training are beyond the scope of this work. Nevertheless, given that this 
technique is one option developed for background flagging, I will briefly outline the 
fundamental principles involved.

When searching for patterns in a multivariable space, a common procedure involves 
defining a set of statistical models that analyze the measured variables and estimate 
the probability that these are consistent with the sought pattern. Once the variables 
are selected, the MVA is trained to recognize patterns by evaluating examples known 
to the trainer, allowing for feedback to refine the pattern identification process.

In the context of identifying $\delta$-electrons, the most relevant variables include 
the timing, position, and spread of the ComboHit (in the $x-y$ plane and along the 
$z$-axis). A characteristic feature of $\delta$-electrons is their tendency to form small, 
dense clusters of hits. For this reason an algorithm based 
on a clustering approach was developed: concentrated clusters in the $x-y$ plane are sought using a 
clustering algorithm, as $\delta$-electrons are highly likely to reside in such clusters. 
The $FlagBkgHits$ algorithm employs a hit-level MVA to ensure that hits within a 
$\delta$-dominant cluster truly belong there, and a cluster-level MVA to differentiate 
$\delta$-dominant clusters from those predominantly containing conversion electrons. 
The combination of these techniques, along with carefully selected cuts, maximizes 
background rejection while preserving signal efficiency.

The current resolution of straw drift tube position measurements is limited to a few 
cm. However, an improved position measurement can be achieved by leveraging the 
multiple straw layers that significantly overlap in the transverse plane. By 
obtaining two hit measurements from a pair of intersecting straws and ensuring 
they fall within a time window on the order of the maximum drift time, one can 
deduce that these hits were produced by the same particle and occurred at the 
intersection of the two straws in the projected plane. This method, involving 
two-dimensional information from the two straws, is referred to as the stereo 
information method and offers a much higher precision.

The clustering process uses the $x$ and $y$ coordinates, as $\delta$ hit positions 
essentially form lines along the $z$-direction, making $z$ clustering unnecessary. 
A random hit is chosen to define the initial cluster, after which the following 
iterative steps are applied. The centroid of each cluster is computed, and hits 
whose distances from a cluster centroid fall within a specified inner threshold 
and time window are added to that cluster. Hits with distances from all existing 
clusters greater than an outer threshold are used to seed new clusters, while hits 
that fall between the two thresholds remain unassigned. This process generates a 
new set of clusters, preparing the system for the next iteration. During each iteration, 
every hit is reconsidered as a potential new point in a cluster, including those already 
assigned. The clustering process continues until convergence is achieved, i.e., 
when an iteration no longer results in any changes. 

\subsection{$DeltaFinder$ Algorithm}
$DeltaFinder$ is an algorithm designed to 
identify $\delta$-electron hit patterns 
rather than CE hits. This algorithm relies 
on the fact that $\delta$-electrons usually 
form a straight line in the $r-z$ plane 
(cyan lines in Figure \ref{fig:yzviewdelta}) and appear 
as a spot in the $x-y$ plane, 
while CE hits create entirely different 
patterns (red lines in Figure \ref{fig:yzviewdelta}). 
The goal of $DeltaFinder$ is to recognize 
$\delta$ electrons in the Mu2e tracker 
with high efficiency, minimizing false positives. 
In the process, all hits associated with 
detected objects are treated as $\delta$-electron hits.
\begin{figure}[!h]
    \centering
    \includegraphics[width =0.6\textwidth]{figures/png/Screenshot_20240811_123048.png}
    \caption[$\delta$-electrons $y-z$ plane pattern.]{    }
    \label{fig:yzviewdelta}
\end{figure}
\subsubsection{Step 1: identifying $\delta$-electron segments}
$DeltaFinder$ first seeks to identify $\delta$-electron 
track segments within each station separately. 
Hits from the same electron should be close in both time 
and space. $\delta$-electrons may hit multiple straws 
within the same panel. The algorithm clusters these 
straw hits in space-time, applying various cleanup 
cuts to ensure that the selected patterns resemble what 
$\delta$-electron hits would look like. 
It does not use explicit stereo hit reconstruction 
for all hits; however, it intersects hit wires to determine 
the position of the $\delta$-electron segment in 3D.

As first step, the algorithm tries to reconstruct a $seed$, that is 
a particle cluster, in time and space, in one station, parallel 
to the beamline axis. 
The seeds are reconstructed as following:
1)It takes two hits close in time and space (two, three or four) 
in one station
The requirement is performed looking and performing a chisquare

2)It reconstruct the intersection between two, 
three or four straws referred to those hits.
With more than two hits, the intersection is the centre of gravity between 
two by two straws

se sono 2 straws-> e' una stereo
se sono 3 straws-> e' baricentro tra i 
3) it exstends the seeds requiring hits to be close 
enough to the intersection point in time and space and check whether 
there already is a seed containing both hits.
The requirement is performed looking and performing a chisquare

4) Per seed it computes the mean energy deposited
5)Considering the mean deposited energy by a seed in the station, 
the algorithm considers as proton 
hits those with mean deposited energy above 5 keV. 
After that manages to connect seeds between each other and tries to find 
the $\delta$-electrons. 
Those particles left are then tried to cluster in time with protons 
and if it is in the range between 3 keV and 5 keV, it is considered 
as a proton and a $\delta$-electron in the first iteration. 
This selection enables optimize the data 
processing, a hit energy threshold can be 
applied to reduce the total number of hits that need to be analyzed. 
\begin{figure}[!h]
    \centering
    \includegraphics[width =0.6\textwidth]{figures/png/Screenshot_20240811_115854.png}
    \caption[$\delta-seed$s from $StrawHit$s.]{A $\delta-seed$ found out from the    }
    \label{fig:deltaseeds}
\end{figure}
\subsubsection{Step 2: connecting $seed$s}
$DeltaFinder$ then connects segments close in 
both the $x-y$ plane and 
time across different stations to form $\delta$-electron candidates. 
A valid candidate must have at least two segments and a 
minimum of five straw hits. 
Reconstructed segments of a 100 MeV electron typically 
remain unconnected 
due to their separation in the $x-y$ plane.
$DeltaFinder$ links $\delta$-electron $seed$s across 
stations, attempting to 
associate new $seed$s with existing $\delta$ 
candidates. If no match is found, a new candidate is created.
Good $\delta$ candidates are marked, and their hits are flagged 
to avoid their inclusion in proton candidate searches.


it allows to have holes in the middle
\subsubsection{Step 3: identifying proton candidates}
Finally, $DeltaFinder$ identifies proton candidates using 
all seeds with a mean deposited energy above 3 keV. 
First, it checks if $seed$ is consistent in time 
with any of existing proton candidates.

by 
clustering high-ionization 
hits in time. The algorithm resolves overlaps 
between proton candidates and 
merges those with consistent segments across stations.


Figure \ref{fig:energydeposited} shows the distribution for the simulated energy loss in the tracker for 
$\delta$s, CEs and protons in case of 1BB pileup. As one can see, it is difficult to 
assess whether a particle hit is from  Only about 4\% of CE hits have energies above 3.5keV (1% above 5 keV)
\begin{figure}[!h]
    \centering
    \includegraphics[width =0.8\textwidth]{figures/png/Screenshot_20240729_151910.png}
\caption[Monte Carlo deposited energy distribution in the Mu2e tracker.]{
   The Monte Carlo deposited energy distribution in the Mu2e tracker. The distribution
   corresponds to 1BB pile-up (Section \ref{pulsedprotonbeam}). The red distribution refers to 
   conversion electrons, while the green and the blue one to $\delta$-electrons and protons respectevely.
}
   \label{fig:energydeposited}
\end{figure}

\section{Analysis of the performance and comparison}
\subsection{Dataset}
\subsection{Low level benchmark}
\subsection{High level benchmark}

\section{Conclusions}
\iffalse

 The detailed background rates for these processes at this energy scale are not very
well studied. In Mu2e, background rates calculated according to figures given in Nuclear Physics of
Muon Capture (D. F. Measday) are used to create simulations. However, due to the lack of
documentation at the energy scales involved in this experiment, there is a large uncertainty in
background rates which requires background removal to be insensitive to the level of backgrounds
This primary purpose of study is to address one prevalent source of background hits: the
secondary ionized electrons, which are referred to as "deltas." Deltas consist primarily of Compton
scattered electrons, pair production electrons, and delta rays, in order of decreasing prevalence.
Compton scattered electrons are produced when photons from various processes strike material
inside the detector. This causes the ejection of electrons that create the background events. The
photons primarily originate from neutron capture by atoms, which leads to an excited nuclear state
that decays via photon emission. These photons generally have energies on the order of a few MeV.
The neutrons, in turn, are created when muons are captured by atoms, creating unstable isotopes that
decay by ejecting neutrons. Pair production electrons, on the other hand, are produced when
processes involving nuclear recoil also produce pairs of electrons and positrons in order to conserve
energy and momentum. Delta rays, or secondary ionized electrons, are produced when charged
energetic particles collide with detector material.






DELTAFINDER:
step 1: Find $\delta$-electron track segments separately within each station
connect segments, allow 1 station wide "gaps"
dominant sources of failures:
I stations with MC particle producing hits in only one face;
I hits with wrong coordinates along the wire - long tails of the $\Delta$T distribution;
I recover hits in "empty" stations to improve efficiency;
I optimization of the algorithm timing performance;
electron hit energy dependence on momentum: path length within the straw depends on momentum
only about 4\% of CE hits have energies above 3.5keV (1\% above 5 keV)
consider several hit energy cutoffs: 3.5 keV, 5keV, 7keV, use all hits




\subsection{Delta electron features}








Figure 2.1 shows the momentum distribution of deltas for 10,000 events.
Most deltas have much lower momenta than conversion electrons. This produces a significant
difference in their trajectories through the tracker, as shown in Figures 2.2 to 2.4. With the naked eye
one can already see dramatic differences.
The most striking quality of deltas is that they tend to create small, dense clusters of hits, as
shown in Figures 2.3 and 2.4. This motivates a "clustering approach"*: we search for such
concentrated clusters in the x-y plane using a clustering algorithm since deltas are very likely to reside
in such a cluster. However, the reverse containment does not hold - clusters can also contain many
conversion electron hits. Removing conversion electron hits can severely harm subsequent
reconstruction, so a filter is necessary. 
low energy (delta, compton, photon conversion) electrons are the largest source of the hits
in the tracker - about 2/3 of the total
language: $\delta$-electron - an electron with P<20 MeV/c
radius of a 10 MeV/c electron in the nominal Mu2e field < 3 cm, close to the resolution
along the wire
very specific topology - multiple hits with the same (X,Y) , within the resolution
\fi