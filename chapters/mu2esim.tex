
\chapter{Mu2e Simulation Framework}\label{mu2eana}

The Mu2e software framework is designed to study the expected performance of the Mu2e detectors, the signal 
reconstruction efficiency, and the characteristics of the backgrounds. The simulation framework is based on 
GEANT4, a widely used simulation toolkit. The simulation relies on Monte Carlo methods.

It is implemented in the C++ programming language and includes a comprehensive set of tools, such as 
tracking, geometry, and physics models. The library provides models of physical processes like particle 
scattering, energy loss, and decay of long-lived particles over a wide energy range. The simulation 
environment implements the Mu2e geometry. Pattern recognition and track reconstruction algorithms are 
incorporated into the software framework.

\section{art and FHiCL}

The Mu2e Offline software is based on the \textit{art} framework. 
This framework is developed and maintained by the Fermilab Scientific 
Computing Division (SCD) and is used by multiple Intensity Frontier experiments at Fermilab.

\textit{art} is a command-line-driven event-processing framework written in 
C++. It operates in a non-interactive mode where it sequences events as directed 
by the user. It has been designed to fulfill a wide range of requirements in 
high-energy physics experiments, including high-level software triggers, online data 
monitoring, calibration, reconstruction, simulation, and analysis.

\textit{art} runtime configuration is written in the Fermilab Hierarchical 
Configuration Language (FHiCL), a data definition language developed at Fermilab. 
A FHiCL file contains definitions of C++ classes that implement the \textit{art} 
services. Within these classes, algorithms ranging from simulation and 
reconstruction to analysis codes are built and integrated into dynamic 
libraries called modules. The FHiCL files declare which modules will be loaded, 
in what order they will run, and which files will be read in input and written in output.

The simulation begins by using a FHiCL file to configure the process, specifying 
the necessary modules and call files containing essential geometry and physics data.

\section{STNTUPLE and ROOT}

STNTUPLE is an n-tuple data format and a lightweight n-tuple analysis framework, 
written in C++. It has been used for many years by the CDF experiment at Fermilab 
and ported to Mu2e. One of the plug-in modules for \textit{art} is specific to the 
usage of the STNTUPLE. Every STNTUPLE file is a ROOT file containing multiple branches, 
each corresponding to a data block. A block is a data container optimized for I/O and 
analysis purposes, storing the Mu2e raw and/or reconstructed data. A STNTUPLE of the 
data saved in an \textit{art} file can be created and stored by using the appropriate 
module in the .fcl \textit{art}-job configuration file. The type of data saved in this 
format is customizable. Once the .stn file is generated, the analysis can be conducted 
using this data format, avoiding the re-running of the reconstruction. STNTUPLE is built 
on top of a data analysis and graphics package ROOT developed at CERN and allows the use 
of all interactive features of ROOT in the analysis.

\section{Multi-stage Simulation}

Mu2e simulation employs a technique known as multi-stage simulation for efficient event 
generation and simulation. This approach involves generating and partially simulating events, 
pausing the simulation, and saving the generated data. Subsequent stages build upon this saved 
data, extending the simulation before saving again. Multiple stages may be utilized to complete 
a simulation. Stages can end when particles reach specific planes or volumes, such as the DS 
region containing the target and detector or the tracker volume. This method is particularly 
useful when early stages consume most of the CPU time in Mu2e simulations.

The motivations for using multi-stage simulation include:

\begin{itemize}
    \item \textbf{Progressive detector definition:} The simulation can define a portion of the 
    detector in later stages. For instance, computationally-intensive tasks like simulating 
    protons on target can be completed in an initial stage, allowing the tracker geometry and 
    algorithms to be inserted later. Once the tracker is ready, simulation can proceed into its volume.
    \item \textbf{Efficient design variation studies:} It facilitates the study of different 
    experimental designs. Simulation stages can end outside a detector, enabling quick testing 
    of various detector designs with no need to repeat the entire prior simulation.
    \item \textbf{Effective error recovery:} It acts like a check-pointing mechanism, enhancing 
    error recovery efficiency. If an error occurs in a later stage, only that stage needs to be 
    redone using the output from the preceding stage. When early stages dominate CPU time, this 
    approach can lead to significant time savings.
    \item \textbf{Resource and limitation handling:} It can handle constraints like job duration 
    and output file size. Techniques like compression, event filtering, and concatenation can be 
    employed to manage resources effectively.
\end{itemize}

Multi-stage simulation also supports two other critical simulation techniques: mixing and resampling. 
These two techniques are necessary to study backgrounds and low statistics processes.

For the study of muon beamline vertical misalignments developed in this Thesis, the simulation 
framework consists of six stages:

\begin{itemize}
    \item \textbf{First stage (s1):} The interactions of 8 GeV protons in the Production Target (PT) 
    are simulated. All produced particles form the beam, which is traced up to the midpoint of the TS, 
    storing information about any particle that makes it that far.
    \item \textbf{Second stage (s2):} It uses the output of s1 and traces the beam up to the entrance 
    of the Detector Solenoid.
    \item \textbf{Third stage (s3):} It propagates the surviving particles from s2 through the 
    upstream portion of the DS vacuum and records muons stopped in the aluminum Stopping Target (ST).
    \item \textbf{Fourth stage (s4):} $\mu^+$ and $\mu^-$ stopped in the ST are separated into two 
    different outputs for separate analysis.
    \item \textbf{Fifth stage (s5):} The detector is finally simulated. Target-stopped muons are 
    forced to undergo Michel decays. The resampling technique is employed to increase the available 
    statistics; each stopped muon is used several times as a starting point to generate different 
    decays. Particles that intersect the tracker or calorimeter are recorded.
    \item \textbf{Sixth stage (s6):} The simulated raw data are converted into simple C++ classes 
    or structs, simulating the digitization of detector raw data.
    \item \textbf{Seventh stage (s7):} The final stage reconstructs the particle tracks and 
    generates the n-tuples.
\end{itemize}

Multi-stage simulation saves computation resources in the study of vertical misalignment 
effects. The production of datasets with displaced COL3 requires only re-running from stage 
2 onwards. This is very convenient as s1 is the most resource-consuming one.

\section{Mu2e Offline Event Reconstruction}

The central objective of the Mu2e reconstruction algorithms is to achieve efficient 
reconstruction of electrons in the range of conversion electrons. To fulfill this objective, 
the algorithms and a few user-defined parameters within them are set to default values optimized 
specifically for this scenario. This section outlines the stages of a Mu2e event reconstruction.

\section{Tracker Hit Reconstruction and Pre-filtering}

The Mu2e reconstruction process starts with hit reconstruction, where digital signals from the 
tracker are converted into physical time and position, creating StrawHits. Adjacent StrawHits in 
a panel, likely from the same particle, are combined into ComboHits to facilitate pattern recognition. 
The FlgBkgHits algorithm flags hits from low-energy electrons or positrons due to scattering, $\gamma$ 
conversion, or $\delta$-rays. This algorithm clusters hits in time and in the $xy$ plane, using 
Multivariate Analysis to distinguish low-energy hits from conversion electron hits, which are stored for subsequent pattern recognition.

\section{Calorimeter Hit Reconstruction}

CaloClusters are formed by combining signals from crystals hit by particles 
in the calorimeter. Clusters are reconstructed by grouping adjacent crystals 
within a 2 ns window and an adjustable energy threshold, currently set at 50 MeV.

\section{Helix Search}

Charged particles in the DS magnetic field follow helical trajectories described 
by the following parameters:

\[
\vec{\eta} \equiv (d_0, \phi_0, \omega, z_0, \tan \lambda);
\]

where $d_0$ is the distance of the point of closest approach to the solenoid axis, 
signed by the particle angular momentum with respect to the origin; $\phi_0$ is 
defined by the momentum direction at the point of closest approach; $\omega = 1/R$ 
is the curvature in the transverse plane; $z_0$ is the $z$-coordinate of the point of 
closest approach; $90^\circ - \lambda$ is the pitch angle between the momentum $p$ and 
the $xy$ plane. Helix search involves time clustering and pattern recognition.

\subsubsection{Time Clustering}

The Time Clustering algorithm groups hits within a narrow time window into TimeClusters. 
Since ComboHits from the same particle tend to cluster in time, for each group of ComboHits a 
TimeCluster is created. To improve the clustering, the time distribution is generated by 
propagating all the hit times to the central plane of the tracker ($z=0$). Finally, the 
TimeCluster time and position in the calorimeter are evaluated. The process is iterated 
until the list of ComboHits associated with the TimeCluster is stable. All TimeClusters with 
more than a programmable number of hits are stored. Pattern recognition follows, utilizing 
information from both tracker and calorimeter through two different algorithms.
\subsubsection{Pattern Recognition}

Pattern recognition in Mu2e can be categorized into two main approaches: 
Tracker-only Pattern Recognition and Calorimeter + Tracker Pattern Recognition.

\paragraph{TrkPatRec: Tracker-only Pattern Recognition}

TrkPatRec performs helix reconstruction in the $xy$ plane and finds the 
track projection on the transverse plane, determining the radius and impact 
parameter with respect to the Stopping Target. It then performs reconstruction 
in the $\phi z$ plane to determine the track pitch. Circle fitting and angular 
position corrections are used for reconstruction.

\paragraph{CalPatRec: Calorimeter + Tracker Pattern Recognition}

CalPatRec employs CaloClusters as initial templates for pattern 
recognition. Assuming the existence of a CaloCluster with reconstructed 
energy above 50 MeV, its time and position are used to filter the collection of 
ComboHits. These hits are expected to lie within the calorimeter acceptance, 
and those within a defined time window around the cluster time are selected. 
Helix fitting is performed using the filtered ComboHits, with track parameters 
iteratively adjusted until convergence is achieved. This combined approach 
enhances track reconstruction efficiency, particularly for electrons in the conversion energy range.

\section{Track Fitting and Momentum Determination}

Once the track candidates have been identified, track fitting algorithms 
are applied to refine the track parameters and estimate the momentum of the 
charged particles. The primary track fitting algorithm used in Mu2e is the Kalman 
Filter, a recursive method that optimally estimates the track parameters by minimizing the chi-squared of the track fit.

The Kalman Filter considers multiple scattering, energy loss, and magnetic 
field variations, making it well-suited for the high-precision tracking 
required in the Mu2e experiment. The momentum resolution achieved is crucial 
for distinguishing conversion electrons from background processes.

\section{Background Rejection and Signal Selection}

Background rejection is a critical aspect of the Mu2e event reconstruction, 
as the experiment aims to detect rare conversion events amidst a significant 
background. Several strategies are employed to enhance signal selection and suppress background events:

\begin{itemize}
    \item \textbf{Timing Cuts:} Events are selected based on their timing relative 
    to the beam structure, reducing beam-related backgrounds.
    \item \textbf{Track Quality Cuts:} Tracks must meet specific quality criteria, 
    such as a minimum number of hits and a maximum chi-squared value, to ensure reliable reconstruction.
    \item \textbf{Calorimeter Matching:} Tracks are matched with CaloClusters based 
    on spatial and temporal criteria to confirm the presence of an electromagnetic 
    shower consistent with a conversion electron.
    \item \textbf{Likelihood-Based Methods:} Multivariate analysis techniques, such 
    as boosted decision trees, are employed to combine various discriminating 
    variables and enhance signal purity.
\end{itemize}

\section{Data Analysis and Event Classification}

After reconstruction and background rejection, the final step in the Mu2e 
simulation framework is data analysis and event classification. This involves 
categorizing events into different classes based on their characteristics and 
analyzing the resulting distributions to extract physics results.

The analysis is performed using ROOT, a powerful data analysis framework that 
provides a wide range of tools for histogramming, fitting, and statistical analysis. 
Custom scripts and macros are developed to automate the analysis process and generate 
plots and tables for interpretation.

\section{Conclusion}

The Mu2e simulation framework is a sophisticated software suite that enables 
comprehensive studies of detector performance, signal reconstruction, and background 
characteristics. By utilizing a combination of multi-stage simulation, advanced 
reconstruction algorithms, and efficient background rejection techniques, the framework 
supports the search for charged lepton flavor violation in the Mu2e experiment. The flexibility 
of the simulation framework allows for thorough investigations of various physics scenarios, 
design optimizations, and detector responses.

Through the integration of state-of-the-art technologies such as GEANT4, \textit{art}, 
and ROOT, the simulation framework provides the necessary tools to achieve the scientific 
goals of the Mu2e collaboration. Ongoing developments and improvements in the framework 
continue to enhance the precision and reliability of the simulation, paving the way for 
groundbreaking discoveries in the field of particle physics.
