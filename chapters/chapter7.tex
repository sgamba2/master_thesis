\chapter{Conclusions}\label{conclusions}
The success of Mu2e depends on many factors, 
one of which is the performance of the tracker. 
The tracker must provide excellent momentum 
resolution, approximately 1 MeV/c, 
in order to distinguish the monochromatic CE 
signal from the background. 
The Mu2e collaboration opted for a detector 
based on the straw tube technology.  
Its distinctive annular geometry is highly 
effective in reducing background. 
The tracker is placed inside the 
DS, downstream from the ST, 
in a uniform 1 T magnetic field. The detector 
has a modular design, consisting of basic 
elements referred to as 
$Panel$s, $Face$s, $Plane$s, and $Station$s. 
The panels contain arrays of 96 straw tubes, 
which serve as the sensitive units, arranged 
like the harp chords. The detector is made of 
18 stations, each with 2 planes, each plane 
with 2 faces, and each face with 3 panels. 

This Thesis provides a comprehensive 
contribution to the Mu2e tracker 
development, covering from its initial stages of 
commissioning to the optimization 
of DAQ and FEE and preliminary calibration 
steps. My work at Fermilab 
focused on thorough testing of the DAQ system 
from both hardware and 
software perspectives, and included 
involvement in the Vertical Slice Test 
(VST) of the tracker. 
The VST encompasses the entire testing 
chain, from the straws to the readout, 
to processed data on disk.
Throughout the chapters, we explored 
critical aspects of the tracker 
system, focusing on the robustness and 
reliability of data acquisition 
processes and the crucial role of 
calibration in ensuring precise measurements. 
Part of my work was dedicated to the offline 
analysis, particularly to pre-pattern recognition 
studies, investigating the optimal methods for 
flagging $\delta$-electrons during data taking. 
I performed the very first systematic study 
comparing the performance of two algorithms 
to determine the best approach for data-taking. 
\section{Commissioning of the tracker DAQ and FEE}
The commissioning of the tracker DAQ and FEE 
systems, described in Chapter 
\ref{commissioning}, has been a crucial step 
in validating the functionality 
and performance of the tracker readout chain.

The first test I performed was to verify the 
correct functioning of 
ROC buffering and to understand its logic. 
During these tests, a single 
ROC was connected to one DTC. Data were 
collected with digi-FPGAs pulsed 
by their internal pulsers, with the ROC set 
in external mode. The ROC's  
readout logic can be emulated with a bit-level 
C++ simulation, which I 
contributed to develop. The Monte Carlo simulation 
and the experimental 
data were compared in two different modes: 
\textit{ROC buffer underflow} and \textit{overflow} 
configuration. These configurations depend on the 
internal hit buffer, which stores up to 255 hits, 
and, depending on \( T_{gen} \) and \( T_{EW} \), 
the total number of hits within the event window 
can be respectively 
less than or equal to 255. 
By studying the timing distribution and delays 
between channels and digi-FPGAs, we 
were able to incorporate realistic delays into 
the simulation. The comparison 
between the data and the bit-level simulation 
demonstrated good agreement, 
validating the simulation's accuracy in reproducing 
the expected ROC behavior under 
the two ROC buffer modes. The agreement between 
the simulation and 
experimental data was highly satisfactory, with 
deviations at the level of \( 10^{-3} \).

The second test focused on evaluating the 
performance of preamplifiers, 
particularly looking for inactive channels, 
cross-talk between channels, 
and unexpected pulse patterns. The first part 
of the test examined channel occupancy, revealing 
dead channels and channels exhibiting a higher 
number of hits than expected. 
Cross-talk was observed in only the first channels, 
with an asymmetry in interference 
between odd and even channels. This phenomenon was 
attributed to the physical proximity 
of preamplifiers on the motherboard and the 
closeness of the first channels to each other. 
Although cross-talk was not fully resolved at 
the time of writing, further investigations 
are ongoing with tracker experts.
Subsequent waveform analysis helped explain 
abnormalities, particularly the inverted 
waveforms observed in some channels, which indicated 
the erroneous triggering of waveform 
on both the leading and trailing edges. The 
waveform analysis 
provided further insights into the charge and 
pulse height distributions, showing two or 
three peaks in their distribution. This was due 
to timing mismatches between the pulser 
and ADC, leading to incorrect readings. Overall, 
these tests provided valuable results 
that will inform future improvements to the 
preamplifier design and its integration into 
the readout chain. They also laid the foundations 
for the development of real-time 
diagnostic tools, along with the identification of 
key performance issues.

In summary, this commissioning phase successfully validated 
the core functionalities 
of the tracker DAQ and FEE systems, ensuring they are ready 
for the next stages of 
testing and integration. The insights gained, particularly 
in handling ROC overflow 
and timing synchronization, will be essential for optimizing 
the system's performance 
during full detector operation. Further work will focus on 
expanding the scope of 
testing with more complex configurations and refining the 
DAQ software for future runs.


\section{First steps towards the station calibration}
Chapter \ref{planning} describes the initial steps towards 
tracker calibration. Our primary objective is to perform a time calibration of 
the first assembled station using cosmic muons, with the aim of achieving 
a resolution on the longitudinal hit position better than 4 cm. This  
requires determining the signal propagation velocity and 
channel-to-channel delays. 

When a charged particle passes through the straw tube gas volume, the 
resulting ionization charge generates an electric signal along the anode wire, 
which propagates to both ends of the straw and reaches the front-end 
electronics, where TDCs measure the signal arrival times, $t_1$ and $t_2$. The primary 
goal of time calibration is to determine the track position along the straw 
by measuring these arrival times. 

I studied the reconstruction of $x_{\text{track}}$ using only the 
information of whether a straw was crossed or not by a cosmic muon. The only 
way to deduce $x_{\text{track}}$ from this "yes-no" information is to use the geometry of the tracker station, 
which consists of a set of rotated panels arranged on four faces orthogonal 
to the detector axis. Since the Mu2e collaboration considered 
that a vertical station calibration might be feasible for our 
goals, due to operational constraints of the station, such as 
gas distribution, fragility, and space, I examined the potential 
biases and systematic errors that might result from this orientation. 

The first step in my analysis involved studying the selection criteria for 
muons crossing a station and then the hit distribuion on the panels 
resulting from the selection. 
The request of having at least one hit per face resulted in 
a non-uniform illumination, with almost no hits 
in the central region of the panel, potentially leading to 
waveform non-linearities. 
Another critical factor is the cosmic event rate, which is 
computed to be approximately 
$R \sim 380$ Hz. The resulting calibration duration would be around 6 days, 
which would be hard to manage. 

To reconstruct the 3D trajectory of the cosmic muon, at least 
two 3D points along its path are needed. The $x$ and $y$ coordinates 
of each point can be determined by the intersection of one pair of 
non-parallel straws. This analysis revealed that a vertical orientation 
for station calibration is not optimal, as the bias range spans 
approximately [-6, 6] cm. 
The 2D distribution of the longitudinal bias versus the 
true position shows four 
distinct spots along the longitudinal axis, each corresponding 
to overlap regions of different straws. The TProfile indicated a systematic effect in determining 
the longitudinal position, with the range extending beyond [-4, 4] cm. 
This suggests that the mean is not a reliable estimator of the bias under 
these conditions, making the calibration process more challenging.

Consequently, the horizontal orientation should be considered to 
optimize the calibration and achieve more accurate results. A new 
mechanical approach is currently under development to address 
the issues outlined. 
In the next months, once the tools to allow the horizontal 
orientation of the station will have been developed, the Mu2e 
collaboration will perform the cosmic muon data-taking and the 
first tracker calibration. 
\section{Pre-pattern recognition studies}

Given the high data volume expected during Mu2e data taking, 
estimated at approximately 7 PBytes per year, optimizing memory usage 
and minimizing CPU consumption are critical. One significant challenge 
is flagging $\delta$-electron hits, since they are the primary source 
of hits in the tracker, without compromising the efficiency of 
CE hit detection and 
track reconstruction. $\delta$-electrons originate from Compton 
scattered electrons, 
pair production electrons and positrons, and delta rays. Compton 
scattered electrons 
are produced when photons from neutron capture interact with the 
detector material, and  
typically have energies of a few MeV. Pair production electrons 
and positrons 
are generated during nuclear recoil processes, and delta rays are produced 
when high-energy charged particles collide with the detector material.

A detailed study of pre-pattern recognition and a comparison of two 
algorithms for 
$\delta$-electron flagging is provided in Chapter \ref{delta} to 
address this issue. 
This study has strong motivations. First, Mu2e's 
primary goal is 
to detect the CE signal: flagging even a small fraction of hits generated 
by the CE as $\delta$-electrons may reduce the CE track 
reconstruction efficiency. 
Second, misidentifying muons and pions as protons or $\delta$-electrons 
can result in inaccurate background estimates. Being able to identify 
protons and count the number of protons in one event is extremely 
important, since proton counting could complement the STM in determining 
the muon stopping rate. 
Simulations show that the majority (approximately 75\%) of hits in 
the tracker 
are generated by $\delta$-electrons, i.e., electrons and positrons 
with momenta below 20 MeV/c.

Two algorithms have been developed in Mu2e collaboration to identify 
and exclude $\delta$-electrons hits from the reconstruction 
process: $FlagBkgHits$ and $DeltaFinder$. 
The $FlagBkgHits$ algorithm initially clusters hits on $x-y$ plane 
and on time and uses an ANN to classify them, 
while $DeltaFinder$ identifies clusters of hits consistent with those 
produced by 
low-momentum particles. Based on the mean energy deposited by a $seed$ 
in a station, hits with energy above 5 keV are considered 
proton hits, while $FlagBkgHits$ lacks the 
capability to identify proton hits.

I conducted a systematic, $two-level$ comparison of the performance 
of the two algorithms:
\begin{enumerate}
    \item \textbf{hit-level comparison}: this study evaluates the accuracy 
    of individual hit flagging, providing a direct method to compare 
    the algorithms' performance. In terms of $\delta$-electron flagging, 
    the two algorithms have a similar performance. However, 
    $DeltaFinder$ performs 
    better in flagging positrons due to the low statistics at energies 
    below 2 MeV/c. This presents a disadvantage for $FlagBkgHits$ since 
    the ANN performance depends on statistics, and it struggles 
    to reconstruct hits 
    that register only a single hit per station. $FlagBkgHits$ flags 
    approximately 70\% more CE hits than $DeltaFinder$, as it analyzes 
    the $x-y$ plane and does not reconstruct hits in the $z$ coordinate. 
    Additionally, $DeltaFinder$ 
    can flag proton hits (approximately 84\%), a task that $FlagBkgHits$ 
    cannot perform. 
    It is crucial that hits corresponding to $p\bar{p}$ annihilation 
    are not flagged. $FlagBkgHits$ has been trained on datasets lacking 
    muon and pion hits and thus flags these particles at rates roughly 
    4 and 3.3 times higher, respectively, than $DeltaFinder$;

\item  \textbf{high-level comparison}: this study 
assesses the algorithms' impact on 
subsequent event reconstruction stages, 
particularly how each contributes to accurate 
track reconstruction. The main difference 
between the two algorithms is in the number 
of reconstructed tracks using CE data samples, 
which is 16\% higher for $FlagBkgHits$. 
This increase is due to $FlagBkgHits$ failing to 
flag protons properly, allowing these 
particles to be sent to the reconstruction stage. 
The difference in the reconstruction of $p\bar{p}$ 
events, measured as the fraction of events with at 
least two reconstructed tracks, is approximately 
22\% higher with $DeltaFinder$.
\end{enumerate}

An important aspect of the study was the 
timing performance of the two algorithms. 
For $FlagBkgHits$, it was necessary to 
account for the time spent creating the $StereoHit$s, 
while $DeltaFinder$ independently reconstructed 
the $seed$s. The timing performance was 
studied across all data samples, and showed that 
the difference in processing time between 
the two algorithms is negligible (with 
$FlagBkgHits$ having an advantage of less than 1 ms), 
considering that the entire reconstruction 
process takes few ms per event.

The primary drawback of $FlagBkgHits$ is its 
reliance on supervised training using 
CE and $\delta$-electron samples. This 
approach fails when other particles, such 
as cosmic muons and those from $p\bar{p}$ 
annihilation, are introduced into the algorithm. 
Furthermore, the training was performed using 
Monte Carlo data rather than real data, 
posing a potential risk when transitioning 
to actual data-taking. Additionally, as an 
ANN-based method, $FlagBkgHits$ performs 
poorly when statistics are very low and lacks 
a well-defined method for proton flagging.
