\chapter{Conclusions}\label{conclusions}
The success of Mu2e depends on many factors and one of them is the 
performance of the tracker. 
The tracker must provide excellent momentum resolution, approximately 1 MeV/c, 
to distinguish the monochromatic conversion electron signal from the background. 
To minimize energy loss, a straw tube tracker will be used, so the 
technology used is the drift tubes. Its 
annular shape is designed to follow the helical trajectories of conversion 
electrons in the magnetic field. The detector has a modular design, consisting 
of basic elements referred to as $Panel$s, $Face$s, $Plane$s, and $Station$s. The 
tracker comprises panels containing arrays of straw tubes, which are the 
sensitive units, arranged like the harp chords.

This Thesis has provided an in-depth examination of the Mu2e 
tracker system, from its initial stages of commissioning to the 
optimization of data acquisition and preliminary calibration steps. 
Throughout the chapters, we explored critical aspects of the tracker 
system, focusing on the robustness and reliability of data acquisition 
processes and the crucial role of calibration in ensuring precise measurements. 

The first phase of this work was integral to the early phases of the Mu2e experiment 
at Fermilab, particularly in the context of the 
tracker system's DAQ and FEE testing. 
The first test I performed has been to verify the correct performance of ROC 
buffering. During these test, a single ROC was connected to the DTC. Data were collected
with digi-FPGAs pulsed by their internal pulsers, with the ROC set in the external
mode. ROC's digital readout logic allows to be emulated with a bit-level C++ simulation,
which I contributed to develop. The Monte Carlo and the data have been compared in two different 
modes: $ROC$ $buffer$ $overflow$ and $underflow$ configuration, at a channel occupancy level, studying 
the timing distribution and delays between channels and digi-FPGAs.