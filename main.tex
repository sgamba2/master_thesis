    %   Il progetto nasce dal template per il frontespizio realizzato da Marco Antonio Corallo, che ringrazio. Seguono alcuni commenti per evidenziare la presenza di alcuni pacchetti che mi sono stati utili per la stesura della tesi. Chiaramente, dipende tutto dal tipo di lavoro che uno vuole eseguire, che determina anche le diverse esigenze. Durante la stesura ho passato molto tempo su siti e forum a cercare di risolvere alcuni probelmi di formattazione, ma in generale Latex è stato piuttosto versatile. 

% Tipo di documento. L'uso di twoside implica che i capitoli inizino sempre con la prima pagina a sinistra, eventualmente lasciando una pagina vuota nel capitolo precedente. Se questa cosa è fastidiosa, è possibile rimuoverlo. 
\documentclass[a4paper, twoside,openright]{report}
\usepackage{tikz}
\usepackage{[caption}
\usetikzlibrary{arrows}
\usetikzlibrary{decorations.markings}
%\usepackage{lineno}

\usetikzlibrary{decorations.pathmorphing}
% \usepackage[absolute,overlay]{textpos}
% \usepackage{onimage}
% Dimensione dei margini
\usepackage[a4paper,top=3cm,bottom=3cm,left=3cm,right=3cm]{geometry} 
% Dimensione del font
\usepackage[fontsize=13pt]{scrextend}
% Lingua del testo
\usepackage[english]{babel}
% Lingua per la bibliografia
\usepackage[fixlanguage]{babelbib}
% Codifica del testo
\usepackage[utf8]{inputenc} 
% Encoding del testo
\usepackage[T1]{fontenc}
\usepackage{caption}
\usepackage{subcaption}
% Permette di generare testo fittizio. Mi è stato utile 
% per capire quale sarebbe stata l'impostazione del 
% testo nella pagina prima che scrivessi un determinato paragrafo
\usepackage{lipsum}
% Per ruotare le immagini
\usepackage{rotating}
% Per modificare l'header delle pagine 
\usepackage{fancyhdr}       \usepackage{tikz-feynman}     

% Librerie matematiche
\usepackage{amssymb}
\usepackage{mathrsfs}
\usepackage{amsmath}
\usepackage{amsthm}         

% Uso delle immagini
\usepackage{graphicx}
% Uso dei colori
%\usepackage[dvipsnames]{xcolor}         
% Uso dei listing per il codice
\usepackage{listings}          
% Per inserire gli hyperlinks tra i vari elementi del testo 
\usepackage{hyperref}     
% Diversi tipi di sottolineature
\usepackage[normalem]{ulem}

% -----------------------------------------------------------------

% Modifica lo stile dell'header
\pagestyle{fancy}
\fancyhf{}
\lhead{\leftmark}
\rhead{\textbf{\thepage}}
\fancyfoot{}
\setlength{\headheight}{12.5pt}

% Rimuove il numero di pagina all'inizio dei capitoli
\fancypagestyle{plain}{
  \fancyfoot{}
  \fancyhead{}
  \renewcommand{\headrulewidth}{0pt}
}

\newcommand\listappendixname{List of Appendices}
\newcommand\appcaption[1]{%
  \addcontentsline{app}{chapter}{#1}}
\makeatletter
\newcommand\listofappendices{%
  \chapter*{\listappendixname}\@starttoc{app}}
\makeatother


% Stile del codice
\lstdefinestyle{codeStyle}{
    % Colore dei commenti
    commentstyle=\color{black},
    % Colore delle keyword
    keywordstyle=\color{black},
    % Stile dei numeri di riga
    numberstyle=\tiny\color{black},
    % Colore delle stringhe
    stringstyle=\color{violet},
    % Dimensione e stile del testo
    basicstyle=\ttfamily\footnotesize,
    % newline solo ai whitespaces
    breakatwhitespace=false,     
    % newline si/no
    breaklines=true,                 
    % Posizione della caption, top/bottom 
    captionpos=b,                    
    % Mantiene gli spazi nel codice, utile per l'indentazione
    keepspaces=true,                 
    % Dove visualizzare i numeri di linea
    numbers=left,                    
    % Distanza tra i numeri di linea
    numbersep=5pt,                  
    % Mostra gli spazi bianchi o meno
    showspaces=false,                
    % Mostra gli spazi bianchi nelle stringhe
    showstringspaces=false,
    % Mostra i tab
    showtabs=false,
    % Dimensione dei tab
    tabsize=2
} \lstset{style=codeStyle}

% Stile di codice per dimensioni maggiori, in cui ho avuto bisogno di un testo più picolo (ad esempio se si vuole inserire del codice che ha linee molto lunghe). Per usare questo stile piuttosto che il precedente, usare 

% \lstset{style=longBlock}
%  % inserire il codice...
% \lstset{style=codeStyle}

% Il secondo comando consente di tornare allo stile precedente 
\lstdefinestyle{longBlock}{
    commentstyle=\color{teal},
    keywordstyle=\color{Magenta},
    numberstyle=\tiny\color{gray},
    stringstyle=\color{violet},
    basicstyle=\ttfamily\scriptsize,
    breakatwhitespace=false,         
    breaklines=true,                 
    captionpos=b,                    
    keepspaces=true,                 
    numbers=left,                    
    numbersep=5pt,                  
    showspaces=false,                
    showstringspaces=false,
    showtabs=false,                  
    tabsize=2
} \lstset{style=codeStyle}

% Togliendo il commento al comando che segue, si inseriscono nella bibliografia anche le fonti presenti in Bibliography.bib ma non citati direttamente con il comando \cite
% \nocite{*}

% Margini prima e dopo blocchi di codice, per avere più distanza
\lstset{aboveskip=20pt,belowskip=20pt}

% Modifica dello stile dei riferimenti, con il testo in cyano
\hypersetup{
    colorlinks,
    linkcolor=black,
    citecolor=black
}

% Aggiunti definizioni, teoremi, linea e listing
\newtheorem{definition}{Definizione}[section]
\newtheorem{theorem}{Teorema}[section]
\providecommand*\definitionautorefname{Definizione}
\providecommand*\theoremautorefname{Teorema}
\providecommand*{\listingautorefname}{Listing}
\providecommand*\lstnumberautorefname{Linea}

\raggedbottom

%\newcommand{\cgs}[1]{{\textcolor{brown}[\textcolor{red}{\bf{GS: }}{ \textcolor{brown}{#1]}}}}             
%\newcommand{\cmc}[1]{{\textcolor{blue}[\textcolor{magenta}{\bf{MC: }}{ \textcolor{blue}{#1]}}}}
\newcounter{apdxsection}
\newcommand\unappendix{\par
  \setcounter{apdxsection}{\value{section}}%
  \setcounter{section}{\value{savesection}}%
  \setcounter{subsection}{0}%
  \gdef\thesection{\@arabic\c@section}}
\setcounter{tocdepth}{4}
\setcounter{secnumdepth}{4}
% -----------------------------------------------------------------
\begin{document}

\begin{titlepage}
    \begin{center}
\begin{figure}[!htb]
    \centering
    \includegraphics[keepaspectratio=true,scale=0.4]{figures/eps/cherubinFrontespizio.eps}
\end{figure}
\begin{figure}[!htb]
    \centering
    \includegraphics[keepaspectratio=true,scale=0.45]{figures/eps/logo_pant541.eps}
\end{figure}

%\begin{center}
    %\LARGE{University of Pisa}
    %\vspace{5mm} \\
    \Large{Department of Physics E. Fermi}
    \vspace{5mm}
    \\ 
    \Large{Master's Degree in Physics}
\end{center}

\vspace{15mm}
\begin{center}
    {\LARGE{\bf Commissioning of the DAQ system,\\ \vspace{3mm} pre-pattern recognition studies and\\ \vspace{5mm} planning for the calibration of\\ \vspace{7mm} the Mu2e tracker at Fermilab}}
\end{center}
\vspace{25mm}

\begin{minipage}[t]{0.47\textwidth}
	{\large{Supervisors:}{\normalsize\vspace{3mm}
	\bf\\ \large{Dr. Pavel Murat} \normalsize\vspace{3mm}\bf \\ \large{Prof. Simone Donati}}}
\end{minipage}
\hfill
\begin{minipage}[t]{0.47\textwidth}\raggedleft
	{\large{Candidate:}{\normalsize\vspace{3mm} \bf\\ \large{Sara Gamba}}}
\end{minipage}

\vspace{25mm}
\hrulefill
\\\centering{\large{Academic Year 2023/2024}}

\end{titlepage}

\cleardoublepage
\thispagestyle{empty}
\vspace*{\stretch{1}}
\begin{flushright}
\itshape 
dediche
\end{flushright}
\vspace{\stretch{2}}
\cleardoublepage

\begin{abstract}
\noindent
The primary objective of the Mu2e experiment is to search for neutrino-less coherent $\mu \rightarrow e$ conversion in the 
field of an aluminum nucleus, $\mu^- \text{Al} \rightarrow e^-\text{Al}$. This process's signal is a monoenergetic electron 
with an energy of approximately 104.97 MeV, Ref. \cite{bartoszek2015mu2e}. 
In the Standard Model (SM), the branching ratio for this process, or for the similarly 
intriguing $\mu \rightarrow e \gamma$ decay, is expected to be of the order of $\mathcal{O}(10^{-54})$. 
These values are well below the sensitivity of any current experimental capabilities, but several models of physics beyond the 
SM predicts a much higher relative rate that could reach an observable level in some super-symmetry scenarios. 
The SINDRUM II experiment established an upper limit on muon conversion 
at $7 \times 10^{-13}$ (90\% C.L.) \cite{SINDRUMII:2006dvw}, and the Mu2e collaboration aims to improve this limit by four orders of magnitude. 
Observing this process would provide unambiguous evidence of physics beyond the Standard Model and its search is complementary to 
the direct searches for new physics currently carried out at the CERN Large Hadron Collider. 

Mu2e will make use of a sophisticated setup to achieve its goals. An 8.9 GeV proton beam from Fermilab will strike a tungsten target, producing $\mu$s through $\pi$s decay. 
These particles are transported via a series of superconducting solenoids to an aluminum stopping target, where the conversion may occur. 
The system includes multiple components to precisely identify the conversion electrons and distinguish them from various backgrounds, including a straw tracker, 
an electromagnetic calorimeter, a Cosmic Ray Veto and a Stopping Target Monitor.

One of the most important Mu2e subsystems is the tracker, which must provide very good momentum resolution to distinguish the monochromatic 
conversion electron signal from the background. Energy loss in the tracker needs to be minimal; thus, the experiment will use a straw 
tube tracker, Ref. \cite{bobbb}. Its annular shape follows the helical trajectories of conversion electrons in the magnetic field. 
The tracker consists of panels containing arrays of straw tubes arranged like hap chords. It is required to measure the conversion electrons with a momentum 
resolution better than 180 keV/c. 

My activity at Fermilab has focused on the commissioning of the Mu2e DAQ system and the Vertical Slice Test (VST) of the tracker. 
The VST encompasses the entire testing chain from the straws to the readout, to processed data on disk. Our ultimate goal was to conduct 
a time calibration of the first assembled station of the tracker, aiming for a longitudinal position resolution better than 4 cm, and to determine the channel-to-channel delay and the drift velocity.

Before proceeding, validating data from the tracker panels is necessary. The tracker's readout system has undergone several 
tests to validate its performance, ensuring accurate and reliable data collection. This includes testing readout logic and firmware 
with simulated and real data to confirm functionality, and developing methods to monitor and validate data accuracy, including identifying and addressing cross-talk and noise issues.

Our objective is to gather data from cosmic rays to develop the time calibration of the tracker. First, it was important to assess the feasibility of 
performing this calibration with a vertical orientation of the first station. For this purpose, straight cosmic rays were simulated in one station to 
understand how the varying luminosity of the straws can impact timing calibration and hit segment reconstruction. This calibration is crucial for achieving 
precise momentum resolution, relying on accurate reconstruction of straw hit positions. Our goal is to determine the drift velocity in the straws and the time 
offset from channel to channel. This involves reconstructing straight cosmic segments within each station by assessing whether a straw has been hit and the 
relative orientations of the straws. From there, we can locate intersection points of straws on a 2D plane, assuming these to be the particle hit coordinates. 
This process is repeated for all pairs of straws, followed by fitting all the points together. The actual position of a hit within a straw can then be reconstructed, 
identifying biases by comparing the MC hits position with the reconstructed ones. Our target is to achieve a longitudinal resolution better than 4 cm, ensuring that every 
bias remains below this threshold. During data collection, each straw will undergo calibration by correlating the time difference between ends with the reconstructed position.
\end{abstract}



\tableofcontents

% Rimuovere se non si vuole la tabella delle figure
\listoffigures
\listoftables
\listofappendices
\chapter{Charged Lepton Flavor Violation}
\textit{This chapter offers a concise overview of the fundamental theoretical and experimental components essential to understand the objectives of the Mu2e experiment at Fermilab and the work conducted for this Thesis. The introduction to the Standard Model and certain extensions serves to justify the investigation of Charged Lepton Flavor Violation (CLFV), since it would be a clear signal of new physics beyond the current theories. Fundamental bibliography for this chapter can be found in Ref. \cite{Bernstein_2013} and \cite{clfv_signorelli}.}
\section{Theoretical Introduction}
\subsection{The Standard Model}
The Standard Model provides an excellent description of elementary particles and their interactions. It describes three out of four the fundamental forces known to this day: electromagnetism interaction, weak interaction and strong interaction. This theory is based on the gauge symmetry group $U(1)_Y \times SU(2)_L \times SU(3)_C$. The first two terms describe the electroweak interaction, $Y$ indicates the hypercharge and $L$ refers to the fact that this acts only on the left handed components of the fields. The last term describes the strong interaction and $C$ indicates the color charge.
The Standard Model contains 25 elementary fields, shown in Fig. \ref{fig:sm}.

\begin{figure}[!h]
\centering
\includegraphics[width =0.8\textwidth]{figures/pdf/Standard_Model_of_Elementary_Particles.pdf}
\caption{Elementary particles of the Standard Model.}
\label{fig:sm}
\end{figure}


\subsubsection{Bosons}\label{bosons}
A boson is a particle with zero or integer spin, which follows Bose-Einstein statistics.
There are twelve fundamental bosons, that are the mediators of interactions as the $\gamma$, $Z$, $W^{\pm}$ for the electroweak interaction and as the eight gluons for the strong interaction.
The Higgs is a complex scalar weak isospin doublet that is responsible for the mechanism through which fermions and
bosons acquire mass and also explains the origin of $U (1)_{EM}$ through the spontaneous
symmetry breaking of the $SU (2)_L \times SU (1)_Y$ in the electroweak sector. Also mesons, that are composed of a quark and antiquark pair, are bosons. Bosons can be either massive, as $Z$, Higgs and $W^{\pm}$, or massless, as the photon and gluons.

\subsubsection{Fermions}
A fermion is a particle characterized by a non-integer spin, i.e. $1/2$, $3/2$ and follows the Fermi-Dirac statistics. The Pauli exclusion principle must be respected. Fermions are divided in two categories, leptons and quarks, depending on the forces through which they interact.  The arrangement of fermions into three generations is dictated by various properties, including mass among other properties, with more massive particles assigned to higher generations, as depicted in Fig.\ref{fig:sm}. Particles of the second and third generations exhibit instability and decay into first-generation particles. Leptons do not interact through strong interaction because they are not color charged, so they interact only via weak and electromagnetic interaction. They are categorized into two groups based on the electric charge: $e$, $\mu$, $\tau$ are the charged leptons and $\nu_e$, $\nu_{\mu}$, $\nu_{\tau}$ are the neutral ones. These particles form doublets of flavour. Neutrinos, due to their neutral nature, interact only weakly so their detection is extremely challenging. The six quarks participate in all known interactions. The known quarks are, in ascending order of mass and generation: up ($u$) and down ($d$), strange ($s$) and charm ($c$), bottom ($b$) and top ($t$) and their antiparticles. They primarily interact with each other through the strong force by gluons exchange. Free quarks have never been observed due to confinement, since they carry color charge. Confinement of quarks is a fundamental aspect of the theory of quantum chromodynamics (QCD) which describes the strong nuclear force. Quarks combine to form color-neutral particles known as hadrons, classified into baryons and mesons. Baryons consist of an odd number of quarks, while mesons, as mentioned in Subsection \ref{bosons}, are composed of a quark and an antiquark. Since quarks have weak electric charge and isospin, they can interact with each other and other fermions through weak and electromagnetic interactions.

\subsection{History of flavor}
The concept of flavor, namely the presence of three duplicates for every family of elementary fermions, is a fundamental aspect in particle physics. This principle is implemented within the Standard Model by introducing three copies of the gauge representations of fermion fields. This view began to take shape in the late 1940s. The origin can be traced back to the experiment conducted by Conversi, Pancini and Piccioni in 1947. These experiment revealed that negative muons, called at that time $mesotrons$, did not undergo nuclear capture as expected. They decayed in electrons, similarly to positive muons, therefore they could not be Yukawa particles. In the same year, Powell and his group identified a two-step decay process ($\pi \rightarrow \mu \rightarrow e$), distinguishing the pion from the muon. Bruno Pontecorvo suggested that the muon could be a sort of $isomer$ of the electron, leading to the idea of a second generation of elementary fermions. Rochester and Butler discovered unusual events in cosmic rays pictures, later identified as $V-particles$ (later discovered that they originated from neutral kaons). This was the first hint of the existence of a second generation of quarks. In 1950 the search for decay $\mu \rightarrow e \gamma$ began. This decay was not found, leading to the principle of conservation of leptons. On the hadron side, the second generation of quarks was established in the mid-70s, involving the GIM mechanism and the discovery of the charm quark. Meanwhile, on the leptonic side, the upper limit on the branching ratio of $\mu^+ \rightarrow  e^+ \gamma$ was set in 1955. The discovery of parity violation in the late 1950s suggested the weak interaction is mediated by bosons. \textit{Feinberg started thinking that $\mu^+ \rightarrow  e^+ \gamma$ could occur at a level of $10^{-4}$ if the bosons existed, through a loop with neutrino and a boson}. This lead to the two-neutrino hypothesis, suggesting that the neutrino coupled to the muon differs from that coupled to the electron, thereby prohibiting $\mu^+ \rightarrow  e^+ \gamma$. The existence of two neutrinos was verified at Brookhaven National Laboratory, with the scattering of two neutrinos coming from $\pi$, that produced only muons. After the observation of CP violation in neutral kaon decay, a third generation of quarks was hypothesized. After the discoveries of the $\tau$ (1976), the $b$ quark (1977), $t$ quark (1995) and $\nu_{\tau}$ (2000), a complete picture was achieved and the concept of flavor was consolidated in the Standard Model.


\subsection{Overview of CLFV}
There are three different lepton flavors: the electron-lepton number $L_e$, the muon-lepton number $L_{\mu}$ and the tau-lepton number $L_{\tau}$. In Table \ref{tab:leptons}, the quantum numbers assigned to each lepton are displayed.
 \begin{center}  
\begin{table}[!h]
\centering
\renewcommand{\arraystretch}{1.5}
\begin{tabular}{c c c c}
\hline
Lepton & $L_e$ & $L_{\mu}$ & $L_{\tau}$\\
\hline
$e^-/e^+$ & $+1 \ /-1$ & 0 & 0 \\
$\nu_{e}/\bar{\nu}_{e}$ & $+1 \ /-1$ & 0 & 0 \\
$\mu^-/\mu^+$ & 0 & $+1 \ /-1$ & 0 \\
$\nu_{\mu}/\bar{\nu}_{\mu}$ & 0 & $+1 \ /-1$ & 0 \\
$\tau^-/\tau^+$ & 0 & 0 & $+1 \ /-1$\\
$\nu_{\tau}/\bar{\nu}_{\tau}$ & 0 & 0 & $+1 \ /-1$ \\
\hline
\end{tabular}
\caption{Lepton numbers assigned to neutrinos and charged leptons.}
\end{table}\label{tab:leptons}
\end{center}
In the Standard Model (SM) defined with massless left-handed neutrinos, Lepton Flavor (LF) is a conserved quantity \cite{universe8060299}. Experimental observations have demonstrated that, as they travel, neutrinos exhibit flavor oscillations, which implies that they must have non-zero masses and mixing angles. This phenomenon represents also a violation of the conservation of the lepton flavor. The Standard Model, while successful in many aspects, fails to explain phenomena like neutrino masses and the consequent flavor oscillations. Since neutrinos get their masses through renormalizable Yukawa interactions
with the Higgs, the predicted CLFV transitions are suppressed by sums over $(\Delta m^2_{i j}/M^2 _W)^2$, as calculated in \cite{MARCIANO1977303} and as shown in Section \ref{massiveneutrinos}, where $\Delta m^2_{ij}$ is mass-squared difference between the neutrino mass eigenstates $i$, $j$ and $M_W$ is the $W$ boson mass. The neutrino mass difference is very small ($\Delta m^2 _{i j} \leq 10^{-3}$ eV$^2$) with respect to the $W$ boson mass so the expected branching ratios reach unmeasurable values, below $10^{-50}$. Experimental studies of the lepton flavor violating process could open a window to new physics. Moreover, lepton flavor constitutes an accidental symmetry within the SM, not related to the gauge structure of the theory but coming
from its particle content, especially from the absence of RH neutrinos. Minor deviations from the Standard Model can easily give rise to extra occurrences of lepton flavor violation, leading to notable rates of CLFV.
There are various extensions of the Standard Model that could potentially be examined in the upcoming experimental searches for CLFV.
In Section \ref{leptonsector}, I will talk about lepton sector and how the lepton numbers are conserved. In Section \ref{massiveneutrinos}, a brief introduction of CLFV with massive neutrinos will be given.
In Section \ref{2higgs} and in Section \ref{susy}, I present a short discussion on Two Higgs Doublet Model and the CLFV in Super-symmetry respectively.
\subsection{Lepton sector in Standard Model}\label{leptonsector}
In the SM, only one Higgs field $\Phi$ exists. The fermions masses and the mixing term arise from the couplings of fermions with Higgs field. In the following, I will call the left-handed $i$-th quarks doublets and leptons doublets as $Q_{L,i}=(u_{L,i} \ d_{L,i})^T$ and $L_{L,i}=(\nu_{L,i} \ e_{L,i})^T$: $u_i$ will be the up-type quark, $d_i$ the down-type quark, $\nu_i$ the neutrino and $e_i$ the charged lepton. These are $SU(2)$ doublets, while $u_{R,j}$, $d_{R,j}$ and $e_{R,j}$ will be the right-handed up-type, down-type quarks and the right charged lepton of the $j$-generation respectively. There is no right-handed neutrino. The Yukawa coupling of fermions with the Higgs field $\mathscr{L}_Y$ is the sum of two terms: $\mathscr{L}_e$ that describes the leptonic component (Eq.\ref{leptoniccomponent}) and $\mathscr{L}_q$ that describes the quark one (Eq.\ref{quarkcomponent}).
\begin{equation}\label{leptoniccomponent}
    -\mathscr{L}_e=\left(Y_e\right)_{i j} \bar{L}_{L i} e_{R j} \Phi+ \text{ h.c. }
\end{equation}
\begin{equation}\label{quarkcomponent}
        -\mathscr{L}_q=\left(Y_u\right)_{i j} \bar{Q}_{L i} u_{R j} \widetilde{\Phi}+\left(Y_d\right)_{i j} \bar{Q}_{L i} d_{R j} \Phi+\text { h.c. }
\end{equation}

where the term $Y_f \ (f  =  u,d,e)$ describe the 3$\times$3 Yukawa complex matrices. $\widetilde{\Phi} \equiv i \tau_2 \Phi^*$ is the conjugate Higgs field. The mass terms of fermions, characterized by the $m_f \bar{f}_L f_R$ form, originate from the breaking of the $SU(2)_L \times U(1)$ symmetry caused by the vacuum expectation value of the Higgs field, as in Eq.\ref{higgs}.
\begin{equation}\label{higgs}
\langle\Phi\rangle=\frac{1}{\sqrt{2}}\left(\begin{array}{l}
0 \\
v
\end{array}\right) \qquad v \simeq 246 \mathrm{GeV}
\end{equation}
The fermion mass is given by:
\begin{equation}
\left(m_f\right)_{i j}=\frac{v}{\sqrt{2}}\left(Y_f\right)_{i j} \qquad f=u, d, e
\end{equation}
Since a right-handed neutrino does not appear in the Lagrangian, neutrinos have no mass, as it is formulated in the SM. The Yukawa matrices can be diagonalized through unitary rotations of the fields, as it follows:
\begin{equation}
Y_f=V_f \hat{Y}_f U_f^{\dagger} \qquad f=u, d, e
\end{equation}
where $ \hat{Y}_f $ is the diagonal Yukawa matrix. It is possible to label fermions in the rotated basis as $f^{\prime}$, so $f_L=V_f f^{\prime}_L$ and $f_R=V_f f^{\prime}_R $. Since $V_f$ and $U_f$ are unitary, the rotation will not affect the neutral interactions term and the kinetic terms, as we can see in $\bar{f}_L \gamma^\mu f_L=\bar{f}_L \gamma^\mu\left(V_f^{\dagger} V_f\right) f_L=\bar{f}_L^{\prime} \gamma^\mu f_L^{\prime}$. The coupling between fermion and Higgs boson will be:
\begin{equation}
-\mathscr{L}_{h \bar{f} f}=\frac{(\hat{m}_f)_{i j}}{v} \bar{f}_{L i}^{\prime}f^{\prime}_{R j} h+\text{h.c.} \qquad f=u, d, e
\end{equation}
where $\hat{m}_f$ denotes the diagonalized mass matrix: it is clear that there are no flavor-violating terms.
In quark sector flavor-violation arises from the rotations in Eq.\ref{quarkviolation} in the charged-current interactions with the $W$ bosons:
\begin{equation}\label{quarkviolation}
\begin{array}{c}
      { \displaystyle 
\mathscr{L}_{C C}  =\frac{g}{\sqrt{2}}\left(\bar{u}_L \gamma^\mu d_L+\bar{\nu}_L \gamma^\mu e_L\right) W_\mu^{+}+\text {h.c.} }\\
 {\displaystyle=\frac{g}{\sqrt{2}}\left(\bar{u}^{\prime}_L \gamma^\mu\left(V_u^{\dagger} V_d\right) d_L^{\prime}+\bar{\nu}^{\prime}_L \gamma^\mu\left(V_\nu^{\dagger} V_e\right) e_L^{\prime}\right) W_\mu^{+}+\text {h.c.}}
\end{array}
\end{equation}

The violation comes from the fact that $V_u \neq V_d$. The mixing is controlled by the Cabibbo-Kobayashi-Maskawa matrix $V_{CKM}\equiv V^{\dagger}_u V_d $ \cite{PhysRevLett.10.531}. Meanwhile, as the lepton sector contains massless neutrinos, $V_{\nu}$ can be arbitrarily chosen as $V_{\nu}=V_e$. From the preceding discussions, it becomes evident that in the SM with massless neutrinos, there is no occurrence of LFV in any form. The Lagrangian $\mathscr{L}_Y$ is invariant under three indipendent global $U(1)$ rotations, resulting in the conservation of three lepton family numbers: $L_e$, $L_\mu$ and $L_\tau$. Furthermore, if the Yukawa coupling Lagrangian includes supplementary terms involving the lepton fields, flavor violation can occur in the lepton sector. Examples of such additions could be a neutrino mass term or a second Higgs doublet.

\subsection{CLFV in the Standard Model with massive neutrinos}\label{massiveneutrinos}
The first evidence against the hypotesis of massless neutrinos emerged with the solar neutrino problem. In the 1960s, the solar neutrino detection experiment at Homestake revealed that the observed number of solar neutrinos, generated by fusion in the Sun, was significantly lower than the anticipated value based on the standard solar model, given that the detector was only sensitive to $\nu_e$ \cite{PhysRevLett.20.1205}. Consistent results were replicated in subsequent experiments employing radiochemical and Cherenkov detectors, discovering neutrino oscillations. These oscillation firmly established non-zero neutrino masses. The lepton flavor-violating neutrino oscillations showed that the global $U(1)$ symmetries associated with the lepton family numbers are not fundamental symmetries. A correction to standard model is needed to include neutrino mass terms. This is possible adding a right-handed neutrino singlet $\nu_R$ or some non-renormalizable operators.
\\
In the first case, an additional term $\Delta \mathscr{L}_D$ to the Yukawa coupling should be introduced:
\begin{equation}
-\Delta \mathscr{L}_D=\left(Y_\nu\right)_{i j} \bar{L}_{L i} \nu_{Rj} \widetilde{\Phi}+\text { h.c. }
\end{equation}
Similarly to other fermions, a Dirac mass term $m_{\nu} \bar{\nu}_L \nu_R$ is generated through simmetry breaking:
\begin{equation}
\left(m_\nu^D\right)_{i j}=\frac{v}{\sqrt{2}}\left(Y_\nu\right)_{i j}
\end{equation}
In this case, the small neutrino masses can be explained only if a very small term $Y_\nu$ is considered ($\leq 10^{-12}$) \cite{clfv_signorelli}. This brings us considering the second hypotesis: adding non-renormalizable operators can introduce Majorana masses for left-handed neutrinos alone. The corresponding $\Delta \mathscr{L}_M$ can be written as:
\begin{equation}
-\Delta \mathscr{L}_M=\frac{1}{2}\left(m_\nu^M\right)_{i j} \overline{\nu_{L i}^C} \nu_{L j}+\text{ h.c.}
\end{equation}
where $\overline{\nu_{L }^C} $ is the charge-conjugated fields. This term violates lepton number and requires an operator of dimension 5 \cite{wein} to be consistent with SM symmetries. A minimal Lagrangian is given by:
\begin{equation}
-\Delta \mathscr{L}_{M \text { eff}}=\frac{\mathcal{C}_{i j}}{\Lambda}\left(\overline{L_{L i}^C} \tau_2 \Phi\right)\left(\Phi^T \tau_2 L_{L j}\right)+\text { h.c.}
\end{equation}
where the $\Lambda$ term represents a mass scale characteristic of extra degrees of freedom and the $C_{ij}$ is an antisymmetric charge conjugation matrix. The corresponding Majorana mass term is:
\begin{equation}
\left(m_\nu^M\right)_{i j}=\frac{\mathcal{C}_{i j} v^2}{\Lambda}
\end{equation}
In this case, the small neutrino masses can be explained only if $\Lambda > > v$: this seems to appear more natural than the previous case. No matter how the extra neutrino mass factor is expressed precisely, a physical basis diagonalizing the mass matrix is determined, resulting in $V_{\nu} \neq V_e$ in Equation \ref{quarkviolation}. Lepton mixing is described by $U_{PMNS} \equiv V_{\nu}^{\dagger} V_e$, which is similar to the CKM matrix. The Pontecorvo-Maki-Nakagawa-Sakata (PMNS) matrix is the term that is typically used to refer to it. As it is in the basis diagonalizing charged lepton masses and it diagonalizes the neutrino mass matrix, $U_{PMNS}$ also explains the mixing between neutrino flavor eigenstates $\nu_{\alpha}$ and mass eigenstates $\nu_i$:
\begin{equation}
\nu_\alpha=\sum_{i=1,2,3}\left(U_{\mathrm{PMNS}}\right)_{l i} \ \nu_i \qquad l=e, \mu, \tau
\end{equation}
In addition to the neutral LFV observed in neutrino oscillations, the mixing outlined by $U_{PMNS}$ can in principle give rise to processes known as Charged Lepton Flavor Violations, i.e. LFV that involves charged leptons. The new Feynman diagrams are loops involving neutrinos and W bosons, as $\mu \rightarrow e \gamma$ in Fig.\ref{fig:mutoegamma} and $\mu N \rightarrow e N$ in Fig.\ref{fig:mutoeN}. 



\begin{figure}[!h]
     \begin{subfigure}[b]{0.4\linewidth}
         \centering
         \includegraphics[scale = 0.2]{figures/png/Screenshot_20240217_171058.png}
         \subcaption{$\mu \rightarrow e \gamma$ process \cite{universe8060299}.}
         \label{fig:mutoegamma}
     \end{subfigure}
     \begin{subfigure}[b]{0.7\linewidth}
         \centering
         \includegraphics[scale = 0.5]{figures/jpg/1_erkKoywyuFzJmMv4PKpc9Q.jpg}
         \subcaption{$\mu N \rightarrow e N$ process.}
         \label{fig:mutoeN}
     \end{subfigure}
     \caption{Some of CLFV processes.}
        \label{fig:three graphs2}
\end{figure}
In the SM, each of these mechanisms is significantly inhibited. Using the example of $\mu \rightarrow  e \gamma $, the branching ratio (BR) of this process may be computed as follows:

\begin{equation}
\begin{aligned}
B R(\mu \rightarrow e \gamma) & =\frac{3 \alpha}{32 \pi}\left|\sum_{i=2,3} U_{\mu i}^* U_{e i} \frac{\Delta m_{1 i}^2}{M_W^2}\right|^2 \\
& =\frac{3 \alpha}{32 \pi}\left(\frac{1}{4}\right) \sin ^2 2 \theta_{13} \sin ^2 \theta_{23}\left|\frac{\Delta m_{13}^2}{M_W^2}\right|^2,
\end{aligned}
\end{equation}

where $\alpha$ is the fine structure constant, $U_{\mu i}$ and $U_{ei}$ are corresponding elements in the PMNS matrix, $\Delta m_{1i}^2$ is the neutrino squared mass differences, $M_W$ is the $W$ boson mass and $\theta_{13}$ and $\theta_{23}$ are rotating angles in PMNS matrix parametrization. The expression yields $B R(\mu \rightarrow e \gamma) \sim \mathcal{O}(10^{-54})$. The big discrepancy in mass between neutrinos and the $W$ boson results in an extraordinarily small value for $|\Delta m_{13}^2/M_W|$. Equivalent suppression mechanisms are evident in other CLFV processes. The rates predicted by the Standard Model are extremely small, making them impractical for detection in any experiment. On the other hand, numerous Beyond the Standard Model (BSM) theories incorporate mechanisms that substantially amplify CLFV rates, a topic to be addressed in the subsequent section. The small value of SM CLFV rates implies that the detection of any CLFV processes in experiments would unequivocally indicate the presence of physics beyond the SM.
\subsection{Beyond the Standard Model}
Numerous Beyond the Standard Model (BSM) theories propose mechanisms that could contribute to CLFV processes, potentially yielding detectable rates in experiments. Here, we highlight a selection of BSM theories known for their CLFV contributions. It is important to note that this list is not comprehensive; for further studies, additional reviews are available in \cite{clfv_signorelli} and \cite{universe8060299}.
\subsubsection{CLFV in Supersymmetry}\label{susy}
Supersymmetry (SUSY) is a theoretical framework that has oriented experiments in the CLFV reasearch for many years. On one hand, models with SUSY broken at energies close to electro-weak scale have given solution to the hierarchy problem, i.e. how to maintain the Higgs mass significantly smaller than the Planck scale ($\sim$10$^{19}$ GeV). On the other hand, the suppression of CLFV processes is due to the wide separation of the neutrinos and $W$ masses, which can be mitigated by introducing SUSY partners of neutrinos and $W$ bosons. This suggests that CLFV processes should have been observable earlier, unless SUSY breaking occurs at or near the electroweak scale ($\sim 10^2$ GeV) \cite{clfv_signorelli}. In this framework, each elementary particle has a superpartner,  with the same quantum numbers except for spin: a boson is the superpartner of a fermion and vice versa. A superpartner of a lepton is called $slepton$. If there is no common eigenstate base between lepton and $slepton$'s mass matrices then a physical $slepton$ will be a superposition of flavors. In this case a loop diagram can lead to CLFV, as shown in Fig.\ref{fig:susy}. Despite the similar topology to that of the SM contribution (Fig.\ref{fig:mutoegamma}), the typical SUSY mass is expected to be much higher than that of the neutrinos. Predictions for the branching ratio of this process vary among different SUSY models, contingent upon specific mechanisms and particle masses. These rates can undergo significant enhancement. For example, in an $SU(5)$ SUSY grand unified theory, the computed branching ratio could reach $\mathcal{O}(10^{-14})$ for a slepton mass on the order of $\mathcal{O}(10^{-14}$ GeV/c$^2)$, a value measurable for upcoming experiments.

\begin{figure}[!h]
\centering
\includegraphics[width =0.4\textwidth]{figures/png/Screenshot_20240218_105920.png}
\caption{SUSY contribution to $l_i \rightarrow l_j\gamma$, through $sleptons$ mass mixing \cite{universe8060299}.}
\label{fig:susy}
\end{figure}


\subsubsection{Two Higgs Doublet Model}\label{2higgs}
Although the Standard Model (SM) incorporates only one Higgs boson, there are no constraints against the presence of additional Higgs fields. One straightforward example of a comprehensive theory featuring multiple Higgs fields is the type III two Higgs doublet model (2HDM), where two Higgs bosons exist, each interacting with fermions and possessing a vacuum expectation value \cite{Harnik_2013}. Generally, the Lagrangian incorporating extra Higgs fields post-electroweak symmetry breaking can be written as:
\begin{equation}
-\mathscr{L}=m_i \bar{f}_{L i} f_{R i}+\left(Y^\alpha\right)_{i j} \bar{f}_{L i} f_R{ }_j h^\alpha+\text { h.c. }+\ldots
\end{equation}
Non renormalizable terms of higher dimensions are omitted. $(Y^\alpha)_{i j}$ represents couplings to a single scalar field and the contributions from different Higgses are summed over. The non-zero-off-diagonal terms in $(Y^\alpha)_{i j}$ give rise to flavor violating Yukawa couplings. From accelerator and precision experiments, constraints on the off-diagonal coupling of the 125 GeV Higgs boson can be obtained.
\subsubsection{Leptoquark Models}
Leptoquarks (LQs) are theoretical particles initially proposed within the Pati-Salam model \cite{PhysRevD.10.275}. Each LQ is linked to both baryon number ($B$) and lepton number ($L$). In various LQ models the quark and lepton sectors are unified. This unification allows for direct coupling between quarks and leptons via the exchange of LQs. Consequently, specific CLFV processes like $K_L^0 \rightarrow e \mu$ and $\mu N \rightarrow e N$ are mediated by LQs. Constraints on LQ models arise from both collider experiments and rare decay searches. Direct searches at ATLAS and CMS have excluded scalar LQs of the first and second generations with masses below $\sim$1 TeV. Indirect searches provide constraints on the mass-coupling plane, where regions with higher couplings and lower LQ masses correspond to higher branching ratios. Additionally, LQ models must satisfy constraints related to proton stability, as some models involve LQs that could mediate proton decay. To mitigate this, the corresponding LQs must either have extremely high masses or their related couplings must be exceedingly small \cite{DORSNER20161}.
\subsubsection{Additional Neutral Gauge Boson}
Grand unified theories (GUTs) are constructed based on extended gauge groups in the pursuit of a more fundamental model. At lower energies, these extended gauge groups are believed to break down to the direct product of the Standard Model (SM) gauge group $SU(3) \times SU(2) \times U(1)$ along with an additional $U(1)$ factor. The neutral gauge boson associated with this $U(1)$ group can mix with the original SM neutral gauge boson, resulting in two mass eigenstates, namely $Z$ and $Z'$. Additionally, extended gauge theories require the introduction of additional fermion fields to cancel anomaly-free currents beyond those of $SU(5)$. These $new$ fermions can mix with the known SM fermions possessing the same electric and color charges, consequently affecting their couplings with gauge bosons. The appearance of off-diagonal terms in neutral current couplings to fermions can lead to flavor-changing couplings to $Z$ and $Z'$. Certain CLFV processes, such as $\mu \rightarrow eee$ and $\mu-e$ conversion, receive tree-level contributions through intermediate $Z$ and $Z'$ bosons. Further insights into the phenomenology of the $Z'$ boson can be found in \cite{Leike_1999}. The search for the existence of $Z'$ bosons is conducted through channels like $Z' \rightarrow \bar{f}f$ at hadron colliders. Mass lower limits of $Z'$ from various specific models are listed in \cite{zyla}, primarily falling within the low TeV range. Particularly, mass lower limits reported in CLFV final states $e\mu$, $e\tau$ and $\mu\tau$ range between 3.5 TeV and 4.5 TeV. Upper limits of $Z \rightarrow l_1 l_2$ couplings to the normal Z boson are also provided in Table \ref{tab:upperlimits}.
%Quello che segue è un esempio di codice. E' possibile modificare il linguaggio per il synyax highlight, aggiungere parole chiave... E' tutto disponibile nella guida del pacchetto \texttt{listings}.

%\lstinputlisting[language=C++]{listings/png/code1.cpp} 
\section{Experiments looking for CLFV}
Charged Lepton Flavor Violation has not been observed yet, despite ongoing efforts to detect such violations across various channels in both dedicated and general-purpose experiments. Some of these efforts are documented in Table \ref{tab:upperlimits}, which presents their respective experimental upper limits.
\begin{center}  
\begin{table}[!h]
\centering
\renewcommand{\arraystretch}{1.5}
\begin{tabular}{c c c c c c}
\hline
Reaction & Present limit & C.L. & Experiment &  Year & Ref.\\
\hline
$\mu^+\ \rightarrow \ e^+ \ \gamma$& $7.5 \times 10^{-13}$ & 90\% & MEG II & 2024 & \cite{megiicollaboration2024search}\\
$\mu^+ \ \rightarrow \ e^+ \ e^+ \ e^-$ & $1.0 \times 10^{-12}$ & 90\% & SINDRUM & 1988 & \cite{SINDRUM:1987nra} \\
$\mu^- \ \text{Ti}\ \rightarrow \ e^- \ \text{Ti}$ &  $6.1 \times 10^{-13}$ & 90\% & SINDRUM II & 1998 & \cite{titanium}\\
$\mu^- \ \text{Au}\ \rightarrow \ e^- \ \text{Au}$ & $7.0 \times 10^{-13}$ & 90\% & SINDRUM II & 2006 & \cite{SINDRUMII:2006dvw} \\
$\mu^+ \ e^- \ \rightarrow \ \mu^- \ e^+$ & $8.3 \times 10^{-11}$ & 90\% & SINDRUM & 1999 & \cite{Willmann:1998gd}\\
$\tau \ \rightarrow \ e \ \gamma$ & $3.3 \times 10^{-8}$ & 90\% & BaBar & 2010 & \cite{Aubert_2010}\\
$\tau \ \rightarrow \ \mu \ \gamma$ & $4.4 \times 10^{-8}$ & 90\% & BaBar & 2010 & \cite{Aubert_2010}\\
$\tau \ \rightarrow \ e \ e \  e$ & $2.7 \times 10^{-8}$ & 90\% & Belle & 2010 & \cite{Hayasaka_2010}\\
$\tau \ \rightarrow \ \mu \ \mu  \ \mu$ & $2.1 \times 10^{-8}$ & 90\% & Belle & 2010 & \cite{Hayasaka_2010} \\
$B^0 \ \rightarrow \ \mu \ e$ & $2.8 \times 10^{-9}$ & 90\% & LHCb & 2013 & \cite{PhysRevLett.111.141801}\\
$B^0 \ \rightarrow \ \tau \ e$ & $2.8 \times 10^{-5}$ & 90\% & BaBar & 2008 & \cite{PhysRevD.77.091104}\\
$B^0 \ \rightarrow \ \tau \ \mu$ & $2.2 \times 10^{-5}$ & 90\% & BaBar & 2008 & \cite{PhysRevD.77.091104}\\
$K_L^0 \ \rightarrow \ \mu \ e$ & $4.7 \times 10^{-12}$& 90\% & BNL E871 & 1998 & \cite{BNL:1998apv}\\
$K^+\ \rightarrow \ \pi^+ \ \mu^+ \ e^-$ & $2.1 \times 10^{-10} $ & 90\% & BNL E865 & 2005 & \cite{PhysRevD.72.012005}\\
$K_L^0 \ \rightarrow \ \pi^0 \ \mu^+ \ e^-$ & $ 4.4 \times 10^{-10}$ & 90\% & KTeV & 2008 & \cite{KTeV:2007cvy}\\
$Z^0 \ \rightarrow \ \mu \ e$ & $1.7 \times 10^{-6}$ & 95\% &  LHC ATLAS & 2014 & \cite{Aad_2014} \\
$Z^0 \ \rightarrow \ \tau \ e$ & $1.7 \times 10^{-6}$ & 95\% &  LEP OPAL & 1995 & \cite{akers}\\
$Z^0 \ \rightarrow \ \tau \ \mu$ & $9.8 \times 10^{-6}$ & 95\% &  LEP DELPHI & 1997 & \cite{abreu}\\
$\pi^0 \ \rightarrow \ \mu \ e$ & $8.6 \times 10^{-9}$ & 90\% & KTeV & 2008 & \cite{KTeV:2007cvy}\\
$\Upsilon (1s) \ \rightarrow \ \mu \ \tau $ & $6.0 \times 10^{-6}$ & 95\% & CLEO & 2008 & \cite{Love_2008}\\
\hline
\end{tabular}
\caption{Experimental upper limits for a variety of CLFV processes, Ref. \cite{clfv_signorelli}.}
\end{table}\label{tab:upperlimits}
\end{center}


\subsection{$\mu$ Channels}
Currently, the most promising channel is the one that includes muon processes. When a proton beam interacts with a target, pions and kaons are produced, that subsequently decay in muons. Muon lifetime is long enough to form a muon beam and we are able to reach intensities of 10$^8 \div 10^{11} \  \mu$/s. There are three primary CLFV channels involving muons, with distinct sensitivities to effective lagrangian terms: $\mu^+ \rightarrow e^+ \gamma$, $\mu^- N \rightarrow e^- N$ and $\mu^+ \rightarrow e^+ e^+ e^-$. The following paragraphs will discuss the experimental challenges and future perspectives for each of these channels. As seen in Table \ref{tab:upperlimits}, these channels have the lowest branching ratio limits. Muons have small mass, that results in a limited number of decay modes. Figure \ref{fig:muchannel} shows how the branching ratio limitations of muon uncommon decays have rapidly improved over the last several decades. The next-generation experiments aim to improve by many orders of magnitude. Muon rare decay studies can also provide theory differentiation power combining results of the three channels. All CLFV extensions to SM can be described by the following Lagrangian, \cite{doi:10.1146/annurev-nucl-100809-131949}:
\begin{equation}\label{LCF}
\mathscr{L}_{C L F V}=\frac{m_\mu}{(1+\kappa) \Lambda^2} \bar{\mu}_R \sigma_{\mu \nu} e_L F^{\mu \nu}+\text{h.c.}+\frac{\kappa}{(1+\kappa) \Lambda^2} \bar{\mu}_L \gamma_\mu e_L\left(\sum_{q=u, d} \bar{q}_L \gamma^\mu \bar{q}_L\right)+\text{h.c.}
\end{equation}
$m_\mu$ is the muon mass and  $F^{\mu \nu}$ is the electromagnetic field tensor. This toy Lagrangian includes two parameters.
$\Gamma$ is the effective energy scale of the new physics and $\kappa$
is the relative strengths of the two operators. The first term in the Lagrangian is a magnetic-moment-type operator and describes all three pro-
cesses mentioned above, it is generated by any loop with some new particle that can be either virtual and real.
The second one corresponds to a four-fermion operator, which mediates $\mu N \rightarrow eN$ at tree level and the other two processes at one-loop level.
The Mu2e experiment can probe an effective mass scale up to $\mathcal{O}$(10$^4$ TeV) with its designed sensitivity assuming $\kappa$ $\gg$ 1.
On the other hand, $\mu \rightarrow e\gamma$ experiments are more sensitive when $\kappa$ $\ll$ 1; the dominant magnetic
moment type term determines the other two processes have lower rates in such a case. In order to learn more about the new
physics, one needs to combine information involving the rates of a different CLFV processes, Ref. \cite{osti_1042577}.
The corresponding parameter space (the $\Gamma-\kappa$ plane) is shown in Figure \ref{fig:muchannelbr}, Ref. \cite{doi:10.1146/annurev-nucl-100809-131949}.
\begin{figure}[!h]
\centering
\includegraphics[width =0.8\textwidth]{figures/png/Screenshot_20240307_161549.png}
\caption{History and outlook of branching ratio limits in muon rare decay modes, Ref. \cite{MARCIANO1977303}.}
\label{fig:muchannel}
\end{figure}
\begin{figure}[!h]
\centering
\includegraphics[width =0.8\textwidth]{figures/png/Screenshot_20240313_120457.png}
\caption{Sensitivity of $\mu \rightarrow e\gamma$ and $\mu N \rightarrow eN$ experiments to the new physics
scale $\Gamma$ as a function of $\kappa$ as defined in Equation \ref{LCF}, Ref. \cite{CGroup:2022tli}. The blue region is the New
Physics phase space excluded by SINDRUM-II, Ref. \cite{SINDRUMII:2006dvw}. The red region represents the
limit set by MEG, Ref. \cite{megi}, while the dashed red line represents the region that is
excluded by MEG-II, Ref. \cite{megiicollaboration2024search}. The solid (dashed) blu line is the
expected limit that would be set by Mu2e, Ref. \cite{universe9010054}.}
\label{fig:muchannelbr}
\end{figure}
\subsubsection{$\mu^+ \rightarrow e^+ \gamma$}
\paragraph{The MEG Experiment}
\cite{megi}
\begin{figure}[!h]
\centering
\includegraphics[width =0.4\textwidth]{figures/png/Screenshot_20240307_150038.png}
\caption{.}
\label{fig:meg}
\end{figure}
\paragraph{MEG II}
\cite{megiicollaboration2024operation}
\cite{megiicollaboration2024search}
\begin{figure}[!h]
\centering
\includegraphics[width =0.4\textwidth]{figures/png/Screenshot_20240307_140235.png}
\caption{.}
\label{fig:meg2}
\end{figure}
\begin{figure}[!h]
\centering
\includegraphics[width =0.4\textwidth]{figures/png/Screenshot_20240307_140116.png}
\caption{.}
\label{fig:meg22}
\end{figure}
\subsubsection{$\mu^+ \rightarrow e^+ e^-  e^+ $}
\paragraph{SINDRUM I}
\cite{sindrumi}
\begin{figure}[!h]
\centering
\includegraphics[width =0.4\textwidth]{figures/png/The-SINDRUM-I-detector-in-the-horizontal-operating-orientation.png}
\caption{.}
\label{fig:sindrumi}
\end{figure}
\paragraph{The Mu3e Experiment}

\begin{figure}[!h]
\centering
\includegraphics[width =0.4\textwidth]{figures/png/Screenshot_20240307_161651.png}
\caption{Longitudinal cross section of the detector. Positrons tracks in red, electron
track in blue \cite{hesketh2022mu3e}}
\label{fig:mu3e}
\end{figure}
\subsubsection{$\mu^- N \rightarrow e^- N $}
\paragraph{SINDRUM II}
\cite{SINDRUMII:2006dvw}
\begin{figure}[!h]
\centering
\includegraphics[width =0.4\textwidth]{figures/png/Screenshot_20240307_163120.png}
\caption{.}
\label{fig:sindrumii}
\end{figure}
\paragraph{Mu2e}
\paragraph{COMET}
\cite{Abramishvili_2020}
\begin{figure}[!h]
\centering
\includegraphics[width =0.4\textwidth]{figures/png/Screenshot_20240307_152133.png}
\caption{Schematic layout of COMET (Phase-II) and COMET Phase-I (not in scale).}
\label{fig:comet}
\end{figure}
\paragraph{Mu2e II}
\cite{dukes}
\subsection{$\tau$ Channels}
The tau lepton is, in principle, a very promising source of CLFV decays. Thanks to the
large tau mass (m $\tau$ $\sim$ 1.777 GeV), many CLFV channels can be investigated: $\tau^\pm \rightarrow \mu^\pm \gamma$ ,
$\tau^\pm\rightarrow e^\pm\gamma$ , $\tau \rightarrow 3l$, $\tau\rightarrow l \ h$, with $h$ a light hadron and $l$ an electron or a muon. Table \ref{tab:upperlimits}
lists the current best limits on the $\tau$ CLFV searches, and Figure \ref{fig:tauchannel} shows the individual results from the
BaBar, Ref. \cite{PhysRevD.77.091104}, Belle [180] and the LHCb [181] experiments, together with their combination.
From the experimental point of view, however, a difficulty immediately arises: the
tau is an unstable particle, with a very short lifetime ($\tau$ = 2.91 $\times$ 10$^{-13}$ s). As a result, $\tau$
beams cannot be realized, and large tau samples must be obtained in intense electron or
proton accelerators, operating in an energy range where the tau production cross section is
large. At $e^+ e^-$ and $pp$ collider machines, the majority of the tau particles are not produced
at rest, which means that, unlike the muon searches discussed before, here we need to deal
with decays-in-flight. Thanks to the boost, the decay products could get energy values up
to several GeV, which experimentally poses the challenge to deliver wide-range calibrations for the detectors (from a few hundreds of MeV to several GeV).
For all these searches, events contain a pair of taus in which one tau undergoes SM decay (tag side),
while the signal side is selected on the basis of the appropriate topology of each individual channel.
The tagging side accepts the leptonic ($\tau$ $\rightarrow$ $l \nu\bar{\nu}$) and 1-prong hadronic decays, while on the signal side,
CLFV candidates are selected on the basis of the appropriate topology of each individual channel.
The following paragraphs discuss the current best limits for some of these experimental
searches from experiments at $B$-factories and $pp$ colliders, Ref. \cite{universe8060299}.
\begin{figure}[!h]
\centering
\includegraphics[width =0.4\textwidth]{figures/png/Screenshot_20240313_133439.png}
\caption{\cite{Amhis_2021}.}
\label{fig:tauchannel}
\end{figure}
\subsubsection{$ \tau \rightarrow l \gamma$}
\iffalse
The $\tau \rightarrow l\gamma$ decay, where $l$ is a light lepton $(e, \mu)$, has been one of the most popular
CLFV tau channels. The signal is characterized by a $l^\pm - \gamma$ pair with an invariant mass
and total energy in the center-of-mass (CM) frame (ECM) close to mτ = 1.777 GeV and
√
s/2, respectively. The dominant irreducible background comes from τ-pair events containing hard photon radiation and one of the τ leptons decaying to a charged lepton. The
remaining backgrounds arise from the relevant radiative processes, e
+e
− → e
+e
−γ and
e
+e
− → µ
+µ
−γ and from hadronic τ decays where a pion is misidentified as an electron
or muon. For this decay channel, the current best limits comes from the BaBar and the
Belle collaborations. BaBar collected (963 ± 7) × 106 τ decays near the Υ(4S), Υ(3S) and
Υ(2S) resonances. In the BaBar detector [179], charged particles are reconstructed as tracks
with a 5-layer silicon vertex tracker and a 40-layer drift chamber inside a 1.5 T solenoidal
magnet. A CsI(Tl) electromagnetic calorimeter is used to identify electrons and photons. A
ring-imaging Cherenkov detector is used to identify charged pions and kaons. The flux
return of the solenoid, instrumented with resistive plate chambers and limited streamer
tubes, is used to identify muons. Signal decays are identified by two kinematic variables:
the energy difference ∆E = ECM −
√
s/2 and the beam energy constrained τ mass obtained
from a kinematic fit after requiring the CM τ energy to be √
s/2 and after assigning the
origin of the γ candidate to the point of closest approach of the signal lepton track to the
e
+e
− collision axis (mBC). Figure 21 shows the distributions of the events for the two decay
channels in mBC vs. ∆E. The red dots are experimental points, the black ellipses are the 2σ
signal contours and the yellow and green regions contain 90\% and 50\% of MC signal events. The searches yield no evidence of signals, and the experiment set upper limits on the
branching fractions of B(τ
± → e
±γ) < 3.3 × 10−8 and B(τ
± → µ
±γ) < 4.4 × 10−8 at 90%
confidence level [140].
The Belle experiment [180] reported comparable limits using a data analysis based
on 988 fb−1
and a strategy similar to that of BaBar. Kinematical selections on missing
momentum and opening angle between particles are used to clean the sample. Figure 22
shows the two-dimensional distribution of∆E/
√
s vs. mBC. The signal events have mBC ∼
mτ and ∆E/
√
s ∼ 0. The most dominant background in the τ
± → µ
±γ (τ± → e
±γ )
search arises from τ
+τ
− events decaying to τ
± → µ
±νµντ (τ
± → e
±νeντ) with a photon
coming from initial-state radiation or beam background. The µ
+µ
−γ and e
+e
−γ events
are subdominant, with their contributions falling below 5\%. Other backgrounds such as
two-photon and e
+e
− → qq¯ (q = u, d, s, c) are negligible in the signal region
No significant excess over background predictions from the Standard Model is observed, and the 90% C.L. upper limits on the branching fractions are set at B(τ
± → µ
±γ) ≤
4.2 × 10−8 and B(τ
± → e
±γ) ≤ 5.6 × 10−8
[183]. With the full dataset expected for the
Belle II experiment [184] (the upgrade of Belle), 50 ab−1
, the upper limit for the branching
fraction of LFV decays τ will be reduced by two orders of magnitude
\fi
\subsubsection{$ \tau \rightarrow 3l $}
\iffalse
The signature for τ → 3l (l = e, µ) is a set of three charged particles, each identified
as either an e or a µ, with an invariant mass and energy equal to that of the parent τ lepton.
In the BaBar [185] and Belle [141] analyses, all the six different combinations were
explored. Events are preselected requiring four reconstructed tracks and zero net charge,
selecting only tracks pointing toward a common region consistent with τ
+τ
− production
and decay. The polar angles of all four tracks in the laboratory frame are required to
be within the calorimeter acceptance range, to ensure good particle identification. The
search strategy consists of forming all possible triplets of charged leptons with the required
total charge and of looking at the distribution of events in the (mBC , ∆E) plane (mBC and
∆E are defined as in the previous section). The backgrounds contaminating the sample
can be divided in three broad categories: low multiplicity e
+e
− → qq¯ (q = u, d, s, c)
events, QED events (Bhabha or µ
+µ
− depending on the specific channel) and SM τ
+τ
−
events. These background classes have distinctive distributions in the (mBC, ∆E) plane.
The e
+e
− → qq¯ (q = u, d, s, c) events tend to populate the plane uniformly, while
QED backgrounds fall in a narrow band at positive values of ∆E, and τ
+τ
− backgrounds
are restricted to negative values of both ∆E and mBC due to the presence of at least one
undetected neutrino. Figure 20 shows the resulting limit for all the combinations to be at
the level of a few 10−8
for both collaborations.
Even if the results are not yet competitive to those from B-factories, it is interesting
to note that experiments at the LHC have also been looking for the τ → 3µ decay. The
ATLAS experiment [186] performed a search for the neutrinoless decay τ
− → µ
−µ
+µ
−
using a sample of W− → τ
−ν¯τ decays from a dataset corresponding to an integrated
luminosity of 20.3 fb−1
collected in 2012 at a center-of-mass energy of 8 TeV. The LHCb
experiment [187] performed the same search using a sample of tau from b and c-hadron
decays from a dataset corresponding to an integrated luminosity of 3.0 fb−1
collected by
the LHCb detector in 2011 and 2012 at center-of-mass energies of 7 and 8 TeV, respectively.
The CMS experiment [188] recently delivered the results for the same search using a sample
of τ leptons produced in both W boson and heavy-flavour hadron decays from a dataset
corresponding to an integrated luminosity of 33.2 fb−1
recorded by the CMS experiment in
2016 [188]. ATLAS, CMS and LHCb reported a 90\% C.L. upper limit on the branching ratio
of 3.76 × 10−7
, 8.0 × 10−8 and 4.6 × 10−8
, respectively. The Belle-II collaboration studied
prospects for the expected sensitivity of this search. This channel has a purely leptonic final
state, thus it is expected to be free of background. This allows to scale the experimental
uncertainties linearly with the luminosity, which means that at least an improvement of a
factor ×50 is expected for Belle-II after accumulating a luminosity of 50 ab−1
[103].
\fi
\subsection{$K$ and other Mesons Channels}
guarda bernestein articolo
\begin{figure}[!h]
\centering
\includegraphics[width =0.4\textwidth]{figures/png/Screenshot_20240307_114258.png}
\caption{.}
\label{fig:Kchannel}
\end{figure}



\chapter{The Mu2e Experiment}\label{mu2echapter}
\textit{
The Mu2e experiment aims to investigate the phenomenon of charged-lepton flavor violating (CLFV) neutrino-less conversion, where a negative muon transitions into an electron within the influence of a Aluminium nucleus. The experiment will measure the ratio between the conversion and the nuclear muon capture rates:
\begin{equation}\label{rmue}
R_{\mu e}=\frac{\mu^{-}+N(Z, A) \rightarrow e^{-}+N(Z, A)}{\mu^{-}+N(Z, A) \rightarrow \nu_\mu+N(Z-1, A)}
\end{equation}
The goal is to improve the current limit, set by the SINDRUM-II experiment, Ref. \cite{SINDRUMII:2006dvw}, by
four orders of magnitude and reach a SES (single-event-sensitivity) of 3 $\times$ 10$^{-17}$ on the
conversion rate, a 90\% CL of 8 $\times$ 10$^{-17}$ and a 5$\sigma$ discovery reach at 2 $\times$ 10$^{-16}$.
Mu2e is presently undergoing commissioning, integration and testing stages at the Fermilab Muon Campus, 
with contributions from an international collaboration. Data taking is planned to begin in 2026. 
This Chapter provides an overview of the employed experimental techniques and infrastructures. 
Fundamental bibliography for this chapter can be found in Ref. \cite{bartoszek2015mu2e}, 
\cite{bobbb}, \cite{Bernstein_2013}, \cite{Kargiantoulakis_2020}, \cite{universe9010054}.}
\section{Experiment concept}
A beam of negative muons, $\mu ^-$, is generated by directing a proton beam at a production target, yielding negative pions along with other mesons and hadrons. 
Pions, with a decay rate of more than 99.9\% due to helicity suppression, undergo $\pi ^- \rightarrow \mu ^- \bar{\nu}_\mu$ decay in flight. 
Accumulating sufficient statistical data within a realistic timeframe necessitates an intense muon beamline. 
Low-momentum secondary muons are trapped by a stopping target to form muonic atoms. Within approximately $10^{-13}$ s, 
a muonic atom transitions to the $1s$ state, Ref. \cite{MEASDAY2001243}. 
Given the brief cascading time compared to the mean muon lifetime in a muonic atom, typically around $10^{-6}$ s, 
instances of muon decay before reaching the ground state are negligible. 
The cascades emit x-ray photons, aiding in estimating the number of muons captured on the stopping target. 
Muonic atoms decay after a specific lifetime determined by the stopping target material. 
In the Mu2e experiment, an Aluminum stopping target is employed, $^{27}$Al, resulting in a lifetime of 
864 ns. Muon decay occurs primarily through two processes: 
nuclear capture $\mu^- N \rightarrow \nu_\mu N'^* $, where $N'^*$ represents an excited 
magnesium nucleus ($^{27}$Mg), and muon decay-in-orbit (DIO), 
that is the three-body decay with neutrinos $\mu ^- N \rightarrow e^- N \nu_\mu \bar{\nu}_e$. 
As shown in Figure \ref{fig:muonicatom}, the ratio between these 
processes varies with the stopping target material. For $^{27}$Al, approximately 60.9\% 
of muonic atoms undergo muon nuclear capture. The remaining 39.1\% undergo muon DIO. 
The experiment aims to identify a third decay mode, neutrinoless muon-to-electron 
conversion $\mu^- N \rightarrow e^- N $. Detectors are designed to detect the signature of a 
single monoenergetic electron: this will be explored further in the next sections.
The Mu2e experiment can detect the lepton-number violation process $\mu^- N \rightarrow e^+ N'$.
This process violates both the charged lepton flavor and the total lepton number ($|\Delta L|$ = 2). 
Conversion positrons tracks helices in the Tracker, but with different chirality from electrons. 
\section{Signals and Backgrounds}\label{sigandbkg}
\subsection{Conversion Electron Signal}
The conversion of a muon to an electron in the field of a nucleus is coherent: 
the muon recoils off the entire nucleus and follows two-body decay kinematics, 
Ref. \cite{bartoszek2015mu2e}. The mass of the nucleus is large compared to the 
mass of the electron, hence the recoil terms are minimal. The outgoing nucleus 
remains in the ground state and as a result the conversion electron (CE) is 
monochromatic with an energy slightly lower than the muon's rest mass:
\begin{equation}
    E_{CE} = m_\mu - E_{recoil}(A) - E_{bind}(Z) 
\end{equation}
where $m_\mu$ is the muon mass, $E_{recoil}\simeq \frac{m^2_\mu}{2 m_N}$ is 
the recoil energy of the target nucleus, with $m_N$ the nucleus mass, and 
$E_{bind}\simeq \frac{Z^2 \alpha^2 m_\mu}{2}$ is the binding energy of the 
$1s$ state of the muonic atom, Ref. \cite{universe9010054}. For the Mu2e 
stopping target material, $^{27}$Al, $E_{CE}$ = 104.97 MeV, Ref. \cite{PhysRevD.84.013006}.
\subsection{Backgrounds}\label{backgrounds}
In order to achieve a sensitivity improvement of four orders of magnitude, 
it is important to know all experimental backgrounds that could interfere with 
the process $\mu^- N \rightarrow e^- N $. Any source capable of producing 105 MeV 
electrons may introduce background. There are five categories of background sources, Ref. \cite{bartoszek2015mu2e}:
\begin{enumerate}
\item electrons or muons coming from cosmic rays;
\item intrinsic phenomena that vary in accordance with beam intensity, as muon Decay-in-Orbit (DIO) and Radiative Muon Capture (RMC);
\item processes that are delayed because of particles that spiral slowly down the muon beam line, such as antiprotons;
\item prompt processes where the detected electron is nearly coincident in time with the arrival of a beam particle at the muon stopping target (e.g. radiative pion capture (RPC), Pion or Muon Decay-in-Fight);
\item reconstruction errors due to accidental activities.
\end{enumerate}
The detailed discussion of these background sources will be provided below, along with the corresponding methods for mitigation. The subsequent section will focus on introducing how these mitigation techniques are applied in the Mu2e experimental design, highlighting the experiment setup.
\subsubsection{Cosmic rays}
Cosmic-ray particles, predominantly cosmic-ray muons, also contribute to the experiment background. Several mechanisms are listed here:
\begin{enumerate}
    \item muon decays occurring within or near the detectors;
    \item muons interacting with nuclei within detectors or surrounding materials;
    \item muons scattering within the detectors and being erroneously identified as electrons;
    \item muons entering the muon beamline with specific initial momentum, either interacting with collimators to produce electrons or proceeding down the muon beamline and being misidentified.
\end{enumerate}

While enhanced reconstruction algorithms can correctly identify cosmic-ray muons in certain instances, there are scenarios where signals induced by cosmic rays are indistinguishable from conversion electrons. The quantity of cosmic-ray-induced backgrounds is proportional to the duration of data collection, with the rate being experiment-specific. Detailed simulations estimated that in the Mu2e experiment, conversion-like signals generated by cosmic rays occur approximately once per day, potentially overwhelming actual conversion signals, Ref. \cite{CRVposter}. Mitigation involves employing a combination of active veto and shielding within the experiment. As shown in Section \ref{CRV}, a high-efficiency veto system is capable of effectively reducing cosmic-ray-induced backgrounds to an acceptable level.
\subsubsection{Intrinsic Backgrounds}
The intrinsic backgrounds in the experiment originate from two physics processes: muon decay-in-orbit (DIO) and radiative muon capture (RMC). In this context, $intrinsic$ denotes that the rates of these backgrounds are directly correlated with the number of muons captured on the stopping target.
\paragraph{Muon Decay-in-Orbit}
Electrons energy coming from the three body decay of a free muon, $\mu^- \rightarrow e^- \bar{\nu}_e \nu_\mu$, is described by the Michel Spectrum. The differential decay rate can be computed, Ref. \cite{michel}:
\begin{equation}
    \frac{\text{d}\Gamma_{\text{free}}}{\text{d}x}= \frac{G^2_F m^5_\mu}{192 \pi^3}x^2(6-4x+\frac{\alpha}{x}f(x)) 
\end{equation}
where $x=\frac{2 E_e}{m_\mu}$, $0\leq x\leq 1$, $G_F$ is the Fermi constant, $\alpha$ is the fine-structure constant, $m_\mu$ is the muon mass, $E_e$ is the electron energy and $f (x)$ represents a complicated radiative correction term, described in Ref. \cite{PhysRev.113.1652}. 
As shown in Figure \ref{fig:linearscalemichel}, the spectrum exhibits a peak energy at 52.8 MeV, constituting half of the muon rest energy  or the energy of the searched conversion electron. However, the presence of an Al nucleus allows the electron to interact with it, exchanging momentum and resulting in a maximum possible energy, $endpoint$ $energy$, that closely matches that of the conversion electron (negligible neutrino energy in this scenario). The difference between the two energy spectra, one neglecting the nuclear recoil and the other influenced by the Al nucleus, is illustrated in Figure \ref{fig:micheldiff}. Figure \ref{fig:logscalemichel} presents a more detailed examination of the high-energy range of the spectrum. When the electron energy $E_e \ \geq$ 85 MeV, the dominant term at the leading order scales with $(E_{\mu e} - E_e)^5$, Ref. \cite{PhysRevD.84.013006}, resulting in a low rate within the energy range very close to the endpoint. However, in actual experimental conditions, due to uncertainties in reconstructions and the energy loss of particles when interacting with stopping targets or detector materials, the monoenergetic conversion electrons tend to be reconstructed with a left-skewed energy distribution, with a full width at half maximum (FWHM) approximately on the order of 1 MeV, Ref. \cite{gaponenko}, as shown in Figure \ref{fig:sensitivity}. Consequently, the high-energy electrons originating from muon DIO and the conversion electrons become indistinguishable. To limit the DIO background, Mu2e detectors are designed to be blind to the majority of low-energy electrons. A particle tracking detector with high momentum resolution also helps to reduce background noise, which will be explored in next sections.
\begin{figure}[!h]
     \begin{subfigure}[b]{0.4\linewidth}
         \centering
         \includegraphics[scale = 0.18]{figures/png/Screenshot_20240222_175415.png}
         \subcaption{Linear scale.}
         \label{fig:linearscalemichel}
     \end{subfigure}
     \begin{subfigure}[b]{0.7\linewidth}
         \centering
         \includegraphics[scale = 0.18]{figures/png/Screenshot_20240222_175446.png}
         \subcaption{Logarithmic scale.}
         \label{fig:logscalemichel}
     \end{subfigure}
     \caption{Electron spectrum for aluminum, Ref. \cite{PhysRevD.84.013006}.}
        \label{fig:michel}
\end{figure}

\begin{figure}[!h]
\centering
\includegraphics[width =0.55\textwidth]{figures/png/Screenshot_20240222_175644.png}
\caption{The electron energy spectrum near to the endpoint. The black line represents the Michel spectrum when neglecting nuclear recoil, while the red dashed line takes into consideration the recoil of the Aluminum nucleus, Ref. \cite{PhysRevD.84.013006}.}
\label{fig:micheldiff}
\end{figure}
\paragraph{Radiative Muon Capture}

The Radiative muon capture (RMC) process differs from ordinary muon nuclear capture by producing an additional photon. In the process $\mu^- N \rightarrow\gamma \nu_\mu N'^* $, the photon can either be real or virtual. The photon, interacting with matter or undergoing pair production, can produce electrons with energies close to $E_{CE}$, introducing background signals to the experiment. The emitted photon's energy follows a spectrum, with its maximum energy, denoted as the kinematic limit $k_{max}$, determined by the equation (Ref. \cite{bartoszek2015mu2e}):

\begin{equation}
k_{max} = m_\mu c^2 - |E_b| - E_{rec} - \Delta M ,
\end{equation}

where $E_b$ represents the muon binding energy on the initial nucleus, $E_{rec}$ denotes 
the recoil energy of the daughter nucleus and $\Delta M$ is the rest energy difference 
between the final and initial nuclides. This formula neglects higher order nuclear effects. 
RMC can be effectively mitigated in the Mu2e experiment by selecting Aluminum as the stopping 
target material. The stopping target is selected so that the daughter nuclide of a muon-capture 
process of any kind is heavier than the original nuclide. For aluminum, the RMC endpoint energy 
is 101.9 MeV, approximately 3.1 MeV below the conversion electron energy, Ref. \cite{bartoszek2015mu2e}. 
The planned FWHM of the conversion peak is around 1 MeV, therefore the RMC background will be 
outside the signal region. However, the RMC background might distort the DIO spectra in the 80-100 MeV 
range, making it difficult to extrapolate to the endpoint. Determining the RMC background from the 
data will be a crucial part of the experiment.
\subsubsection{Prompt Processes}
This type of background sources can generate electrons at roughly the same time as 
the entering beam particles. There are four primary sources: radiative pion capture (RPC), 
pion decay-in-flight ($\pi$-DIF), muon decay-in-flight ($\mu$-DIF) and beam electrons.
\paragraph{Radiative Pion Capture}
The secondary muon beam carries a considerable quantity of pions.  It is not excluded 
that some pions can reach the stopping target. Pions, when captured in materials, can 
produce a high-energy photon, i.e. $\pi^- N(A,Z) \rightarrow \gamma N ^* (A,Z-1)$. 
This phenomenon, called radiative pion capture, is observed in approximately 2\% of 
pions captured in Aluminum, Ref. \cite{PhysRevC.5.1867}. Similar to RMC, the photon 
can internally convert into an electron-positron pair or emit an on-shell photon, 
leading to pair production. The external pair-production depends on the thickness 
of the material. The resulting electrons can contribute to the experiment background. 
Despite its similarity to RMC, RPC is more challenging background to suppress due to 
the fact that the endpoint of the energy spectrum of photons, and consequently the 
resulting electrons, is not constrained by the rest energy of the muon. The mass of 
a pion is 139.6 MeV, which is much higher than the conversion energy. Consequently, 
there exists no energy separation between the search range for conversion electrons 
and the electron energy spectrum originating from RPC photons. The SINDRUM II results 
were limited by the pion-induced background and also by the low intensity of its muon 
beam, Ref. \cite{SINDRUMII:2006dvw}. SINDRUM II employed a primary proton beam with a 
frequency of one pulse every 19.75 ns, lasting approximately 0.3 ns. This interval 
between pulses was shorter than the 26 ns lifetime of pions, Ref. \cite{zyla}, 
ensuring a consistent pion flux. To mitigate RPC, SINDRUM II employed a degrader to suppress 
pions and a veto counter in the beam, resulting in less than 1 out of 109 pions reaching their 
target. However, given the more intense beam, the Mu2e experiment has to change the approach. 
Mu2e employs a pulsed proton beam. Given their brief lifetime, nearly all pions decay or 
interact with materials shortly after the pulse of the proton beam. The RPC background 
can be suppressed by opening a live window for conversion electron search at a right time. 
One important point to note is that in the event of protons coming out of the beam pulse, 
out-of-time, the resultant pions could still contribute to RPC background. Consequently, 
it is important for the pulsed beam to achieve a high extinction level, ensuring that the 
ratio of out-of-time to in-time RPC remains below a specified threshold. Further elaboration 
on the pulsed beam used in Mu2e will be provided in Section \ref{accel}.
\paragraph{$\pi$-DIF and $\mu$-DIF}
The decay-in-flight of the pion ($\pi$-DIF) and the decay-in-flight of the muon ($\mu$-DIF) exhibit quite similar characteristics. Free pions and muons can undergo electron decay while transitioning from the production target to the stopping target, through the processes $\mu^- \rightarrow e^- \nu_\mu \bar{\nu}_e$ and $\pi^- \rightarrow e^- \bar{\nu}_e$. In the center-of-mass frame of the initial particles, electrons originating from the first process exhibit an energy spectrum that reaches an endpoint of 52.8 MeV, while those from the second process have a consistent energy of approximately 70 MeV. As the pions and muons move at relativistic velocities, the energies (and momenta) of the resultant electrons are boosted. For instance, a muon with a momentum of around 79 MeV/c or a pion with momentum close to 70 MeV/c can generate an electron with an energy of 105 MeV, Ref. \cite{bartoszek2015mu2e}. Implementing a pulsed proton beam and employing a delayed live window can help suppress background from $\pi$-DIF and $\mu$-DIF events. Particles with sufficient momentum to boost the daughter electrons to the concerning energies move quickly along the muon beamline and are gone by the time the live search window begins, Ref. \cite{bobbb}. In addition, the shape of the Mu2e detectors contributes to mitigation.
\paragraph{Beam electrons}\label{beamelectrons}
Other mechanisms generate electrons, both at the production target and along the muon beamline. For instance, neutral pions formed at the production target can decay in two photons, after which the photons can either create electron-pairs or interact with nearby materials to generate electrons. Other beam particles can decay or interact at any point before the muon stopping target, producing electrons with energy equal to $E_{CE}$. These sources of background are colled beam electrons and it is possible to reduce them through the pulsed beam and the delayed live window. Moreover, Mu2e uses a set of solenoids to generate a magnetic field across the muon beamline and in the area of the stopping target and detectors. A charged particle follows a spiral trajectory when a magnetic field is applied, where the size and shape of the spiral are determined by the particle's electric charge and its parallel and perpendicular to magnetic field components of momentum. This brings to the installation of collimators along the muon beamline to suppress the number of high-momentum particles exceeding 100 MeV/c, Ref. \cite{bartoszek2015mu2e}. Moreover, the magnetic field is designed with a gradient in proximity of the stopping target. This gradient effectively divides the paths of conversion electrons from those originating upstream, unless they undergo scattering within the stopping target, Ref. \cite{bobbb}. Solenoid system and the magnetic field will be deeply explored in Section \ref{setup}.
\subsubsection{Delayed processes from antiprotons}
Protons, at a given threshold, can generate antiprotons within the production target. This occurs through the process of antiproton production: $pp \rightarrow ppp\bar{p}$. The minimum kinetic energy of the initial proton beam can be found applying 4-momentum conservation principles to the system. 
If we consider that all four particles in the final state are at rest in the center of mass frame, the minimum kinetic energy needed for $p\bar{p}$ production can be found, which is approximately $6 m_pc^2 \sim 5.6$ GeV, where $m_p$ is the mass of the proton. In an ideal scenario, maintaining the beam energy below this threshold could enable us to avoid this background in the experiment. Unfortunately, the Mu2e proton beam goes beyond the threshold for antiproton production. The antiprotons are long-lived and massive. Antiprotons with momenta below 100 MeV/c travel at speeds less than 0.1$c$, requiring several $\mu$s to spiral from the production target to the stopping target, Ref. \cite{bartoszek2015mu2e}. They have the correct charge and momentum to pass through the collimators placed between the production target and the stopping target. Annihilating or undergoing interactions with other materials, they have the capability to release a substantial amount of energy, generating numerous secondary electrons. The time delay connected with these interactions significantly exceeds the muon lifetime, leading to a continual flow of antiprotons reaching the stopping target. The pulsed beam and the delayed live window fail to suppress the antiproton background. The best approach is to prevent the antiprotons from reaching the region where the stopping target is located. A thin absorber is positioned in the muon beamline to capture the antiprotons. Its design was developed to find a compromise between increasing antiproton absorption and decreasing muon beam loss.
\subsubsection{Accidental activity}
This final background category arises not from the physical interactions of a specific particle, but rather from the accidental reconstruction of extra events in the detectors, miming conversion-like signals. During the muon-capture process, nuclei in excited states can expel protons, neutrons and photons. These expelled protons have high ionization potential, producing large signals in the detectors and increasing the likelihood of reconstruction errors. Additionally, other coincidences, such as multiple muon decay-in-orbit occurring in close succession, can also contribute to reconstruction errors. To suppress the flux of protons from the muon-capture process reaching the detectors, additional polyethylene absorbers are used to surround the stopping target in the Mu2e experiment. A systematic uncertainty is evaluated as part of the background analysis to take into account these accidental events.

\subsection{Backgrounds estimates and Signal Sensitivity}
Mu2e run plan is divided in two different phases. Run I will take place in 2026, before a 2 years shutdown due to the planned accelerator upgrade for the long baseline neutrino program. In Run I phase one, a low intensity proton beam, $1.6 \times 10^7$ protons/pulse, will be used. In Run I phase two, the mean intensity will be increased up to $3.9 \times 10^7$ protons/pulse. The total number of stopped muons will be $6 \times 10^{16}$, corresponding to the 10\% of the number required to satisfy the Mu2e goals in the complete data-taking. As discussed in Ref. \cite{universe9010054}, the discovery $R_{\mu e}$, Ref. \ref{rmue}, corresponding to a 50\% probability of observing the conversion signal at a 5$\sigma$ significance level is $R_{\mu e}^{5 \sigma}= 1.2 \times 10^{-15} $. Reaching the 5$\sigma$ significance level requires observing 5 $\mu\rightarrow e$ events in the two-dimensional search window 103.60 < $p$ < 104.90 MeV/c, 640 < $T_0$ < 1650 ns. One of the
parameters characterizing the sensitivity of an experiment to a process of interest is its
single event sensitivity (SES), defined as:
\begin{equation}
    SES \equiv \frac{1}{N_{POT} \cdot P_{\mu \ stop} \cdot \epsilon_{CE} \cdot BR_{capture}}
\end{equation} 
$N_{POT}$ means the number of protons on target in the experiment, $P_{\mu \ stop}$ is the number of muons stopped on target per proton, $\epsilon_{CE}$ is the conversion electron acceptance, which is a product of detector efficiency (dependent on the momentum signal region) and fraction of muons interacting in the live time window and $BR_{capture}$ is the branching ratio of muon captures, which is 60.9\%. The optimized Mu2e signal window corresponds to a SES of $2.4 \times 10^{-16}$. The background estimates after the sensitivity optimization are summarized in Table \ref{tab:summarybkg}, resulting in a total background of approximately $\sim$ 0.1 event/year. Figure \ref{fig:sensitivity} shows the momentum and time distributions for CE signal and background processes corresponding to the optimized signal window for Run I. A detailed analysis and estimate of the Mu2e expected backgrounds for Run I can be found in Ref. \cite{universe9010054}.
\begin{center}  
\begin{table}[!h]
\centering
\renewcommand{\arraystretch}{1.2}
\begin{tabular}{| c | c |}
\hline
\textbf{Channel} & \textbf{Mu2e Run I}\\
\hline
SES & 2.4 $\times \ 10^{-16}$ \\
\hline
Cosmic rays & 0.046 $\pm$ 0.010 (stat.) $\pm$ 0.009 (syst.) \\
DIO & 0.038 $\pm$ 0.002 (stat.) $ ^{+ \ 0.025} _{- \ 0.015}$ (syst.)\\
Antiprotons & 0.010 $\pm$ 0.003 (stat.) $\pm$ 0.010 (syst.) \\
RPC in-time & 0.010 $\pm$ 0.002 (stat.) $ ^{+ \ 0.001} _{- \ 0.003}$ (syst.)\\
RPC out-of-time & (1.2 $\pm$ 0.1  (stat.) $ ^{+ \ 0.1} _{- \ 0.3}$ (syst.)) $\times$ $10^{-3}$ \\
RMC & $<$ 2.4 $\times$ $10^{-3}$ \\
Decays in flight & $<$ 2 $\times$ $10^{-3}$ \\
Beam electrons & $<$ 1 $\times$ $10^{-3}$ \\
\hline
Total &  0.105 $\pm$ 0.032\\
\hline
\end{tabular}
\caption{Summary of the several background sources to the conversion electron search as expected in Mu2e Run I,  Ref. \cite{universe9010054}. The table also shows the corresponding single event sensitivity (SES). This is is defined as the $R_{\mu e}$ ratio when there is one signal event.}
\end{table}\label{tab:summarybkg}
\end{center}
\begin{figure}[!h]
\centering
\includegraphics[width =0.93\textwidth]{figures/png/Screenshot_20240225_102708.png}
\caption{Left: Momentum distribution of the conversion electron signal and expected backgrounds. Right: Time distribution of the conversion electron signal and expected backgrounds. The arrows show the signal region selected for the analysis,103.60 $< \ p \ < $ 104.90 MeV/c and 640 $< \ T_0 \ < $ 1650 ns, Section \ref{pulsedprotonbeam}. The $CE$ signal distributions correspond to $R_{\mu e} = 1 \times 10^{-15}$, Ref. \cite{universe9010054}.}
\label{fig:sensitivity}
\end{figure}
\section{Experimental setup}\label{setup}
Figure \ref{fig:mu2escheme} shows a schematic overview of the Mu2e experiment, illustrating the trajectory of the pulsed proton beam directed towards the production target indicated by the red arrow. The experiment uses a solenoid system to generate magnetic fields essential for its operations. The Production Solenoid (PS) surrounds the production target, while further downstream, the Transport Solenoid (TS) provides the magnetic field for the muon beamline. The TS, configured in an S-shape, incorporates collimators and proton absorbers strategically positioned to minimize experimental backgrounds. The stopping target is located at the beginning of the Detector Solenoid (DS). Proton absorbers surround the stopping target, not shown in the figure. The Tracker and the Calorimeter are housed in the DS, enabling momentum and energy measurements, respectively. Additionally, a Stopping Target Monitor (STM) positioned downstream at the DS's end, not shown, monitors the stopping target's condition and estimates the captured muon count. Not shown in the figure is the Mu2e Cosmic Ray Veto (CRV) system: it surrounds the DS and half of the TS. In next sections, a more detailed description of systems of the Mu2e experiment will be given.
\begin{figure}[!h]
\centering
\includegraphics[width =\textwidth]{figures/png/Screenshot_20240301_143105.png}
\caption{Schematic view of the Mu2e apparatus. The center of the Mu2e reference frame is located at the COL3 collimator center, its y-axis points upwards, the z-axis is parallel to the DS axis and points downstream, and the x-axis completes the right-handed reference frame, Ref. \cite{universe9010054}.}
\label{fig:mu2escheme}
\end{figure}
\section{Accelerator system and Proton Beam}\label{accel}
\subsection{Pulsed Proton Beam}\label{pulsedprotonbeam}
As previously discussed in Section  \ref{backgrounds}, the Mu2e experiment employs a pulsed proton beam to reduce background from prompt processes, as pion capture. The 8 GeV, 8 kW beam originates from the Fermilab Booster, Ref. \cite{PhysRevAccelBeams.20.111003}. A pulsed beam with a 1695 ns gap between two pulses is needed. Figure \ref{fig:accell} illustrates the Fermilab accelerator facilities involved in generating and delivering the pulsed proton beam. The Fermilab Booster delivers 8 GeV protons in 20 batches throughout a 1.4 s Main Injector cycle at 15 Hz, as shown in Figure \ref{fig:deliver}. Thus, the accelerator timeline is described using a fundamental time unit of 1 tick that corresponds to 66.7 ms.

\begin{figure}[!h]
     \begin{subfigure}[b]{0.4\linewidth}
         \centering
         \includegraphics[scale = 0.3]{figures/png/Screenshot_20240301_151449.png}
         \subcaption{\centering Fermilab accelerator facilities involved in producing and delivering the pulsed proton beam.}
         \label{fig:accell}
     \end{subfigure}
     \begin{subfigure}[b]{0.7\linewidth}
         \centering
         \includegraphics[scale = 0.39]{figures/png/Screenshot_20240301_151418.png}
         \subcaption{\centering Proton beam delivery to Mu2e.}
         \label{fig:deliver}
     \end{subfigure}
     \caption{Pulsed proton beam delivery, Ref. \cite{accelerator}.}
        \label{fig:three graphs1}
\end{figure}
The two Mu2e batches, represented by the two blue bars at ticks 1 and 2, are injected into the Recycler Ring, each containing $4 \times 10^{12}$ protons. The protons from these batches are reorganized within the Recycler using a 2.5 MHz radio frequency (RF) system into 8 bunches. These bunches are then extracted individually from the Recycler and transported to the Delivery Ring every 48.1 ms, as shown in the middle part of Figure \ref{fig:deliver}. Once inside the Delivery Ring, a single bunch of $1 \times 10^{12}$ protons undergoes gradual extraction. This process results in the extraction of a small fraction of the bunch per revolution, delivered to the Mu2e experiment. The complete bunch is extracted over a span of 43.1 ms, across $\sim$25000 turns around the Delivery Ring, as shown in the bottom part of Figure \ref{fig:deliver}.
\begin{figure}[!h]
\centering
\includegraphics[width =\textwidth]{figures/png/Screenshot_20240301_151148.png}
\caption{Proton beam profile at the Mu2e Proton Target, Ref. \cite{accelerator}.}
\label{fig:beamprofile}
\end{figure}
Figure \ref{fig:beamprofile} illustrates the temporal profile of the beam at the Mu2e Proton Target. Consecutive proton pulses are spaced by 1695 ns. Each pulse lasts for 250 ns and contains $(3.9 \pm 2.0 )\times 10^7$ protons. The 1695 ns pulse separation is highly advantageous for the Mu2e experiment. Figure \ref{fig:beamwindow} shows the beam pulse, the simulated pion flux, the muon capture rate on the stopping target and the muon decay rate. The active window for detecting conversion electrons begins at $\sim$640 ns and extends for mor or less 1 $\mu$s. This selection finds a compromise between reducing pion-induced backgrounds  and increasing the rate of muon decays. Because of the brief lifetime of the pions, they are gone by the time the active window opens, resulting in a suppression of the pion count by $\mathcal{O}(10^{11})$. The 1695 ns pulse separation exceeds twice the muon lifetime, allowing sufficient temporal separation between prompt backgrounds and the live window without excessively compromising beam intensity.
\begin{figure}[!h]
\centering
\includegraphics[width =\textwidth]{figures/png/Screenshot_20240301_164649.png}
\caption{Proton pulses arrive at the production solenoid every 1695 ns. A delayed live-time window can suppress the beam-related background, Ref. \cite{universe9010054}.}
\label{fig:beamwindow}
\end{figure}
\subsection{Proton Beam Extinction and Extinction Monitor}
As mentioned in the previous section, the Mu2e experiment requires the extinction level of the incoming proton beam, to reduce backgrounds caused by out-of-time protons. The extinction rate, defined as the ratio between the number of out-of-time protons and the number of the in-time protons, should be lower than $10^{-10}$, Ref. \cite{bartoszek2015mu2e}. The structure of the beam leads to an extinction level $2.1 \times 10^{-5}$. To take into account the fact that some beam will leak out of two consecutive proton pulses, an Extinction Insert is deployed in the M4 beamline between the Delivery Ring and the Mu2e experiment. The out-of-time beam particles are swept into a collimator system by a set of oscillating dipoles, called AC dipoles. These AC dipoles are simulated to offer an additional extinction factor of $5\times 10^{-8}$, reducing the overall extinction to $1.1 \times 10^{-12}$, leaving a margin of 10$^2$, Ref. \cite{accelerator}. An Extinction Monitor is positioned downstream of the production target along the proton beamline, Figure \ref{fig:extintion}. It monitors the extinction level of the incoming beam striking the Mu2e production target and delivers a measurement at a precision better than $10^{-10}$. The Extinction Monitor consists of a collimator and magnetic filter system, a pixel telescope, a system of trigger scintillators and a range-stack. The collimator and magnetic filtter system transport a small quantity of the particles generated at the production target to the Extinction Monitor. The pixel telescope tracks the trajectory of charged particles coming from the collimator. The pixel telescope consists of a permanent magnet and 8 scintillators, as shown in Figure \ref{fig:extintionmonitor}. The system uses a permanent magnet to separate two sets of four scintillator planes, allowing for momentum measurements of entering particles. The range stack is located further downstream from the pixel telescope. Steel absorber plates separate scintillators, distinguishing between hadrons and muons based on their penetrating capacities.
\begin{figure}[!h]
\centering
\includegraphics[width =\textwidth]{figures/png/800px-Extinction_filter.png}
\caption{The Extinction Monitor is located downstream of the
production target, Ref. \cite{Prebys:IPAC2015-THPF121}.}
\label{fig:extintion}
\end{figure}
\begin{figure}[!h]
\centering
\includegraphics[width =0.9\textwidth]{figures/png/Screenshot_20240306_184720.png}
\caption{The tracking spectrometer of the Mu2e experiment, consisting of eight planes of pixel detectors and a permanent magnet spectrometer, Ref. \cite{Prebys:IPAC2015-THPF121}.}
\label{fig:extintionmonitor}
\end{figure}
\subsection{Production Target}
The Mu2e production target is an additional essential element of the accelerator systems. It is suspended in the middle of the PS bore, Ref. \cite{bartoszek2015mu2e}. It is made of tungsten. The tungsten has a high pion production cross-section, capable of producing the necessary number of stopped muons. The refractory material allows the target to be cooled radiatively without the need of any extra cooling equipment. A compact design minimizes pion reabsorption on the production target.
\section{Solenoids System}
The Mu2e Solenoid system consists of three magnetically coupled systems: the \textbf{Production Solenoid} (PS), the \textbf{Transport Solenoid} (TS) and the \textbf{Detector Solenoid} (DS), shown in Figure \ref{fig:mu2escheme}. Each system contains multiple module and each one is made of superconducting coils wound with aluminum stabilized Nb-Ti Rutherford cables, located in a 4.5 m long cryostat. The shape of solenoids is designed in order to efficiently transmit muons and to suppress other particles. The resulting magnetic field is $\mathcal{O}$(1 T): its highest value is 4.6 T at the upstream end of the experiment and the lowest value is 1 T at the downstream end. The muons are guided toward the stopping target and the DS by lowering the magnetic field. Local magnetic field minima are avoided to avoid trapping particles in these areas. In the PS, the magnetic field decreases from 4.6 T to 2.5 T at the entrance of the TS. The large gradient helps to collect the secondary pions and muons and to direct them towards the DS. The magnetic field across the S-shaped TS changes its value only by a factor of 0.5 T. The shape of the TS allows to not transmit photons and other neutral particles in the DS and its dimension was set to avoid transmitting particles with large momentum. These particles either spiral with a large helical radius (large initial momenta perpendicular to the field) or cannot create an S-shaped bend (large initial momenta parallel to the magnetic field), resulting in collisions with solenoid walls or collimators. As particle drifts in the solenoid field is dependent on the particle charge, positively and negatively charged muons drift in opposite vertical directions and are separated, as explained in Appendix \ref{appendix1}. In the upstream curved solenoid portion, as shown in Figure \ref{fig:collimators}, the spiraling positive (blue) and negative (red) muons are deflected downwards and upwards respectively. Positive muons are stopped in collimators COL3u and COL3d. The TS is long enough for pion decay, suppressing RPC backgrounds. It also protects the detectors from radioactively hot areas around the primary beam and the production target. The magnetic field in the first half of the DS is reduced from 2 T to 1 T. In a smoothly gradient field, the adiabatic invariance of the magnetic flux can be used. Assuming a constant $p^2_\perp/B$, there is:
\begin{equation}
    v^2_{\parallel}=v^2_0-v^2_{\perp 0}\frac{B(z)}{B_0}
\end{equation}
Here, $\perp$ is referred with respect to the magnetic field $B$ and $\parallel$ is referred to the $z$ direction. The subscript 0’s indicates the initial state. The gradient pitches electrons forward into the tracker's acceptance while rejecting higher velocity electron, as mentioned in \ref{beamelectrons}. 
The second half of the DS containing the Tracker and the Calorimeter has a uniform magnetic field of 1 T. This allows particle trajectories and momenta to be reliably measured.
\begin{figure}[!h]
\centering
\includegraphics[width =0.9\textwidth]{figures/png/Screenshot_20240303_152845.png}
\caption{The muon beamline, composed of the Production Solenoid (PS), the Transport Solenoid (TS) and the Dector Solenoids (DS), Ref. \cite{ginther}. The PS is 4 m long. The role of the PS, shown in Figure \ref{fig:PS}, is to collect pions, kaons and their decay products muons. The TS is S-shaped and it is divided in five parts. The first is called TS1 that contains the first collimator (COL1), made of copper wedges. It filters particles from their momentum and reduces the radiation damage for coils of the upstream part of the TS (TSu). The TS2 contains a toroidal field in order to select negative muons from the positive ones in TS3 with two rotatable collimators (COLu and COLd). The rotatable feature allows selection of $\mu^-$'s instead of $\mu^+$'s, which can be used for detector calibration. The TS4 role is to put the beam in the center of the solenoid. TS5 connects TS with DS and it contains a collimator (COL5) made of polyethylene, which will serve as a shield from neutrons. A thin window assembly is installed at the beginning of the TS and also between the rotatable collimators to absorb antiprotons in the beam.
The DS, shown in Figure \ref{fig:DS}, is a cylindrical system of approximately 11 m in length and 2 m in radius, which houses the Stopping Target and the main Mu2e detectors. The system is divided into two sections: a 4 m gradient section following the TS and a 6 m spectrometer section. The gradient region allows to separate conversion electrons from beam electrons. In the spectrometer region, the uniform magnetic field allows a precise measurement of the particle momentum. The upstream part of the beamline accounts for the Production Solenoid (PS) and the first part of the Transport Solenoid (TSu). The downstream part is composed of the last portion of the Transport Solenoids (TSd) and the Detector Solenoid (DS). Protons from muon captures in the Stopping Target are partially suppressed by a 0.5 mm thick cylindrical-shaped polyethylene proton absorber, placed halfway between the Stopping Target and the tracker. Finally, the IFB is a plate at the end of the DS, maintaining the DS vacuum and providing a path for the services and signals between vacuum and hall air.
}
\label{fig:muonbeamline}
\end{figure}
\begin{figure}[!h]
\centering
\includegraphics[width =0.95\textwidth]{figures/png/800px-MuonBeamlineCollimators2.png}
\caption{The view of the Transport Solenoids and the collimators COL3u and COL3d showing the offset apertures in those collimators. The upper spiraling negative muons (red) pass through the aperture while the positive muons (blue) are stopped by these collimators, Ref. \cite{tsview}.}
\label{fig:collimators}
\end{figure}
\begin{figure}[!h]
\centering
\includegraphics[width =0.95\textwidth]{figures/png/800px-Production_solenoid.png}
\caption{Cross-section of the production solenoid, Ref. \cite{6376120}.
The production target is placed approximately at the center of the superconducting coils.}
\label{fig:PS}
\end{figure}
\begin{figure}[!h]
\centering
\includegraphics[width =0.95\textwidth]{figures/png/Screenshot_20240306_225639.png}
\caption{Overall structure of the Detector Solenoid coils and cryostat, Ref. \cite{bobbb}.}
\label{fig:DS}
\end{figure}
\section{Stopping target}
The Mu2e muon stopping target is composed of 37 annular aluminum foils  with a purity of above 99.99\%, Ref. \cite{bobbb}. The foils are 100 $\mu$m thick, minimizing energy losses of the conversion electrons. This design narrows the reconstructed conversion electron momentum distribution and separates it from the DIO electron momentum distribution. The annular design reduces interactions with the beam electrons and other particle released in muon nuclear captures, which can create particle sprays and damage to the tracker's internal components. The central hole does not affect the target capacity to halt muons, which move in helical patterns. Muons passing through the hole of an upstream foil will stop in a downstream layer. There are various factors to consider while selecting aluminum as the stopping target material. First, as described in Section \ref{backgrounds}, the aluminum target has a low RMC background. Moreover, the muonic aluminium atom has a quite long lifetime, as shown in Figure \ref{fig:muonicatom}. The long lifetime allows separation between prompt backgrounds and a live window with a good decay rate. The muon DIO endpoint energy, further, depends on the type of nucleus, as shown in Figure \ref{fig:endpoint}. Aluminum has a high endpoint energy, so when muons are captured on other detector materials with a higher atomic number $Z$, they have lower endpoint energies and do not contribute to background. Aluminum is a suitable stopping target material for muon-to-electron conversion searches.
Moreover, the branching ratio ($BR$) of the conversion varies depending on the stopping target material due to differences in atomic number ($Z$) and mass number ($A$). By comparing $BR$s on different nuclei normalized to aluminum, it's possible to identify the dominating operator type, such as scalar ($S$), dipole ($D$), vector of transition charge radius type and vector of effective $Z$-penguin type, ($V(\gamma)$) and ($V(Z)$) respectively. Despite challenges in separating prompt backgrounds from signals due to short lifetimes of muonic atoms, materials with higher $Z$ offer better model differentiation. If the Mu2e experiment observes conversion signals, a subsequent search, Mu2e-II, could employ titanium as the stopping target. A more detailed discussion can be found in Ref. \cite{PhysRevD.80.013002}, \cite{PhysRevD.76.059902} and \cite{abusalma2018expression}.
\begin{figure}[!h]
\centering
\includegraphics[width =0.5\textwidth]{figures/png/lifetime_mu_matter.png}
\caption{ The mean lifetimes and free decay branches of the muonic $1s$ state versus the atomic number of the nucleus to which the negative muon is bound \cite{TYamazaki_1975}.}
\label{fig:muonicatom}
\end{figure}
\begin{figure}[!h]
\centering
\includegraphics[width =0.5\textwidth]{figures/png/endopint.png}
\caption{ The dependence of electron energy spectrum endpoint from muon DIO, Ref. \cite{dukes}.}
\label{fig:endpoint}
\end{figure}


\section{Detectors}
\subsection{The Tracker}\label{trackersec}
The Mu2e Straw Tube Tracker will measure conversion electrons momentum.
The Tracker must have as good resolution as is reasonably achievable
in order to minimize backgrounds, especially from the DIO electrons, Ref. \cite{bobbb}.
Since $dE/dx$ in the stopping target spreads the 
monochromatic energy distribution of the conversion electron downwards, 
it is needed to stop as many muons as possible. DIO events are smeared near the conversion 
peak as well, but this is a stochastic process. Conversion electrons with high energy loss 
might end up in the DIO spectrum with lower energy loss. Energy loss in the tracker needs 
to be as small as possible, since the stopping target only stops about 40\% of muons. 
To achieve excellent resolution, small high-side tails and minimal energy loss, the experiment 
will employ a straw tube tracker, Ref. \cite{bobbb}.
The Tracker is located inside the DS, downstream of the stopping target. 
Its shape follows the helical trajectories of conversion electrons 
in the magnetic field. 96 straw tubes are arranged in 
two staggered layers to form a panel, Ref. \cite{bartoszek2015mu2e}. 
The basic standalone module both mechanically and electrically is 
shown in Figure \ref{fig:trkpanel}, upper left. Each panel, that is 
harp-shaped, spans 120°. The straw tubes are arranged like harp chords. 
The electronics is stored in the outer volume. The Mu2e Tracker has a 
very limited material component, particularly in the straw region, 
which reduces the probability of scattering conversion electrons 
and increases the momentum resolution. The simulated resolution is shown in Figure \ref{fig:trkres}. 
The panels are combined 
using the approach indicated in Figure \ref{fig:trkpanel}, upper right, 
Ref. \cite{trk}. Three panels create a full circle and two layers of six 
panels, rotated by 30°, form a plane. A station is made up of two planes 
that have been rotated by 60°. One or more panels cover the entire annular 
region. The entire Tracker consists of 18 identical stations, shown in 
Figure \ref{fig:trkpanel}, bottom. It is 3.2 metres long and 1.7 metres in diameter. 
\begin{figure}[!h]
\centering
\includegraphics[width =0.7\textwidth]{figures/png/Screenshot_20240306_222803.png}
\caption{(Upper left) Tracker panel. (Upper right) Isometric view 
of a tracker plane (left) with three panels each on the front and 
back face and a station (right) consisting of two planes. (Bottom) 
The assembled tracker, with 18 stations. Stations are shown in 
grey and support structure in gold, Ref. \cite{bartoszek2015mu2e}.}
\label{fig:trkpanel}
\end{figure}
\begin{figure}[!h]
    \centering
    \includegraphics[width =0.7\textwidth]{figures/png/Screenshot_20240330_104830.png}
    \caption{The simulated resolution of the Mu2e tracker for electrons at the
    conversion energy. The asymmetric low-side tail is caused by the tracker 
    stochastic energy loss mechanism. The high-side tail, where decay-in-orbit 
    events would be much more present with respect to the signal region, is small. 
    The simulation take into account all beam, tracker, electronics and DAQ 
    properties, Ref. \cite{bobbb}.}
    \label{fig:trkres}
    \end{figure}
Figure \ref{fig:sttrk} shows that the Tracker straws are only 
in the active region with radius ranging from 380 mm to 700 mm. 
The central part of the Tracker is not instrumented. A similar annular 
design can also be found in the Mu2e Calorimeter, in Section \ref{calorimeter}. 
The geometry of the detector is specifically designed to reduce background. 
Detectors are blind to muon beams and associated activity, including $\pi$-DIF, 
$\mu$-DIF, beam electrons and other particles emitted during nuclear captures. 
Its shape prevents the detection of low-energy electrons released during muon DIO. 
Figure \ref{fig:sttrk} shows the electrons trajectory as coloured circles passing 
through the stopping target. For a homogeneous magnetic field, the radius of a 
helix follows the equation \ref{partincamp}, so it is proportional to transverse 
momentum of the particle. Low momentum electrons, $\lesssim$ 53 MeV/c, can pass 
through detectors without causing a hit. This corresponds to the green trajectory 
in Figure \ref{fig:sttrk}, whereas the green circle represents an example 
trajectory. Electrons with higher momenta can leave some hits on the 
detectors, but few hits are insufficient to reconstruct the electron 
trajectory, orange trajectory shown in Figure \ref{fig:sttrk}. The tracks 
are only reconstructible when the electron momentum is high enough: $\gtrsim$ 
90 MeV/c. It is expected that only muon DIO events will be recorded in the 
Tracker. The low rate simplifies front-end electronics (FEE) requirements 
and minimises reconstruction mistakes caused by accidental activity. 
Chapter \ref{chaptertrk} will provide a detailed discussion of the straw 
tube Tracker panel, including its detection mechanism, mechanical design, 
front-end electronics and data acquisition (DAQ) system, as the focus of 
my thesis is on the Vertical Slice Test of the straw tracker.
\begin{figure}[!h]
\centering
\includegraphics[width =0.7\textwidth]{figures/png/Screenshot_20240306_214911.png}
\caption{The annular design of Mu2e detectors: a view of the Tracker, Ref. \cite{trk}.}
\label{fig:sttrk}
\end{figure}
\subsection{The Electromagnetic Calorimeter}\label{calorimeter}
The Mu2e Electromagnetic Calorimeter serves multiple purposes, Ref. \cite{em4}:
\begin{itemize}
    \item the energy deposition measured in the calorimeter separates $e$ and $\mu$ with the same momentum.
    The separation in energy-to-momentum ratio ($E/p$) can be used to identify particles and suppresses
    part of the cosmic background caused by cosmic-ray muons;
    \item the calorimeter signals can improve the quality of track reconstruction. The additional information can
    be used for consistency check, reducing reconstruction errors caused by accidental activities;
    \item it provides triggers for the experiment independent of the Tracker, Ref. \cite{em6}. 
\end{itemize} 
The calorimeter sits downstream of the Straw Tube Tracker in the 1 T region in the Detector Solenoid and in a vacuum of $10^{-4}$ Torr. 
It consists of two annular detection disks, separated by 70 cm, Ref. \cite{em7}. 
The separation was chosen to match half of the distance between two periods in the helical trajectory of a
typical 105 MeV conversion electron. This maximizes the detection probability of conversion electrons in the
calorimeter. Electrons that pass through the hole of the first disk will be detected by the second one.
The disks inner radius is 35 cm and the outer one is 66 cm. A schematic view of the calorimeter is shown in Figure \ref{fig:calo1}.
\begin{figure}[!h]
    \centering
    \includegraphics[width =0.7\textwidth]{figures/png/Screenshot_20240322_122050.png}
    \caption{The Mu2e Calorimeter annular disks, Ref. \cite{em7}.}
    \label{fig:calo1}
\end{figure}
Each calorimeter disk is filled with 674 undoped cesium iodide (CsI) crystals, Ref. \cite{em6}, wrapped with a 150 $\mu$m foil of Tyvek. 
Tyvek is pure CsI, with a wavelenght peak emission at 315 nm and a scintillation time of 20 ns: it was chosen because of its
fast emission and low cost. The crystals are 34 $\times$ 34 $\times$ 200 mm$^3$ in dimensions, as shown in Figure \ref{fig:calo2} (top right). 
To improve reliability, light collection and resolution, each crystal is readout by two custom design SiPMs array, Ref. \cite{em1}. 
Magnetic field resistance was the factor in choosing between SIPMs and PMTs.
The crystal have fast emission time (decay time better than 40 ns) and an acceptable light yield, 20 photoelectrons/MeV per SiPM.
Undoped CsI crystals were also chosen because they are sufficiently radiation hard.
CsI radiation lenght is $X_0 \sim $2 cm and the crystal length is about 10 $X_0$:
it is sufficient to contain the 105 MeV CE showers since the average incident angle is 50°. 
To reduce thermal coupling between crystals and electronics, a 2 mm air gap is maintained between the crystal and the readout sensor. 
Since the main scintillation component has a wavelength of 310 nm to match the SiPM
photon detection efficiency as a function of wavelength, each crystal is coupled with a readout 
module consisting of two ultraviolet (UV)-extended Silicon Photomultiplier (SiPM) arrays
and the corresponding analog front-end electronics (FEE) boards, as illustrated in Figure \ref{fig:calo2} (bottom right), 
Ref. \cite{em5}, \cite{em2} and \cite{em3}. 
\begin{figure}[!h]
    \centering
    \includegraphics[width =0.7\textwidth]{figures/png/Screenshot_20240322_121000.png}
    \caption{Left: the CAD of the two calorimeter disks. Top right: the calorimeter parallelepiped shaped crystals.
            Bottom right: two SiPM arrays glued onto copper holder on the left and one readout 
            unit formed by two SiPM arrays and two FEE boards mounted on its
            copper holder and on the right, Ref. \cite{em4}.}
    \label{fig:calo2}
\end{figure}
\begin{figure}[!h]
        \centering
        \includegraphics[width =0.8\textwidth]{figures/png/Screenshot_20240322_121017.png}
        \caption{Left: the breakout of calorimeter mechanical components. Top right: the breakdown of front
        panel plate. Bottom right: the mezzanine and DiRAC boards, Ref. \cite{em4}.}
        \label{fig:calo3}
        \end{figure}

Each SiPM is composed of a 2 $\times$ 3 array of individual 6 mm $\times$ 6 mm cells. 
The array dimensions were chosen to maximise light collecting, with an active area of 1.8 $\times$ 1.2 cm$^2$ 
and a small total capacitance due to the parallel structure, resulting in a signal width of less than 200 ns.
Each SiPM is linked to its own power supply and high voltage in Front-End Electronics (FEE) board.
The crystal readout unit includes 2 SiPMs, 2 FEE boards, mechanical support for SiPM cooling, a Faraday cage to 
decrease noise and a fiber for laser calibration. A Mezzanine Board (MB) controls 20 amplifiers. Signals are sent from MB to the ADCs in the 
digitizer board (DiRAC). Signals will be sampled at 200 Msps, 5 ns binning, by the digitization system. 
As shown in Figure \ref{fig:calo3}, each board is positioned within the electronic crates that sorround the calorimeter disks.
Zero suppressed data are sent from the DiRAC board to the DAQ via optical fibers. 
The calorimeter is designed to have good energy resolution ($\sigma_E/E$ < 10\%), 
timing resolution ($\sigma_t \sim$500 ps for 100 MeV electrons) and position resolution
($\sim$6 mm). A 51-crystal prototype module, called Module-0, was exposed
to a test beam and results for energy and time resolution are
shown in Figure \ref{fig:calores}, Ref. \cite{bobbb} and \cite{calo95}.

\begin{figure}[!h]
    \centering
    \includegraphics[width =0.8\textwidth]{figures/png/Screenshot_20240330_105520.png}
    \caption{The energy and time resolution of the Module-0 for the Mu2e calorimeter for electrons at the conversion energy. (Top left) The resolution for electrons striking the
    array at normal incidence. (Top right) Resolution for electrons striking the array at 55° with respect to the face. (Bottom) The time resolution. 
    All measurements are compared with simulations (red line), Ref. \cite{bobbb} and \cite{calo95}.}
                \label{fig:calores}
                \end{figure}
\subsection{The Cosmic Ray Veto}\label{CRV}
As stated in Section \ref{backgrounds}, Mu2e expects backgrounds generated by cosmic rays 
at a rate of one per day. Although the EM calorimeter can detect some backgrounds created 
by muons with the correct momenta, there are still problematic cases. To reduce these 
backgrounds, Mu2e employs an active veto system combined with shielding. An overview 
of the Cosmic Ray Veto (CRV) is shown in Figure \ref{fig:crv}. It covers the entire 
DS and half of the TS (there is no veto at the bottom), for a total area of 327 m$^2$.
The CRV modules are manufactured from plastic scintillator extrusions. Each extrusion 
has a cross-sectional area of $5 \times 2$ cm$^2$, with different lengths. The extrusions 
are coated with titanium dioxide to improve internal reflections and so the light yield. 
Two grooves are extruded inside the scintillator bar throughout its length, containing 
1.4 mm diameter wavelength changing fibres. They send light to the extrusion ends, 
where each fibre is detected by a $2 \times 2$ mm$^2$ SiPM on each end. Figure \ref{fig:crvmodule} 
illustrates the cross-section of the CRV module. Each module has four overlapping layers of plastic 
scintillator counters to reduce the effect of gaps. The layers are separated by $\sim$ 10 mm aluminium 
plates used as absorbers. When three of the four layers of the CRV are activated, a veto window of $\pm$125 ns 
is provided. A conversion-like signal observed during this time window is assumed to be produced by a cosmic-ray muon. 
The anticipated total dead time is $\sim$5\%. The CRV system should be highly efficient. 
The success of the experiment depends on a detection effectiveness of 99.99\% or higher.

\begin{figure}[!h]
\centering
\includegraphics[width =0.85\textwidth]{figures/jpg/Crv_downstream.jpg}
\caption{The cosmic ray veto covering the Detector Solenoid looking upstream, showing the downstream (CRV-D), left (CRV-L), top (CRV-T) sectors, as well as the hole where the transport solenoid enters the enclosure, Ref. \cite{bartoszek2015mu2e}.}
\label{fig:crv}
\end{figure}

\begin{figure}[!h]
\centering
\includegraphics[width =0.85\textwidth]{figures/png/Crv_module_geometry.png}
\caption{The CRV module geometry and nomenclature. Internal gaps are those between the counters in a di-counter, Ref. \cite{Giovannella_2020}.}
\label{fig:crvmodule}
\end{figure}
\subsection{The Stopping Target Monitor}
A Muon Beam Stop (MBS) is installed at the downstream end of the DS to absorb muons that are not stopped in the stopping target, Ref. \cite{bartoszek2015mu2e}. The MBS is intended to limit the effects of muon decay and capture inside the stop. The magnetic field gradient prevents the majority of the low-energy charged particles created in the MBS from moving upstream. It is made from high-$Z$ minerals and polyethylene. Muons have an extremely short lifetime in high-$Z$ materials, as shown in Figure \ref{fig:muonicatom}, enabling activities to take place before the live window starting at 640 ns. Polyethylene, on the other hand, absorbs protons and neutrons emitted by excited nuclei created during muon capture.
The Stopping Target Monitor, STM, is placed downstream of the MBS to measure the number of muons collected on the stopping target, because of the extremely high x-ray and gamma rates, as shown in Figure \ref{fig:stm}. This number will be used as denominator in $R_{\mu e}$. Mu2e detects the x-ray spectrum released when muons are stopped on a stopping target and fall to the ground state. The experiment uses 357 keV x-ray photons created when muons go from the $2p \rightarrow 1s$ state, Ref. \cite{bobbb}, and aims to measure the number of stopped muons with the 10\% of accuracy. The STM employs an x-ray detection device, a high purity germanium (HPGe) solid-state detector. The detector should only view the target, if possible. A requirement is a good collimation ahead of the detector. It sees the stopping target through a stainless steel pipe 10 cm in diameter and 3 m long, tunneling through the MBS near the DS axis, Ref. \cite{stm}. Collimators inside the pipe give the STM a clear view of the stopping target while obscuring other components, such as the downstream collimator in the TS. HPGe devices are too slow to track individual occurrences. In addition, the germanium lattice is sensitive to radiation damage. To keep radiation damage and rates under control, the STM is approximately 35 m from the stopping target and is well-insulated. An alternative method measures the delayed photons released during the beta decay of $^{27}$Mg, which occurs in 13\% of muon catches. The excited $^{27}$Mg decays to an excited state of $^{27}$Al with a half-life of 9.5 min, emitting an 844 keV photon within ps that can be detected. This method uses the pulsed beam macrostructure (the 1.4 s cycle) to detect only proton batches that are not transmitted to Mu2e.
\begin{figure}[!h]
\centering
\includegraphics[width =\textwidth]{figures/png/Screenshot_20240306_180910.png}
\caption{The Stopping-Target Monitor geometry showing the DS region (left), the End Cap Shielding, sweeper magnet and STM field-of-view collimator. At the far end of the hall (right) is the final $spot-size$ collimator and the STM detector.}
\label{fig:stm}
\end{figure}


\chapter{The Mu2e tracker}\label{chaptertrk}

\textit{The following Chapter will focus on the Mu2e tracker, one of the most important detectors in Mu2e. 
It will give further information about the straw tube tracker panels, 
including the detection mechanism, the mechanical design and the Front End Electronics (FEE).
As previously pointed out, the Mu2e tracker is made up of 216 identical panels. 
Each panel is a mechanical and electrical standalone module. Each panel has 96 drift 
tubes used as fundamental detecting components. I will start by discussing the 
theoretical characteristics of drift tubes and then I will go more in detail with Mu2e tracker features. 
Most of the discussion is based on \cite{kola} and \cite{bobbb}.}

\section{Drift Tubes}
Gas detectors are capable of measuring charged particle coordinates. 
They provide great spatial resolution and detection efficiency at a low cost, Ref. \cite{kola}. 
There are many different gas ionizazion detectors, one of these is the drift tube.
The basic configuration of a drift tube is shown in Figure \ref{fig:drifttube}.
The cylindrical conducting tube, the cathode, is grounded.
A hollow cylindrical conducting tube is grounded and serves as the cathode.
The tube is filled with a combination of noble gas, often Argon, and quench gas. 
A thin sensing wire is suspended along the cylindrical cathode axis. 
The wire, called anode, receives a high voltage. Long and thin drift tubes, known as $straw$ $tubes$, have a similar form.
\begin{figure}[!h]
    \centering
    \includegraphics[width =0.8\textwidth]{figures/png/Screenshot_20240324_232621.png}
    \caption{Schematics of a drift tube, Ref. \cite{kola}.}
    \label{fig:drifttube}
    \end{figure}
Assuming an anode radius $a$, a cathode radius $b$ and using Gauss theorem, the electric field is:
\begin{equation}\label{avalanche}
    E(r)=\frac{1}{r}\frac{\lambda}{2\pi \epsilon}=\frac{1}{r}\frac{V}{ \text{ln}(b/a)} \qquad (a<r<b)
\end{equation}
Where $\lambda$ is linear charge density on the wire and $\epsilon$ is the dielectric constant of the gas.
Thinner wires are typically preferred in drift tubes. A greater electric field near the wire increases the amplification factor 
of the drift tube at the same voltage. Additionally, smaller wires improve spatial resolution, Ref. \cite{kola}. 
\subsection{Gas Ionization}
Two are the interactions that can deposit energy when particles traverse the gas volume: ionization and excitation.
Collisions between a charged particle $C$ and an atom $A$ can result in the ejection of one or more electrons: 
$A \ C \rightarrow A^+ e^- C$, or $A \ C \rightarrow A^{++} e^- e^- C$, if more than one electron is released.
This process is called primary ionisation. The mean energy loss per path length can be determined using the Bethe-Bloch formula.
A noble gas atom $A$ can turn also into an excited state $A^*$ through the interaction $A \ C \rightarrow A^* C$. 
If the excitation energy of $A^*$ is higher than the ionization potential of another species $B$ in the gas, the quencher, the
Penning Effect can produce ionization through $A^*B \rightarrow A B^+ e^-$. In addition, the noble gases can
also form molecular ions through processes such as $A^* A \rightarrow A_{2}^{*} \rightarrow A^+_2 e^-$.
A secondary ionisation can occurr through these processes or through electrons that have sufficient energy for generating more ions.
To compute the average number of electron-ion pairs produced by an initial particle, divide its total energy loss 
by the average energy required to make an electron-ion pair. Due to the energy lost during excitation, this average 
does not match the gas ionisation potential. Measurements showed an average of one electron-ion pair every 
30 eV, with variations depending on gas composition and starting particle. Except for very slow particles, 
this value remains constant regardless of their initial energy.
Without an electric field, electrons and ions created during ionisations spread uniformly. 
Collisions cause them to lose energy and eventually reach thermal equilibrium with the surrounding gas. 
They eventually recombine. An electron maximal range during ionisation is correlated with its initial kinetic 
energy. In a normal temperature and pressure gas, a 10 keV electron may be stopped in approximately 1 mm. 
Ionised electrons often have reduced kinetic energy, leading to a shorter range.
\subsection{Drift of Ions and Electrons}
The electric field accelerates free electrons and ions towards the anode and cathode 
along the field lines. As charges accelerate, they scatter on other particles in the gas, 
losing energy. The directions of motion are randomised, and maximum speeds are set. As a result, 
charges move uniformly along the electric field. This is referred to as the drift velocity of the 
charge. It is superimposed with the thermal motions.
Drift velocities for ions and electrons varies based on several parameters. Ions have a bigger mass 
than electrons and their masses are comparable to those of gas molecules. During collisions, gas 
molecules absorb a significant portion of the energy gained from ions during acceleration. In a drift 
tube detector just a little amount of electric field energy enters the energy associated with ion motion, 
making it comparable to the initial thermal energy before acceleration. 
The ion drift velocity $v_i$ is proportional to the reduced electric field $E/N$ where N is the number density of the gas 
and it is typically $\mathcal{O}(10^3)$ cm/ns, except in the region near to the anode wire where a stronger 
electric field is present. The ion thermal velocity is typically $\mathcal{O}(10^4)$ cm/ns at room temperature.
On the opposite, only a small fraction of the energy is released during elastic collision from electrons, 
so they acquire more energy from the electric field than their thermal energy.
Many different factors impact electron drift velocity. Some gas molecules, such as H$_2$O or CO$_2$, 
can interact with electrons to produce negative ions due to their great electron affinity. In rare situations, 
electrons gather enough energy to exceed gas molecule excitation threshold, resulting in inelastic collisions. 
Electron drift velocity is a complex function of electric field intensity due to several variables influencing 
electron collisions across a large energy range. Figure \ref{fig:drift} shows the electron drift
velocity at different electric field strengths in argon-carbon dioxide mixtures (Ar-CO$_2$) of different
proportions, Ref. \cite{ZHAO1994485}.
\begin{figure}[!h]
    \centering
    \includegraphics[width =0.7\textwidth]{figures/png/Screenshot_20240330_102206.png}
    \caption{Electron drift velocity versus electric field in Ar:CO$_2$ mixtures of different proportions, Ref. \cite{ZHAO1994485}. 
    80\%:20\% Ar:CO$_2$ mixture is the gas used in the Mu2e tracker.}
    \label{fig:drift}
\end{figure}
Electron drift velocities in drift tubes are typically $\mathcal{O}(10^6)$ cm/ns.
This is significantly higher than ion drift velocities and comparable to electron thermal velocity under the same conditions. 
The radial coordinate of an ionisation can be computed using the electron drift velocity and time.
Since drifting electrons and ions are scattered on gas molecules, they also diffuse along their trajectory. 
Electrons diffuse significantly quicker than ions because of their high velocity. Electron diffusion limits 
the intrinsic resolution of drift tubes used to measure incoming particle coordinates. CO$_2$ has internal 
degrees of freedom at low collision energies, preventing electron energies from exceeding thermal energy until 
field strengths above about 2 kV/cm. This improves intrinsic spatial resolution.
\subsection{Avalanche Multiplication}
Electrons can ionise when they face a high electric field near the anode wire. 
The released secondary electrons form tertiary electrons, and so on. Free electrons rapidly multiply, 
resulting in an avalanche. In a drift tubes, where the electron mean free path is about the order of $\mu$m, 
an avalanche develops when the electric field approaches $\mathcal{O}(10)$ kV/cm. 
According to Equation \ref{avalanche}, using $a \sim \mathcal{O}(10^{-3})$ cm, $b \sim \mathcal{O}(1)$ cm and
a normal voltage of 1-2 kV, the avalanches can occur within $\mathcal{O}(100) \ \mu$m from the anode wire. 
Electrons from the avalanche are collected on the anode wire within 1 ns, while positively charged ions move towards the cathode.
Drifting ions mostly generate signals in electrodes via induction. Figure \ref{fig:avalanche} shows the model of an ionisation avalanche.
\begin{figure}[!h]
    \centering
    \includegraphics[width =0.7\textwidth]{figures/png/Screenshot_20240330_182509.png}
    \caption{The model of an ionisation avalanche forming at the anode wire of a proportional tube or chamber, Ref. \cite{kola}. 
    (a) In the drift volume, electrons and ions are generated and drift to their corresponding electrodes. 
    (b) Near the wire, the electron achieves a high enough field to induce secondary ionisation, resulting in an avalanche. 
    (c) The electric field separates charges created during an avalanche. 
    (d) Electrons have higher lateral diffusion than ions, causing the avalanche to expand 
    around the wire and produce a positive charge cloud in the shape of a drop. 
    (e) Electrons from the avalanche reach the anode within nanoseconds, but ions take longer, up to ms, to reach the cathode.}
    \label{fig:avalanche}
\end{figure}
\subsubsection{Avalanche Gain}
When an avalanche develops, the amplification factor is around $10^4-10^6$. The number of electrons freed per unit path length is given
by the first Townsend ionization coefficient $\alpha=\sigma_{ion}n=1/\lambda_{ion}$ and this depends on the electric field $E$, 
as a higher electric field corresponds to a higher kinetic energy of the electron, that increases the ionization cross-section.
The increase $dN$ of the number of electron-ion pairs over a path length $ds$ is, Ref. \cite{kola}:
\begin{equation}
    dN=\alpha(E)Nds
\end{equation}
Solving this equation, we can easily obtain the gas amplification $G$:
\begin{equation}\label{av}
    G=\frac{N(s_a)}{N_0}=\text{exp} \left( \int_{s_0}^{s_a} \alpha(E(s)) ds \right)=\text{exp} \left( \int_{E_{min}}^{E(a)} \frac{\alpha(E(s))}{dE/ds} dE \right)
\end{equation}
where $N_0$ corresponds to unamplified electrons in $s=s_0$ and $E_{min}$ corresponds to the minimum energy 
for ionisation to occurr. The energy distribution depends on the electric field which is position dependent. 
Since the free path is inversely proportional to the particle density in the gas, $E_{min}(\rho)=E_{min}(\rho_0)\rho/\rho_0$.
It is reasonable to say that the coefficient is proportional to the field strength, $\alpha= \beta E$, in the low field region. 
Adding this relation with Equation \ref{avalanche} and \ref{av}:
\begin{equation}
     \text{ln}(G)=\beta \ a \ E(a) \ \text{ln}\left( \frac{E(a)}{E_{min}}\right)
\end{equation}
where $\beta$ can be related to $w_i$, that is the energy spent for one ionisation and its value is equal to $e \Delta V$.
As the voltage drop per unit path length is $dV = E(s)ds = (\alpha/\beta)ds$, we obtain $dN=N \beta dV$. Integrating, we can see that
$\beta= \text{ln}(2)/\Delta V$, so the gain in a drift tube is:
\begin{equation}
     \text{ln}(G)=\frac{ \text{ln}(2)}{\Delta V} \ a \ E(a)  \ \text{ln}\left( \frac{E(a)\rho_0}{E_{min}(\rho_0)\rho}\right) \qquad E(a)=\frac{V}{a ln(b/a)}
\end{equation}
which is the Diethorn's formula. 
Gain measurements with variable $\rho$/$\rho_0$, $a$, and $E(a)$ can provide the parameters $E_{min} (\rho_0)$ and $\Delta V$. 
The gas temperature $T$, pressure $P$ and operating voltage $V$ significantly impact the gain of a drift tube with a particular shape and gas mixture. 
\subsubsection{Quench Gas}
To avoid subsequent avalanches, the drift tube gas combination may contain a quench gas, such as CO$_2$, 
methane, or other hydrocarbons. During an avalanche, photons are created by gas deexcitation and electron 
attachment to electronegative species, resulting in negative ions. Photons can generate ionisations 
outside the primary avalanche zone or create free electrons on the cathode surface, resulting in secondary avalanches.
The difficulty arises when the signal created is not proportional to the deposited energy by the original particle 
and is no longer localised to the energy deposition point. Enough intense photons can induce a chain reaction of 
secondary avalanches, leading to a continuous discharge. The use of quench gas prevents subsequent 
avalanches by absorbing ionising photons before they travel far. A tiny quantity of quench gas during normal operation 
can significantly decrease secondary avalanches and breakdowns.
\subsubsection{Operation Modes of Gaseous Ionization Detectors}
In drift tube detectors, the number of electron-ion pairs formed during an avalanche is 
proportional to the starting number of electrons, as shown in the gain computation. To 
operate in a proportional mode, an appropriate voltage is needed to reduce the effects 
of avalanche charges on the electric field. Figure \ref{fig:gaseous} illustrates how a 
gaseous ionisation detector may work in multiple modes based on the operating voltage. Higher operating 
voltage leads to higher charges on the electrodes. Low voltage causes ionisation charges to recombine 
before reaching electrodes, leading to no signal collection. At higher voltages, in ionisation chamber 
region, charges can drift to electrodes, but the electric field is insufficient for avalanches to occur. 
Increasing the operational voltage leads to drift tubes and proportional counters. When the voltage 
becomes high enough, proportionality is lost. When electrons from an avalanche are collected, 
the high density of positive ions near the anode might affect the electric field. 
Electrons in future avalanches that enter the area between the positive ion cloud and the wire face a 
decreased electric field, resulting in lower amplification. The electric field becomes greater in the 
tail of the avalanche, which is far from the wire than the ion cloud. This range of voltage is called region of 
limited proportionality. When the operating voltage reaches high values, 
avalanches create sufficiently energy photons to cause secondary avalanches 
that propagate across the detector, independently from the quench gas. 
This results in detector saturating the output. This way of operating is 
called breakdown mode, commonly known as the Geiger-Muller mode.
\begin{figure}[!h]
    \centering
    \includegraphics[width =0.5\textwidth]{figures/png/Screenshot_20240330_203416.png}
    \caption{The dependence of particle gain on applied voltage in gaseous ionisation detectors. 
    The numbers on the axes are for orders of magnitude only and they depend on the device geometry and gas concentration. 
    Drift tubes operate in the proportional mode, Ref. \cite{kola}.}
    \label{fig:gaseous}
    \end{figure}
\subsection{Signal Creation and Propagation}
Drift tube signals do not originate from avalanche charges. In this case, the anode wire would 
receive the complete signal within a few ns. Signal pulses are created by charges on electrodes 
caused by electron and ion mobility. The Shockley-Ramo theorem, Ref. \cite{kola}, can be 
used to determine the induced charge and current. 
The Shockley-Ramo theorem yields various important results. The total induced charge of a moving charge 
$q$ is determined by the initial and final positions only.  A charge pair induces the same amount of 
charge on an electrode as the charge collected on it. Furthermore, if all electrodes are treated as 
an unity, their weighted potential will be one. 
If one electrode completely encloses the others, the weighted field in the contained region is always zero. 
This means that the total induced current across all electrodes is always zero. The Shockley-Ramo theorem can 
be applied to the drift tube. In an avalanche with $N$ electron-ion pairs, we evaluate the induced current 
signal on the anode wire. The current on the cathode has an additional negative sign. The weighted 
potential and field in the straw depend on the radial coordinate $r$:
\begin{equation}
    \psi_w(r)=\frac{\text{ln}(b/r)}{\text{ln}(b/a)}
\end{equation}
\begin{equation}
    \textbf{E}_w(r)=\frac{1}{\text{ln}(b/a)}\hat{\textbf{r}}
\end{equation}
As previously mentioned, the drift velocity of ions, $u$, is proportional to the electric field intensity $E$. Therefore, $u$ = $mu \ E$. 
Then the ion trajectory fulfils:
\begin{equation}
    u=\frac{dr(t)}{dt}=\frac{\mu V_0}{r(t)\text{ln}(b/a)}
\end{equation}
When ionisation occurs at the anode, the starting condition can be approximated as $r(0) = a$. So, the solution to the given differential equation is:
 \begin{equation}
    r(t)=a \sqrt{1+\frac{t}{t_0}} \qquad t_0=\frac{a^2 \ln (b / a)}{2 \mu V_0}
    \end{equation}
Here, $t_0$ represents the characteristic time, which is generally on the order of 1 ns. 
The time $t$ falls within the range $0 < t < t_{max}$, where $t_{max} = t_0 [(b/a)^2 - 1]$. 
The induced current and charge from the wire are expressed as:
 \begin{equation}
    I_{i o n}^{i n d}(t)=-(e N) E_w(r) \frac{\mathrm{d} r(t)}{\mathrm{d} t}=-\frac{e N}{2 \ln (b / a)} \frac{1}{t+t_0}
    \end{equation}
    \begin{equation}
        Q_{\text {ion }}^{\text {ind }}(t)=\int_0^t I^{\text {ind }}\left(t^{\prime}\right) \mathrm{d} t^{\prime}=-\frac{e N}{2 \ln (b / a)} \ln \left(1+\frac{t}{t_0}\right)
        \end{equation}
The total charge created on the wire by the travelling electrons is:
\begin{equation}
    Q_e^{i n d}=-e N\left[\psi_w(a)-\psi_w\left(r_{a v g}\right)\right]=-e N \frac{\ln \left(r_{a v g} / a\right)}{\ln (b / a)}
    \end{equation}
    where $r_{avg}$ is the average position of ionisations in the avalanche. When compared to the total charge 
    induced $Q^{ind}_{tot}(t_{max}) = -eN$, electron mobility in the avalanche accounts for just 1-2\%. 
    Positive ion drift from the avalanche accounts for the majority of the signal. A signal propagates 
    to both ends of the drift tube from the avalanche location.
    Signals with distinct frequency components can propagate at varying velocities. This causes the signal to disperse.
    When determining the longitudinal coordinate of an avalanche using the signal arrival time difference between straw tube ends, various complications emerge.
%For low-frequency components of the signal (for a meter-long drift tube, this translates to a frequency much less than 300 MHz), 
%a quasi-electrostatic approach is sufficient. The signals seen at tubes ends are influenced by impedances between and on the electrodes. 
%On the other hand, for the high-frequency components of the signal, the tube needs to be treated as a transmission line. 
%The propagation speed of a signal wave with frequency $\omega$ is then $c/\sqrt{\omega}$ $\epsilon$($\omega$), 
%where $c$ is the speed of light and $\epsilon$ is the dielectric constant of the gas. As $\epsilon$ is a function of $\omega$, 
%components of different frequencies propagate at slightly different velocities. This leads to a dispersion (widening) of the signal. It brings some subtleties when using signal arrival time difference between straw tube ends to determine the longitudinal coordinate of an avalanche.
\section{The tracker panels}
The Mu2e tracker straw tube arrays use the same detection principles as the gaseous ionisation detectors, Ref. \cite{kola}, 
but it has a significant different design and manufacturing improvements to meet the experiment precise requirements.
\begin{figure}[!h]
    \centering
    \includegraphics[width =0.5\textwidth]{figures/png/Screenshot_20240327_000000.png}
    \caption{A picture of the Mu2e straw tube
    compared to a pencil, Ref. \cite{trk}.}
    \label{fig:trkpencil}
    \end{figure}
    \begin{figure}[!h]
        \centering
        \includegraphics[width =0.5\textwidth]{figures/png/Screenshot_20240327_000131.png}
        \caption{The fully assembled tracker panel with its straws, Ref. \cite{trk}.}
        \label{fig:strawtubes}
        \end{figure}
The straw tubes used in the Mu2e tracker \ref{fig:trkpencil} are 5 mm in diameter and 0.33 - 1.17 m in length, 
Ref. \cite{bartoszek2015mu2e}. The straws are wound with two layers of 6 $\mu$m-thick metallized Mylar and a 3 
$\mu$m layer of glue in between. The straw wall is 15 $\mu$m thick: this helps to minimize the amount of materials 
in the tracker, lowering the total energy loss from the observed electrons. Furthermore, it minimises the likelihood 
of significant deflections in electron trajectories, facilitating track reconstruction. As a result, the experiment 
will have a very good momentum resolution. The straw tube anodes are composed of gold-plated tungsten wires 
with a diameter of 25 $\mu$m. The straws and anode wires are tensioned and work-hardened to avoid sagging.
\begin{figure}[!h]
    \centering
    \includegraphics[width =0.6\textwidth]{figures/png/Screenshot_20240326_234405.png}
    \caption{Straw arrangement in a tracker panel, Ref. \cite{trk}.}
    \label{fig:trktubes}
    \end{figure}
To create a perfect electric field with adequate spatial resolutions, the anode wires are oriented 
to the panels with a precision of at least 25 $mu$m in the radial direction and 50 $mu$m in the 
perpendicular direction. All panels are x-ray scanned to accurately measure and document the wire 
locations of straw tube channels. Figure \ref{fig:trkpencil} shows the 
termination mechanism that holds the wire in place.
To increase mechanical strength, a brass tube is joined to either end of the straw using silver epoxied, Ref. \cite{bartoszek2015mu2e}. 
To ensure electrical isolation, a Kapton sleeve is put within the brass tube. An injection-molded plastic insert is then connected 
inside the sleeve. The insert contains a semicylindrical duct that allows gas to enter and exit the straw tube. The insert 
has a groove along its axis and a U-shaped brass anode pin inserted at the end. To avoid slippage, the anode wire is 
epoxied into the groove and soldered to the anode pin.
A T-shaped pin protector protects the anode pin from breaking by covering the groove in the plastic insert. 
The pin protector is epoxied to the insert, with an extra brass ring connecting them. A ground clip 
is silver-epoxied to two adjacent straws on the brass tubes and rings, providing a shared ground connection 
for the straws. 
Signals are sent to a common pre-amplifier (preamp) board via the anode pin pair and grounding clip. 
The FEE will be introduced later. Figure \ref{fig:strawtubes} shows completely built straws in a tracker panel. 
Each panel has 96 straw tubes organised in two staggered layers for improved tracking. Figure \ref{fig:trktubes} 
show the detailed spatial arrangement of the straws, Ref. \cite{trk}. Channels are numbered from 0 (longest) 
to 95 (shortest), starting with the radially innermost straw.
Straws are spaced by 1.25 mm and can expand under gas pressure, Ref. \cite{bartoszek2015mu2e}.
The Mu2e tracker panels employ a gas flow of 80\%:20\% Ar:CO$_2$. 
The vacuum environment within the Detector Solenoid reduces the impact of electron scattering on trajectories. 
However, the tracker panels, particularly the straws, must endure pressure differences. Under normal temperature 
and pressure, each panel must have an average leak rate of 0.014 cm$^3$/min. The nominal operating voltage of 
the straw tube channels is 1450 V. An earlier investigation on a prototype of the tracker showed the straw tube 
gain as 1.25 $\times$10$^4$ at 1250 V and 7 $\times$ 10$^4$at 1425 V. According to Diethorn's calculation, Equation \ref{XXX}, 
the gas gain at 1450 V is around 1 $\times$10$^5$.

\section{The tracker Front-End Electronics}
Before being accessible to the Mu2e Data Acquisition (DAQ) system, the analog signals 
from the tracker straw tube channels need amplification, digitization and packaging. 
The tracker front-end electronics (FEEs) are designed to achieve these goals. The FEEs 
consist of multiple Printed Circuit Boards (PCBs) as shown in Figure \ref{fig:trackerfee}, Ref. \cite{vadimmu2e}.
All PCBs are situated in the outer section of the panel. Pulse timing is measured at the 
end of each straw, in order to be able to measure electron position on the wire. A measure 
of $dE/dx$ is provided, so pattern recognition can be possible, Ref. \cite{bartoszek2015mu2e}. For this purpose each straw has:
\begin{itemize}
    \item two preamp channels, one for each end;
    \item two TDC channels, one for each end;
    \item one ADC channel, measuring sum of both ends;
    \item one High Voltage feed.
\end{itemize}
There are 46,080 preamps and TDCs and 23,040 ADC channels. There are two sides of the panel, one called HV side and one called CAL side. 
\begin{figure}[!h]
\centering
\includegraphics[width =0.8\textwidth]{figures/png/Screenshot_20240131_111836.png}
\caption{An overview of the tracker front-end electronics (FEE), Ref.  
\cite{vadimmu2e}. The preamps and the Analog Motherboard (AMB) 
on the other side of the straws are not shown in the figure.}
\label{fig:trackerfee}
\end{figure}
On either side, there are an Analog Mother Board (AMB) and a Jumper board, 
which task consist of directing signals from the preamps towards the Digital 
Motherboard (DMB) positioned at the center, then to the Digitizer Readout and 
Assembler Controller (DRAC) (mounted on top of the DMB) to be processed and 
temporarily stored. Both the AMBs and the DMB handle low voltage distribution 
and the HV side of the AMB distributes high voltage to the straw anode wires, 
reason why it is called this way. On the AMBs and the DMB there are sensors, 
monitoring environmental variables such as temperature, pressure and humidity. 
The low voltage power supply is fed into the panel through the KEY. The KEY 
contains an optical fiber link and a JTAG interface for communication. The 
frontends components were chosen to sustain high level of radiation.
\subsection{Pre-Amplifiers}
The pre-amplifiers (preamps) are responsible for the initial readout of 
signals from both ends of the straw tube channels. As previously explained, 
the channels are read out from both ends and two adjacent channels are linked 
to the same preamp. Each tracker panel contains 48 preamps on the HV side, 
while an additional 48 preamps are located on the opposite side, the CAL one. 
Preamps are mounted vertically on the AMBs. Preamps are required to have a 
matching 300 $\Omega$ input impedance with the straw, in order to avoid signal 
reflections. The preamps convert the straw tube current signals into voltage signals. 
Signals are amplified and shaped. The preamps on the CAL and HV sides aren't 
exactly the same. The CAL preamps have circuitry that can inject calibration 
pulses into the channel. This capability enables the channel readout electronics 
to be tested without a high voltage source. The preamps on the HV side distribute 
the high voltage supply to individual straw tubes. 
%The voltage gains of individual channels, different from the gas gain, are set by control signal from the DRAC. A bias voltage, adjustable for each side of a channel, is applied to the signals. 
\subsection{Digitizer Readout and Assembler Controller}\label{DRAC}
The brain of a tracker panel is called Digitizer Readout and Assembler Controller (DRAC). 
The DRAC is responsible for digitizing, packaging and temporary storage. It also controls 
panel operations. The schematics of the DRAC board is shown in Figure \ref{fig:drac}. In 
this figure Analog to Digital Converters (ADCs), DDR3 memories and compators are shown. 
The large chips in the centers are the Field-Programmable Gate Arrays (FPGAs). The one 
in the center is the Readout Controller (ROC), which manages communications, monitors 
slow control variables and controls panel operations \ref{ROC}. The left and the right 
ones, which are referred to as the digi FPGAs, are responsible for monitoring data output, 
buffering data and assembling data packages. Each of them refers to 48 straw channels. 
\begin{figure}[!h]
\centering
\includegraphics[width =\textwidth]{figures/png/Screenshot_20240204_115052.png}
\caption{Digitizer Readout and Assembler Controller (DRAC) board schematics, Ref. \cite{drac}. 
The DRAC board is the brain of the tracker panel. In this figure Analog to Digital Converters (ADCs), FPGAs, DDR3 memories and compators are shown.}
\label{fig:drac}
\end{figure}
In figure \ref{fig:flowfee}, signal flow in the tracker FEEs is reported. At the 
beginning the signal coming from both ends of the straw is routed to the preamps. 
After that, the analog signal is sent through the microstrip transmission line to 
the digitizers. In the DRAC, the two biased signals are fed independently to 
zero-crossing comparators, which produce square pulses when the signals exceed 
their respective thresholds. The squared pulses are sent to Time-to-Digital Converters 
(TDCs, firmware based, 16 bits each) implemented in the digi FPGAs, that have the task 
of digitizing the trigger timings, including the arrival time and the time over threshold, 
at a rate of about 62.5 MHz. Besides drift time, TDCs measure time difference across the 
straw to estimate position along the wire and the intrinsic time resolution of TDC is about 
25 ps, adding comparator jitter, noise and other external effects the final resolution is 
$\sim$ 70 ps for time division. Furthermore, an integrator adds voltage signals coming 
from the two straw ends. In data collection, a hit occurs when both ends of a straw channel 
are simultaneously activated. The total is digitized by a 12-bit (10-bit ENOB) ADC at 40 MHz
and then sent to the digi FPGA. The digi FPGA creates a data packet for each hit based on 
TDC and ADC information. This suppresses false triggers caused by random electrical noise. 
The Digitizer receives signals from both ends of four straws and multiplexes them into one 
output buffer before sending a packet of data to the ROC. Signals are sent to the ROC at 
200 MHz. The data packets are then briefly stored in DDR3 memory for further use by the 
DAQ system. It is important to save also the voltage signal, since the pulse height 
information can help us to reject proton background, a significant source of noise 
or to distinguish muons from electrons. The proton signal will appear as a saturated flag, 
since the proton $dE/dx$ is $\times$50 the electron one.
\begin{figure}[!h]
\centering
\includegraphics[width =0.8\textwidth]{figures/png/Screenshot_20240203_135048.png}
\caption{Signal flow through front end electronics, Ref. \cite{bartoszek2015mu2e}.}
\label{fig:flowfee}
\end{figure}

\subsection{Read-Out Controller}\label{ROC} 
The main job of the ROCs (one per panel, 216 in total) 
is to collect data from the digitizer boards, buffer data and 
send them to DAQ. They are based on an FPGA architecture. They 
continuously stream out the zero-suppressed data collected between 
two proton pulses from the detectors, in this case the tracker, to 
the Data Transfer Controllers (DTCs), Ref. \cite{GIOIOSA2023167732}. The buffer 
stage is fundamental, since during the beam inter-spill time (836 ms 
out of each 1333 ms), we want us to be able to take data from cosmic 
rays, even if the rate will be very low. For this purpose, the ROCs 
include external DRAM. The communication is flexible, thanks to the 
programmable nature of digitizer, ROC and DAQ. 
\section{Requirements on tracker Performance}
Here a resume of the requirements that the tracker must satisfy to ensure the success of the experiment is presented, Ref. \cite{trkreq}.
A momentum resolution less than 180 keV/c for a 105 MeV/c electron is needed in the nominal
1 Tesla solenoidal field, as measured at the front face of the tracker volume (before
passing through any tracker related material). Non-Gaussian tails, particularly any
high-side tail, must be controlled such that the DIO background results in much less
than one event at design sensitivity. To reach this, simulation results indicate that 
a single straw requires around 4 cm of longitudinal and 200 $\mu$m of transverse resolution for drift path lengths. 
It must have an acceptance of approximately 20\% for conversion electrons.
The tracker must operate in an ambient vacuum (< $10^{-4}$ Torr).
It should be able to withstand a rate of 5 MHz per straw (highest rate straw) 500 ns after the spill. This is
for background studies. The nominal experiment live time starts 700 ns after the spill.

\appendix

\chapter{Charged particle in a magnetic field}\label{appendix1}
The motion of a charged particle in a magnetic field can be described by the Lorentz force:
\begin{equation}
    \mathbf{F}=q \ \mathbf{v}\times\mathbf{B}
\end{equation}
where $q$ is the particle charge, $\mathbf{v}$ is the particle speed and $\mathbf{B}$ is 
the magnetic field.
A charged particle moving in a uniform solenoidal field describes the 
combination of a free trajectory and a circular motion, namely a helix, with the property:
\begin{equation}\label{partincamp}
    |\mathbf{B}|\rho=\frac{p_\perp}{|q|}
\end{equation}
where $\rho$ is the helix radius, $\p_\perp$ is the transverse momentum component.
This is the simplest example, although more complex magnetic fields can 
produce a more complex trajectories. In this appendix, I will go over how to 
use a gradient to accelerate particles and also how particles perpendicularly 
drift in a curved magnetic field. The intensity of a magnetic field with a 
non-null gradient changes with the position. The force derived from the gradient is 
proportional to the particle magnetic momentum $\mu$, where $E$ represents 
energy. It can be written as:
\begin{equation}
   \mathbf{F}=-\mu \nabla \mathbf{B} \qquad \qquad \mu=\frac{c^2 p_\perp^2}{2 E B}
\end{equation}
The force alters the direction of momentum, not the particle's energy. If the 
gradient is strong enough, it can reverse the direction of motion and reflect 
the particle like a magnetic mirror. Complex gradients can trap particles in 
certain regions or prevent them from staying there for too long.
The other important feature is the use of curved magnetic fields. In a curved 
solenoid, the particle orbit drifts perpendicular to the bending plane. 
The drift velocity $v_D$ and total displacement $D$ can be determined from 
the path along the curved solenoid $S$:
\begin{equation}
    v_D=\frac{m \gamma c}{e B R}(v_\parallel ^2+\frac{1}{2}v_\perp ^2)
\end{equation}
\begin{equation}
    D \propto p S (\frac{1}{cos \theta} + cos \theta)
\end{equation}
In the equations above, parallel and perpendicular refer to the magnetic field direction 
and $R$ represents the solenoid bending radius. $\theta$ is 
the angle of the helix axis and the magnetic field and the sign of the drift 
relies on the sign of the charge. These characteristics show that a curved 
solenoid can be used to separate particle beams of opposite charge.
\appcaption{Appendix A: Charged particle in
magnetic field}
\chapter*{Acknowledgements}
I would like to extend my deepest gratitude to my supervisor, P. Murat, for his invaluable guidance.
His teachings went far beyond textbooks, showing me how to think critically and like a physicist.
His patience and support have been essential throughout my time at Fermilab.
\vspace{1.5mm}
\\
My sincere thanks also go to Professor S. Donati for his support and understanding during this year.
I am grateful to him for giving me the opportunity to have such a significant experience at Fermilab,
and for his impartial guidance in helping me shape my career.
\vspace{1.5mm}
\\
I would like to express my immense gratitude to Namitha and Sridhar,
who have always helped me with work and software issues, but most importantly,
for their friendship, which has been a truly pleasant discovery.
\vspace{1.5mm}
\\
Vorrei ringraziare la mia famiglia materna e paterna, per l'immenso sostegno, l'affetto e la considerazione che mi hanno sempre dato.
Grazie per aver condiviso con gioia i traguardi importanti e per avermi aiutato nei momenti complessi. Grazie per avermi fatto sentire
vicino a voi nonostante la grande lontananza da casa.
\vspace{1.5mm}
\\
Grazie a tutti gli amici: a quelli dell'infanzia, delle superiori, del mare, al gruppo della pallavolo, a 
quelli dell'Università, ai bimbi del Fermilab, a chi ho conosciuto più
recentemente, a quelli che sono stati importanti punti di riferimento nella mia vita 
e che l'hanno profondamente segnata, a chi ho avuto modo di conoscere meno,
a quelli con cui ho passato momenti felici, brutti, a coloro con cui ho riso, 
pianto, litigato, viaggiato. Vi porto sempre con me.
Un ringraziamento particolare alle mie amiche, Giulia, Rebe, Giulia, Eli, Silvi, 
Vane, Marti, soprattutto per questo ultimo anno (e non solo), per le parole di conforto, il sostegno e le risate.
\vspace{1.5mm}
\\
Vorrei esprimere la mia più profonda gratitudine a Gianluca e Giuseppina,
che si sono presi cura di me in modo davvero impeccabile e 
che mi hanno dato gli strumenti per smettere di vedere il mondo solo in nero.
\vspace{1.5mm}
\\
Questo ultimo ringraziamento sarà compreso solo da chi ha avuto un animale nella vita. Un grande grazie a Baghy e Chloè, per avermi ascoltato ripetere 
dalle elementari fino, con qualche pausa nel frangente, all'Università e per avermi dato conforto nei momenti di solitudine. In particolare vorrei menzionare 
la profonda protezione di Baghy in qualsiasi momento difficile.
\vspace{0.5mm}
\\
Alle persone che ho menzionato e non, grazie per avermi reso chi sono oggi.
\vspace{\stretch{2}}
\cleardoublepage


\bibliographystyle{unsrt} %{plain}
\bibliography{chapters/Bibliografia.bib}

\end{document}
% -----------------------------------------------------------------
