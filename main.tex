%   Il progetto nasce dal template per il frontespizio realizzato da Marco Antonio Corallo, che ringrazio. Seguono alcuni commenti per evidenziare la presenza di alcuni pacchetti che mi sono stati utili per la stesura della tesi. Chiaramente, dipende tutto dal tipo di lavoro che uno vuole eseguire, che determina anche le diverse esigenze. Durante la stesura ho passato molto tempo su siti e forum a cercare di risolvere alcuni probelmi di formattazione, ma in generale Latex è stato piuttosto versatile. 

% Tipo di documento. L'uso di twoside implica che i capitoli inizino sempre con la prima pagina a sinistra, eventualmente lasciando una pagina vuota nel capitolo precedente. Se questa cosa è fastidiosa, è possibile rimuoverlo. 
\documentclass[a4paper, twoside,openright]{report}


\usepackage{[caption}
\usepackage{tikz}
%\usepackage{lineno}
\usepackage{notoccite}


% \usepackage[absolute,overlay]{textpos}
% \usepackage{onimage}
% Dimensione dei margini
\usepackage[a4paper,top=3cm,bottom=3cm,left=3cm,right=3cm]{geometry} 
% Dimensione del font
\usepackage[fontsize=13pt]{scrextend}
% Lingua del testo
\usepackage[english]{babel}
% Lingua per la bibliografia
\usepackage[square,sort,comma,numbers]{natbib}
% Codifica del testo
\usepackage[utf8]{inputenc} 
% Encoding del testo
\usepackage[T1]{fontenc}
\usepackage{caption}
\usepackage{subcaption}
% Permette di generare testo fittizio. Mi è stato utile 
% per capire quale sarebbe stata l'impostazione del 
% testo nella pagina prima che scrivessi un determinato paragrafo

% Per ruotare le immagini
\usepackage{rotating}
% Per modificare l'header delle pagine 
\usepackage{fancyhdr}     

% Librerie matematiche
\usepackage{amssymb}
\usepackage{mathrsfs}
\usepackage{amsmath}
\usepackage{amsthm}         

% Uso delle immagini
\usepackage{graphicx}
% Uso dei colori
\usepackage[normalem]{ulem}  % for striking out
\usepackage{xcolor}         
% \usepackage{comment}         
\usepackage{todonotes}         
\usepackage{soul}         
% Uso dei listing per il codice
\usepackage{listings}          
% Per inserire gli hyperlinks tra i vari elementi del testo 
\usepackage{hyperref}     
% Diversi tipi di sottolineature

% -----------------------------------------------------------------

% Modifica lo stile dell'header
\pagestyle{fancy}
\fancyhf{}
\lhead{\leftmark}
\rhead{\textbf{\thepage}}
\fancyfoot{}
\setlength{\headheight}{12.5pt}

% Rimuove il numero di pagina all'inizio dei capitoli
\fancypagestyle{plain}{
  \fancyfoot{}
  \fancyhead{}
  \renewcommand{\headrulewidth}{0pt}
}


% Stile del codice
\lstdefinestyle{codeStyle}{
    % Colore dei commenti
    commentstyle=\color{black},
    % Colore delle keyword
    keywordstyle=\color{black},
    % Stile dei numeri di riga
    numberstyle=\tiny\color{black},
    % Colore delle stringhe
    stringstyle=\color{violet},
    % Dimensione e stile del testo
    basicstyle=\ttfamily\footnotesize,
    % newline solo ai whitespaces
    breakatwhitespace=false,     
    % newline si/no
    breaklines=true,                 
    % Posizione della caption, top/bottom 
    captionpos=b,                    
    % Mantiene gli spazi nel codice, utile per l'indentazione
    keepspaces=true,                 
    % Dove visualizzare i numeri di linea
    numbers=left,                    
    % Distanza tra i numeri di linea
    numbersep=5pt,                  
    % Mostra gli spazi bianchi o meno
    showspaces=false,                
    % Mostra gli spazi bianchi nelle stringhe
    showstringspaces=false,
    % Mostra i tab
    showtabs=false,
    % Dimensione dei tab
    tabsize=2
} \lstset{style=codeStyle}

% Stile di codice per dimensioni maggiori, in cui ho avuto bisogno di un testo più picolo (ad esempio se si vuole inserire del codice che ha linee molto lunghe). Per usare questo stile piuttosto che il precedente, usare 

% \lstset{style=longBlock}
%  % inserire il codice...
% \lstset{style=codeStyle}

% Il secondo comando consente di tornare allo stile precedente 
\lstdefinestyle{longBlock}{
    commentstyle=\color{teal},
    keywordstyle=\color{Magenta},
    numberstyle=\tiny\color{gray},
    stringstyle=\color{violet},
    basicstyle=\ttfamily\scriptsize,
    breakatwhitespace=false,         
    breaklines=true,                 
    captionpos=b,                    
    keepspaces=true,                 
    numbers=left,                    
    numbersep=5pt,                  
    showspaces=false,                
    showstringspaces=false,
    showtabs=false,                  
    tabsize=2
} \lstset{style=codeStyle}

% Togliendo il commento al comando che segue, si inseriscono nella bibliografia anche le fonti presenti in Bibliography.bib ma non citati direttamente con il comando \cite
% \nocite{*}

% Margini prima e dopo blocchi di codice, per avere più distanza
\lstset{aboveskip=20pt,belowskip=20pt}

% Modifica dello stile dei riferimenti, con il testo in cyano
\hypersetup{
    linkcolor=black,
    citecolor=black
}

% Aggiunti definizioni, teoremi, linea e listing
\newtheorem{definition}{Definizione}[section]
\newtheorem{theorem}{Teorema}[section]
\providecommand*\definitionautorefname{Definizione}
\providecommand*\theoremautorefname{Teorema}
\providecommand*{\listingautorefname}{Listing}
\providecommand*\lstnumberautorefname{Linea}

\raggedbottom

%\newcommand{\cgs}[1]{{\textcolor{brown}[\textcolor{red}{\bf{GS: }}{ \textcolor{brown}{#1]}}}}             
%\newcommand{\cmc}[1]{{\textcolor{blue}[\textcolor{magenta}{\bf{MC: }}{ \textcolor{blue}{#1]}}}}
\newcounter{apdxsection}
\newcommand\unappendix{\par
  \setcounter{apdxsection}{\value{section}}%
  \setcounter{section}{\value{savesection}}%
  \setcounter{subsection}{0}%
  \gdef\thesection{\@arabic\c@section}}
\setcounter{tocdepth}{4}
\setcounter{secnumdepth}{4}


%%%%%%%%%%%%%%%%%%%%%%%%%%%%%%%%%%%%%%%%%%%%%%%%%%%%%%%%%%%%%%%%%%%%%%%%%%%%%%
% commands
%%%%%%%%%%%%%%%%%%%%%%%%%%%%%%%%%%%%%%%%%%%%%%%%%%%%%%%%%%%%%%%%%%%%%%%%%%%%%%
\newcommand {\blue}      {\color{blue}}
\newcommand {\green}     {\color{ForestGreen}}
\newcommand {\red}       {\color{red}}
\newcommand {\purple}    {\color{purple}}
\newcommand {\violet}    {\color{violet}}

%%%%%%%%%%%%%%%%%%%%%%%%%%%%%%%%%%%%%%%%%%%%%%%%%%%%%%%%%%%%%%%%%%%%%%%%%%%%%%
% editing commands
%%%%%%%%%%%%%%%%%%%%%%%%%%%%%%%%%%%%%%%%%%%%%%%%%%%%%%%%%%%%%%%%%%%%%%%%%%%%%%
\newcommand {\hlite}[1]    {{\red \ul{#1}}}       % needs xcolor, soul
\newcommand {\del}[1]    {{\blue \sout{#1}}}
\newcommand {\add}[1]    {{\red #1}}

% -----------------------------------------------------------------
\begin{document}

\begin{titlepage}
    \begin{center}
\begin{figure}[!htb]
    \centering
    \includegraphics[keepaspectratio=true,scale=0.4]{figures/eps/cherubinFrontespizio.eps}
\end{figure}
\begin{figure}[!htb]
    \centering
    \includegraphics[keepaspectratio=true,scale=0.45]{figures/eps/logo_pant541.eps}
\end{figure}

%\begin{center}
    %\LARGE{University of Pisa}
    %\vspace{5mm} \\
    \Large{Department of Physics E. Fermi}
    \vspace{5mm}
    \\ 
    \Large{Master's Degree in Physics}
\end{center}

\vspace{15mm}
\begin{center}
    {\LARGE{\bf Commissioning of the DAQ system,\\ \vspace{3mm} pre-pattern recognition studies and\\ \vspace{5mm} planning for the calibration of\\ \vspace{7mm} the Mu2e tracker at Fermilab}}
\end{center}
\vspace{25mm}

\begin{minipage}[t]{0.47\textwidth}
	{\large{Supervisors:}{\normalsize\vspace{3mm}
	\bf\\ \large{Dr. Pavel Murat} \normalsize\vspace{3mm}\bf \\ \large{Prof. Simone Donati}}}
\end{minipage}
\hfill
\begin{minipage}[t]{0.47\textwidth}\raggedleft
	{\large{Candidate:}{\normalsize\vspace{3mm} \bf\\ \large{Sara Gamba}}}
\end{minipage}

\vspace{25mm}
\hrulefill
\\\centering{\large{Academic Year 2023/2024}}

\end{titlepage}

\cleardoublepage
\thispagestyle{empty}
\vspace*{\stretch{1}}
\begin{flushright}
\itshape 
dediche
\end{flushright}
\vspace{\stretch{2}}
\cleardoublepage

\chapter*{Acknowledgements}
I would like to extend my deepest gratitude to my supervisor, P. Murat, for his invaluable guidance.
His teachings went far beyond textbooks, showing me how to think critically and like a physicist.
His patience and support have been essential throughout my time at Fermilab.
\vspace{1.5mm}
\\
My sincere thanks also go to Professor S. Donati for his support and understanding during this year.
I am grateful to him for giving me the opportunity to have such a significant experience at Fermilab,
and for his impartial guidance in helping me shape my career.
\vspace{1.5mm}
\\
I would like to express my immense gratitude to Namitha and Sridhar,
who have always helped me with work and software issues, but most importantly,
for their friendship, which has been a truly pleasant discovery.
\vspace{1.5mm}
\\
Vorrei ringraziare la mia famiglia materna e paterna, per l'immenso sostegno, l'affetto e la considerazione che mi hanno sempre dato.
Grazie per aver condiviso con gioia i traguardi importanti e per avermi aiutato nei momenti complessi. Grazie per avermi fatto sentire
vicino a voi nonostante la grande lontananza da casa.
\vspace{1.5mm}
\\
Grazie a tutti gli amici: a quelli dell'infanzia, delle superiori, del mare, al gruppo della pallavolo, a 
quelli dell'Università, ai bimbi del Fermilab, a chi ho conosciuto più
recentemente, a quelli che sono stati importanti punti di riferimento nella mia vita 
e che l'hanno profondamente segnata, a chi ho avuto modo di conoscere meno,
a quelli con cui ho passato momenti felici, brutti, a coloro con cui ho riso, 
pianto, litigato, viaggiato. Vi porto sempre con me.
Un ringraziamento particolare alle mie amiche, Giulia, Rebe, Giulia, Eli, Silvi, 
Vane, Marti, soprattutto per questo ultimo anno (e non solo), per le parole di conforto, il sostegno e le risate.
\vspace{1.5mm}
\\
Vorrei esprimere la mia più profonda gratitudine a Gianluca e Giuseppina,
che si sono presi cura di me in modo davvero impeccabile e 
che mi hanno dato gli strumenti per smettere di vedere il mondo solo in nero.
\vspace{1.5mm}
\\
Questo ultimo ringraziamento sarà compreso solo da chi ha avuto un animale nella vita. Un grande grazie a Baghy e Chloè, per avermi ascoltato ripetere 
dalle elementari fino, con qualche pausa nel frangente, all'Università e per avermi dato conforto nei momenti di solitudine. In particolare vorrei menzionare 
la profonda protezione di Baghy in qualsiasi momento difficile.
\vspace{0.5mm}
\\
Alle persone che ho menzionato e non, grazie per avermi reso chi sono oggi.
\vspace{\stretch{2}}
\cleardoublepage

\begin{abstract}
\noindent
The primary objective of the Mu2e experiment is to search for neutrino-less coherent $\mu \rightarrow e$ conversion in the 
field of an aluminum nucleus, $\mu^- \text{Al} \rightarrow e^-\text{Al}$. This process's signal is a monoenergetic electron 
with an energy of approximately 104.97 MeV, Ref. \cite{bartoszek2015mu2e}. 
In the Standard Model (SM), the branching ratio for this process, or for the similarly 
intriguing $\mu \rightarrow e \gamma$ decay, is expected to be of the order of $\mathcal{O}(10^{-54})$. 
These values are well below the sensitivity of any current experimental capabilities, but several models of physics beyond the 
SM predicts a much higher relative rate that could reach an observable level in some super-symmetry scenarios. 
The SINDRUM II experiment established an upper limit on muon conversion 
at $7 \times 10^{-13}$ (90\% C.L.) \cite{SINDRUMII:2006dvw}, and the Mu2e collaboration aims to improve this limit by four orders of magnitude. 
Observing this process would provide unambiguous evidence of physics beyond the Standard Model and its search is complementary to 
the direct searches for new physics currently carried out at the CERN Large Hadron Collider. 

Mu2e will make use of a sophisticated setup to achieve its goals. An 8.9 GeV proton beam from Fermilab will strike a tungsten target, producing $\mu$s through $\pi$s decay. 
These particles are transported via a series of superconducting solenoids to an aluminum stopping target, where the conversion may occur. 
The system includes multiple components to precisely identify the conversion electrons and distinguish them from various backgrounds, including a straw tracker, 
an electromagnetic calorimeter, a Cosmic Ray Veto and a Stopping Target Monitor.

One of the most important Mu2e subsystems is the tracker, which must provide very good momentum resolution to distinguish the monochromatic 
conversion electron signal from the background. Energy loss in the tracker needs to be minimal; thus, the experiment will use a straw 
tube tracker, Ref. \cite{bobbb}. Its annular shape follows the helical trajectories of conversion electrons in the magnetic field. 
The tracker consists of panels containing arrays of straw tubes arranged like hap chords. It is required to measure the conversion electrons with a momentum 
resolution better than 180 keV/c. 

My activity at Fermilab has focused on the commissioning of the Mu2e DAQ system and the Vertical Slice Test (VST) of the tracker. 
The VST encompasses the entire testing chain from the straws to the readout, to processed data on disk. Our ultimate goal was to conduct 
a time calibration of the first assembled station of the tracker, aiming for a longitudinal position resolution better than 4 cm, and to determine the channel-to-channel delay and the drift velocity.

Before proceeding, validating data from the tracker panels is necessary. The tracker's readout system has undergone several 
tests to validate its performance, ensuring accurate and reliable data collection. This includes testing readout logic and firmware 
with simulated and real data to confirm functionality, and developing methods to monitor and validate data accuracy, including identifying and addressing cross-talk and noise issues.

Our objective is to gather data from cosmic rays to develop the time calibration of the tracker. First, it was important to assess the feasibility of 
performing this calibration with a vertical orientation of the first station. For this purpose, straight cosmic rays were simulated in one station to 
understand how the varying luminosity of the straws can impact timing calibration and hit segment reconstruction. This calibration is crucial for achieving 
precise momentum resolution, relying on accurate reconstruction of straw hit positions. Our goal is to determine the drift velocity in the straws and the time 
offset from channel to channel. This involves reconstructing straight cosmic segments within each station by assessing whether a straw has been hit and the 
relative orientations of the straws. From there, we can locate intersection points of straws on a 2D plane, assuming these to be the particle hit coordinates. 
This process is repeated for all pairs of straws, followed by fitting all the points together. The actual position of a hit within a straw can then be reconstructed, 
identifying biases by comparing the MC hits position with the reconstructed ones. Our target is to achieve a longitudinal resolution better than 4 cm, ensuring that every 
bias remains below this threshold. During data collection, each straw will undergo calibration by correlating the time difference between ends with the reconstructed position.
\end{abstract}

\tableofcontents
\listoffigures
\chapter{Charged Lepton Flavor Violation}
\textit{This chapter offers a concise overview of the fundamental theoretical and experimental components essential to understand the objectives of the Mu2e experiment at Fermilab and the work conducted for this Thesis. The introduction to the Standard Model and certain extensions serves to justify the investigation of Charged Lepton Flavor Violation (CLFV), since it would be a clear signal of new physics beyond the current theories. Fundamental bibliography for this chapter can be found in Ref. \cite{Bernstein_2013} and \cite{clfv_signorelli}.}
\section{Theoretical Introduction}
\subsection{The Standard Model}
The Standard Model provides an excellent description of elementary particles and their interactions. It describes three out of four the fundamental forces known to this day: electromagnetism interaction, weak interaction and strong interaction. This theory is based on the gauge symmetry group $U(1)_Y \times SU(2)_L \times SU(3)_C$. The first two terms describe the electroweak interaction, $Y$ indicates the hypercharge and $L$ refers to the fact that this acts only on the left handed components of the fields. The last term describes the strong interaction and $C$ indicates the color charge.
The Standard Model contains 25 elementary fields, shown in Fig. \ref{fig:sm}.

\begin{figure}[!h]
\centering
\includegraphics[width =0.8\textwidth]{figures/pdf/Standard_Model_of_Elementary_Particles.pdf}
\caption{Elementary particles of the Standard Model.}
\label{fig:sm}
\end{figure}


\subsubsection{Bosons}\label{bosons}
A boson is a particle with zero or integer spin, which follows Bose-Einstein statistics.
There are twelve fundamental bosons, that are the mediators of interactions as the $\gamma$, $Z$, $W^{\pm}$ for the electroweak interaction and as the eight gluons for the strong interaction.
The Higgs is a complex scalar weak isospin doublet that is responsible for the mechanism through which fermions and
bosons acquire mass and also explains the origin of $U (1)_{EM}$ through the spontaneous
symmetry breaking of the $SU (2)_L \times SU (1)_Y$ in the electroweak sector. Also mesons, that are composed of a quark and antiquark pair, are bosons. Bosons can be either massive, as $Z$, Higgs and $W^{\pm}$, or massless, as the photon and gluons.

\subsubsection{Fermions}
A fermion is a particle characterized by a non-integer spin, i.e. $1/2$, $3/2$ and follows the Fermi-Dirac statistics. The Pauli exclusion principle must be respected. Fermions are divided in two categories, leptons and quarks, depending on the forces through which they interact.  The arrangement of fermions into three generations is dictated by various properties, including mass among other properties, with more massive particles assigned to higher generations, as depicted in Fig.\ref{fig:sm}. Particles of the second and third generations exhibit instability and decay into first-generation particles. Leptons do not interact through strong interaction because they are not color charged, so they interact only via weak and electromagnetic interaction. They are categorized into two groups based on the electric charge: $e$, $\mu$, $\tau$ are the charged leptons and $\nu_e$, $\nu_{\mu}$, $\nu_{\tau}$ are the neutral ones. These particles form doublets of flavour. Neutrinos, due to their neutral nature, interact only weakly so their detection is extremely challenging. The six quarks participate in all known interactions. The known quarks are, in ascending order of mass and generation: up ($u$) and down ($d$), strange ($s$) and charm ($c$), bottom ($b$) and top ($t$) and their antiparticles. They primarily interact with each other through the strong force by gluons exchange. Free quarks have never been observed due to confinement, since they carry color charge. Confinement of quarks is a fundamental aspect of the theory of quantum chromodynamics (QCD) which describes the strong nuclear force. Quarks combine to form color-neutral particles known as hadrons, classified into baryons and mesons. Baryons consist of an odd number of quarks, while mesons, as mentioned in Subsection \ref{bosons}, are composed of a quark and an antiquark. Since quarks have weak electric charge and isospin, they can interact with each other and other fermions through weak and electromagnetic interactions.

\subsection{History of flavor}
The concept of flavor, namely the presence of three duplicates for every family of elementary fermions, is a fundamental aspect in particle physics. This principle is implemented within the Standard Model by introducing three copies of the gauge representations of fermion fields. This view began to take shape in the late 1940s. The origin can be traced back to the experiment conducted by Conversi, Pancini and Piccioni in 1947. These experiment revealed that negative muons, called at that time $mesotrons$, did not undergo nuclear capture as expected. They decayed in electrons, similarly to positive muons, therefore they could not be Yukawa particles. In the same year, Powell and his group identified a two-step decay process ($\pi \rightarrow \mu \rightarrow e$), distinguishing the pion from the muon. Bruno Pontecorvo suggested that the muon could be a sort of $isomer$ of the electron, leading to the idea of a second generation of elementary fermions. Rochester and Butler discovered unusual events in cosmic rays pictures, later identified as $V-particles$ (later discovered that they originated from neutral kaons). This was the first hint of the existence of a second generation of quarks. In 1950 the search for decay $\mu \rightarrow e \gamma$ began. This decay was not found, leading to the principle of conservation of leptons. On the hadron side, the second generation of quarks was established in the mid-70s, involving the GIM mechanism and the discovery of the charm quark. Meanwhile, on the leptonic side, the upper limit on the branching ratio of $\mu^+ \rightarrow  e^+ \gamma$ was set in 1955. The discovery of parity violation in the late 1950s suggested the weak interaction is mediated by bosons. \textit{Feinberg started thinking that $\mu^+ \rightarrow  e^+ \gamma$ could occur at a level of $10^{-4}$ if the bosons existed, through a loop with neutrino and a boson}. This lead to the two-neutrino hypothesis, suggesting that the neutrino coupled to the muon differs from that coupled to the electron, thereby prohibiting $\mu^+ \rightarrow  e^+ \gamma$. The existence of two neutrinos was verified at Brookhaven National Laboratory, with the scattering of two neutrinos coming from $\pi$, that produced only muons. After the observation of CP violation in neutral kaon decay, a third generation of quarks was hypothesized. After the discoveries of the $\tau$ (1976), the $b$ quark (1977), $t$ quark (1995) and $\nu_{\tau}$ (2000), a complete picture was achieved and the concept of flavor was consolidated in the Standard Model.


\subsection{Overview of CLFV}
There are three different lepton flavors: the electron-lepton number $L_e$, the muon-lepton number $L_{\mu}$ and the tau-lepton number $L_{\tau}$. In Table \ref{tab:leptons}, the quantum numbers assigned to each lepton are displayed.
 \begin{center}  
\begin{table}[!h]
\centering
\renewcommand{\arraystretch}{1.5}
\begin{tabular}{c c c c}
\hline
Lepton & $L_e$ & $L_{\mu}$ & $L_{\tau}$\\
\hline
$e^-/e^+$ & $+1 \ /-1$ & 0 & 0 \\
$\nu_{e}/\bar{\nu}_{e}$ & $+1 \ /-1$ & 0 & 0 \\
$\mu^-/\mu^+$ & 0 & $+1 \ /-1$ & 0 \\
$\nu_{\mu}/\bar{\nu}_{\mu}$ & 0 & $+1 \ /-1$ & 0 \\
$\tau^-/\tau^+$ & 0 & 0 & $+1 \ /-1$\\
$\nu_{\tau}/\bar{\nu}_{\tau}$ & 0 & 0 & $+1 \ /-1$ \\
\hline
\end{tabular}
\caption{Lepton numbers assigned to neutrinos and charged leptons.}
\end{table}\label{tab:leptons}
\end{center}
In the Standard Model (SM) defined with massless left-handed neutrinos, Lepton Flavor (LF) is a conserved quantity \cite{universe8060299}. Experimental observations have demonstrated that, as they travel, neutrinos exhibit flavor oscillations, which implies that they must have non-zero masses and mixing angles. This phenomenon represents also a violation of the conservation of the lepton flavor. The Standard Model, while successful in many aspects, fails to explain phenomena like neutrino masses and the consequent flavor oscillations. Since neutrinos get their masses through renormalizable Yukawa interactions
with the Higgs, the predicted CLFV transitions are suppressed by sums over $(\Delta m^2_{i j}/M^2 _W)^2$, as calculated in \cite{MARCIANO1977303} and as shown in Section \ref{massiveneutrinos}, where $\Delta m^2_{ij}$ is mass-squared difference between the neutrino mass eigenstates $i$, $j$ and $M_W$ is the $W$ boson mass. The neutrino mass difference is very small ($\Delta m^2 _{i j} \leq 10^{-3}$ eV$^2$) with respect to the $W$ boson mass so the expected branching ratios reach unmeasurable values, below $10^{-50}$. Experimental studies of the lepton flavor violating process could open a window to new physics. Moreover, lepton flavor constitutes an accidental symmetry within the SM, not related to the gauge structure of the theory but coming
from its particle content, especially from the absence of RH neutrinos. Minor deviations from the Standard Model can easily give rise to extra occurrences of lepton flavor violation, leading to notable rates of CLFV.
There are various extensions of the Standard Model that could potentially be examined in the upcoming experimental searches for CLFV.
In Section \ref{leptonsector}, I will talk about lepton sector and how the lepton numbers are conserved. In Section \ref{massiveneutrinos}, a brief introduction of CLFV with massive neutrinos will be given.
In Section \ref{2higgs} and in Section \ref{susy}, I present a short discussion on Two Higgs Doublet Model and the CLFV in Super-symmetry respectively.
\subsection{Lepton sector in Standard Model}\label{leptonsector}
In the SM, only one Higgs field $\Phi$ exists. The fermions masses and the mixing term arise from the couplings of fermions with Higgs field. In the following, I will call the left-handed $i$-th quarks doublets and leptons doublets as $Q_{L,i}=(u_{L,i} \ d_{L,i})^T$ and $L_{L,i}=(\nu_{L,i} \ e_{L,i})^T$: $u_i$ will be the up-type quark, $d_i$ the down-type quark, $\nu_i$ the neutrino and $e_i$ the charged lepton. These are $SU(2)$ doublets, while $u_{R,j}$, $d_{R,j}$ and $e_{R,j}$ will be the right-handed up-type, down-type quarks and the right charged lepton of the $j$-generation respectively. There is no right-handed neutrino. The Yukawa coupling of fermions with the Higgs field $\mathscr{L}_Y$ is the sum of two terms: $\mathscr{L}_e$ that describes the leptonic component (Eq.\ref{leptoniccomponent}) and $\mathscr{L}_q$ that describes the quark one (Eq.\ref{quarkcomponent}).
\begin{equation}\label{leptoniccomponent}
    -\mathscr{L}_e=\left(Y_e\right)_{i j} \bar{L}_{L i} e_{R j} \Phi+ \text{ h.c. }
\end{equation}
\begin{equation}\label{quarkcomponent}
        -\mathscr{L}_q=\left(Y_u\right)_{i j} \bar{Q}_{L i} u_{R j} \widetilde{\Phi}+\left(Y_d\right)_{i j} \bar{Q}_{L i} d_{R j} \Phi+\text { h.c. }
\end{equation}

where the term $Y_f \ (f  =  u,d,e)$ describe the 3$\times$3 Yukawa complex matrices. $\widetilde{\Phi} \equiv i \tau_2 \Phi^*$ is the conjugate Higgs field. The mass terms of fermions, characterized by the $m_f \bar{f}_L f_R$ form, originate from the breaking of the $SU(2)_L \times U(1)$ symmetry caused by the vacuum expectation value of the Higgs field, as in Eq.\ref{higgs}.
\begin{equation}\label{higgs}
\langle\Phi\rangle=\frac{1}{\sqrt{2}}\left(\begin{array}{l}
0 \\
v
\end{array}\right) \qquad v \simeq 246 \mathrm{GeV}
\end{equation}
The fermion mass is given by:
\begin{equation}
\left(m_f\right)_{i j}=\frac{v}{\sqrt{2}}\left(Y_f\right)_{i j} \qquad f=u, d, e
\end{equation}
Since a right-handed neutrino does not appear in the Lagrangian, neutrinos have no mass, as it is formulated in the SM. The Yukawa matrices can be diagonalized through unitary rotations of the fields, as it follows:
\begin{equation}
Y_f=V_f \hat{Y}_f U_f^{\dagger} \qquad f=u, d, e
\end{equation}
where $ \hat{Y}_f $ is the diagonal Yukawa matrix. It is possible to label fermions in the rotated basis as $f^{\prime}$, so $f_L=V_f f^{\prime}_L$ and $f_R=V_f f^{\prime}_R $. Since $V_f$ and $U_f$ are unitary, the rotation will not affect the neutral interactions term and the kinetic terms, as we can see in $\bar{f}_L \gamma^\mu f_L=\bar{f}_L \gamma^\mu\left(V_f^{\dagger} V_f\right) f_L=\bar{f}_L^{\prime} \gamma^\mu f_L^{\prime}$. The coupling between fermion and Higgs boson will be:
\begin{equation}
-\mathscr{L}_{h \bar{f} f}=\frac{(\hat{m}_f)_{i j}}{v} \bar{f}_{L i}^{\prime}f^{\prime}_{R j} h+\text{h.c.} \qquad f=u, d, e
\end{equation}
where $\hat{m}_f$ denotes the diagonalized mass matrix: it is clear that there are no flavor-violating terms.
In quark sector flavor-violation arises from the rotations in Eq.\ref{quarkviolation} in the charged-current interactions with the $W$ bosons:
\begin{equation}\label{quarkviolation}
\begin{array}{c}
      { \displaystyle 
\mathscr{L}_{C C}  =\frac{g}{\sqrt{2}}\left(\bar{u}_L \gamma^\mu d_L+\bar{\nu}_L \gamma^\mu e_L\right) W_\mu^{+}+\text {h.c.} }\\
 {\displaystyle=\frac{g}{\sqrt{2}}\left(\bar{u}^{\prime}_L \gamma^\mu\left(V_u^{\dagger} V_d\right) d_L^{\prime}+\bar{\nu}^{\prime}_L \gamma^\mu\left(V_\nu^{\dagger} V_e\right) e_L^{\prime}\right) W_\mu^{+}+\text {h.c.}}
\end{array}
\end{equation}

The violation comes from the fact that $V_u \neq V_d$. The mixing is controlled by the Cabibbo-Kobayashi-Maskawa matrix $V_{CKM}\equiv V^{\dagger}_u V_d $ \cite{PhysRevLett.10.531}. Meanwhile, as the lepton sector contains massless neutrinos, $V_{\nu}$ can be arbitrarily chosen as $V_{\nu}=V_e$. From the preceding discussions, it becomes evident that in the SM with massless neutrinos, there is no occurrence of LFV in any form. The Lagrangian $\mathscr{L}_Y$ is invariant under three indipendent global $U(1)$ rotations, resulting in the conservation of three lepton family numbers: $L_e$, $L_\mu$ and $L_\tau$. Furthermore, if the Yukawa coupling Lagrangian includes supplementary terms involving the lepton fields, flavor violation can occur in the lepton sector. Examples of such additions could be a neutrino mass term or a second Higgs doublet.

\subsection{CLFV in the Standard Model with massive neutrinos}\label{massiveneutrinos}
The first evidence against the hypotesis of massless neutrinos emerged with the solar neutrino problem. In the 1960s, the solar neutrino detection experiment at Homestake revealed that the observed number of solar neutrinos, generated by fusion in the Sun, was significantly lower than the anticipated value based on the standard solar model, given that the detector was only sensitive to $\nu_e$ \cite{PhysRevLett.20.1205}. Consistent results were replicated in subsequent experiments employing radiochemical and Cherenkov detectors, discovering neutrino oscillations. These oscillation firmly established non-zero neutrino masses. The lepton flavor-violating neutrino oscillations showed that the global $U(1)$ symmetries associated with the lepton family numbers are not fundamental symmetries. A correction to standard model is needed to include neutrino mass terms. This is possible adding a right-handed neutrino singlet $\nu_R$ or some non-renormalizable operators.
\\
In the first case, an additional term $\Delta \mathscr{L}_D$ to the Yukawa coupling should be introduced:
\begin{equation}
-\Delta \mathscr{L}_D=\left(Y_\nu\right)_{i j} \bar{L}_{L i} \nu_{Rj} \widetilde{\Phi}+\text { h.c. }
\end{equation}
Similarly to other fermions, a Dirac mass term $m_{\nu} \bar{\nu}_L \nu_R$ is generated through simmetry breaking:
\begin{equation}
\left(m_\nu^D\right)_{i j}=\frac{v}{\sqrt{2}}\left(Y_\nu\right)_{i j}
\end{equation}
In this case, the small neutrino masses can be explained only if a very small term $Y_\nu$ is considered ($\leq 10^{-12}$) \cite{clfv_signorelli}. This brings us considering the second hypotesis: adding non-renormalizable operators can introduce Majorana masses for left-handed neutrinos alone. The corresponding $\Delta \mathscr{L}_M$ can be written as:
\begin{equation}
-\Delta \mathscr{L}_M=\frac{1}{2}\left(m_\nu^M\right)_{i j} \overline{\nu_{L i}^C} \nu_{L j}+\text{ h.c.}
\end{equation}
where $\overline{\nu_{L }^C} $ is the charge-conjugated fields. This term violates lepton number and requires an operator of dimension 5 \cite{wein} to be consistent with SM symmetries. A minimal Lagrangian is given by:
\begin{equation}
-\Delta \mathscr{L}_{M \text { eff}}=\frac{\mathcal{C}_{i j}}{\Lambda}\left(\overline{L_{L i}^C} \tau_2 \Phi\right)\left(\Phi^T \tau_2 L_{L j}\right)+\text { h.c.}
\end{equation}
where the $\Lambda$ term represents a mass scale characteristic of extra degrees of freedom and the $C_{ij}$ is an antisymmetric charge conjugation matrix. The corresponding Majorana mass term is:
\begin{equation}
\left(m_\nu^M\right)_{i j}=\frac{\mathcal{C}_{i j} v^2}{\Lambda}
\end{equation}
In this case, the small neutrino masses can be explained only if $\Lambda > > v$: this seems to appear more natural than the previous case. No matter how the extra neutrino mass factor is expressed precisely, a physical basis diagonalizing the mass matrix is determined, resulting in $V_{\nu} \neq V_e$ in Equation \ref{quarkviolation}. Lepton mixing is described by $U_{PMNS} \equiv V_{\nu}^{\dagger} V_e$, which is similar to the CKM matrix. The Pontecorvo-Maki-Nakagawa-Sakata (PMNS) matrix is the term that is typically used to refer to it. As it is in the basis diagonalizing charged lepton masses and it diagonalizes the neutrino mass matrix, $U_{PMNS}$ also explains the mixing between neutrino flavor eigenstates $\nu_{\alpha}$ and mass eigenstates $\nu_i$:
\begin{equation}
\nu_\alpha=\sum_{i=1,2,3}\left(U_{\mathrm{PMNS}}\right)_{l i} \ \nu_i \qquad l=e, \mu, \tau
\end{equation}
In addition to the neutral LFV observed in neutrino oscillations, the mixing outlined by $U_{PMNS}$ can in principle give rise to processes known as Charged Lepton Flavor Violations, i.e. LFV that involves charged leptons. The new Feynman diagrams are loops involving neutrinos and W bosons, as $\mu \rightarrow e \gamma$ in Fig.\ref{fig:mutoegamma} and $\mu N \rightarrow e N$ in Fig.\ref{fig:mutoeN}. 



\begin{figure}[!h]
     \begin{subfigure}[b]{0.4\linewidth}
         \centering
         \includegraphics[scale = 0.2]{figures/png/Screenshot_20240217_171058.png}
         \subcaption{$\mu \rightarrow e \gamma$ process \cite{universe8060299}.}
         \label{fig:mutoegamma}
     \end{subfigure}
     \begin{subfigure}[b]{0.7\linewidth}
         \centering
         \includegraphics[scale = 0.5]{figures/jpg/1_erkKoywyuFzJmMv4PKpc9Q.jpg}
         \subcaption{$\mu N \rightarrow e N$ process.}
         \label{fig:mutoeN}
     \end{subfigure}
     \caption{Some of CLFV processes.}
        \label{fig:three graphs2}
\end{figure}
In the SM, each of these mechanisms is significantly inhibited. Using the example of $\mu \rightarrow  e \gamma $, the branching ratio (BR) of this process may be computed as follows:

\begin{equation}
\begin{aligned}
B R(\mu \rightarrow e \gamma) & =\frac{3 \alpha}{32 \pi}\left|\sum_{i=2,3} U_{\mu i}^* U_{e i} \frac{\Delta m_{1 i}^2}{M_W^2}\right|^2 \\
& =\frac{3 \alpha}{32 \pi}\left(\frac{1}{4}\right) \sin ^2 2 \theta_{13} \sin ^2 \theta_{23}\left|\frac{\Delta m_{13}^2}{M_W^2}\right|^2,
\end{aligned}
\end{equation}

where $\alpha$ is the fine structure constant, $U_{\mu i}$ and $U_{ei}$ are corresponding elements in the PMNS matrix, $\Delta m_{1i}^2$ is the neutrino squared mass differences, $M_W$ is the $W$ boson mass and $\theta_{13}$ and $\theta_{23}$ are rotating angles in PMNS matrix parametrization. The expression yields $B R(\mu \rightarrow e \gamma) \sim \mathcal{O}(10^{-54})$. The big discrepancy in mass between neutrinos and the $W$ boson results in an extraordinarily small value for $|\Delta m_{13}^2/M_W|$. Equivalent suppression mechanisms are evident in other CLFV processes. The rates predicted by the Standard Model are extremely small, making them impractical for detection in any experiment. On the other hand, numerous Beyond the Standard Model (BSM) theories incorporate mechanisms that substantially amplify CLFV rates, a topic to be addressed in the subsequent section. The small value of SM CLFV rates implies that the detection of any CLFV processes in experiments would unequivocally indicate the presence of physics beyond the SM.
\subsection{Beyond the Standard Model}
Numerous Beyond the Standard Model (BSM) theories propose mechanisms that could contribute to CLFV processes, potentially yielding detectable rates in experiments. Here, we highlight a selection of BSM theories known for their CLFV contributions. It is important to note that this list is not comprehensive; for further studies, additional reviews are available in \cite{clfv_signorelli} and \cite{universe8060299}.
\subsubsection{CLFV in Supersymmetry}\label{susy}
Supersymmetry (SUSY) is a theoretical framework that has oriented experiments in the CLFV reasearch for many years. On one hand, models with SUSY broken at energies close to electro-weak scale have given solution to the hierarchy problem, i.e. how to maintain the Higgs mass significantly smaller than the Planck scale ($\sim$10$^{19}$ GeV). On the other hand, the suppression of CLFV processes is due to the wide separation of the neutrinos and $W$ masses, which can be mitigated by introducing SUSY partners of neutrinos and $W$ bosons. This suggests that CLFV processes should have been observable earlier, unless SUSY breaking occurs at or near the electroweak scale ($\sim 10^2$ GeV) \cite{clfv_signorelli}. In this framework, each elementary particle has a superpartner,  with the same quantum numbers except for spin: a boson is the superpartner of a fermion and vice versa. A superpartner of a lepton is called $slepton$. If there is no common eigenstate base between lepton and $slepton$'s mass matrices then a physical $slepton$ will be a superposition of flavors. In this case a loop diagram can lead to CLFV, as shown in Fig.\ref{fig:susy}. Despite the similar topology to that of the SM contribution (Fig.\ref{fig:mutoegamma}), the typical SUSY mass is expected to be much higher than that of the neutrinos. Predictions for the branching ratio of this process vary among different SUSY models, contingent upon specific mechanisms and particle masses. These rates can undergo significant enhancement. For example, in an $SU(5)$ SUSY grand unified theory, the computed branching ratio could reach $\mathcal{O}(10^{-14})$ for a slepton mass on the order of $\mathcal{O}(10^{-14}$ GeV/c$^2)$, a value measurable for upcoming experiments.

\begin{figure}[!h]
\centering
\includegraphics[width =0.4\textwidth]{figures/png/Screenshot_20240218_105920.png}
\caption{SUSY contribution to $l_i \rightarrow l_j\gamma$, through $sleptons$ mass mixing \cite{universe8060299}.}
\label{fig:susy}
\end{figure}


\subsubsection{Two Higgs Doublet Model}\label{2higgs}
Although the Standard Model (SM) incorporates only one Higgs boson, there are no constraints against the presence of additional Higgs fields. One straightforward example of a comprehensive theory featuring multiple Higgs fields is the type III two Higgs doublet model (2HDM), where two Higgs bosons exist, each interacting with fermions and possessing a vacuum expectation value \cite{Harnik_2013}. Generally, the Lagrangian incorporating extra Higgs fields post-electroweak symmetry breaking can be written as:
\begin{equation}
-\mathscr{L}=m_i \bar{f}_{L i} f_{R i}+\left(Y^\alpha\right)_{i j} \bar{f}_{L i} f_R{ }_j h^\alpha+\text { h.c. }+\ldots
\end{equation}
Non renormalizable terms of higher dimensions are omitted. $(Y^\alpha)_{i j}$ represents couplings to a single scalar field and the contributions from different Higgses are summed over. The non-zero-off-diagonal terms in $(Y^\alpha)_{i j}$ give rise to flavor violating Yukawa couplings. From accelerator and precision experiments, constraints on the off-diagonal coupling of the 125 GeV Higgs boson can be obtained.
\subsubsection{Leptoquark Models}
Leptoquarks (LQs) are theoretical particles initially proposed within the Pati-Salam model \cite{PhysRevD.10.275}. Each LQ is linked to both baryon number ($B$) and lepton number ($L$). In various LQ models the quark and lepton sectors are unified. This unification allows for direct coupling between quarks and leptons via the exchange of LQs. Consequently, specific CLFV processes like $K_L^0 \rightarrow e \mu$ and $\mu N \rightarrow e N$ are mediated by LQs. Constraints on LQ models arise from both collider experiments and rare decay searches. Direct searches at ATLAS and CMS have excluded scalar LQs of the first and second generations with masses below $\sim$1 TeV. Indirect searches provide constraints on the mass-coupling plane, where regions with higher couplings and lower LQ masses correspond to higher branching ratios. Additionally, LQ models must satisfy constraints related to proton stability, as some models involve LQs that could mediate proton decay. To mitigate this, the corresponding LQs must either have extremely high masses or their related couplings must be exceedingly small \cite{DORSNER20161}.
\subsubsection{Additional Neutral Gauge Boson}
Grand unified theories (GUTs) are constructed based on extended gauge groups in the pursuit of a more fundamental model. At lower energies, these extended gauge groups are believed to break down to the direct product of the Standard Model (SM) gauge group $SU(3) \times SU(2) \times U(1)$ along with an additional $U(1)$ factor. The neutral gauge boson associated with this $U(1)$ group can mix with the original SM neutral gauge boson, resulting in two mass eigenstates, namely $Z$ and $Z'$. Additionally, extended gauge theories require the introduction of additional fermion fields to cancel anomaly-free currents beyond those of $SU(5)$. These $new$ fermions can mix with the known SM fermions possessing the same electric and color charges, consequently affecting their couplings with gauge bosons. The appearance of off-diagonal terms in neutral current couplings to fermions can lead to flavor-changing couplings to $Z$ and $Z'$. Certain CLFV processes, such as $\mu \rightarrow eee$ and $\mu-e$ conversion, receive tree-level contributions through intermediate $Z$ and $Z'$ bosons. Further insights into the phenomenology of the $Z'$ boson can be found in \cite{Leike_1999}. The search for the existence of $Z'$ bosons is conducted through channels like $Z' \rightarrow \bar{f}f$ at hadron colliders. Mass lower limits of $Z'$ from various specific models are listed in \cite{zyla}, primarily falling within the low TeV range. Particularly, mass lower limits reported in CLFV final states $e\mu$, $e\tau$ and $\mu\tau$ range between 3.5 TeV and 4.5 TeV. Upper limits of $Z \rightarrow l_1 l_2$ couplings to the normal Z boson are also provided in Table \ref{tab:upperlimits}.
%Quello che segue è un esempio di codice. E' possibile modificare il linguaggio per il synyax highlight, aggiungere parole chiave... E' tutto disponibile nella guida del pacchetto \texttt{listings}.

%\lstinputlisting[language=C++]{listings/png/code1.cpp} 
\section{Experiments looking for CLFV}
Charged Lepton Flavor Violation has not been observed yet, despite ongoing efforts to detect such violations across various channels in both dedicated and general-purpose experiments. Some of these efforts are documented in Table \ref{tab:upperlimits}, which presents their respective experimental upper limits.
\begin{center}  
\begin{table}[!h]
\centering
\renewcommand{\arraystretch}{1.5}
\begin{tabular}{c c c c c c}
\hline
Reaction & Present limit & C.L. & Experiment &  Year & Ref.\\
\hline
$\mu^+\ \rightarrow \ e^+ \ \gamma$& $7.5 \times 10^{-13}$ & 90\% & MEG II & 2024 & \cite{megiicollaboration2024search}\\
$\mu^+ \ \rightarrow \ e^+ \ e^+ \ e^-$ & $1.0 \times 10^{-12}$ & 90\% & SINDRUM & 1988 & \cite{SINDRUM:1987nra} \\
$\mu^- \ \text{Ti}\ \rightarrow \ e^- \ \text{Ti}$ &  $6.1 \times 10^{-13}$ & 90\% & SINDRUM II & 1998 & \cite{titanium}\\
$\mu^- \ \text{Au}\ \rightarrow \ e^- \ \text{Au}$ & $7.0 \times 10^{-13}$ & 90\% & SINDRUM II & 2006 & \cite{SINDRUMII:2006dvw} \\
$\mu^+ \ e^- \ \rightarrow \ \mu^- \ e^+$ & $8.3 \times 10^{-11}$ & 90\% & SINDRUM & 1999 & \cite{Willmann:1998gd}\\
$\tau \ \rightarrow \ e \ \gamma$ & $3.3 \times 10^{-8}$ & 90\% & BaBar & 2010 & \cite{Aubert_2010}\\
$\tau \ \rightarrow \ \mu \ \gamma$ & $4.4 \times 10^{-8}$ & 90\% & BaBar & 2010 & \cite{Aubert_2010}\\
$\tau \ \rightarrow \ e \ e \  e$ & $2.7 \times 10^{-8}$ & 90\% & Belle & 2010 & \cite{Hayasaka_2010}\\
$\tau \ \rightarrow \ \mu \ \mu  \ \mu$ & $2.1 \times 10^{-8}$ & 90\% & Belle & 2010 & \cite{Hayasaka_2010} \\
$B^0 \ \rightarrow \ \mu \ e$ & $2.8 \times 10^{-9}$ & 90\% & LHCb & 2013 & \cite{PhysRevLett.111.141801}\\
$B^0 \ \rightarrow \ \tau \ e$ & $2.8 \times 10^{-5}$ & 90\% & BaBar & 2008 & \cite{PhysRevD.77.091104}\\
$B^0 \ \rightarrow \ \tau \ \mu$ & $2.2 \times 10^{-5}$ & 90\% & BaBar & 2008 & \cite{PhysRevD.77.091104}\\
$K_L^0 \ \rightarrow \ \mu \ e$ & $4.7 \times 10^{-12}$& 90\% & BNL E871 & 1998 & \cite{BNL:1998apv}\\
$K^+\ \rightarrow \ \pi^+ \ \mu^+ \ e^-$ & $2.1 \times 10^{-10} $ & 90\% & BNL E865 & 2005 & \cite{PhysRevD.72.012005}\\
$K_L^0 \ \rightarrow \ \pi^0 \ \mu^+ \ e^-$ & $ 4.4 \times 10^{-10}$ & 90\% & KTeV & 2008 & \cite{KTeV:2007cvy}\\
$Z^0 \ \rightarrow \ \mu \ e$ & $1.7 \times 10^{-6}$ & 95\% &  LHC ATLAS & 2014 & \cite{Aad_2014} \\
$Z^0 \ \rightarrow \ \tau \ e$ & $1.7 \times 10^{-6}$ & 95\% &  LEP OPAL & 1995 & \cite{akers}\\
$Z^0 \ \rightarrow \ \tau \ \mu$ & $9.8 \times 10^{-6}$ & 95\% &  LEP DELPHI & 1997 & \cite{abreu}\\
$\pi^0 \ \rightarrow \ \mu \ e$ & $8.6 \times 10^{-9}$ & 90\% & KTeV & 2008 & \cite{KTeV:2007cvy}\\
$\Upsilon (1s) \ \rightarrow \ \mu \ \tau $ & $6.0 \times 10^{-6}$ & 95\% & CLEO & 2008 & \cite{Love_2008}\\
\hline
\end{tabular}
\caption{Experimental upper limits for a variety of CLFV processes, Ref. \cite{clfv_signorelli}.}
\end{table}\label{tab:upperlimits}
\end{center}


\subsection{$\mu$ Channels}
Currently, the most promising channel is the one that includes muon processes. When a proton beam interacts with a target, pions and kaons are produced, that subsequently decay in muons. Muon lifetime is long enough to form a muon beam and we are able to reach intensities of 10$^8 \div 10^{11} \  \mu$/s. There are three primary CLFV channels involving muons, with distinct sensitivities to effective lagrangian terms: $\mu^+ \rightarrow e^+ \gamma$, $\mu^- N \rightarrow e^- N$ and $\mu^+ \rightarrow e^+ e^+ e^-$. The following paragraphs will discuss the experimental challenges and future perspectives for each of these channels. As seen in Table \ref{tab:upperlimits}, these channels have the lowest branching ratio limits. Muons have small mass, that results in a limited number of decay modes. Figure \ref{fig:muchannel} shows how the branching ratio limitations of muon uncommon decays have rapidly improved over the last several decades. The next-generation experiments aim to improve by many orders of magnitude. Muon rare decay studies can also provide theory differentiation power combining results of the three channels. All CLFV extensions to SM can be described by the following Lagrangian, \cite{doi:10.1146/annurev-nucl-100809-131949}:
\begin{equation}\label{LCF}
\mathscr{L}_{C L F V}=\frac{m_\mu}{(1+\kappa) \Lambda^2} \bar{\mu}_R \sigma_{\mu \nu} e_L F^{\mu \nu}+\text{h.c.}+\frac{\kappa}{(1+\kappa) \Lambda^2} \bar{\mu}_L \gamma_\mu e_L\left(\sum_{q=u, d} \bar{q}_L \gamma^\mu \bar{q}_L\right)+\text{h.c.}
\end{equation}
$m_\mu$ is the muon mass and  $F^{\mu \nu}$ is the electromagnetic field tensor. This toy Lagrangian includes two parameters.
$\Gamma$ is the effective energy scale of the new physics and $\kappa$
is the relative strengths of the two operators. The first term in the Lagrangian is a magnetic-moment-type operator and describes all three pro-
cesses mentioned above, it is generated by any loop with some new particle that can be either virtual and real.
The second one corresponds to a four-fermion operator, which mediates $\mu N \rightarrow eN$ at tree level and the other two processes at one-loop level.
The Mu2e experiment can probe an effective mass scale up to $\mathcal{O}$(10$^4$ TeV) with its designed sensitivity assuming $\kappa$ $\gg$ 1.
On the other hand, $\mu \rightarrow e\gamma$ experiments are more sensitive when $\kappa$ $\ll$ 1; the dominant magnetic
moment type term determines the other two processes have lower rates in such a case. In order to learn more about the new
physics, one needs to combine information involving the rates of a different CLFV processes, Ref. \cite{osti_1042577}.
The corresponding parameter space (the $\Gamma-\kappa$ plane) is shown in Figure \ref{fig:muchannelbr}, Ref. \cite{doi:10.1146/annurev-nucl-100809-131949}.
\begin{figure}[!h]
\centering
\includegraphics[width =0.8\textwidth]{figures/png/Screenshot_20240307_161549.png}
\caption{History and outlook of branching ratio limits in muon rare decay modes, Ref. \cite{MARCIANO1977303}.}
\label{fig:muchannel}
\end{figure}
\begin{figure}[!h]
\centering
\includegraphics[width =0.8\textwidth]{figures/png/Screenshot_20240313_120457.png}
\caption{Sensitivity of $\mu \rightarrow e\gamma$ and $\mu N \rightarrow eN$ experiments to the new physics
scale $\Gamma$ as a function of $\kappa$ as defined in Equation \ref{LCF}, Ref. \cite{CGroup:2022tli}. The blue region is the New
Physics phase space excluded by SINDRUM-II, Ref. \cite{SINDRUMII:2006dvw}. The red region represents the
limit set by MEG, Ref. \cite{megi}, while the dashed red line represents the region that is
excluded by MEG-II, Ref. \cite{megiicollaboration2024search}. The solid (dashed) blu line is the
expected limit that would be set by Mu2e, Ref. \cite{universe9010054}.}
\label{fig:muchannelbr}
\end{figure}
\subsubsection{$\mu^+ \rightarrow e^+ \gamma$}
\paragraph{The MEG Experiment}
\cite{megi}
\begin{figure}[!h]
\centering
\includegraphics[width =0.4\textwidth]{figures/png/Screenshot_20240307_150038.png}
\caption{.}
\label{fig:meg}
\end{figure}
\paragraph{MEG II}
\cite{megiicollaboration2024operation}
\cite{megiicollaboration2024search}
\begin{figure}[!h]
\centering
\includegraphics[width =0.4\textwidth]{figures/png/Screenshot_20240307_140235.png}
\caption{.}
\label{fig:meg2}
\end{figure}
\begin{figure}[!h]
\centering
\includegraphics[width =0.4\textwidth]{figures/png/Screenshot_20240307_140116.png}
\caption{.}
\label{fig:meg22}
\end{figure}
\subsubsection{$\mu^+ \rightarrow e^+ e^-  e^+ $}
\paragraph{SINDRUM I}
\cite{sindrumi}
\begin{figure}[!h]
\centering
\includegraphics[width =0.4\textwidth]{figures/png/The-SINDRUM-I-detector-in-the-horizontal-operating-orientation.png}
\caption{.}
\label{fig:sindrumi}
\end{figure}
\paragraph{The Mu3e Experiment}

\begin{figure}[!h]
\centering
\includegraphics[width =0.4\textwidth]{figures/png/Screenshot_20240307_161651.png}
\caption{Longitudinal cross section of the detector. Positrons tracks in red, electron
track in blue \cite{hesketh2022mu3e}}
\label{fig:mu3e}
\end{figure}
\subsubsection{$\mu^- N \rightarrow e^- N $}
\paragraph{SINDRUM II}
\cite{SINDRUMII:2006dvw}
\begin{figure}[!h]
\centering
\includegraphics[width =0.4\textwidth]{figures/png/Screenshot_20240307_163120.png}
\caption{.}
\label{fig:sindrumii}
\end{figure}
\paragraph{Mu2e}
\paragraph{COMET}
\cite{Abramishvili_2020}
\begin{figure}[!h]
\centering
\includegraphics[width =0.4\textwidth]{figures/png/Screenshot_20240307_152133.png}
\caption{Schematic layout of COMET (Phase-II) and COMET Phase-I (not in scale).}
\label{fig:comet}
\end{figure}
\paragraph{Mu2e II}
\cite{dukes}
\subsection{$\tau$ Channels}
The tau lepton is, in principle, a very promising source of CLFV decays. Thanks to the
large tau mass (m $\tau$ $\sim$ 1.777 GeV), many CLFV channels can be investigated: $\tau^\pm \rightarrow \mu^\pm \gamma$ ,
$\tau^\pm\rightarrow e^\pm\gamma$ , $\tau \rightarrow 3l$, $\tau\rightarrow l \ h$, with $h$ a light hadron and $l$ an electron or a muon. Table \ref{tab:upperlimits}
lists the current best limits on the $\tau$ CLFV searches, and Figure \ref{fig:tauchannel} shows the individual results from the
BaBar, Ref. \cite{PhysRevD.77.091104}, Belle [180] and the LHCb [181] experiments, together with their combination.
From the experimental point of view, however, a difficulty immediately arises: the
tau is an unstable particle, with a very short lifetime ($\tau$ = 2.91 $\times$ 10$^{-13}$ s). As a result, $\tau$
beams cannot be realized, and large tau samples must be obtained in intense electron or
proton accelerators, operating in an energy range where the tau production cross section is
large. At $e^+ e^-$ and $pp$ collider machines, the majority of the tau particles are not produced
at rest, which means that, unlike the muon searches discussed before, here we need to deal
with decays-in-flight. Thanks to the boost, the decay products could get energy values up
to several GeV, which experimentally poses the challenge to deliver wide-range calibrations for the detectors (from a few hundreds of MeV to several GeV).
For all these searches, events contain a pair of taus in which one tau undergoes SM decay (tag side),
while the signal side is selected on the basis of the appropriate topology of each individual channel.
The tagging side accepts the leptonic ($\tau$ $\rightarrow$ $l \nu\bar{\nu}$) and 1-prong hadronic decays, while on the signal side,
CLFV candidates are selected on the basis of the appropriate topology of each individual channel.
The following paragraphs discuss the current best limits for some of these experimental
searches from experiments at $B$-factories and $pp$ colliders, Ref. \cite{universe8060299}.
\begin{figure}[!h]
\centering
\includegraphics[width =0.4\textwidth]{figures/png/Screenshot_20240313_133439.png}
\caption{\cite{Amhis_2021}.}
\label{fig:tauchannel}
\end{figure}
\subsubsection{$ \tau \rightarrow l \gamma$}
\iffalse
The $\tau \rightarrow l\gamma$ decay, where $l$ is a light lepton $(e, \mu)$, has been one of the most popular
CLFV tau channels. The signal is characterized by a $l^\pm - \gamma$ pair with an invariant mass
and total energy in the center-of-mass (CM) frame (ECM) close to mτ = 1.777 GeV and
√
s/2, respectively. The dominant irreducible background comes from τ-pair events containing hard photon radiation and one of the τ leptons decaying to a charged lepton. The
remaining backgrounds arise from the relevant radiative processes, e
+e
− → e
+e
−γ and
e
+e
− → µ
+µ
−γ and from hadronic τ decays where a pion is misidentified as an electron
or muon. For this decay channel, the current best limits comes from the BaBar and the
Belle collaborations. BaBar collected (963 ± 7) × 106 τ decays near the Υ(4S), Υ(3S) and
Υ(2S) resonances. In the BaBar detector [179], charged particles are reconstructed as tracks
with a 5-layer silicon vertex tracker and a 40-layer drift chamber inside a 1.5 T solenoidal
magnet. A CsI(Tl) electromagnetic calorimeter is used to identify electrons and photons. A
ring-imaging Cherenkov detector is used to identify charged pions and kaons. The flux
return of the solenoid, instrumented with resistive plate chambers and limited streamer
tubes, is used to identify muons. Signal decays are identified by two kinematic variables:
the energy difference ∆E = ECM −
√
s/2 and the beam energy constrained τ mass obtained
from a kinematic fit after requiring the CM τ energy to be √
s/2 and after assigning the
origin of the γ candidate to the point of closest approach of the signal lepton track to the
e
+e
− collision axis (mBC). Figure 21 shows the distributions of the events for the two decay
channels in mBC vs. ∆E. The red dots are experimental points, the black ellipses are the 2σ
signal contours and the yellow and green regions contain 90\% and 50\% of MC signal events. The searches yield no evidence of signals, and the experiment set upper limits on the
branching fractions of B(τ
± → e
±γ) < 3.3 × 10−8 and B(τ
± → µ
±γ) < 4.4 × 10−8 at 90%
confidence level [140].
The Belle experiment [180] reported comparable limits using a data analysis based
on 988 fb−1
and a strategy similar to that of BaBar. Kinematical selections on missing
momentum and opening angle between particles are used to clean the sample. Figure 22
shows the two-dimensional distribution of∆E/
√
s vs. mBC. The signal events have mBC ∼
mτ and ∆E/
√
s ∼ 0. The most dominant background in the τ
± → µ
±γ (τ± → e
±γ )
search arises from τ
+τ
− events decaying to τ
± → µ
±νµντ (τ
± → e
±νeντ) with a photon
coming from initial-state radiation or beam background. The µ
+µ
−γ and e
+e
−γ events
are subdominant, with their contributions falling below 5\%. Other backgrounds such as
two-photon and e
+e
− → qq¯ (q = u, d, s, c) are negligible in the signal region
No significant excess over background predictions from the Standard Model is observed, and the 90% C.L. upper limits on the branching fractions are set at B(τ
± → µ
±γ) ≤
4.2 × 10−8 and B(τ
± → e
±γ) ≤ 5.6 × 10−8
[183]. With the full dataset expected for the
Belle II experiment [184] (the upgrade of Belle), 50 ab−1
, the upper limit for the branching
fraction of LFV decays τ will be reduced by two orders of magnitude
\fi
\subsubsection{$ \tau \rightarrow 3l $}
\iffalse
The signature for τ → 3l (l = e, µ) is a set of three charged particles, each identified
as either an e or a µ, with an invariant mass and energy equal to that of the parent τ lepton.
In the BaBar [185] and Belle [141] analyses, all the six different combinations were
explored. Events are preselected requiring four reconstructed tracks and zero net charge,
selecting only tracks pointing toward a common region consistent with τ
+τ
− production
and decay. The polar angles of all four tracks in the laboratory frame are required to
be within the calorimeter acceptance range, to ensure good particle identification. The
search strategy consists of forming all possible triplets of charged leptons with the required
total charge and of looking at the distribution of events in the (mBC , ∆E) plane (mBC and
∆E are defined as in the previous section). The backgrounds contaminating the sample
can be divided in three broad categories: low multiplicity e
+e
− → qq¯ (q = u, d, s, c)
events, QED events (Bhabha or µ
+µ
− depending on the specific channel) and SM τ
+τ
−
events. These background classes have distinctive distributions in the (mBC, ∆E) plane.
The e
+e
− → qq¯ (q = u, d, s, c) events tend to populate the plane uniformly, while
QED backgrounds fall in a narrow band at positive values of ∆E, and τ
+τ
− backgrounds
are restricted to negative values of both ∆E and mBC due to the presence of at least one
undetected neutrino. Figure 20 shows the resulting limit for all the combinations to be at
the level of a few 10−8
for both collaborations.
Even if the results are not yet competitive to those from B-factories, it is interesting
to note that experiments at the LHC have also been looking for the τ → 3µ decay. The
ATLAS experiment [186] performed a search for the neutrinoless decay τ
− → µ
−µ
+µ
−
using a sample of W− → τ
−ν¯τ decays from a dataset corresponding to an integrated
luminosity of 20.3 fb−1
collected in 2012 at a center-of-mass energy of 8 TeV. The LHCb
experiment [187] performed the same search using a sample of tau from b and c-hadron
decays from a dataset corresponding to an integrated luminosity of 3.0 fb−1
collected by
the LHCb detector in 2011 and 2012 at center-of-mass energies of 7 and 8 TeV, respectively.
The CMS experiment [188] recently delivered the results for the same search using a sample
of τ leptons produced in both W boson and heavy-flavour hadron decays from a dataset
corresponding to an integrated luminosity of 33.2 fb−1
recorded by the CMS experiment in
2016 [188]. ATLAS, CMS and LHCb reported a 90\% C.L. upper limit on the branching ratio
of 3.76 × 10−7
, 8.0 × 10−8 and 4.6 × 10−8
, respectively. The Belle-II collaboration studied
prospects for the expected sensitivity of this search. This channel has a purely leptonic final
state, thus it is expected to be free of background. This allows to scale the experimental
uncertainties linearly with the luminosity, which means that at least an improvement of a
factor ×50 is expected for Belle-II after accumulating a luminosity of 50 ab−1
[103].
\fi
\subsection{$K$ and other Mesons Channels}
guarda bernestein articolo
\begin{figure}[!h]
\centering
\includegraphics[width =0.4\textwidth]{figures/png/Screenshot_20240307_114258.png}
\caption{.}
\label{fig:Kchannel}
\end{figure}



\chapter{The Mu2e Experiment}\label{mu2echapter}
\textit{
The Mu2e experiment aims to investigate the phenomenon of charged-lepton flavor violating (CLFV) neutrino-less conversion, where a negative muon transitions into an electron within the influence of a Aluminium nucleus. The experiment will measure the ratio between the conversion and the nuclear muon capture rates:
\begin{equation}\label{rmue}
R_{\mu e}=\frac{\mu^{-}+N(Z, A) \rightarrow e^{-}+N(Z, A)}{\mu^{-}+N(Z, A) \rightarrow \nu_\mu+N(Z-1, A)}
\end{equation}
The goal is to improve the current limit, set by the SINDRUM-II experiment, Ref. \cite{SINDRUMII:2006dvw}, by
four orders of magnitude and reach a SES (single-event-sensitivity) of 3 $\times$ 10$^{-17}$ on the
conversion rate, a 90\% CL of 8 $\times$ 10$^{-17}$ and a 5$\sigma$ discovery reach at 2 $\times$ 10$^{-16}$.
Mu2e is presently undergoing commissioning, integration and testing stages at the Fermilab Muon Campus, 
with contributions from an international collaboration. Data taking is planned to begin in 2026. 
This Chapter provides an overview of the employed experimental techniques and infrastructures. 
Fundamental bibliography for this chapter can be found in Ref. \cite{bartoszek2015mu2e}, 
\cite{bobbb}, \cite{Bernstein_2013}, \cite{Kargiantoulakis_2020}, \cite{universe9010054}.}
\section{Experiment concept}
A beam of negative muons, $\mu ^-$, is generated by directing a proton beam at a production target, yielding negative pions along with other mesons and hadrons. 
Pions, with a decay rate of more than 99.9\% due to helicity suppression, undergo $\pi ^- \rightarrow \mu ^- \bar{\nu}_\mu$ decay in flight. 
Accumulating sufficient statistical data within a realistic timeframe necessitates an intense muon beamline. 
Low-momentum secondary muons are trapped by a stopping target to form muonic atoms. Within approximately $10^{-13}$ s, 
a muonic atom transitions to the $1s$ state, Ref. \cite{MEASDAY2001243}. 
Given the brief cascading time compared to the mean muon lifetime in a muonic atom, typically around $10^{-6}$ s, 
instances of muon decay before reaching the ground state are negligible. 
The cascades emit x-ray photons, aiding in estimating the number of muons captured on the stopping target. 
Muonic atoms decay after a specific lifetime determined by the stopping target material. 
In the Mu2e experiment, an Aluminum stopping target is employed, $^{27}$Al, resulting in a lifetime of 
864 ns. Muon decay occurs primarily through two processes: 
nuclear capture $\mu^- N \rightarrow \nu_\mu N'^* $, where $N'^*$ represents an excited 
magnesium nucleus ($^{27}$Mg), and muon decay-in-orbit (DIO), 
that is the three-body decay with neutrinos $\mu ^- N \rightarrow e^- N \nu_\mu \bar{\nu}_e$. 
As shown in Figure \ref{fig:muonicatom}, the ratio between these 
processes varies with the stopping target material. For $^{27}$Al, approximately 60.9\% 
of muonic atoms undergo muon nuclear capture. The remaining 39.1\% undergo muon DIO. 
The experiment aims to identify a third decay mode, neutrinoless muon-to-electron 
conversion $\mu^- N \rightarrow e^- N $. Detectors are designed to detect the signature of a 
single monoenergetic electron: this will be explored further in the next sections.
The Mu2e experiment can detect the lepton-number violation process $\mu^- N \rightarrow e^+ N'$.
This process violates both the charged lepton flavor and the total lepton number ($|\Delta L|$ = 2). 
Conversion positrons tracks helices in the Tracker, but with different chirality from electrons. 
\section{Signals and Backgrounds}\label{sigandbkg}
\subsection{Conversion Electron Signal}
The conversion of a muon to an electron in the field of a nucleus is coherent: 
the muon recoils off the entire nucleus and follows two-body decay kinematics, 
Ref. \cite{bartoszek2015mu2e}. The mass of the nucleus is large compared to the 
mass of the electron, hence the recoil terms are minimal. The outgoing nucleus 
remains in the ground state and as a result the conversion electron (CE) is 
monochromatic with an energy slightly lower than the muon's rest mass:
\begin{equation}
    E_{CE} = m_\mu - E_{recoil}(A) - E_{bind}(Z) 
\end{equation}
where $m_\mu$ is the muon mass, $E_{recoil}\simeq \frac{m^2_\mu}{2 m_N}$ is 
the recoil energy of the target nucleus, with $m_N$ the nucleus mass, and 
$E_{bind}\simeq \frac{Z^2 \alpha^2 m_\mu}{2}$ is the binding energy of the 
$1s$ state of the muonic atom, Ref. \cite{universe9010054}. For the Mu2e 
stopping target material, $^{27}$Al, $E_{CE}$ = 104.97 MeV, Ref. \cite{PhysRevD.84.013006}.
\subsection{Backgrounds}\label{backgrounds}
In order to achieve a sensitivity improvement of four orders of magnitude, 
it is important to know all experimental backgrounds that could interfere with 
the process $\mu^- N \rightarrow e^- N $. Any source capable of producing 105 MeV 
electrons may introduce background. There are five categories of background sources, Ref. \cite{bartoszek2015mu2e}:
\begin{enumerate}
\item electrons or muons coming from cosmic rays;
\item intrinsic phenomena that vary in accordance with beam intensity, as muon Decay-in-Orbit (DIO) and Radiative Muon Capture (RMC);
\item processes that are delayed because of particles that spiral slowly down the muon beam line, such as antiprotons;
\item prompt processes where the detected electron is nearly coincident in time with the arrival of a beam particle at the muon stopping target (e.g. radiative pion capture (RPC), Pion or Muon Decay-in-Fight);
\item reconstruction errors due to accidental activities.
\end{enumerate}
The detailed discussion of these background sources will be provided below, along with the corresponding methods for mitigation. The subsequent section will focus on introducing how these mitigation techniques are applied in the Mu2e experimental design, highlighting the experiment setup.
\subsubsection{Cosmic rays}
Cosmic-ray particles, predominantly cosmic-ray muons, also contribute to the experiment background. Several mechanisms are listed here:
\begin{enumerate}
    \item muon decays occurring within or near the detectors;
    \item muons interacting with nuclei within detectors or surrounding materials;
    \item muons scattering within the detectors and being erroneously identified as electrons;
    \item muons entering the muon beamline with specific initial momentum, either interacting with collimators to produce electrons or proceeding down the muon beamline and being misidentified.
\end{enumerate}

While enhanced reconstruction algorithms can correctly identify cosmic-ray muons in certain instances, there are scenarios where signals induced by cosmic rays are indistinguishable from conversion electrons. The quantity of cosmic-ray-induced backgrounds is proportional to the duration of data collection, with the rate being experiment-specific. Detailed simulations estimated that in the Mu2e experiment, conversion-like signals generated by cosmic rays occur approximately once per day, potentially overwhelming actual conversion signals, Ref. \cite{CRVposter}. Mitigation involves employing a combination of active veto and shielding within the experiment. As shown in Section \ref{CRV}, a high-efficiency veto system is capable of effectively reducing cosmic-ray-induced backgrounds to an acceptable level.
\subsubsection{Intrinsic Backgrounds}
The intrinsic backgrounds in the experiment originate from two physics processes: muon decay-in-orbit (DIO) and radiative muon capture (RMC). In this context, $intrinsic$ denotes that the rates of these backgrounds are directly correlated with the number of muons captured on the stopping target.
\paragraph{Muon Decay-in-Orbit}
Electrons energy coming from the three body decay of a free muon, $\mu^- \rightarrow e^- \bar{\nu}_e \nu_\mu$, is described by the Michel Spectrum. The differential decay rate can be computed, Ref. \cite{michel}:
\begin{equation}
    \frac{\text{d}\Gamma_{\text{free}}}{\text{d}x}= \frac{G^2_F m^5_\mu}{192 \pi^3}x^2(6-4x+\frac{\alpha}{x}f(x)) 
\end{equation}
where $x=\frac{2 E_e}{m_\mu}$, $0\leq x\leq 1$, $G_F$ is the Fermi constant, $\alpha$ is the fine-structure constant, $m_\mu$ is the muon mass, $E_e$ is the electron energy and $f (x)$ represents a complicated radiative correction term, described in Ref. \cite{PhysRev.113.1652}. 
As shown in Figure \ref{fig:linearscalemichel}, the spectrum exhibits a peak energy at 52.8 MeV, constituting half of the muon rest energy  or the energy of the searched conversion electron. However, the presence of an Al nucleus allows the electron to interact with it, exchanging momentum and resulting in a maximum possible energy, $endpoint$ $energy$, that closely matches that of the conversion electron (negligible neutrino energy in this scenario). The difference between the two energy spectra, one neglecting the nuclear recoil and the other influenced by the Al nucleus, is illustrated in Figure \ref{fig:micheldiff}. Figure \ref{fig:logscalemichel} presents a more detailed examination of the high-energy range of the spectrum. When the electron energy $E_e \ \geq$ 85 MeV, the dominant term at the leading order scales with $(E_{\mu e} - E_e)^5$, Ref. \cite{PhysRevD.84.013006}, resulting in a low rate within the energy range very close to the endpoint. However, in actual experimental conditions, due to uncertainties in reconstructions and the energy loss of particles when interacting with stopping targets or detector materials, the monoenergetic conversion electrons tend to be reconstructed with a left-skewed energy distribution, with a full width at half maximum (FWHM) approximately on the order of 1 MeV, Ref. \cite{gaponenko}, as shown in Figure \ref{fig:sensitivity}. Consequently, the high-energy electrons originating from muon DIO and the conversion electrons become indistinguishable. To limit the DIO background, Mu2e detectors are designed to be blind to the majority of low-energy electrons. A particle tracking detector with high momentum resolution also helps to reduce background noise, which will be explored in next sections.
\begin{figure}[!h]
     \begin{subfigure}[b]{0.4\linewidth}
         \centering
         \includegraphics[scale = 0.18]{figures/png/Screenshot_20240222_175415.png}
         \subcaption{Linear scale.}
         \label{fig:linearscalemichel}
     \end{subfigure}
     \begin{subfigure}[b]{0.7\linewidth}
         \centering
         \includegraphics[scale = 0.18]{figures/png/Screenshot_20240222_175446.png}
         \subcaption{Logarithmic scale.}
         \label{fig:logscalemichel}
     \end{subfigure}
     \caption{Electron spectrum for aluminum, Ref. \cite{PhysRevD.84.013006}.}
        \label{fig:michel}
\end{figure}

\begin{figure}[!h]
\centering
\includegraphics[width =0.55\textwidth]{figures/png/Screenshot_20240222_175644.png}
\caption{The electron energy spectrum near to the endpoint. The black line represents the Michel spectrum when neglecting nuclear recoil, while the red dashed line takes into consideration the recoil of the Aluminum nucleus, Ref. \cite{PhysRevD.84.013006}.}
\label{fig:micheldiff}
\end{figure}
\paragraph{Radiative Muon Capture}

The Radiative muon capture (RMC) process differs from ordinary muon nuclear capture by producing an additional photon. In the process $\mu^- N \rightarrow\gamma \nu_\mu N'^* $, the photon can either be real or virtual. The photon, interacting with matter or undergoing pair production, can produce electrons with energies close to $E_{CE}$, introducing background signals to the experiment. The emitted photon's energy follows a spectrum, with its maximum energy, denoted as the kinematic limit $k_{max}$, determined by the equation (Ref. \cite{bartoszek2015mu2e}):

\begin{equation}
k_{max} = m_\mu c^2 - |E_b| - E_{rec} - \Delta M ,
\end{equation}

where $E_b$ represents the muon binding energy on the initial nucleus, $E_{rec}$ denotes 
the recoil energy of the daughter nucleus and $\Delta M$ is the rest energy difference 
between the final and initial nuclides. This formula neglects higher order nuclear effects. 
RMC can be effectively mitigated in the Mu2e experiment by selecting Aluminum as the stopping 
target material. The stopping target is selected so that the daughter nuclide of a muon-capture 
process of any kind is heavier than the original nuclide. For aluminum, the RMC endpoint energy 
is 101.9 MeV, approximately 3.1 MeV below the conversion electron energy, Ref. \cite{bartoszek2015mu2e}. 
The planned FWHM of the conversion peak is around 1 MeV, therefore the RMC background will be 
outside the signal region. However, the RMC background might distort the DIO spectra in the 80-100 MeV 
range, making it difficult to extrapolate to the endpoint. Determining the RMC background from the 
data will be a crucial part of the experiment.
\subsubsection{Prompt Processes}
This type of background sources can generate electrons at roughly the same time as 
the entering beam particles. There are four primary sources: radiative pion capture (RPC), 
pion decay-in-flight ($\pi$-DIF), muon decay-in-flight ($\mu$-DIF) and beam electrons.
\paragraph{Radiative Pion Capture}
The secondary muon beam carries a considerable quantity of pions.  It is not excluded 
that some pions can reach the stopping target. Pions, when captured in materials, can 
produce a high-energy photon, i.e. $\pi^- N(A,Z) \rightarrow \gamma N ^* (A,Z-1)$. 
This phenomenon, called radiative pion capture, is observed in approximately 2\% of 
pions captured in Aluminum, Ref. \cite{PhysRevC.5.1867}. Similar to RMC, the photon 
can internally convert into an electron-positron pair or emit an on-shell photon, 
leading to pair production. The external pair-production depends on the thickness 
of the material. The resulting electrons can contribute to the experiment background. 
Despite its similarity to RMC, RPC is more challenging background to suppress due to 
the fact that the endpoint of the energy spectrum of photons, and consequently the 
resulting electrons, is not constrained by the rest energy of the muon. The mass of 
a pion is 139.6 MeV, which is much higher than the conversion energy. Consequently, 
there exists no energy separation between the search range for conversion electrons 
and the electron energy spectrum originating from RPC photons. The SINDRUM II results 
were limited by the pion-induced background and also by the low intensity of its muon 
beam, Ref. \cite{SINDRUMII:2006dvw}. SINDRUM II employed a primary proton beam with a 
frequency of one pulse every 19.75 ns, lasting approximately 0.3 ns. This interval 
between pulses was shorter than the 26 ns lifetime of pions, Ref. \cite{zyla}, 
ensuring a consistent pion flux. To mitigate RPC, SINDRUM II employed a degrader to suppress 
pions and a veto counter in the beam, resulting in less than 1 out of 109 pions reaching their 
target. However, given the more intense beam, the Mu2e experiment has to change the approach. 
Mu2e employs a pulsed proton beam. Given their brief lifetime, nearly all pions decay or 
interact with materials shortly after the pulse of the proton beam. The RPC background 
can be suppressed by opening a live window for conversion electron search at a right time. 
One important point to note is that in the event of protons coming out of the beam pulse, 
out-of-time, the resultant pions could still contribute to RPC background. Consequently, 
it is important for the pulsed beam to achieve a high extinction level, ensuring that the 
ratio of out-of-time to in-time RPC remains below a specified threshold. Further elaboration 
on the pulsed beam used in Mu2e will be provided in Section \ref{accel}.
\paragraph{$\pi$-DIF and $\mu$-DIF}
The decay-in-flight of the pion ($\pi$-DIF) and the decay-in-flight of the muon ($\mu$-DIF) exhibit quite similar characteristics. Free pions and muons can undergo electron decay while transitioning from the production target to the stopping target, through the processes $\mu^- \rightarrow e^- \nu_\mu \bar{\nu}_e$ and $\pi^- \rightarrow e^- \bar{\nu}_e$. In the center-of-mass frame of the initial particles, electrons originating from the first process exhibit an energy spectrum that reaches an endpoint of 52.8 MeV, while those from the second process have a consistent energy of approximately 70 MeV. As the pions and muons move at relativistic velocities, the energies (and momenta) of the resultant electrons are boosted. For instance, a muon with a momentum of around 79 MeV/c or a pion with momentum close to 70 MeV/c can generate an electron with an energy of 105 MeV, Ref. \cite{bartoszek2015mu2e}. Implementing a pulsed proton beam and employing a delayed live window can help suppress background from $\pi$-DIF and $\mu$-DIF events. Particles with sufficient momentum to boost the daughter electrons to the concerning energies move quickly along the muon beamline and are gone by the time the live search window begins, Ref. \cite{bobbb}. In addition, the shape of the Mu2e detectors contributes to mitigation.
\paragraph{Beam electrons}\label{beamelectrons}
Other mechanisms generate electrons, both at the production target and along the muon beamline. For instance, neutral pions formed at the production target can decay in two photons, after which the photons can either create electron-pairs or interact with nearby materials to generate electrons. Other beam particles can decay or interact at any point before the muon stopping target, producing electrons with energy equal to $E_{CE}$. These sources of background are colled beam electrons and it is possible to reduce them through the pulsed beam and the delayed live window. Moreover, Mu2e uses a set of solenoids to generate a magnetic field across the muon beamline and in the area of the stopping target and detectors. A charged particle follows a spiral trajectory when a magnetic field is applied, where the size and shape of the spiral are determined by the particle's electric charge and its parallel and perpendicular to magnetic field components of momentum. This brings to the installation of collimators along the muon beamline to suppress the number of high-momentum particles exceeding 100 MeV/c, Ref. \cite{bartoszek2015mu2e}. Moreover, the magnetic field is designed with a gradient in proximity of the stopping target. This gradient effectively divides the paths of conversion electrons from those originating upstream, unless they undergo scattering within the stopping target, Ref. \cite{bobbb}. Solenoid system and the magnetic field will be deeply explored in Section \ref{setup}.
\subsubsection{Delayed processes from antiprotons}
Protons, at a given threshold, can generate antiprotons within the production target. This occurs through the process of antiproton production: $pp \rightarrow ppp\bar{p}$. The minimum kinetic energy of the initial proton beam can be found applying 4-momentum conservation principles to the system. 
If we consider that all four particles in the final state are at rest in the center of mass frame, the minimum kinetic energy needed for $p\bar{p}$ production can be found, which is approximately $6 m_pc^2 \sim 5.6$ GeV, where $m_p$ is the mass of the proton. In an ideal scenario, maintaining the beam energy below this threshold could enable us to avoid this background in the experiment. Unfortunately, the Mu2e proton beam goes beyond the threshold for antiproton production. The antiprotons are long-lived and massive. Antiprotons with momenta below 100 MeV/c travel at speeds less than 0.1$c$, requiring several $\mu$s to spiral from the production target to the stopping target, Ref. \cite{bartoszek2015mu2e}. They have the correct charge and momentum to pass through the collimators placed between the production target and the stopping target. Annihilating or undergoing interactions with other materials, they have the capability to release a substantial amount of energy, generating numerous secondary electrons. The time delay connected with these interactions significantly exceeds the muon lifetime, leading to a continual flow of antiprotons reaching the stopping target. The pulsed beam and the delayed live window fail to suppress the antiproton background. The best approach is to prevent the antiprotons from reaching the region where the stopping target is located. A thin absorber is positioned in the muon beamline to capture the antiprotons. Its design was developed to find a compromise between increasing antiproton absorption and decreasing muon beam loss.
\subsubsection{Accidental activity}
This final background category arises not from the physical interactions of a specific particle, but rather from the accidental reconstruction of extra events in the detectors, miming conversion-like signals. During the muon-capture process, nuclei in excited states can expel protons, neutrons and photons. These expelled protons have high ionization potential, producing large signals in the detectors and increasing the likelihood of reconstruction errors. Additionally, other coincidences, such as multiple muon decay-in-orbit occurring in close succession, can also contribute to reconstruction errors. To suppress the flux of protons from the muon-capture process reaching the detectors, additional polyethylene absorbers are used to surround the stopping target in the Mu2e experiment. A systematic uncertainty is evaluated as part of the background analysis to take into account these accidental events.

\subsection{Backgrounds estimates and Signal Sensitivity}
Mu2e run plan is divided in two different phases. Run I will take place in 2026, before a 2 years shutdown due to the planned accelerator upgrade for the long baseline neutrino program. In Run I phase one, a low intensity proton beam, $1.6 \times 10^7$ protons/pulse, will be used. In Run I phase two, the mean intensity will be increased up to $3.9 \times 10^7$ protons/pulse. The total number of stopped muons will be $6 \times 10^{16}$, corresponding to the 10\% of the number required to satisfy the Mu2e goals in the complete data-taking. As discussed in Ref. \cite{universe9010054}, the discovery $R_{\mu e}$, Ref. \ref{rmue}, corresponding to a 50\% probability of observing the conversion signal at a 5$\sigma$ significance level is $R_{\mu e}^{5 \sigma}= 1.2 \times 10^{-15} $. Reaching the 5$\sigma$ significance level requires observing 5 $\mu\rightarrow e$ events in the two-dimensional search window 103.60 < $p$ < 104.90 MeV/c, 640 < $T_0$ < 1650 ns. One of the
parameters characterizing the sensitivity of an experiment to a process of interest is its
single event sensitivity (SES), defined as:
\begin{equation}
    SES \equiv \frac{1}{N_{POT} \cdot P_{\mu \ stop} \cdot \epsilon_{CE} \cdot BR_{capture}}
\end{equation} 
$N_{POT}$ means the number of protons on target in the experiment, $P_{\mu \ stop}$ is the number of muons stopped on target per proton, $\epsilon_{CE}$ is the conversion electron acceptance, which is a product of detector efficiency (dependent on the momentum signal region) and fraction of muons interacting in the live time window and $BR_{capture}$ is the branching ratio of muon captures, which is 60.9\%. The optimized Mu2e signal window corresponds to a SES of $2.4 \times 10^{-16}$. The background estimates after the sensitivity optimization are summarized in Table \ref{tab:summarybkg}, resulting in a total background of approximately $\sim$ 0.1 event/year. Figure \ref{fig:sensitivity} shows the momentum and time distributions for CE signal and background processes corresponding to the optimized signal window for Run I. A detailed analysis and estimate of the Mu2e expected backgrounds for Run I can be found in Ref. \cite{universe9010054}.
\begin{center}  
\begin{table}[!h]
\centering
\renewcommand{\arraystretch}{1.2}
\begin{tabular}{| c | c |}
\hline
\textbf{Channel} & \textbf{Mu2e Run I}\\
\hline
SES & 2.4 $\times \ 10^{-16}$ \\
\hline
Cosmic rays & 0.046 $\pm$ 0.010 (stat.) $\pm$ 0.009 (syst.) \\
DIO & 0.038 $\pm$ 0.002 (stat.) $ ^{+ \ 0.025} _{- \ 0.015}$ (syst.)\\
Antiprotons & 0.010 $\pm$ 0.003 (stat.) $\pm$ 0.010 (syst.) \\
RPC in-time & 0.010 $\pm$ 0.002 (stat.) $ ^{+ \ 0.001} _{- \ 0.003}$ (syst.)\\
RPC out-of-time & (1.2 $\pm$ 0.1  (stat.) $ ^{+ \ 0.1} _{- \ 0.3}$ (syst.)) $\times$ $10^{-3}$ \\
RMC & $<$ 2.4 $\times$ $10^{-3}$ \\
Decays in flight & $<$ 2 $\times$ $10^{-3}$ \\
Beam electrons & $<$ 1 $\times$ $10^{-3}$ \\
\hline
Total &  0.105 $\pm$ 0.032\\
\hline
\end{tabular}
\caption{Summary of the several background sources to the conversion electron search as expected in Mu2e Run I,  Ref. \cite{universe9010054}. The table also shows the corresponding single event sensitivity (SES). This is is defined as the $R_{\mu e}$ ratio when there is one signal event.}
\end{table}\label{tab:summarybkg}
\end{center}
\begin{figure}[!h]
\centering
\includegraphics[width =0.93\textwidth]{figures/png/Screenshot_20240225_102708.png}
\caption{Left: Momentum distribution of the conversion electron signal and expected backgrounds. Right: Time distribution of the conversion electron signal and expected backgrounds. The arrows show the signal region selected for the analysis,103.60 $< \ p \ < $ 104.90 MeV/c and 640 $< \ T_0 \ < $ 1650 ns, Section \ref{pulsedprotonbeam}. The $CE$ signal distributions correspond to $R_{\mu e} = 1 \times 10^{-15}$, Ref. \cite{universe9010054}.}
\label{fig:sensitivity}
\end{figure}
\section{Experimental setup}\label{setup}
Figure \ref{fig:mu2escheme} shows a schematic overview of the Mu2e experiment, illustrating the trajectory of the pulsed proton beam directed towards the production target indicated by the red arrow. The experiment uses a solenoid system to generate magnetic fields essential for its operations. The Production Solenoid (PS) surrounds the production target, while further downstream, the Transport Solenoid (TS) provides the magnetic field for the muon beamline. The TS, configured in an S-shape, incorporates collimators and proton absorbers strategically positioned to minimize experimental backgrounds. The stopping target is located at the beginning of the Detector Solenoid (DS). Proton absorbers surround the stopping target, not shown in the figure. The Tracker and the Calorimeter are housed in the DS, enabling momentum and energy measurements, respectively. Additionally, a Stopping Target Monitor (STM) positioned downstream at the DS's end, not shown, monitors the stopping target's condition and estimates the captured muon count. Not shown in the figure is the Mu2e Cosmic Ray Veto (CRV) system: it surrounds the DS and half of the TS. In next sections, a more detailed description of systems of the Mu2e experiment will be given.
\begin{figure}[!h]
\centering
\includegraphics[width =\textwidth]{figures/png/Screenshot_20240301_143105.png}
\caption{Schematic view of the Mu2e apparatus. The center of the Mu2e reference frame is located at the COL3 collimator center, its y-axis points upwards, the z-axis is parallel to the DS axis and points downstream, and the x-axis completes the right-handed reference frame, Ref. \cite{universe9010054}.}
\label{fig:mu2escheme}
\end{figure}
\section{Accelerator system and Proton Beam}\label{accel}
\subsection{Pulsed Proton Beam}\label{pulsedprotonbeam}
As previously discussed in Section  \ref{backgrounds}, the Mu2e experiment employs a pulsed proton beam to reduce background from prompt processes, as pion capture. The 8 GeV, 8 kW beam originates from the Fermilab Booster, Ref. \cite{PhysRevAccelBeams.20.111003}. A pulsed beam with a 1695 ns gap between two pulses is needed. Figure \ref{fig:accell} illustrates the Fermilab accelerator facilities involved in generating and delivering the pulsed proton beam. The Fermilab Booster delivers 8 GeV protons in 20 batches throughout a 1.4 s Main Injector cycle at 15 Hz, as shown in Figure \ref{fig:deliver}. Thus, the accelerator timeline is described using a fundamental time unit of 1 tick that corresponds to 66.7 ms.

\begin{figure}[!h]
     \begin{subfigure}[b]{0.4\linewidth}
         \centering
         \includegraphics[scale = 0.3]{figures/png/Screenshot_20240301_151449.png}
         \subcaption{\centering Fermilab accelerator facilities involved in producing and delivering the pulsed proton beam.}
         \label{fig:accell}
     \end{subfigure}
     \begin{subfigure}[b]{0.7\linewidth}
         \centering
         \includegraphics[scale = 0.39]{figures/png/Screenshot_20240301_151418.png}
         \subcaption{\centering Proton beam delivery to Mu2e.}
         \label{fig:deliver}
     \end{subfigure}
     \caption{Pulsed proton beam delivery, Ref. \cite{accelerator}.}
        \label{fig:three graphs1}
\end{figure}
The two Mu2e batches, represented by the two blue bars at ticks 1 and 2, are injected into the Recycler Ring, each containing $4 \times 10^{12}$ protons. The protons from these batches are reorganized within the Recycler using a 2.5 MHz radio frequency (RF) system into 8 bunches. These bunches are then extracted individually from the Recycler and transported to the Delivery Ring every 48.1 ms, as shown in the middle part of Figure \ref{fig:deliver}. Once inside the Delivery Ring, a single bunch of $1 \times 10^{12}$ protons undergoes gradual extraction. This process results in the extraction of a small fraction of the bunch per revolution, delivered to the Mu2e experiment. The complete bunch is extracted over a span of 43.1 ms, across $\sim$25000 turns around the Delivery Ring, as shown in the bottom part of Figure \ref{fig:deliver}.
\begin{figure}[!h]
\centering
\includegraphics[width =\textwidth]{figures/png/Screenshot_20240301_151148.png}
\caption{Proton beam profile at the Mu2e Proton Target, Ref. \cite{accelerator}.}
\label{fig:beamprofile}
\end{figure}
Figure \ref{fig:beamprofile} illustrates the temporal profile of the beam at the Mu2e Proton Target. Consecutive proton pulses are spaced by 1695 ns. Each pulse lasts for 250 ns and contains $(3.9 \pm 2.0 )\times 10^7$ protons. The 1695 ns pulse separation is highly advantageous for the Mu2e experiment. Figure \ref{fig:beamwindow} shows the beam pulse, the simulated pion flux, the muon capture rate on the stopping target and the muon decay rate. The active window for detecting conversion electrons begins at $\sim$640 ns and extends for mor or less 1 $\mu$s. This selection finds a compromise between reducing pion-induced backgrounds  and increasing the rate of muon decays. Because of the brief lifetime of the pions, they are gone by the time the active window opens, resulting in a suppression of the pion count by $\mathcal{O}(10^{11})$. The 1695 ns pulse separation exceeds twice the muon lifetime, allowing sufficient temporal separation between prompt backgrounds and the live window without excessively compromising beam intensity.
\begin{figure}[!h]
\centering
\includegraphics[width =\textwidth]{figures/png/Screenshot_20240301_164649.png}
\caption{Proton pulses arrive at the production solenoid every 1695 ns. A delayed live-time window can suppress the beam-related background, Ref. \cite{universe9010054}.}
\label{fig:beamwindow}
\end{figure}
\subsection{Proton Beam Extinction and Extinction Monitor}
As mentioned in the previous section, the Mu2e experiment requires the extinction level of the incoming proton beam, to reduce backgrounds caused by out-of-time protons. The extinction rate, defined as the ratio between the number of out-of-time protons and the number of the in-time protons, should be lower than $10^{-10}$, Ref. \cite{bartoszek2015mu2e}. The structure of the beam leads to an extinction level $2.1 \times 10^{-5}$. To take into account the fact that some beam will leak out of two consecutive proton pulses, an Extinction Insert is deployed in the M4 beamline between the Delivery Ring and the Mu2e experiment. The out-of-time beam particles are swept into a collimator system by a set of oscillating dipoles, called AC dipoles. These AC dipoles are simulated to offer an additional extinction factor of $5\times 10^{-8}$, reducing the overall extinction to $1.1 \times 10^{-12}$, leaving a margin of 10$^2$, Ref. \cite{accelerator}. An Extinction Monitor is positioned downstream of the production target along the proton beamline, Figure \ref{fig:extintion}. It monitors the extinction level of the incoming beam striking the Mu2e production target and delivers a measurement at a precision better than $10^{-10}$. The Extinction Monitor consists of a collimator and magnetic filter system, a pixel telescope, a system of trigger scintillators and a range-stack. The collimator and magnetic filtter system transport a small quantity of the particles generated at the production target to the Extinction Monitor. The pixel telescope tracks the trajectory of charged particles coming from the collimator. The pixel telescope consists of a permanent magnet and 8 scintillators, as shown in Figure \ref{fig:extintionmonitor}. The system uses a permanent magnet to separate two sets of four scintillator planes, allowing for momentum measurements of entering particles. The range stack is located further downstream from the pixel telescope. Steel absorber plates separate scintillators, distinguishing between hadrons and muons based on their penetrating capacities.
\begin{figure}[!h]
\centering
\includegraphics[width =\textwidth]{figures/png/800px-Extinction_filter.png}
\caption{The Extinction Monitor is located downstream of the
production target, Ref. \cite{Prebys:IPAC2015-THPF121}.}
\label{fig:extintion}
\end{figure}
\begin{figure}[!h]
\centering
\includegraphics[width =0.9\textwidth]{figures/png/Screenshot_20240306_184720.png}
\caption{The tracking spectrometer of the Mu2e experiment, consisting of eight planes of pixel detectors and a permanent magnet spectrometer, Ref. \cite{Prebys:IPAC2015-THPF121}.}
\label{fig:extintionmonitor}
\end{figure}
\subsection{Production Target}
The Mu2e production target is an additional essential element of the accelerator systems. It is suspended in the middle of the PS bore, Ref. \cite{bartoszek2015mu2e}. It is made of tungsten. The tungsten has a high pion production cross-section, capable of producing the necessary number of stopped muons. The refractory material allows the target to be cooled radiatively without the need of any extra cooling equipment. A compact design minimizes pion reabsorption on the production target.
\section{Solenoids System}
The Mu2e Solenoid system consists of three magnetically coupled systems: the \textbf{Production Solenoid} (PS), the \textbf{Transport Solenoid} (TS) and the \textbf{Detector Solenoid} (DS), shown in Figure \ref{fig:mu2escheme}. Each system contains multiple module and each one is made of superconducting coils wound with aluminum stabilized Nb-Ti Rutherford cables, located in a 4.5 m long cryostat. The shape of solenoids is designed in order to efficiently transmit muons and to suppress other particles. The resulting magnetic field is $\mathcal{O}$(1 T): its highest value is 4.6 T at the upstream end of the experiment and the lowest value is 1 T at the downstream end. The muons are guided toward the stopping target and the DS by lowering the magnetic field. Local magnetic field minima are avoided to avoid trapping particles in these areas. In the PS, the magnetic field decreases from 4.6 T to 2.5 T at the entrance of the TS. The large gradient helps to collect the secondary pions and muons and to direct them towards the DS. The magnetic field across the S-shaped TS changes its value only by a factor of 0.5 T. The shape of the TS allows to not transmit photons and other neutral particles in the DS and its dimension was set to avoid transmitting particles with large momentum. These particles either spiral with a large helical radius (large initial momenta perpendicular to the field) or cannot create an S-shaped bend (large initial momenta parallel to the magnetic field), resulting in collisions with solenoid walls or collimators. As particle drifts in the solenoid field is dependent on the particle charge, positively and negatively charged muons drift in opposite vertical directions and are separated, as explained in Appendix \ref{appendix1}. In the upstream curved solenoid portion, as shown in Figure \ref{fig:collimators}, the spiraling positive (blue) and negative (red) muons are deflected downwards and upwards respectively. Positive muons are stopped in collimators COL3u and COL3d. The TS is long enough for pion decay, suppressing RPC backgrounds. It also protects the detectors from radioactively hot areas around the primary beam and the production target. The magnetic field in the first half of the DS is reduced from 2 T to 1 T. In a smoothly gradient field, the adiabatic invariance of the magnetic flux can be used. Assuming a constant $p^2_\perp/B$, there is:
\begin{equation}
    v^2_{\parallel}=v^2_0-v^2_{\perp 0}\frac{B(z)}{B_0}
\end{equation}
Here, $\perp$ is referred with respect to the magnetic field $B$ and $\parallel$ is referred to the $z$ direction. The subscript 0’s indicates the initial state. The gradient pitches electrons forward into the tracker's acceptance while rejecting higher velocity electron, as mentioned in \ref{beamelectrons}. 
The second half of the DS containing the Tracker and the Calorimeter has a uniform magnetic field of 1 T. This allows particle trajectories and momenta to be reliably measured.
\begin{figure}[!h]
\centering
\includegraphics[width =0.9\textwidth]{figures/png/Screenshot_20240303_152845.png}
\caption{The muon beamline, composed of the Production Solenoid (PS), the Transport Solenoid (TS) and the Dector Solenoids (DS), Ref. \cite{ginther}. The PS is 4 m long. The role of the PS, shown in Figure \ref{fig:PS}, is to collect pions, kaons and their decay products muons. The TS is S-shaped and it is divided in five parts. The first is called TS1 that contains the first collimator (COL1), made of copper wedges. It filters particles from their momentum and reduces the radiation damage for coils of the upstream part of the TS (TSu). The TS2 contains a toroidal field in order to select negative muons from the positive ones in TS3 with two rotatable collimators (COLu and COLd). The rotatable feature allows selection of $\mu^-$'s instead of $\mu^+$'s, which can be used for detector calibration. The TS4 role is to put the beam in the center of the solenoid. TS5 connects TS with DS and it contains a collimator (COL5) made of polyethylene, which will serve as a shield from neutrons. A thin window assembly is installed at the beginning of the TS and also between the rotatable collimators to absorb antiprotons in the beam.
The DS, shown in Figure \ref{fig:DS}, is a cylindrical system of approximately 11 m in length and 2 m in radius, which houses the Stopping Target and the main Mu2e detectors. The system is divided into two sections: a 4 m gradient section following the TS and a 6 m spectrometer section. The gradient region allows to separate conversion electrons from beam electrons. In the spectrometer region, the uniform magnetic field allows a precise measurement of the particle momentum. The upstream part of the beamline accounts for the Production Solenoid (PS) and the first part of the Transport Solenoid (TSu). The downstream part is composed of the last portion of the Transport Solenoids (TSd) and the Detector Solenoid (DS). Protons from muon captures in the Stopping Target are partially suppressed by a 0.5 mm thick cylindrical-shaped polyethylene proton absorber, placed halfway between the Stopping Target and the tracker. Finally, the IFB is a plate at the end of the DS, maintaining the DS vacuum and providing a path for the services and signals between vacuum and hall air.
}
\label{fig:muonbeamline}
\end{figure}
\begin{figure}[!h]
\centering
\includegraphics[width =0.95\textwidth]{figures/png/800px-MuonBeamlineCollimators2.png}
\caption{The view of the Transport Solenoids and the collimators COL3u and COL3d showing the offset apertures in those collimators. The upper spiraling negative muons (red) pass through the aperture while the positive muons (blue) are stopped by these collimators, Ref. \cite{tsview}.}
\label{fig:collimators}
\end{figure}
\begin{figure}[!h]
\centering
\includegraphics[width =0.95\textwidth]{figures/png/800px-Production_solenoid.png}
\caption{Cross-section of the production solenoid, Ref. \cite{6376120}.
The production target is placed approximately at the center of the superconducting coils.}
\label{fig:PS}
\end{figure}
\begin{figure}[!h]
\centering
\includegraphics[width =0.95\textwidth]{figures/png/Screenshot_20240306_225639.png}
\caption{Overall structure of the Detector Solenoid coils and cryostat, Ref. \cite{bobbb}.}
\label{fig:DS}
\end{figure}
\section{Stopping target}
The Mu2e muon stopping target is composed of 37 annular aluminum foils  with a purity of above 99.99\%, Ref. \cite{bobbb}. The foils are 100 $\mu$m thick, minimizing energy losses of the conversion electrons. This design narrows the reconstructed conversion electron momentum distribution and separates it from the DIO electron momentum distribution. The annular design reduces interactions with the beam electrons and other particle released in muon nuclear captures, which can create particle sprays and damage to the tracker's internal components. The central hole does not affect the target capacity to halt muons, which move in helical patterns. Muons passing through the hole of an upstream foil will stop in a downstream layer. There are various factors to consider while selecting aluminum as the stopping target material. First, as described in Section \ref{backgrounds}, the aluminum target has a low RMC background. Moreover, the muonic aluminium atom has a quite long lifetime, as shown in Figure \ref{fig:muonicatom}. The long lifetime allows separation between prompt backgrounds and a live window with a good decay rate. The muon DIO endpoint energy, further, depends on the type of nucleus, as shown in Figure \ref{fig:endpoint}. Aluminum has a high endpoint energy, so when muons are captured on other detector materials with a higher atomic number $Z$, they have lower endpoint energies and do not contribute to background. Aluminum is a suitable stopping target material for muon-to-electron conversion searches.
Moreover, the branching ratio ($BR$) of the conversion varies depending on the stopping target material due to differences in atomic number ($Z$) and mass number ($A$). By comparing $BR$s on different nuclei normalized to aluminum, it's possible to identify the dominating operator type, such as scalar ($S$), dipole ($D$), vector of transition charge radius type and vector of effective $Z$-penguin type, ($V(\gamma)$) and ($V(Z)$) respectively. Despite challenges in separating prompt backgrounds from signals due to short lifetimes of muonic atoms, materials with higher $Z$ offer better model differentiation. If the Mu2e experiment observes conversion signals, a subsequent search, Mu2e-II, could employ titanium as the stopping target. A more detailed discussion can be found in Ref. \cite{PhysRevD.80.013002}, \cite{PhysRevD.76.059902} and \cite{abusalma2018expression}.
\begin{figure}[!h]
\centering
\includegraphics[width =0.5\textwidth]{figures/png/lifetime_mu_matter.png}
\caption{ The mean lifetimes and free decay branches of the muonic $1s$ state versus the atomic number of the nucleus to which the negative muon is bound \cite{TYamazaki_1975}.}
\label{fig:muonicatom}
\end{figure}
\begin{figure}[!h]
\centering
\includegraphics[width =0.5\textwidth]{figures/png/endopint.png}
\caption{ The dependence of electron energy spectrum endpoint from muon DIO, Ref. \cite{dukes}.}
\label{fig:endpoint}
\end{figure}


\section{Detectors}
\subsection{The Tracker}\label{trackersec}
The Mu2e Straw Tube Tracker will measure conversion electrons momentum.
The Tracker must have as good resolution as is reasonably achievable
in order to minimize backgrounds, especially from the DIO electrons, Ref. \cite{bobbb}.
Since $dE/dx$ in the stopping target spreads the 
monochromatic energy distribution of the conversion electron downwards, 
it is needed to stop as many muons as possible. DIO events are smeared near the conversion 
peak as well, but this is a stochastic process. Conversion electrons with high energy loss 
might end up in the DIO spectrum with lower energy loss. Energy loss in the tracker needs 
to be as small as possible, since the stopping target only stops about 40\% of muons. 
To achieve excellent resolution, small high-side tails and minimal energy loss, the experiment 
will employ a straw tube tracker, Ref. \cite{bobbb}.
The Tracker is located inside the DS, downstream of the stopping target. 
Its shape follows the helical trajectories of conversion electrons 
in the magnetic field. 96 straw tubes are arranged in 
two staggered layers to form a panel, Ref. \cite{bartoszek2015mu2e}. 
The basic standalone module both mechanically and electrically is 
shown in Figure \ref{fig:trkpanel}, upper left. Each panel, that is 
harp-shaped, spans 120°. The straw tubes are arranged like harp chords. 
The electronics is stored in the outer volume. The Mu2e Tracker has a 
very limited material component, particularly in the straw region, 
which reduces the probability of scattering conversion electrons 
and increases the momentum resolution. The simulated resolution is shown in Figure \ref{fig:trkres}. 
The panels are combined 
using the approach indicated in Figure \ref{fig:trkpanel}, upper right, 
Ref. \cite{trk}. Three panels create a full circle and two layers of six 
panels, rotated by 30°, form a plane. A station is made up of two planes 
that have been rotated by 60°. One or more panels cover the entire annular 
region. The entire Tracker consists of 18 identical stations, shown in 
Figure \ref{fig:trkpanel}, bottom. It is 3.2 metres long and 1.7 metres in diameter. 
\begin{figure}[!h]
\centering
\includegraphics[width =0.7\textwidth]{figures/png/Screenshot_20240306_222803.png}
\caption{(Upper left) Tracker panel. (Upper right) Isometric view 
of a tracker plane (left) with three panels each on the front and 
back face and a station (right) consisting of two planes. (Bottom) 
The assembled tracker, with 18 stations. Stations are shown in 
grey and support structure in gold, Ref. \cite{bartoszek2015mu2e}.}
\label{fig:trkpanel}
\end{figure}
\begin{figure}[!h]
    \centering
    \includegraphics[width =0.7\textwidth]{figures/png/Screenshot_20240330_104830.png}
    \caption{The simulated resolution of the Mu2e tracker for electrons at the
    conversion energy. The asymmetric low-side tail is caused by the tracker 
    stochastic energy loss mechanism. The high-side tail, where decay-in-orbit 
    events would be much more present with respect to the signal region, is small. 
    The simulation take into account all beam, tracker, electronics and DAQ 
    properties, Ref. \cite{bobbb}.}
    \label{fig:trkres}
    \end{figure}
Figure \ref{fig:sttrk} shows that the Tracker straws are only 
in the active region with radius ranging from 380 mm to 700 mm. 
The central part of the Tracker is not instrumented. A similar annular 
design can also be found in the Mu2e Calorimeter, in Section \ref{calorimeter}. 
The geometry of the detector is specifically designed to reduce background. 
Detectors are blind to muon beams and associated activity, including $\pi$-DIF, 
$\mu$-DIF, beam electrons and other particles emitted during nuclear captures. 
Its shape prevents the detection of low-energy electrons released during muon DIO. 
Figure \ref{fig:sttrk} shows the electrons trajectory as coloured circles passing 
through the stopping target. For a homogeneous magnetic field, the radius of a 
helix follows the equation \ref{partincamp}, so it is proportional to transverse 
momentum of the particle. Low momentum electrons, $\lesssim$ 53 MeV/c, can pass 
through detectors without causing a hit. This corresponds to the green trajectory 
in Figure \ref{fig:sttrk}, whereas the green circle represents an example 
trajectory. Electrons with higher momenta can leave some hits on the 
detectors, but few hits are insufficient to reconstruct the electron 
trajectory, orange trajectory shown in Figure \ref{fig:sttrk}. The tracks 
are only reconstructible when the electron momentum is high enough: $\gtrsim$ 
90 MeV/c. It is expected that only muon DIO events will be recorded in the 
Tracker. The low rate simplifies front-end electronics (FEE) requirements 
and minimises reconstruction mistakes caused by accidental activity. 
Chapter \ref{chaptertrk} will provide a detailed discussion of the straw 
tube Tracker panel, including its detection mechanism, mechanical design, 
front-end electronics and data acquisition (DAQ) system, as the focus of 
my thesis is on the Vertical Slice Test of the straw tracker.
\begin{figure}[!h]
\centering
\includegraphics[width =0.7\textwidth]{figures/png/Screenshot_20240306_214911.png}
\caption{The annular design of Mu2e detectors: a view of the Tracker, Ref. \cite{trk}.}
\label{fig:sttrk}
\end{figure}
\subsection{The Electromagnetic Calorimeter}\label{calorimeter}
The Mu2e Electromagnetic Calorimeter serves multiple purposes, Ref. \cite{em4}:
\begin{itemize}
    \item the energy deposition measured in the calorimeter separates $e$ and $\mu$ with the same momentum.
    The separation in energy-to-momentum ratio ($E/p$) can be used to identify particles and suppresses
    part of the cosmic background caused by cosmic-ray muons;
    \item the calorimeter signals can improve the quality of track reconstruction. The additional information can
    be used for consistency check, reducing reconstruction errors caused by accidental activities;
    \item it provides triggers for the experiment independent of the Tracker, Ref. \cite{em6}. 
\end{itemize} 
The calorimeter sits downstream of the Straw Tube Tracker in the 1 T region in the Detector Solenoid and in a vacuum of $10^{-4}$ Torr. 
It consists of two annular detection disks, separated by 70 cm, Ref. \cite{em7}. 
The separation was chosen to match half of the distance between two periods in the helical trajectory of a
typical 105 MeV conversion electron. This maximizes the detection probability of conversion electrons in the
calorimeter. Electrons that pass through the hole of the first disk will be detected by the second one.
The disks inner radius is 35 cm and the outer one is 66 cm. A schematic view of the calorimeter is shown in Figure \ref{fig:calo1}.
\begin{figure}[!h]
    \centering
    \includegraphics[width =0.7\textwidth]{figures/png/Screenshot_20240322_122050.png}
    \caption{The Mu2e Calorimeter annular disks, Ref. \cite{em7}.}
    \label{fig:calo1}
\end{figure}
Each calorimeter disk is filled with 674 undoped cesium iodide (CsI) crystals, Ref. \cite{em6}, wrapped with a 150 $\mu$m foil of Tyvek. 
Tyvek is pure CsI, with a wavelenght peak emission at 315 nm and a scintillation time of 20 ns: it was chosen because of its
fast emission and low cost. The crystals are 34 $\times$ 34 $\times$ 200 mm$^3$ in dimensions, as shown in Figure \ref{fig:calo2} (top right). 
To improve reliability, light collection and resolution, each crystal is readout by two custom design SiPMs array, Ref. \cite{em1}. 
Magnetic field resistance was the factor in choosing between SIPMs and PMTs.
The crystal have fast emission time (decay time better than 40 ns) and an acceptable light yield, 20 photoelectrons/MeV per SiPM.
Undoped CsI crystals were also chosen because they are sufficiently radiation hard.
CsI radiation lenght is $X_0 \sim $2 cm and the crystal length is about 10 $X_0$:
it is sufficient to contain the 105 MeV CE showers since the average incident angle is 50°. 
To reduce thermal coupling between crystals and electronics, a 2 mm air gap is maintained between the crystal and the readout sensor. 
Since the main scintillation component has a wavelength of 310 nm to match the SiPM
photon detection efficiency as a function of wavelength, each crystal is coupled with a readout 
module consisting of two ultraviolet (UV)-extended Silicon Photomultiplier (SiPM) arrays
and the corresponding analog front-end electronics (FEE) boards, as illustrated in Figure \ref{fig:calo2} (bottom right), 
Ref. \cite{em5}, \cite{em2} and \cite{em3}. 
\begin{figure}[!h]
    \centering
    \includegraphics[width =0.7\textwidth]{figures/png/Screenshot_20240322_121000.png}
    \caption{Left: the CAD of the two calorimeter disks. Top right: the calorimeter parallelepiped shaped crystals.
            Bottom right: two SiPM arrays glued onto copper holder on the left and one readout 
            unit formed by two SiPM arrays and two FEE boards mounted on its
            copper holder and on the right, Ref. \cite{em4}.}
    \label{fig:calo2}
\end{figure}
\begin{figure}[!h]
        \centering
        \includegraphics[width =0.8\textwidth]{figures/png/Screenshot_20240322_121017.png}
        \caption{Left: the breakout of calorimeter mechanical components. Top right: the breakdown of front
        panel plate. Bottom right: the mezzanine and DiRAC boards, Ref. \cite{em4}.}
        \label{fig:calo3}
        \end{figure}

Each SiPM is composed of a 2 $\times$ 3 array of individual 6 mm $\times$ 6 mm cells. 
The array dimensions were chosen to maximise light collecting, with an active area of 1.8 $\times$ 1.2 cm$^2$ 
and a small total capacitance due to the parallel structure, resulting in a signal width of less than 200 ns.
Each SiPM is linked to its own power supply and high voltage in Front-End Electronics (FEE) board.
The crystal readout unit includes 2 SiPMs, 2 FEE boards, mechanical support for SiPM cooling, a Faraday cage to 
decrease noise and a fiber for laser calibration. A Mezzanine Board (MB) controls 20 amplifiers. Signals are sent from MB to the ADCs in the 
digitizer board (DiRAC). Signals will be sampled at 200 Msps, 5 ns binning, by the digitization system. 
As shown in Figure \ref{fig:calo3}, each board is positioned within the electronic crates that sorround the calorimeter disks.
Zero suppressed data are sent from the DiRAC board to the DAQ via optical fibers. 
The calorimeter is designed to have good energy resolution ($\sigma_E/E$ < 10\%), 
timing resolution ($\sigma_t \sim$500 ps for 100 MeV electrons) and position resolution
($\sim$6 mm). A 51-crystal prototype module, called Module-0, was exposed
to a test beam and results for energy and time resolution are
shown in Figure \ref{fig:calores}, Ref. \cite{bobbb} and \cite{calo95}.

\begin{figure}[!h]
    \centering
    \includegraphics[width =0.8\textwidth]{figures/png/Screenshot_20240330_105520.png}
    \caption{The energy and time resolution of the Module-0 for the Mu2e calorimeter for electrons at the conversion energy. (Top left) The resolution for electrons striking the
    array at normal incidence. (Top right) Resolution for electrons striking the array at 55° with respect to the face. (Bottom) The time resolution. 
    All measurements are compared with simulations (red line), Ref. \cite{bobbb} and \cite{calo95}.}
                \label{fig:calores}
                \end{figure}
\subsection{The Cosmic Ray Veto}\label{CRV}
As stated in Section \ref{backgrounds}, Mu2e expects backgrounds generated by cosmic rays 
at a rate of one per day. Although the EM calorimeter can detect some backgrounds created 
by muons with the correct momenta, there are still problematic cases. To reduce these 
backgrounds, Mu2e employs an active veto system combined with shielding. An overview 
of the Cosmic Ray Veto (CRV) is shown in Figure \ref{fig:crv}. It covers the entire 
DS and half of the TS (there is no veto at the bottom), for a total area of 327 m$^2$.
The CRV modules are manufactured from plastic scintillator extrusions. Each extrusion 
has a cross-sectional area of $5 \times 2$ cm$^2$, with different lengths. The extrusions 
are coated with titanium dioxide to improve internal reflections and so the light yield. 
Two grooves are extruded inside the scintillator bar throughout its length, containing 
1.4 mm diameter wavelength changing fibres. They send light to the extrusion ends, 
where each fibre is detected by a $2 \times 2$ mm$^2$ SiPM on each end. Figure \ref{fig:crvmodule} 
illustrates the cross-section of the CRV module. Each module has four overlapping layers of plastic 
scintillator counters to reduce the effect of gaps. The layers are separated by $\sim$ 10 mm aluminium 
plates used as absorbers. When three of the four layers of the CRV are activated, a veto window of $\pm$125 ns 
is provided. A conversion-like signal observed during this time window is assumed to be produced by a cosmic-ray muon. 
The anticipated total dead time is $\sim$5\%. The CRV system should be highly efficient. 
The success of the experiment depends on a detection effectiveness of 99.99\% or higher.

\begin{figure}[!h]
\centering
\includegraphics[width =0.85\textwidth]{figures/jpg/Crv_downstream.jpg}
\caption{The cosmic ray veto covering the Detector Solenoid looking upstream, showing the downstream (CRV-D), left (CRV-L), top (CRV-T) sectors, as well as the hole where the transport solenoid enters the enclosure, Ref. \cite{bartoszek2015mu2e}.}
\label{fig:crv}
\end{figure}

\begin{figure}[!h]
\centering
\includegraphics[width =0.85\textwidth]{figures/png/Crv_module_geometry.png}
\caption{The CRV module geometry and nomenclature. Internal gaps are those between the counters in a di-counter, Ref. \cite{Giovannella_2020}.}
\label{fig:crvmodule}
\end{figure}
\subsection{The Stopping Target Monitor}
A Muon Beam Stop (MBS) is installed at the downstream end of the DS to absorb muons that are not stopped in the stopping target, Ref. \cite{bartoszek2015mu2e}. The MBS is intended to limit the effects of muon decay and capture inside the stop. The magnetic field gradient prevents the majority of the low-energy charged particles created in the MBS from moving upstream. It is made from high-$Z$ minerals and polyethylene. Muons have an extremely short lifetime in high-$Z$ materials, as shown in Figure \ref{fig:muonicatom}, enabling activities to take place before the live window starting at 640 ns. Polyethylene, on the other hand, absorbs protons and neutrons emitted by excited nuclei created during muon capture.
The Stopping Target Monitor, STM, is placed downstream of the MBS to measure the number of muons collected on the stopping target, because of the extremely high x-ray and gamma rates, as shown in Figure \ref{fig:stm}. This number will be used as denominator in $R_{\mu e}$. Mu2e detects the x-ray spectrum released when muons are stopped on a stopping target and fall to the ground state. The experiment uses 357 keV x-ray photons created when muons go from the $2p \rightarrow 1s$ state, Ref. \cite{bobbb}, and aims to measure the number of stopped muons with the 10\% of accuracy. The STM employs an x-ray detection device, a high purity germanium (HPGe) solid-state detector. The detector should only view the target, if possible. A requirement is a good collimation ahead of the detector. It sees the stopping target through a stainless steel pipe 10 cm in diameter and 3 m long, tunneling through the MBS near the DS axis, Ref. \cite{stm}. Collimators inside the pipe give the STM a clear view of the stopping target while obscuring other components, such as the downstream collimator in the TS. HPGe devices are too slow to track individual occurrences. In addition, the germanium lattice is sensitive to radiation damage. To keep radiation damage and rates under control, the STM is approximately 35 m from the stopping target and is well-insulated. An alternative method measures the delayed photons released during the beta decay of $^{27}$Mg, which occurs in 13\% of muon catches. The excited $^{27}$Mg decays to an excited state of $^{27}$Al with a half-life of 9.5 min, emitting an 844 keV photon within ps that can be detected. This method uses the pulsed beam macrostructure (the 1.4 s cycle) to detect only proton batches that are not transmitted to Mu2e.
\begin{figure}[!h]
\centering
\includegraphics[width =\textwidth]{figures/png/Screenshot_20240306_180910.png}
\caption{The Stopping-Target Monitor geometry showing the DS region (left), the End Cap Shielding, sweeper magnet and STM field-of-view collimator. At the far end of the hall (right) is the final $spot-size$ collimator and the STM detector.}
\label{fig:stm}
\end{figure}


\chapter{The Mu2e tracker}\label{chaptertrk}

\textit{The following Chapter will focus on the Mu2e tracker, one of the most important detectors in Mu2e. 
It will give further information about the straw tube tracker panels, 
including the detection mechanism, the mechanical design and the Front End Electronics (FEE).
As previously pointed out, the Mu2e tracker is made up of 216 identical panels. 
Each panel is a mechanical and electrical standalone module. Each panel has 96 drift 
tubes used as fundamental detecting components. I will start by discussing the 
theoretical characteristics of drift tubes and then I will go more in detail with Mu2e tracker features. 
Most of the discussion is based on \cite{kola} and \cite{bobbb}.}

\section{Drift Tubes}
Gas detectors are capable of measuring charged particle coordinates. 
They provide great spatial resolution and detection efficiency at a low cost, Ref. \cite{kola}. 
There are many different gas ionizazion detectors, one of these is the drift tube.
The basic configuration of a drift tube is shown in Figure \ref{fig:drifttube}.
The cylindrical conducting tube, the cathode, is grounded.
A hollow cylindrical conducting tube is grounded and serves as the cathode.
The tube is filled with a combination of noble gas, often Argon, and quench gas. 
A thin sensing wire is suspended along the cylindrical cathode axis. 
The wire, called anode, receives a high voltage. Long and thin drift tubes, known as $straw$ $tubes$, have a similar form.
\begin{figure}[!h]
    \centering
    \includegraphics[width =0.8\textwidth]{figures/png/Screenshot_20240324_232621.png}
    \caption{Schematics of a drift tube, Ref. \cite{kola}.}
    \label{fig:drifttube}
    \end{figure}
Assuming an anode radius $a$, a cathode radius $b$ and using Gauss theorem, the electric field is:
\begin{equation}\label{avalanche}
    E(r)=\frac{1}{r}\frac{\lambda}{2\pi \epsilon}=\frac{1}{r}\frac{V}{ \text{ln}(b/a)} \qquad (a<r<b)
\end{equation}
Where $\lambda$ is linear charge density on the wire and $\epsilon$ is the dielectric constant of the gas.
Thinner wires are typically preferred in drift tubes. A greater electric field near the wire increases the amplification factor 
of the drift tube at the same voltage. Additionally, smaller wires improve spatial resolution, Ref. \cite{kola}. 
\subsection{Gas Ionization}
Two are the interactions that can deposit energy when particles traverse the gas volume: ionization and excitation.
Collisions between a charged particle $C$ and an atom $A$ can result in the ejection of one or more electrons: 
$A \ C \rightarrow A^+ e^- C$, or $A \ C \rightarrow A^{++} e^- e^- C$, if more than one electron is released.
This process is called primary ionisation. The mean energy loss per path length can be determined using the Bethe-Bloch formula.
A noble gas atom $A$ can turn also into an excited state $A^*$ through the interaction $A \ C \rightarrow A^* C$. 
If the excitation energy of $A^*$ is higher than the ionization potential of another species $B$ in the gas, the quencher, the
Penning Effect can produce ionization through $A^*B \rightarrow A B^+ e^-$. In addition, the noble gases can
also form molecular ions through processes such as $A^* A \rightarrow A_{2}^{*} \rightarrow A^+_2 e^-$.
A secondary ionisation can occurr through these processes or through electrons that have sufficient energy for generating more ions.
To compute the average number of electron-ion pairs produced by an initial particle, divide its total energy loss 
by the average energy required to make an electron-ion pair. Due to the energy lost during excitation, this average 
does not match the gas ionisation potential. Measurements showed an average of one electron-ion pair every 
30 eV, with variations depending on gas composition and starting particle. Except for very slow particles, 
this value remains constant regardless of their initial energy.
Without an electric field, electrons and ions created during ionisations spread uniformly. 
Collisions cause them to lose energy and eventually reach thermal equilibrium with the surrounding gas. 
They eventually recombine. An electron maximal range during ionisation is correlated with its initial kinetic 
energy. In a normal temperature and pressure gas, a 10 keV electron may be stopped in approximately 1 mm. 
Ionised electrons often have reduced kinetic energy, leading to a shorter range.
\subsection{Drift of Ions and Electrons}
The electric field accelerates free electrons and ions towards the anode and cathode 
along the field lines. As charges accelerate, they scatter on other particles in the gas, 
losing energy. The directions of motion are randomised, and maximum speeds are set. As a result, 
charges move uniformly along the electric field. This is referred to as the drift velocity of the 
charge. It is superimposed with the thermal motions.
Drift velocities for ions and electrons varies based on several parameters. Ions have a bigger mass 
than electrons and their masses are comparable to those of gas molecules. During collisions, gas 
molecules absorb a significant portion of the energy gained from ions during acceleration. In a drift 
tube detector just a little amount of electric field energy enters the energy associated with ion motion, 
making it comparable to the initial thermal energy before acceleration. 
The ion drift velocity $v_i$ is proportional to the reduced electric field $E/N$ where N is the number density of the gas 
and it is typically $\mathcal{O}(10^3)$ cm/ns, except in the region near to the anode wire where a stronger 
electric field is present. The ion thermal velocity is typically $\mathcal{O}(10^4)$ cm/ns at room temperature.
On the opposite, only a small fraction of the energy is released during elastic collision from electrons, 
so they acquire more energy from the electric field than their thermal energy.
Many different factors impact electron drift velocity. Some gas molecules, such as H$_2$O or CO$_2$, 
can interact with electrons to produce negative ions due to their great electron affinity. In rare situations, 
electrons gather enough energy to exceed gas molecule excitation threshold, resulting in inelastic collisions. 
Electron drift velocity is a complex function of electric field intensity due to several variables influencing 
electron collisions across a large energy range. Figure \ref{fig:drift} shows the electron drift
velocity at different electric field strengths in argon-carbon dioxide mixtures (Ar-CO$_2$) of different
proportions, Ref. \cite{ZHAO1994485}.
\begin{figure}[!h]
    \centering
    \includegraphics[width =0.7\textwidth]{figures/png/Screenshot_20240330_102206.png}
    \caption{Electron drift velocity versus electric field in Ar:CO$_2$ mixtures of different proportions, Ref. \cite{ZHAO1994485}. 
    80\%:20\% Ar:CO$_2$ mixture is the gas used in the Mu2e tracker.}
    \label{fig:drift}
\end{figure}
Electron drift velocities in drift tubes are typically $\mathcal{O}(10^6)$ cm/ns.
This is significantly higher than ion drift velocities and comparable to electron thermal velocity under the same conditions. 
The radial coordinate of an ionisation can be computed using the electron drift velocity and time.
Since drifting electrons and ions are scattered on gas molecules, they also diffuse along their trajectory. 
Electrons diffuse significantly quicker than ions because of their high velocity. Electron diffusion limits 
the intrinsic resolution of drift tubes used to measure incoming particle coordinates. CO$_2$ has internal 
degrees of freedom at low collision energies, preventing electron energies from exceeding thermal energy until 
field strengths above about 2 kV/cm. This improves intrinsic spatial resolution.
\subsection{Avalanche Multiplication}
Electrons can ionise when they face a high electric field near the anode wire. 
The released secondary electrons form tertiary electrons, and so on. Free electrons rapidly multiply, 
resulting in an avalanche. In a drift tubes, where the electron mean free path is about the order of $\mu$m, 
an avalanche develops when the electric field approaches $\mathcal{O}(10)$ kV/cm. 
According to Equation \ref{avalanche}, using $a \sim \mathcal{O}(10^{-3})$ cm, $b \sim \mathcal{O}(1)$ cm and
a normal voltage of 1-2 kV, the avalanches can occur within $\mathcal{O}(100) \ \mu$m from the anode wire. 
Electrons from the avalanche are collected on the anode wire within 1 ns, while positively charged ions move towards the cathode.
Drifting ions mostly generate signals in electrodes via induction. Figure \ref{fig:avalanche} shows the model of an ionisation avalanche.
\begin{figure}[!h]
    \centering
    \includegraphics[width =0.7\textwidth]{figures/png/Screenshot_20240330_182509.png}
    \caption{The model of an ionisation avalanche forming at the anode wire of a proportional tube or chamber, Ref. \cite{kola}. 
    (a) In the drift volume, electrons and ions are generated and drift to their corresponding electrodes. 
    (b) Near the wire, the electron achieves a high enough field to induce secondary ionisation, resulting in an avalanche. 
    (c) The electric field separates charges created during an avalanche. 
    (d) Electrons have higher lateral diffusion than ions, causing the avalanche to expand 
    around the wire and produce a positive charge cloud in the shape of a drop. 
    (e) Electrons from the avalanche reach the anode within nanoseconds, but ions take longer, up to ms, to reach the cathode.}
    \label{fig:avalanche}
\end{figure}
\subsubsection{Avalanche Gain}
When an avalanche develops, the amplification factor is around $10^4-10^6$. The number of electrons freed per unit path length is given
by the first Townsend ionization coefficient $\alpha=\sigma_{ion}n=1/\lambda_{ion}$ and this depends on the electric field $E$, 
as a higher electric field corresponds to a higher kinetic energy of the electron, that increases the ionization cross-section.
The increase $dN$ of the number of electron-ion pairs over a path length $ds$ is, Ref. \cite{kola}:
\begin{equation}
    dN=\alpha(E)Nds
\end{equation}
Solving this equation, we can easily obtain the gas amplification $G$:
\begin{equation}\label{av}
    G=\frac{N(s_a)}{N_0}=\text{exp} \left( \int_{s_0}^{s_a} \alpha(E(s)) ds \right)=\text{exp} \left( \int_{E_{min}}^{E(a)} \frac{\alpha(E(s))}{dE/ds} dE \right)
\end{equation}
where $N_0$ corresponds to unamplified electrons in $s=s_0$ and $E_{min}$ corresponds to the minimum energy 
for ionisation to occurr. The energy distribution depends on the electric field which is position dependent. 
Since the free path is inversely proportional to the particle density in the gas, $E_{min}(\rho)=E_{min}(\rho_0)\rho/\rho_0$.
It is reasonable to say that the coefficient is proportional to the field strength, $\alpha= \beta E$, in the low field region. 
Adding this relation with Equation \ref{avalanche} and \ref{av}:
\begin{equation}
     \text{ln}(G)=\beta \ a \ E(a) \ \text{ln}\left( \frac{E(a)}{E_{min}}\right)
\end{equation}
where $\beta$ can be related to $w_i$, that is the energy spent for one ionisation and its value is equal to $e \Delta V$.
As the voltage drop per unit path length is $dV = E(s)ds = (\alpha/\beta)ds$, we obtain $dN=N \beta dV$. Integrating, we can see that
$\beta= \text{ln}(2)/\Delta V$, so the gain in a drift tube is:
\begin{equation}
     \text{ln}(G)=\frac{ \text{ln}(2)}{\Delta V} \ a \ E(a)  \ \text{ln}\left( \frac{E(a)\rho_0}{E_{min}(\rho_0)\rho}\right) \qquad E(a)=\frac{V}{a ln(b/a)}
\end{equation}
which is the Diethorn's formula. 
Gain measurements with variable $\rho$/$\rho_0$, $a$, and $E(a)$ can provide the parameters $E_{min} (\rho_0)$ and $\Delta V$. 
The gas temperature $T$, pressure $P$ and operating voltage $V$ significantly impact the gain of a drift tube with a particular shape and gas mixture. 
\subsubsection{Quench Gas}
To avoid subsequent avalanches, the drift tube gas combination may contain a quench gas, such as CO$_2$, 
methane, or other hydrocarbons. During an avalanche, photons are created by gas deexcitation and electron 
attachment to electronegative species, resulting in negative ions. Photons can generate ionisations 
outside the primary avalanche zone or create free electrons on the cathode surface, resulting in secondary avalanches.
The difficulty arises when the signal created is not proportional to the deposited energy by the original particle 
and is no longer localised to the energy deposition point. Enough intense photons can induce a chain reaction of 
secondary avalanches, leading to a continuous discharge. The use of quench gas prevents subsequent 
avalanches by absorbing ionising photons before they travel far. A tiny quantity of quench gas during normal operation 
can significantly decrease secondary avalanches and breakdowns.
\subsubsection{Operation Modes of Gaseous Ionization Detectors}
In drift tube detectors, the number of electron-ion pairs formed during an avalanche is 
proportional to the starting number of electrons, as shown in the gain computation. To 
operate in a proportional mode, an appropriate voltage is needed to reduce the effects 
of avalanche charges on the electric field. Figure \ref{fig:gaseous} illustrates how a 
gaseous ionisation detector may work in multiple modes based on the operating voltage. Higher operating 
voltage leads to higher charges on the electrodes. Low voltage causes ionisation charges to recombine 
before reaching electrodes, leading to no signal collection. At higher voltages, in ionisation chamber 
region, charges can drift to electrodes, but the electric field is insufficient for avalanches to occur. 
Increasing the operational voltage leads to drift tubes and proportional counters. When the voltage 
becomes high enough, proportionality is lost. When electrons from an avalanche are collected, 
the high density of positive ions near the anode might affect the electric field. 
Electrons in future avalanches that enter the area between the positive ion cloud and the wire face a 
decreased electric field, resulting in lower amplification. The electric field becomes greater in the 
tail of the avalanche, which is far from the wire than the ion cloud. This range of voltage is called region of 
limited proportionality. When the operating voltage reaches high values, 
avalanches create sufficiently energy photons to cause secondary avalanches 
that propagate across the detector, independently from the quench gas. 
This results in detector saturating the output. This way of operating is 
called breakdown mode, commonly known as the Geiger-Muller mode.
\begin{figure}[!h]
    \centering
    \includegraphics[width =0.5\textwidth]{figures/png/Screenshot_20240330_203416.png}
    \caption{The dependence of particle gain on applied voltage in gaseous ionisation detectors. 
    The numbers on the axes are for orders of magnitude only and they depend on the device geometry and gas concentration. 
    Drift tubes operate in the proportional mode, Ref. \cite{kola}.}
    \label{fig:gaseous}
    \end{figure}
\subsection{Signal Creation and Propagation}
Drift tube signals do not originate from avalanche charges. In this case, the anode wire would 
receive the complete signal within a few ns. Signal pulses are created by charges on electrodes 
caused by electron and ion mobility. The Shockley-Ramo theorem, Ref. \cite{kola}, can be 
used to determine the induced charge and current. 
The Shockley-Ramo theorem yields various important results. The total induced charge of a moving charge 
$q$ is determined by the initial and final positions only.  A charge pair induces the same amount of 
charge on an electrode as the charge collected on it. Furthermore, if all electrodes are treated as 
an unity, their weighted potential will be one. 
If one electrode completely encloses the others, the weighted field in the contained region is always zero. 
This means that the total induced current across all electrodes is always zero. The Shockley-Ramo theorem can 
be applied to the drift tube. In an avalanche with $N$ electron-ion pairs, we evaluate the induced current 
signal on the anode wire. The current on the cathode has an additional negative sign. The weighted 
potential and field in the straw depend on the radial coordinate $r$:
\begin{equation}
    \psi_w(r)=\frac{\text{ln}(b/r)}{\text{ln}(b/a)}
\end{equation}
\begin{equation}
    \textbf{E}_w(r)=\frac{1}{\text{ln}(b/a)}\hat{\textbf{r}}
\end{equation}
As previously mentioned, the drift velocity of ions, $u$, is proportional to the electric field intensity $E$. Therefore, $u$ = $mu \ E$. 
Then the ion trajectory fulfils:
\begin{equation}
    u=\frac{dr(t)}{dt}=\frac{\mu V_0}{r(t)\text{ln}(b/a)}
\end{equation}
When ionisation occurs at the anode, the starting condition can be approximated as $r(0) = a$. So, the solution to the given differential equation is:
 \begin{equation}
    r(t)=a \sqrt{1+\frac{t}{t_0}} \qquad t_0=\frac{a^2 \ln (b / a)}{2 \mu V_0}
    \end{equation}
Here, $t_0$ represents the characteristic time, which is generally on the order of 1 ns. 
The time $t$ falls within the range $0 < t < t_{max}$, where $t_{max} = t_0 [(b/a)^2 - 1]$. 
The induced current and charge from the wire are expressed as:
 \begin{equation}
    I_{i o n}^{i n d}(t)=-(e N) E_w(r) \frac{\mathrm{d} r(t)}{\mathrm{d} t}=-\frac{e N}{2 \ln (b / a)} \frac{1}{t+t_0}
    \end{equation}
    \begin{equation}
        Q_{\text {ion }}^{\text {ind }}(t)=\int_0^t I^{\text {ind }}\left(t^{\prime}\right) \mathrm{d} t^{\prime}=-\frac{e N}{2 \ln (b / a)} \ln \left(1+\frac{t}{t_0}\right)
        \end{equation}
The total charge created on the wire by the travelling electrons is:
\begin{equation}
    Q_e^{i n d}=-e N\left[\psi_w(a)-\psi_w\left(r_{a v g}\right)\right]=-e N \frac{\ln \left(r_{a v g} / a\right)}{\ln (b / a)}
    \end{equation}
    where $r_{avg}$ is the average position of ionisations in the avalanche. When compared to the total charge 
    induced $Q^{ind}_{tot}(t_{max}) = -eN$, electron mobility in the avalanche accounts for just 1-2\%. 
    Positive ion drift from the avalanche accounts for the majority of the signal. A signal propagates 
    to both ends of the drift tube from the avalanche location.
    Signals with distinct frequency components can propagate at varying velocities. This causes the signal to disperse.
    When determining the longitudinal coordinate of an avalanche using the signal arrival time difference between straw tube ends, various complications emerge.
%For low-frequency components of the signal (for a meter-long drift tube, this translates to a frequency much less than 300 MHz), 
%a quasi-electrostatic approach is sufficient. The signals seen at tubes ends are influenced by impedances between and on the electrodes. 
%On the other hand, for the high-frequency components of the signal, the tube needs to be treated as a transmission line. 
%The propagation speed of a signal wave with frequency $\omega$ is then $c/\sqrt{\omega}$ $\epsilon$($\omega$), 
%where $c$ is the speed of light and $\epsilon$ is the dielectric constant of the gas. As $\epsilon$ is a function of $\omega$, 
%components of different frequencies propagate at slightly different velocities. This leads to a dispersion (widening) of the signal. It brings some subtleties when using signal arrival time difference between straw tube ends to determine the longitudinal coordinate of an avalanche.
\section{The tracker panels}
The Mu2e tracker straw tube arrays use the same detection principles as the gaseous ionisation detectors, Ref. \cite{kola}, 
but it has a significant different design and manufacturing improvements to meet the experiment precise requirements.
\begin{figure}[!h]
    \centering
    \includegraphics[width =0.5\textwidth]{figures/png/Screenshot_20240327_000000.png}
    \caption{A picture of the Mu2e straw tube
    compared to a pencil, Ref. \cite{trk}.}
    \label{fig:trkpencil}
    \end{figure}
    \begin{figure}[!h]
        \centering
        \includegraphics[width =0.5\textwidth]{figures/png/Screenshot_20240327_000131.png}
        \caption{The fully assembled tracker panel with its straws, Ref. \cite{trk}.}
        \label{fig:strawtubes}
        \end{figure}
The straw tubes used in the Mu2e tracker \ref{fig:trkpencil} are 5 mm in diameter and 0.33 - 1.17 m in length, 
Ref. \cite{bartoszek2015mu2e}. The straws are wound with two layers of 6 $\mu$m-thick metallized Mylar and a 3 
$\mu$m layer of glue in between. The straw wall is 15 $\mu$m thick: this helps to minimize the amount of materials 
in the tracker, lowering the total energy loss from the observed electrons. Furthermore, it minimises the likelihood 
of significant deflections in electron trajectories, facilitating track reconstruction. As a result, the experiment 
will have a very good momentum resolution. The straw tube anodes are composed of gold-plated tungsten wires 
with a diameter of 25 $\mu$m. The straws and anode wires are tensioned and work-hardened to avoid sagging.
\begin{figure}[!h]
    \centering
    \includegraphics[width =0.6\textwidth]{figures/png/Screenshot_20240326_234405.png}
    \caption{Straw arrangement in a tracker panel, Ref. \cite{trk}.}
    \label{fig:trktubes}
    \end{figure}
To create a perfect electric field with adequate spatial resolutions, the anode wires are oriented 
to the panels with a precision of at least 25 $mu$m in the radial direction and 50 $mu$m in the 
perpendicular direction. All panels are x-ray scanned to accurately measure and document the wire 
locations of straw tube channels. Figure \ref{fig:trkpencil} shows the 
termination mechanism that holds the wire in place.
To increase mechanical strength, a brass tube is joined to either end of the straw using silver epoxied, Ref. \cite{bartoszek2015mu2e}. 
To ensure electrical isolation, a Kapton sleeve is put within the brass tube. An injection-molded plastic insert is then connected 
inside the sleeve. The insert contains a semicylindrical duct that allows gas to enter and exit the straw tube. The insert 
has a groove along its axis and a U-shaped brass anode pin inserted at the end. To avoid slippage, the anode wire is 
epoxied into the groove and soldered to the anode pin.
A T-shaped pin protector protects the anode pin from breaking by covering the groove in the plastic insert. 
The pin protector is epoxied to the insert, with an extra brass ring connecting them. A ground clip 
is silver-epoxied to two adjacent straws on the brass tubes and rings, providing a shared ground connection 
for the straws. 
Signals are sent to a common pre-amplifier (preamp) board via the anode pin pair and grounding clip. 
The FEE will be introduced later. Figure \ref{fig:strawtubes} shows completely built straws in a tracker panel. 
Each panel has 96 straw tubes organised in two staggered layers for improved tracking. Figure \ref{fig:trktubes} 
show the detailed spatial arrangement of the straws, Ref. \cite{trk}. Channels are numbered from 0 (longest) 
to 95 (shortest), starting with the radially innermost straw.
Straws are spaced by 1.25 mm and can expand under gas pressure, Ref. \cite{bartoszek2015mu2e}.
The Mu2e tracker panels employ a gas flow of 80\%:20\% Ar:CO$_2$. 
The vacuum environment within the Detector Solenoid reduces the impact of electron scattering on trajectories. 
However, the tracker panels, particularly the straws, must endure pressure differences. Under normal temperature 
and pressure, each panel must have an average leak rate of 0.014 cm$^3$/min. The nominal operating voltage of 
the straw tube channels is 1450 V. An earlier investigation on a prototype of the tracker showed the straw tube 
gain as 1.25 $\times$10$^4$ at 1250 V and 7 $\times$ 10$^4$at 1425 V. According to Diethorn's calculation, Equation \ref{XXX}, 
the gas gain at 1450 V is around 1 $\times$10$^5$.

\section{The tracker Front-End Electronics}
Before being accessible to the Mu2e Data Acquisition (DAQ) system, the analog signals 
from the tracker straw tube channels need amplification, digitization and packaging. 
The tracker front-end electronics (FEEs) are designed to achieve these goals. The FEEs 
consist of multiple Printed Circuit Boards (PCBs) as shown in Figure \ref{fig:trackerfee}, Ref. \cite{vadimmu2e}.
All PCBs are situated in the outer section of the panel. Pulse timing is measured at the 
end of each straw, in order to be able to measure electron position on the wire. A measure 
of $dE/dx$ is provided, so pattern recognition can be possible, Ref. \cite{bartoszek2015mu2e}. For this purpose each straw has:
\begin{itemize}
    \item two preamp channels, one for each end;
    \item two TDC channels, one for each end;
    \item one ADC channel, measuring sum of both ends;
    \item one High Voltage feed.
\end{itemize}
There are 46,080 preamps and TDCs and 23,040 ADC channels. There are two sides of the panel, one called HV side and one called CAL side. 
\begin{figure}[!h]
\centering
\includegraphics[width =0.8\textwidth]{figures/png/Screenshot_20240131_111836.png}
\caption{An overview of the tracker front-end electronics (FEE), Ref.  
\cite{vadimmu2e}. The preamps and the Analog Motherboard (AMB) 
on the other side of the straws are not shown in the figure.}
\label{fig:trackerfee}
\end{figure}
On either side, there are an Analog Mother Board (AMB) and a Jumper board, 
which task consist of directing signals from the preamps towards the Digital 
Motherboard (DMB) positioned at the center, then to the Digitizer Readout and 
Assembler Controller (DRAC) (mounted on top of the DMB) to be processed and 
temporarily stored. Both the AMBs and the DMB handle low voltage distribution 
and the HV side of the AMB distributes high voltage to the straw anode wires, 
reason why it is called this way. On the AMBs and the DMB there are sensors, 
monitoring environmental variables such as temperature, pressure and humidity. 
The low voltage power supply is fed into the panel through the KEY. The KEY 
contains an optical fiber link and a JTAG interface for communication. The 
frontends components were chosen to sustain high level of radiation.
\subsection{Pre-Amplifiers}
The pre-amplifiers (preamps) are responsible for the initial readout of 
signals from both ends of the straw tube channels. As previously explained, 
the channels are read out from both ends and two adjacent channels are linked 
to the same preamp. Each tracker panel contains 48 preamps on the HV side, 
while an additional 48 preamps are located on the opposite side, the CAL one. 
Preamps are mounted vertically on the AMBs. Preamps are required to have a 
matching 300 $\Omega$ input impedance with the straw, in order to avoid signal 
reflections. The preamps convert the straw tube current signals into voltage signals. 
Signals are amplified and shaped. The preamps on the CAL and HV sides aren't 
exactly the same. The CAL preamps have circuitry that can inject calibration 
pulses into the channel. This capability enables the channel readout electronics 
to be tested without a high voltage source. The preamps on the HV side distribute 
the high voltage supply to individual straw tubes. 
%The voltage gains of individual channels, different from the gas gain, are set by control signal from the DRAC. A bias voltage, adjustable for each side of a channel, is applied to the signals. 
\subsection{Digitizer Readout and Assembler Controller}\label{DRAC}
The brain of a tracker panel is called Digitizer Readout and Assembler Controller (DRAC). 
The DRAC is responsible for digitizing, packaging and temporary storage. It also controls 
panel operations. The schematics of the DRAC board is shown in Figure \ref{fig:drac}. In 
this figure Analog to Digital Converters (ADCs), DDR3 memories and compators are shown. 
The large chips in the centers are the Field-Programmable Gate Arrays (FPGAs). The one 
in the center is the Readout Controller (ROC), which manages communications, monitors 
slow control variables and controls panel operations \ref{ROC}. The left and the right 
ones, which are referred to as the digi FPGAs, are responsible for monitoring data output, 
buffering data and assembling data packages. Each of them refers to 48 straw channels. 
\begin{figure}[!h]
\centering
\includegraphics[width =\textwidth]{figures/png/Screenshot_20240204_115052.png}
\caption{Digitizer Readout and Assembler Controller (DRAC) board schematics, Ref. \cite{drac}. 
The DRAC board is the brain of the tracker panel. In this figure Analog to Digital Converters (ADCs), FPGAs, DDR3 memories and compators are shown.}
\label{fig:drac}
\end{figure}
In figure \ref{fig:flowfee}, signal flow in the tracker FEEs is reported. At the 
beginning the signal coming from both ends of the straw is routed to the preamps. 
After that, the analog signal is sent through the microstrip transmission line to 
the digitizers. In the DRAC, the two biased signals are fed independently to 
zero-crossing comparators, which produce square pulses when the signals exceed 
their respective thresholds. The squared pulses are sent to Time-to-Digital Converters 
(TDCs, firmware based, 16 bits each) implemented in the digi FPGAs, that have the task 
of digitizing the trigger timings, including the arrival time and the time over threshold, 
at a rate of about 62.5 MHz. Besides drift time, TDCs measure time difference across the 
straw to estimate position along the wire and the intrinsic time resolution of TDC is about 
25 ps, adding comparator jitter, noise and other external effects the final resolution is 
$\sim$ 70 ps for time division. Furthermore, an integrator adds voltage signals coming 
from the two straw ends. In data collection, a hit occurs when both ends of a straw channel 
are simultaneously activated. The total is digitized by a 12-bit (10-bit ENOB) ADC at 40 MHz
and then sent to the digi FPGA. The digi FPGA creates a data packet for each hit based on 
TDC and ADC information. This suppresses false triggers caused by random electrical noise. 
The Digitizer receives signals from both ends of four straws and multiplexes them into one 
output buffer before sending a packet of data to the ROC. Signals are sent to the ROC at 
200 MHz. The data packets are then briefly stored in DDR3 memory for further use by the 
DAQ system. It is important to save also the voltage signal, since the pulse height 
information can help us to reject proton background, a significant source of noise 
or to distinguish muons from electrons. The proton signal will appear as a saturated flag, 
since the proton $dE/dx$ is $\times$50 the electron one.
\begin{figure}[!h]
\centering
\includegraphics[width =0.8\textwidth]{figures/png/Screenshot_20240203_135048.png}
\caption{Signal flow through front end electronics, Ref. \cite{bartoszek2015mu2e}.}
\label{fig:flowfee}
\end{figure}

\subsection{Read-Out Controller}\label{ROC} 
The main job of the ROCs (one per panel, 216 in total) 
is to collect data from the digitizer boards, buffer data and 
send them to DAQ. They are based on an FPGA architecture. They 
continuously stream out the zero-suppressed data collected between 
two proton pulses from the detectors, in this case the tracker, to 
the Data Transfer Controllers (DTCs), Ref. \cite{GIOIOSA2023167732}. The buffer 
stage is fundamental, since during the beam inter-spill time (836 ms 
out of each 1333 ms), we want us to be able to take data from cosmic 
rays, even if the rate will be very low. For this purpose, the ROCs 
include external DRAM. The communication is flexible, thanks to the 
programmable nature of digitizer, ROC and DAQ. 
\section{Requirements on tracker Performance}
Here a resume of the requirements that the tracker must satisfy to ensure the success of the experiment is presented, Ref. \cite{trkreq}.
A momentum resolution less than 180 keV/c for a 105 MeV/c electron is needed in the nominal
1 Tesla solenoidal field, as measured at the front face of the tracker volume (before
passing through any tracker related material). Non-Gaussian tails, particularly any
high-side tail, must be controlled such that the DIO background results in much less
than one event at design sensitivity. To reach this, simulation results indicate that 
a single straw requires around 4 cm of longitudinal and 200 $\mu$m of transverse resolution for drift path lengths. 
It must have an acceptance of approximately 20\% for conversion electrons.
The tracker must operate in an ambient vacuum (< $10^{-4}$ Torr).
It should be able to withstand a rate of 5 MHz per straw (highest rate straw) 500 ns after the spill. This is
for background studies. The nominal experiment live time starts 700 ns after the spill.

\chapter{DAQ readout testing}
\textit{In this Chapter, I present the initial results of the tracker DAQ commissioning. Before reading out the detector, 
    it was necessary to understand the readout process. This Chapter is divided into three sections: the first relates 
    to the validation of ROC readout through Monte Carlo simulation, subsequently, I will discuss the initial and primitive tracker panel data quality monitoring, 
    and in the third part, I will address high-rate software testing.}
  \section{Description of teststand setup}
    The tracker test stand,  called TS1, shown in Figure \ref{fig:TS1}, included one DRAC card \ref{DRAC} connected via an optical fiber
    to the DTC installed in the DAQ computer, called mu2edaq09. The teststand included 96 channels in total.
    The ROC \ref{ROC} could be operated in two different data readout modes:
    \begin{itemize}
    \item  First mode: the ROC was emulating the data itself without reading FPGAs (a pattern readout mode);
    \item  Second mode: the ROC was reading digi FPGAs.
    \end{itemize}
    \begin{figure}[!h]
        \centering
        \includegraphics[width =0.8\textwidth]{figures/jpg/IMG_20240219_090538.jpg}
        \caption{The tracker test stand, TS1, featuring one DRAC card linked via optical fiber to the DTC in the DAQ computer (mu2edaq09). The setup includes a total of 96 channels.}
        \label{fig:TS1}
        \end{figure}
    Most of the data were taken operating in the second mode, with digi FPGAs, pulsed by their internal pulsers.
    The pulser has two different frequencies,  31.29 MHz/(2$^7$+1), or approximately 250 kHz, 
    and 31.29 MHz/(2$^9$+1), or approximately 60 kHz.
    Event window is the time interval between two heartbeats (HB's). 
    A timing diagram of a single channel readout is shown in Fig. \ref{fig:3}.
    Pulses, separated by $T_{gen}=1/f_{gen}$, where $f_{gen}$ is the generator frequency
    are represented by gray triangles.
    The event window, with the width of $T_{EW}$, that represents the distance between the proton pulses, 
    was varied from 700 ns to 50 $\mu$s, to test the system and to increase flexibility during tests, but during Mu2e data taking it will be approximately 1.7 $\mu$s. 
    The ROC firmware has an internal hit buffer which stores up to 255 hits.
    Depending on $T_{gen}$ and $T_{EW}$, the data taking can proceed in two different
    scenarios:
    \begin{itemize}
    \item
      The event window is large enough , so the total number of generated hits is greater than 255. In this case
      the ROC hit buffer always gets filled up, and only the first 255 hits are read out;
    \item
      The total number of hits within the event window is less than 255.
      In this case the ROC hit buffer doesn't get filled up and the total number of hits may vary from one event to another.
    \end{itemize}
    
    Each digi FPGA has its own pulse generator and the pulse sequences from the two
    generators are offset with respect to each other by a time interval $\Delta t$.
    The offset is constant for as long as the DRAC board is powered up and varies randomly between 0 and $T_{gen}$ when DRAC is powercycled.
    The timing of the readout is uncorrelated with the generator timing sequences,
      so the number of pulses within the readout window can vary from one event to another, as shown
      in Figure \ref{fig:3}.
    
    Relative timing offsets of the channels within the same FPGA are of the order of a few ns.
    The channel readout sequence is fixed.
    
    \begin{figure}[!h]
    \centering
    \includegraphics[width =\textwidth]{figures/png/finalimg.png}
    \caption{Graphic illustration of pulses in an event window.}
    \label{fig:3}
    \end{figure}
    
    %The data taking has been performed $\mu$sing OTSDAQ+ARTDAQ software, and for each run the output data have been stored in an art file moved to {\bf /exp/mu2e/data/projects/tracker/vst} area mounted on Mu2e central platforms.
    
 \section{XXX da preamp a dato} 
\section{Validation of ROC readout through Monte Carlo simulation}

%%%%%%%%%%%%%%%%%%%%%%%%%%%%%%%%%%%%%%%%%%%%%%%%%%%%%%%%%%%%%%%%%%%%%%%%%%%%%%
\subsection{Monte Carlo simulation}\label{MonteCarlo}
 
The ROC readout logic is purely digital, so the readout process can be simulated. 
The logic of the simulation is as follows.
The simulated parameters for each event are the number of hits in each channel
and the total number of readout hits.

In the following sections, we call $occupancy$ the total number of hits
recorded in a given channel during the test run.

Given that the total number of hits per event doesn't exceed 255, the simulation follows these steps:
\begin{itemize}
\item
  The event window starts at $t=0$s;
\item
  In a given FPGA, the timing of the first pulse is generated randomly from 0 to $T_{gen}$
  by sampling a uniform distribution;
\item
  The subsequent pulses are added, separated from each other by $T_{gen}$,
  until the absolute time
  of the next pulse is greater than $T_{EW}$;
\item
  In the readout part of the simulation, the pulses are read out following the readout sequence;
\item
  the readout ``continues'' until all simulated hits are included or
  the total number of read out hits reaches the maximum threshold of 255. 
\end{itemize}

The simulation also takes into account the offset between the two FPGA timing sequences
and the individual channel-to-channel timing offsets. 
In the following sections, results of the data taking are compared with the simulation.
\subsection{``ROC buffer overflow'' mode}
%%%%%%%%%%%%%%%%%%%%%%%%%%%%%%%%%%%%%%%%%%%%%%%%%%%%%%%%%%%%%%%%%%%%%%%%%%%%%%
In the ``ROC buffer overflow'' configuration, which will be referred to as RUN281, the event window size is 50 $\mu$s
and the pulser rate is 60 kHz.

\subsubsection{Hit timing and occupancy}\label{over}
The first distributions to look at are the time distributions of hits in 
different channels and the distribution of the total number of hits
in a given channel (occupancy) as a function of the channel number.
The timing distributions of hits in channel 0 of the first FPGA
and in channel 2 of the second FPGA are shown in Fig.\ref{fig:1}.
The left distribution is, as expected, uniform, however the right one looks
less trivial.

\begin{figure}[!h]
  \hspace{-0.5in}
  \begin{tikzpicture}
    \node[anchor=south west,inner sep=0] at (0,0.) {
      % \node[shift={(0 cm,0.cm)},inner sep=0,rotate={90}] at (0,0) {}
      % \makebox[\textwidth][c] {
      \includegraphics[width=0.6\textwidth]{figures/pdf/figure_00007_timedistr_roc_simulation_ch0_281.pdf}
      % }
    };
    \node[anchor=south west,inner sep=0] at (9,0.) {
      % \node[shift={(0 cm,0.cm)},inner sep=0,rotate={90}] at (0,0) {}
      % \makebox[\textwidth][c] {
      \includegraphics[width=0.6\textwidth]{figures/pdf/figure_00003_timedistr_roc_simulation_ch2_281.pdf}
      % }
    };
  \end{tikzpicture}
  \caption{
    \label{fig:1}
    Left: time distribution of hits in the channel 0 in the first FPGA. Right: time distribution of hits in the channel 2 in the second FPGA.
    }
\end{figure}

The distributions in Fig.\ref{fig:1} are easier to understand by looking at the occupancy plot in Fig.\ref{fig:2} (left).
The channel ordering in this plot corresponds to the readout order.
Channels in the beginning of the readout sequence always have all their hits read out,
  however that is not true for the channels in the end of the readout sequence.
\begin{figure}[!h]
  \hspace{-0.5in}
  \begin{tikzpicture}
    \node[anchor=south west,inner sep=0] at (0,0.) {
      % \node[shift={(0 cm,0.cm)},inner sep=0,rotate={90}] at (0,0) {}
      % \makebox[\textwidth][c] {
      \includegraphics[width=0.6\textwidth]{figures/pdf/figure_00004_nhitsvschannel_roc_simulation_281.pdf}
      % }
    };
    \node[anchor=south west,inner sep=0] at (9,0.) {
      % \node[shift={(0 cm,0.cm)},inner sep=0,rotate={90}] at (0,0) {}
      % \makebox[\textwidth][c] {
      \includegraphics[width=0.6\textwidth]{figures/pdf/figure_00014_nhitsvschannel_roc_simulation_281.pdf}
      % }
    };
  \end{tikzpicture}
  \caption{
    \label{fig:2}
    Left: number of hits versus the channel number. The channels are numbered in the readout order.
    Right: zoom on the last channels in the readout sequence. The data and MC distributions
    differ from each other by $\sim$ 10$^{-3}$.
  }
\end{figure}

Figure \ref{fig:66} shows the distribution of the number of hits in channel 0.
For the event window of 50 $\mu$s and the time between the pulses of 16 $\mu$s,
the number of hits could be 3 or 4,
depending on the timing offset of a given readout window with respect to the generated timing sequence.
This distribution plays a key role in understanding of the occupancy plot.
\begin{figure}[!h]
\centering
\includegraphics[width =0.8\textwidth]{figures/pdf/figure_00066_nhits_ch00_run281.pdf}
\caption{
  The distribution of the number of hits in the channel 0 of the first digi FPGA (RUN281).
}
\label{fig:66}
\end{figure}

In the distribution shown in Figure \ref{fig:2} (left),
the first 68 channels are the ones with 4 hits per channel in the first FPGA
and three hits per channel in the second FPGA, 
resulting in the total of 255 hits.
The second plateau extending from 68 to 75 corresponds to the channels
with 3 hits per channel in the first FPGA and 4 hits per channel in the second one.
  The ``dent'' in the end of the second plateau is due to the fact that the 48 channels of the first FPGA
  yield 144 hits, so the second FPGA contributes 111 hits. The first 27 channels of the second FPGA contribute
  4 hits per channel each, but as 111 is not an integer of 4, the three hits from channel 28 in the readout sequence
  fill up the total ROC buffer of 255 hits.
There is a big step at the end of this plateau, because if we count the number of hits
in the first FPGA we get 144, so in the second FPGA we have 111 hits in total,
due to the fact that the maximum number of hits in total is 255.
111 is not divisible by 4, so the first 27 channels in the second FPGA will have 4 hits
and the last one will have 3 hits.
The last plateau corresponds to events with 3 hits per channel from the first FPGA
and 3 hits per channel from the second FPGA.

A zoom on last channels is shown on the right picture of Fig.\ref{fig:2}.
The relative difference between the data and the MC distributions is at a level of $10^{-3}$,
which is a very good agreement.
Coming back to Fig.\ref{fig:1}, the first channels in the readout sequence
always have all their hits read out,
while the channels in the end of the readout sequence do not,
as the ROC hit buffer gets filled up after
the first 255 hits are read out.
This results in a uniform time distribution for the first channels readout and in a non-uniform
time distribution for the last readout channels, depending on $T_{gen}$ and $T_{EW}$.
The dips in the hit timing distribution for channel 2 are defined by the timing offset
between the two FPGA pulsers. 


%%%%%%%%%%%%%%%%%%%%%%%%%%%%%%%%%%%%%%%%%%%%%%%%%%%%%%%%%%%%%%%%%%%%%%%%%%%%%%
\subsubsection{Number of hits}
Fig. \ref{fig:3} shows that in the ``buffer overflow'' mode all events,
as expected, have 255 hits read out per event.

\begin{figure}[!h]
\centering
\includegraphics[width =0.8\textwidth]{figures/pdf/figure_00008_nhits_281.pdf}
\caption{
  The distribution of the total number of hits read out per event.
}
\label{fig:3}
\end{figure}
\subsection{The ``regular'' mode }
In the ``regular'' configuration, which will be referred to as RUN105038, the event window size is 25 $\mu$s
and the pulser rate is 60 kHz.

\subsubsection{Time distribution and occupancy}

Similar to Section \ref{over}, Figure \ref{fig:4} shows the distributions
of the number of hits in two channels, one from the 
first and another one - from the second FPGA. 
In this readout configuration, the expected number of pulses in a given channel
within the event window is one or two, and the total number of pulses is always below 255.

\begin{figure}[!h]
  \hspace{-0.5in}
  \begin{tikzpicture}
    \node[anchor=south west,inner sep=0] at (0,0.) {
      % \node[shift={(0 cm,0.cm)},inner sep=0,rotate={90}] at (0,0) {}
      % \makebox[\textwidth][c] {
      \includegraphics[width=0.6\textwidth]{figures/pdf/figure_00001_timedistr_roc_simulation_10538.pdf}
      % }
    };
    \node[anchor=south west,inner sep=0] at (9,0.) {
      % \node[shift={(0 cm,0.cm)},inner sep=0,rotate={90}] at (0,0) {}
      % \makebox[\textwidth][c] {
      \includegraphics[width=0.6\textwidth]{figures/pdf/figure_00012_timedistr_roc_simulation_ch2_105038.pdf}
      % }
    };
  \end{tikzpicture}
  \caption{
    Left: the hit time distribution for this in channel 2, the second FPGA. Right: the hit time distribution for hits in channel 0, the first FPGA.
  }
  \label{fig:4}

\end{figure}
In this mode, the readout of a given channel is not affected by the readout of previous
channels and the ``occupancy'' distributions shown in Figure \ref{fig:5} are, as expected, uniform.
\begin{figure}[!h]
\centering
\includegraphics[width =0.8\textwidth]{figures/pdf/figure_00002_nhitsvschannel_roc_simulation_2.pdf}
\caption{Occupancy: the number of hits versus the channel number for RUN105038.}
\label{fig:5}
\end{figure}


Fig.\ref{fig:67} shows the distribution of the number of hits in the channel 0 (first FPGA).
\begin{figure}[!h]
\centering
\includegraphics[width =0.8\textwidth]{figures/pdf/figure_00067_nhits_ch00_run105038.pdf}
\caption{
  The distribution of the number of hits in channel 0, first FPGA, for RUN105038.
  Entries in the $n$(hits)=0 bin are due to the readout errors.
}
\label{fig:67}
\end{figure}

%%%%%%%%%%%%%%%%%%%%%%%%%%%%%%%%%%%%%%%%%%%%%%%%%%%%%%%%%%%%%%%%%%%%%%%%%%%%%%
\subsubsection{Number of hits}
Compared to RUN281, the event window in RUN105038 was twice shorter
and the ROC readout buffer wasn't getting filled up.
The total number of hits within the event window depends on the relative offset
of the event window with respect to the FPGA pulsers, and varies from
144 to 192, as shown in Figure \ref{fig:6}.

\begin{figure}[!h]
\centering
\includegraphics[width =0.8\textwidth]{figures/pdf/figure_00009_nhits_105038.pdf}
\caption{
  The distribution of the total number of hits per event in ``non-saturated'' mode.
}
\label{fig:6}
\end{figure}
\section{Primitive Data Quality Monitoring}

\section{High rate software testing}
\subsection{Firmware-Based Data Acquisition}
The Mu2e Trigger and Data Acquisition (TDAQ) system collects digitized data from the Tracker, Calorimeter, Cosmic Ray Veto and Beam Monitoring components (Stopping Target Monitor and Extinction Monitor) and delivers that data to online and offline processing for analysis. It must merge data from $\sim$450 subsystems and apply filters to reduce data volume by a factor of 100 before storing it offline. It is also responsible for detector synchronization, control, monitoring and operator interfaces. The Mu2e DAQ system $\mu$ses a $streaming$ readout technique, which means that all detector data from the experiment is digitized and zero-suppressed in their respective Front End Electronics (FEEs) before being transferred. This strategy results in a high data flow in the DAQ system while providing greater flexibility in data selection and analysis.
\subsubsection{Expected rate}
The data-taking periods will be divided in two modes: on-spill and off-spill. The on-spill mode covers periods when 8 GeV proton bunch are colliding the production target. The off-spill mode covers all other periods: between bunches, calibration periods, commissioning. 
Calling what is written in section XX, it is possible to estimate the on-spill event contribution:
\begin{itemize}
    \item 43.1ms (time of one spill)/ 1695ns(digitization time) = 25K pulses per spill
    \item 8(number of spills)*25K=200K on spill events /cycle
    \item 200K/ 1.4s(cycle time) = 145K ON Spill events/s
\end{itemize}
di questi 1.4s, 1.055s si riferisce all'offspill e 0.4s all'onspill.
Meanwhile, the off-spill event contribution is:
\begin{itemize}
    \item 145K ON Spill events/s*0.4s=58Kevents
    \item 55 K /1.4s= 41K off spill events per second per cycle
\end{itemize}
buffering: 0.4/1.4
\\
TOTAL INPUT: 186Kevents/s (Hz).
\\
INPUT EVENT SIZE:150KBytes/bunch(1695ns)=90GB/s
\\
buffering:0.4s(on spill)/1.4s(off spill)
\\
TOTAL: 90GB/s*0.4s(on spill)/1.4s(off spill)= 28GBytes/s
\\
online processing factor: 100(trigger that is after the farm manager and it is an approximation because it depends on the the input rate)
\\
TOTAL OUTPUT (that goes to boardreaders): 1.5kHz
\\
TOTAL: 1.5Kevents /s (Hz)*150KBytes/event=225MBytes/s
\\
The detector will generate $\sim$150 KB of zero-suppressed data per bunch (in un bunch ci sono 1695 ns), Section \ref{accel}, for an average data rate of $\sim$90 GB/s when beam is present. To reduce DAQ bandwidth requirements, this data is buffered in Readout
Controller (ROC) memory during the spill period and transmitted to the DAQ over the full supercycle for an average data rate of $\sim$28 GBytes/sec.

In Figure \ref{fig:linktodaq}, the general design of the Mu2e DAQ system is shown.
\begin{figure}[!h]
\centering
\includegraphics[width =0.8\textwidth]{figures/png/Screenshot_20240206_144803.png}
\caption{Mu2e DAQ Architecture.}
\label{fig:linktodaq}
\end{figure}
The left blocks represent the Readout Controllers (ROCs) in different detectors. The center block houses the DAQ system's online components, which include the Run Control Host, 40 DAQ servers, the Detector Control System (DCS) and the Event Building Switch. The Run Control Host receives beam status and timing information from the Accelerator Controls network and operator commands from the remote control room. The Detector Control System (DCS) is the window onto the status and health of the Mu2e detector. The Event Building (EVB) function combines these subsets to form a complete detector data set for analysis by an online processor, Ref. \cite{bartoszek2015mu2e}. Event building is typically done in a switching network to sustain high rates. The right block houses the DAQ system's offline components, which are $\mu$sed for data storage and processing. During an active spill (the first approximately 43 ms of the 48 ms bunch extraction cycle outlined in Chapter \ref{accel}), the experiment receives RF Zero-Crossing Markers from the Accelerator that are synchronized to the 1695 ns proton pulse cycles (the event windows). Based on these markers, the Command Fan-Out (CFO) module within the Run Control Host generates a 40 MHz system clock and encodes Event Window Markers (EWMs) in the system clock to indicate the start of the event windows. The CFO then sends the encoded system clock, along with run control packets, to Data Transfer Controllers (DTCs) in the DAQ servers. The DTCs\footnote{The Mu2e Data Transfer Controller (DTC), Ref. \cite{ryan} takes data from various Read-Out Controllers and may conduct event construction and data preprocessing. The DTC module connects a maximum of six ROCs to the Trigger and Data Acquisition (TDAQ) servers, which execute the TDAQ online software framework.} then transfer the encoded clock to the detectors' ROCs, where the EWMs are recovered with fixed delay relative to the original RF Zero-Crossing Markers and $\mu$sed in the local ROCs to discriminate data acquired during consecutive event windows. The Tracker generates a DDR3 memory address at the beginning of each event window. The relevant memory area is designated to hold Tracker hits received during that event window. Data requests trigger data readouts from the ROCs to the DAQ system. The Data Requests are modifiable through the CFO as described by the Run Plan, although they are initially given to the Tracker and Calorimeter ROCs via the DTCs following each event window.

\subsubsection{TDAQ software: artdaq}





\subsubsection{Expected rate}





Each hit is composed of a data packet having a fixed length of 128 bits (16 bytes):
\begin{itemize}
    \item 16 bit header - it contains information as a packet header, a channel identifier to specify the channel so the ROC can assign the hit to a wire number and a packet checksum;
    \item 16 bit - TDC left straw end;
    \item 16 bit - TDC right straw end;
    \item 8$\times$10 bit ADC.
\end{itemize}
A packet of 128 bits can be transferred every 640 ns (200 Mbps).An additional 32
bits must be added as an end-of-file marker after the data $\mu$spill hit data is buffered.
The ROC-to-DAQ connection is made via fiber optic links arranged in rings, with multiple ROC per ring, as shown in Figure \ref{fig:linktodaq}. This is possible since a single optical link can handle 2.6 Gbps, while the ROC output is around 230 Mbps. This value comes from the fact that the highest rate for any 4 straws group (corresponding to one digitizer data line to the ROC) is 240 kHz or 30 Mbps (at 128 bits/hit) and the Main Injector supplies Mu2e beam only 32\% of the time. The ROC monitors slow control variables and controls panel operations too.


XXXXXXXXX: mettere le caratteristiche di Ref. \cite{vadi}  ?????

\chapter{Pre-pattern recognition studies}\label{delta}
\textit{In the context of DAQ testing and performance analysis during Mu2e data-taking, 
where the data volume is expected to reach approximately 7 PBytes per year, optimizing 
memory usage and minimizing CPU consumption is crucial. The primary source of hits in the 
Mu2e tracker will be $\delta$-electrons. Therefore, it is essential to effectively flag these 
hits without compromising the efficiency of CE hit detection and track reconstruction. This 
Chapter presents a comparison between two algorithms designed for $\delta$-electron flagging.}

\section{$\delta$-electrons as source of background}

The primary source of hits in Mu2e is due to $\delta$-electrons, 
which mainly consist of Compton-scattered electrons, pair-production 
electrons and positrons, and delta rays, listed in decreasing 
order of prevalence. Compton-scattered electrons are produced 
when photons, generated by various processes, interact with the 
detector material. These photons primarily originate from 
neutron capture by atoms, which excites the nucleus and leads 
to photon emission during decay. Typically, these photons have 
energies of a few MeV. The neutrons are produced when muons are 
captured by atoms, resulting in unstable isotopes that decay by 
emitting neutrons. Pair-production electrons and positrons are 
generated during nuclear recoil processes, where pairs of electrons 
and positrons are created to conserve energy and momentum. Delta rays, 
or secondary ionization electrons, are generated when high-energy 
charged particles collide with the detector material.


\subsection{Compton scattering}
The Compton effect (Figure \ref{fig:compt}) refers to the 
scattering of a photon by a free or quasi-free electron. 
An electron is considered "quasi-free" when the energy of 
the incoming photon is significantly higher than the 
electron's binding energy ($E_\gamma \gg E_B$). The 
scattering process is termed Compton scattering if the 
electron is ejected from the atom, carrying away the 
recoil momentum. This effect is most prominent in an 
extended energy region around 1 MeV, with the region 
being much larger for low $Z$ materials compared to high $Z$ materials.

\begin{figure}[!h]
    \centering
    \includegraphics[width =0.4\textwidth]{figures/png/Screenshot_20240812_204345.png}
    \caption[The Compton effect.]{
    The Compton effect \cite{kola}.}
    \label{fig:compt}
\end{figure}

Since the photon scatters quasi-elastically off the electron, 
the energy and angle of the scattered photon are interdependent. 
To describe this relationship, we use the 4-momenta defined as 
follows: $k = (E_\gamma, \mathbf{k}c)$ and $p_e = (m_e c^2, 0)$ 
represent the 4-momenta of the photon and the electron (at rest) 
before scattering, and $k' = (E'_\gamma, \mathbf{k}'c)$ and 
$p'_e = (E'_e, \mathbf{p}'_e c)$ represent the 4-momenta 
after scattering. The angle between the scattered photon and 
the incident photon is denoted as $\theta_\gamma$, while the 
angle of the electron is denoted as $\theta_e$. By applying 
energy-momentum conservation:

\begin{equation}\label{compcons}
k + p_e = k' + p'_e
\end{equation}
\begin{equation}\label{compcons2}
(k - k')^2 = (p'_e - p_e)^2 \Rightarrow -k \cdot k' = m_e^2 c^4 - p'_e \cdot p_e
\end{equation}
\begin{equation}
\Rightarrow E_\gamma E'_\gamma (1 - \cos \theta_\gamma) = m_e c^2 \left(E'_e - m_e c^2\right) = m_e c^2 \left(E_\gamma - E'_\gamma \right)
\end{equation}

The right-hand side of the last equation uses the kinetic energy of the electron:

\begin{equation}
T = E'_e - m_e c^2 = E_\gamma - E'_\gamma
\end{equation}

which follows from the energy part of equation \ref{compcons}. 
The energy of the scattered photon as a function of the scattering 
angle is derived from equation \ref{compcons2}:

\begin{equation}\label{diffeq}
E'_\gamma = \frac{E_\gamma}{1 + \epsilon (1 - \cos \theta_\gamma)}
\end{equation}

where $\epsilon = \frac{E_\gamma}{m_e c^2}$.

The differential cross section per (free) electron, known as the 
Klein-Nishina formula, is calculated using methods from quantum electrodynamics:

\begin{equation}\label{kleinnishina}
\frac{d\sigma}{d\Omega} = \frac{r_e^2}{2} \frac{1 + \epsilon (1 - \cos \theta_\gamma)}{[1 + \epsilon (1 - \cos \theta_\gamma)]^2} \left(1 + \cos^2 \theta_\gamma + \frac{\epsilon^2 (1 - \cos \theta_\gamma)^2}{1 + \epsilon (1 - \cos \theta_\gamma)} \right)
\end{equation}

An electron bound in an atom can only be considered quasi-free 
if the photon's energy is significantly higher than the electron's 
binding energy. As the photon energy increases, more shell electrons 
become quasi-free, leading to the Compton cross section per atom 
approaching proportionality to $Z$, with individual electrons 
contributing incoherently:

\begin{equation}
\sigma_C^{\text{atom}} = Z\sigma_C
\end{equation}

where $\sigma_C$ is the Klein-Nishina cross section for a 
single free electron. The Compton cross section decreases at 
lower energies, where coherent scattering (Rayleigh scattering) 
off the entire atom (without ionizing the electron shell) becomes dominant.

By reformulating the Klein-Nishina formula, one can obtain the 
differential dependence of the Compton cross section on the 
kinetic energy of the recoil electron $T = E_\gamma - E'_\gamma$:

\begin{equation}
\frac{d\sigma}{dT} = \frac{\pi r_e^2}{m_e c^2 \epsilon^2} \left[2 + \frac{t^2}{\epsilon^2 (1 - t)^2} + \frac{t}{1 - t}\left(t - \frac{2}{\epsilon}\right)\right]
\end{equation}

where $t = T/E_\gamma$. Because the scattering process is 
elastic, there is a one-to-one relationship between the 
energy and angle $\theta_e$ of the electron:

\begin{equation}
\cos \theta_e = \frac{T(E_\gamma + m_e c^2)}{E_\gamma \sqrt{T^2 + 2m_ec^2 T}} = \frac{1 + \epsilon}{\sqrt{\epsilon^2 + 2\epsilon/t}}
\end{equation}

The maximum energy transfer to the electron is obtained 
from equation \ref{diffeq} for backward scattering of the 
photon ($\theta_\gamma = 180^\circ$), corresponding to 
forward scattering of the electron ($\theta_e = 0^\circ$). 
The electron's kinetic energy reaches its maximum value in 
this case, $T \rightarrow T_{\text{max}}$. In the measured 
energy spectrum, this leads to the so-called "Compton edge" at:

\begin{equation}
T_{\text{max}} = \frac{E_\gamma \cdot 2\epsilon}{1 + 2\epsilon}
\end{equation}

which lies slightly below the photopeak. The energy difference 
between the photopeak and the Compton edge $E'_\gamma(\theta = \pi)$ 
decreases with increasing $E_\gamma$ and approaches:

\begin{equation}
E'_\gamma(\theta = \pi) \approx \frac{m_e c^2}{2} \text{ for } E_\gamma \gg m_e c^2
\end{equation}

\subsection{Pair production}
In the Coulomb field of a charge, a photon can 
convert into an electron-positron pair (Figure 
\ref{fig:pprod})\footnote{Photon emission by an 
electron (bremsstrahlung) and pair production are closely 
related processes. By modifying the bremsstrahlung diagram-changing 
the outgoing photon to an incoming one and the incoming electron to 
an outgoing positron-one obtains the pair production 
diagram. The matrix elements of these processes are 
related, at least in the lowest order. Consequently, both 
processes are treated together in the foundational work by 
Bethe and Heitler, often referred to as the 'Bethe-Heitler processes'.}.

\begin{figure}[!h]
    \centering
    \includegraphics[width=0.4\textwidth]{figures/png/Screenshot_20240812_204755.png}
    \caption[The pair production.]{The pair production \cite{kola}.}
    \label{fig:pprod}
\end{figure}

The energy of the photon must exceed twice the electron 
mass plus the recoil energy transferred to the field-producing 
charge. For most elements, pair production predominantly 
occurs in the Coulomb field of the nucleus. For nuclei, 
the recoil energy is usually negligible, leading to a 
threshold energy for pair production of:
\begin{equation}
    E_{\gamma} \geq 2m_e c^2 + 2 \frac{m_e^2}{m_{\text{nucleus}}} c^2
\end{equation}

If the nuclear charge is not screened by atomic electrons 
(for low energies, the photon must come relatively close to 
the nucleus to make pair production probable, meaning it 
interacts with the "bare" nucleus),
\begin{equation}
    1 \ll \epsilon \ll \frac{1}{\alpha Z^{1/3}}
\end{equation}
the pair-production cross section is given by:

\begin{equation}
    \sigma_{\text{pair}} = 4 \alpha r_e^2 Z^2 \left(\frac{7}{9} \ln 2 \epsilon - \frac{109}{54}\right) \text{ cm}^2/\text{atom}
\end{equation}

However, for complete screening of the nuclear charge ($\epsilon \gg 1/\alpha Z^{1/3}$):
\begin{equation}\label{sigmapair}
    \sigma_{\text{pair}} = 4 \alpha r_e^2 Z^2 \left(\frac{7}{9} \ln \frac{183}{Z^{1/3}} - \frac{1}{54}\right) \text{ cm}^2/\text{atom}
\end{equation}

At high energies, pair production can occur even 
at relatively large impact parameters between the 
photon and the nucleus. In this case, the screening 
effect of atomic electrons must be considered. For 
large photon energies, the pair-production cross 
section approaches an energy-independent value as given by
 Equation \ref{sigmapair}. Ignoring the small term in the equation, 
 the asymptotic value of $1/54$ is expressed as:
\begin{equation}
    \sigma_{\text{pair}} \approx \frac{7}{9} \cdot 4 \alpha r_e^2 Z^2 \ln\left(\frac{183}{Z^{1/3}}\right) \approx \frac{7}{9} \cdot \frac{1}{X_0} \cdot \frac{A}{N_A \rho}
    \label{eq:paircross_radiationlength}
\end{equation}

The energy is uniformly distributed between the produced 
electrons and positrons at low and medium energies, but becomes 
slightly asymmetric at high energies.

The field of the nucleus is formed by the coherent sum of $Z$ 
nucleon charges, leading to the $Z^2$ dependence of the pair 
production cross section.

Even with large momentum transfers $\Delta p$ to the nucleus, 
the energy transfer $(\Delta p)^2/2M$ remains small due to the 
large nuclear mass $M$. After pair creation, the remaining 
energy is equally divided between the $e^+$ and the $e^-$.

\subsection{Delta rays}
High-energy $\delta$-rays, or $knock-on$ electrons, 
are produced when a projectile particle collides 
centrally with shell electrons, resulting in 
significant energy transfers. These electrons 
gain high kinetic energy and can be described 
through elastic collisions with quasi-free electrons. 
By considering the energy-momentum conservation relation 
and using the Lorentz factors $\gamma$ and $\beta$, the 
relationship between the kinetic energy $T$ of the 
$\delta$-ray and the emission angle $\theta$ can be derived as:

\begin{equation}
\cos \theta = \frac{T(\gamma + m_e / M)}{\gamma \beta \sqrt{T^2 + 2T m_e c^2}}
\end{equation}

\begin{equation}
T(\theta) = \frac{2 m_e c^2 \beta^2 \gamma^2 \cos^2 \theta}{\gamma^2(1 - \beta^2 \cos^2 \theta) + 2 \gamma m_e / M + m_e^2 / M^2}
\end{equation}

The maximum energy transfer $T_{\text{max}}$ occurs at $\theta = 0^\circ$, 
while the minimum energy, $T_{\text{min}}$, occurs at $\theta = 90^\circ$. 
At highly relativistic energies ($\gamma \gg 1$ and $\theta \gg 1/\gamma$), 
the energy-angle relationship becomes independent of the incoming particle's properties.

The rate of $\delta$-rays per energy interval $dT$ and path length $dx$ is given by:

\begin{equation}
\frac{d^2 N}{dx \, dT} = n_e \frac{d\sigma}{dT}
\end{equation}

which, when combined with the electron density and the differential cross section, becomes:

\begin{equation}
\frac{d^2 N}{dx \, dT} = \frac{1}{2} z^2 \frac{Z}{A} K \rho \frac{1}{\beta^2} \frac{F(T)}{T^2}
\end{equation}

Here, $K$ is the constant from the Bethe-Bloch formula, 
and $F(T)$ is a function accounting for spin dependence. 
Integration over $T$ and $x$ provides the number of $\delta$-rays in a medium of thickness $\Delta x$:

\begin{equation}
N = \frac{1}{2} z^2 \frac{Z}{A} K \rho \Delta x \frac{1}{\beta^2} \left(\frac{1}{T_{\text{min}}} - \frac{1}{T_{\text{max}}}\right) \approx 0.077 \frac{\text{MeV cm}^2}{\text{g}} z^2 \rho \Delta x \frac{1}{T_{\text{min}}}
\end{equation}

The emission angle dependence is given by:

\begin{equation}
\frac{dT}{d \cos \theta} = 4 m_e c^2 \frac{\cos \theta}{\sin^4 \theta}
\end{equation}

Substituting this into the rate equation yields:

\begin{equation}
\frac{d^2 N}{dx \, d \cos \theta} = \frac{1}{2} z^2 \frac{Z}{A} K \rho \frac{1}{\cos^3 \theta} \frac{1}{m_e c^2} \approx 0.15 \frac{\text{cm}^2}{\text{g}} z^2 \rho \frac{1}{\cos^3 \theta}
\end{equation}

This expression diverges as $\theta$ approaches $90^\circ$, 
where $T$ approaches zero, indicating a limitation in the 
assumption of a free electron. The resulting distributions 
suggest that $\delta$-rays emitted at small angles can 
significantly affect the spatial resolution in detectors, 
particularly through ionization clusters that broaden the 
track of the mother particle.



\section{Monte Carlo samples}\label{datasample}
For our studies, we used three distinct Monte Carlo data samples: two were 
related to CEs combined with two different proton pulse intensity, while the third was 
dedicated to antiproton annihilation events. The production of particles within 
the PT and their tracking from the PS to the DS is managed using the Mu2e Offline software 
(Appendix \ref{mu2eana}). The simulation of particle interactions is based on 
GEANT4, while the event processing is handled by the art framework and data 
management is governed by the SAM system.

The Mu2e simulation employs a technique known as multi-stage simulation 
to generate and simulate events efficiently. This method involves generating 
events, partially simulating them, and then pausing the process to save the 
data generated thus far. Later stages resume the simulation from the saved 
state. This approach optimizes both the time required for event generation 
and the usage of computing resources. The simulation is divided into seven 
stages, S0 through S6, which will be briefly discussed in the context of our specific needs:

\begin{itemize}
    \item \textbf{Stage 0}: primary beam protons are generated and propagated 
    inside the PT. When an inelastic interaction occurs at the target, the vertex 
    position is recorded, and the propagation is halted. The probability of 
    inelastic interactions is determined by the total inelastic cross-section 
    ($\sigma_{\text{tot.inelastic}}$) of protons on tungsten (W), which follows the 
    empirical cross section $\sigma_{\text{tot.inelastic}}(\text{W}) = (1710 \pm 30) \ \text{mb}$.
    The probability that a proton from the beam induces an inelastic interaction with the tungsten target is computed using a Monte Carlo approach 
    ($    \frac{N_{\text{inelastic}}}{N_{\text{POT}}} = 0.792$). Regarding the $\bar{p}$ 
    data sample, the number of antiprotons produced per proton on target (POT) at the production target is given by:
    $$
    N_{\bar{p}_{\text{POT}}} = \frac{\sigma_p}{\sigma_{\text{tot.inelastic}}} \times N_{\text{inelastic}} = \frac{0.5 \times 0.2824 \, \text{mb}}{1710} \times 0.792 = 6.5 \times 10^{-5}.
    $$
    Antiprotons are generated with a flat momentum distribution between 0 and 5 GeV/c and 
    are assumed to have isotropic directions;
    
    \item \textbf{Stage 1}: events are propagated through the TS and PS. 
    Only those particles that reach a virtual detector (VD) in front 
    of the TS entrance are saved as input for the next stage;

    \item \textbf{Stage 2}: in this stage, $\pi^-$, $\mu^-$, and $\bar{p}$ 
    particles are stored and tracked. To enhance statistics, a resample factor 
    is applied here: each $\bar{p}$ from S1 is traced multiple times, creating 
    independent events due to statistical fluctuations in energy losses within 
    the absorbers. A resample factor of $10^5$ keeps oversampling under control;

    \item \textbf{Stage 3}: particle transportation from the COL3u exit to the TS5 entrance;

    \item \textbf{Stage 4}: particle transportation from TS5 to the ST, 
    where the positions of stopped pions and muons are stored;

    \item \textbf{Stage 5}: to optimize simulation efficiency, $3 \cdot 10^7$ 
    stationary pions and antiprotons are generated at rest at the stopping 
    target, using the recorded time and position of the stops. These particles 
    are then propagated to the detectors. For the CE dataset, a fraction of muons 
    are assumed to decay into conversion electrons. For the $\bar{p}$ dataset, 
    antiproton annihilation at rest in the ST is simulated based on the positions 
    and times of the stopped antiprotons. Background electrons from annihilation 
    result from decays such as $\pi^0 \to \gamma \gamma$, followed by photon 
    conversions, and $\pi^- \to \mu^- \nu$, followed by $\mu^-$ decays. During 
    this stage, the raw simulated data are digitized into simple C++ classes or 
    structs, mimicking the detector's raw data;

    \item \textbf{Stage 6}: Finally, reconstruction from detector hits is performed.
\end{itemize}

Regarding CPU time, Stages 0 and 1 are the most time-consuming. 
Stage 2 can also be demanding when a high resampling factor is used, 
while the later stages are significantly faster by an order of magnitude.

A typical Mu2e event includes multiple hits from particles produced 
by muon captures in the ST, as well as particles entering the DS from 
the TS. These hits are called pileup hits, which make 
up the majority of the detector's hits. The pileup level 
depends on the proton pulse intensity. The Mu2e pileup 
simulation assumes that the pulse intensity varies on a 
time scale much longer than 2 $\mu$s, meaning all proton 
pulses around the simulated one share the same intensity. 
Under this assumption, a transformation is applied to hits 
with time $T_i > 1695$ ns outside the microbunch limits, 
assigning them a new time $t_i = \frac{T_i}{1695}$ ns, 
effectively accounting for late hits that would otherwise 
be attributed to later microbunches. 

For low-intensity (1BB) mode, with a mean intensity of 
$1.6 \times 10^7$ protons/pulse, approximately 25,000 
muons stop in the ST per pulse. In the high-intensity 
(2BB) mode, this number is about 2.5 times higher (Section \ref{pulsedprotonbeam}).

The datasets used for conversion electron plus pileup 
analysis will be referred to as $CELE1BB$ and $CELE2BB$ 
for 1BB and 2BB pileup, respectively. The dataset for 
antiproton analysis without pileup will be referred to as $PBAR0BB$.
For datasets with pileup, 
the pileup hits are explicitly added to the hits from 
the signal process. In this case, however, no pileup hits 
are added, meaning the data represents pure signal. 
\section{$\delta$-electrons producing hits in the Mu2e tracker}

In the Mu2e experiment, low-energy electrons and positrons, referred to as $\delta$-electrons, 
with a momentum below 20 MeV/c, constitute the majority of the tracker hits. Managing these 
hits is crucial for optimizing memory efficiency and CPU consumption. From a physics point 
of view, there are different reasons why it is important to identify those hits:
\begin{itemize}
    \item the signal we want to observe is the CE. Flagging even a small fraction of 
    potential CE hits is extremely dangerous, as it could result in the failure to 
    reconstruct the corresponding tracks.
    As shown in the histogram in Figure \ref{fig:momhits}, the simulated CE hits make up 
    just 1\% of the total hits in the tracker;
    \item flagging protons would assist the STM in counting the muon stopping rate. 
    It is possible to estimate the number of muons captured in the stopping target 
    by counting the number of protons produced during nuclear muon capture, 
    which is one of the possible processes a stopped muon can undergo (61\%);
    \item misidentifying muons and pions as protons or $\delta$-electrons could lead to a 
    incorrect estimate of the background events. This is particularly significant for the antiproton background.
    $p\bar{p}$ annihilation at rest in the ST can produce events with more than one track, 
    each with a momentum around 100-200 MeV/c. For $p\bar{p}$ annihilation events 
    in the ST, the rate of multi-track events is about 500 times higher 
    than the rate of events with a single signal-like electron. 
    For $10^4$ $p\bar{p}$ annihilation events generated, about 3.7\% of 
    the events contained two reconstructable particle tracks. Therefore, 
    the identification and reconstruction of multi-track events could be 
    used to constrain the $\bar{p}$ background. Thus, it is crucial
     not to flag muons or pions, as the fraction of multi-track events is very low.
\end{itemize}
Figure \ref{fig:momhits} shows the true momentum distribution of 
hits that make at least one hit in the tracker in the case of a 
dataset containing at least one CE per event. Each momentum bin is filled with 
the number of hits corresponding to a specific Monte Carlo particle. 
The data sample description used to perform this distribution 
is presented in Section \ref{datasample}.
The distribution reveals that the majority of hits originate from 
low-energy electrons and positrons (orange), constituting approximately 
75\% of the total number of hits. This histogram corresponds to 
the 1BB pileup scenario. 
There is an asymmetry between the number of hits below 20 MeV/c 
produced by electrons and positrons: electrons account for 
71\% of all $\delta$-electron hits in the tracker, while positrons 
contribute only 4\%. This discrepancy arises because only electrons 
undergo Compton scattering, which is the primary source of hits at 
energies around 1 MeV. This difference will be crucial in 
the following sections when discussing the $\delta$ flagging efficiency.

As evident from the histogram, 14\% of the total hits are due to protons, 
which are produced by nuclear processes. 
In particular, the first peak in protons momentum distribution arises from 
inelastic neutron scattering and while larger momentum values correspond muon capture at rest.
Their kinetic energy ranges from about 5 to 20 MeV, resulting in low $\beta \gamma$ values, 
making them heavily ionizing particles. The deposited energy will be 
one of the variables used to discriminate protons. 

It is important to note that the bump around 50 MeV/c in 
the positron distribution should not be present. According 
to the Monte Carlo truth, this is due to muon Decay-In-Flight, 
and we expect $N(\mu^+ \rightarrow e^+ )/N(\mu^- \rightarrow e^- )$ to be about $10^{-3}$ 
for muons entering the DS. The Decay-In-Orbit on the IPA 
(Section \ref{detectorsolenoid}) should also be around 
$10^{-3}$ compared to the DIF of negative muons. 
The simulation of $\mu^+$ may contain some errors, 
which we have reported to the simulation specialists. 
However, this issue is not problematic for the analysis 
of low-momentum electrons and positrons, as the momentum ranges are different.

\begin{figure}[!h]
        \centering
        \includegraphics[width =0.95\textwidth]{figures/png/Screenshot_20240812_152905.png}
    \caption[Monte Carlo momentum distribution of particles producing hits in the Mu2e tracker (conversion electron dataset and pileup).]{
       The Monte Carlo momentum distribution of particles producing at 
       least one hit in the Mu2e tracker (conversion electron dataset and pileup). The distribution
       corresponds to 1BB pile-up. The momentum distribution 
       of all particles making hits is depicted in dark blue, with electrons 
       shown in pink, positrons in light blue, $\delta$s in orange, protons and deuterons in 
       light green, and CEs in dark green. }
       \label{fig:momhits}
\end{figure}


Figure \ref{fig:pbar} shows the true momentum distribution of 
particles produced in the $p\bar{p}$ annihilation at the ST. 
The data sample description used to perform this distribution 
is presented in Section \ref{datasample}.
Each momentum bin contains the number of hits corresponding 
to a specific Monte Carlo particle. The dataset used to generate 
this histogram has no pileup (0BB). 
The particles produced in the $p\bar{p}$ annihilation 
are mostly pions, muons, and a few electrons. 
It can be observed that the momentum distribution 
peaks in the 100-200 MeV/c range. Photons are also 
produced, and they can undergo Compton scattering and 
pair production, which explains the presence of a 
$\delta$-electron peak that is about $\sim$100 
times lower than the peak in Figure \ref{fig:momhits}.

\begin{figure}[!h]
    \centering
    \includegraphics[width =0.9\textwidth]{figures/png/Screenshot_20240815_124710.png}
\caption[Monte Carlo momentum distribution of particles producing hits in the Mu2e tracker ($\bar{p}$ dataset and no pileup).]{
   The Monte Carlo momentum distribution of particles producing at 
   least one hit in the Mu2e tracker ($\bar{p}$ dataset and no pileup). 
   The momentum distribution 
   of all particles making hits is depicted in dark blue, with electrons 
   shown in pink, positrons in light blue, $\delta$s in orange, protons and deuterons in green, pions in light brown and muons 
   in black. }
   \label{fig:pbar}
\end{figure}

As shown in Figure \ref{fig:momhits}, the majority of the recorded hits 
originate from $\delta$-electrons, with protons being the second most 
common source. Figure \ref{fig:afbef} presents an example of comparison of an 
event before and after the background hits have been flagged. 
Flagging these hits is crucial for several reasons: it prevents unnecessary 
data from being sent to the pattern recognition algorithms, thereby conserving 
CPU resources and reducing processing time. Moreover, it is important to avoid 
storing these hits on tape, as doing so would lead to inefficiencies in data storage.

\begin{figure}[!h]
    \begin{subfigure}[b]{0.4\linewidth}
        \centering
        \includegraphics[scale = 0.3]{figures/png/Screenshot_20240811_123612.png}
        \subcaption{Before.}
        \label{fig:bef}
    \end{subfigure}
    \begin{subfigure}[b]{0.7\linewidth}
        \centering
        \includegraphics[scale = 0.3]{figures/png/Screenshot_20240811_124245.png}
        \subcaption{After.}
        \label{fig:af}
    \end{subfigure}
    \caption{Before and After background hits flagging. The transverse $x-y$ views of a CE 
    event with 2BB pile-up (Section \ref{pulsedprotonbeam}). The segments are the $hit$ tracker straws. The hits marked in
    red are from electrons and the ones in blue are from positrons.}
    \label{fig:afbef} 
\end{figure}






\section{Mu2e $\delta$-electrons rejection algorithms}
The Mu2e Offline software tool is illustrated in 
Appendix \ref{mu2eana}, and the reconstruction 
process is described in Appendix \ref{eventreco}. 

The flagging of $\delta$-electrons is a step in the 
Mu2e reconstruction process that occurs before time clustering and pattern recognition.

In Mu2e Offline, there are two types of flagging algorithms:
\begin{itemize}
    \item $FlagBkgHits$, described in Section \ref{flagbkghits};
    \item $DeltaFinder$, described in Section \ref{deltafinder}.
\end{itemize}

\subsection{The $FlagBkgHits$ Algorithm}\label{flagbkghits}
The detailed description of multivariate analysis (MVA) and the 
process of MVA training are beyond the scope of this work. 
Nevertheless, since this technique is one of the options 
developed for background flagging, I will briefly outline 
the fundamental principles involved.

When searching for patterns in a multivariable space, 
a common procedure involves defining a set of statistical 
models that analyze the measured variables and estimate 
the probability that these are consistent with the 
sought pattern. Once the variables are selected, 
the MVA is trained to recognize patterns by 
evaluating examples known to the trainer, 
allowing for feedback to refine the 
pattern identification process.

The first selection concerns the mean energy deposited in  
the $ComboHit$s used to create a $StereoHit$ and excludes 
those with a deposited energy above 5 keV. 
The algorithm then classify as protons all hits with a deposited energy above 4.5 keV.
Other selections concern the timing, position, and spread of the $ComboHit$ 
in the $x$-$y$ plane. The maximum cluster timing 
and diameter are 20 ns and 5 mm, respectively.

Since $\delta$-electrons tend to form small, dense 
clusters of hits on $x$-$y$ plane, an algorithm based on a clustering 
approach was developed: concentrated clusters in the 
$x$-$y$ plane are sought using a clustering algorithm, 
as $\delta$-electrons are highly likely to reside 
in such clusters. The $FlagBkgHits$ algorithm employs 
a hit-level MVA to ensure that hits within a 
$\delta$-dominant cluster truly belong there, 
and a cluster-level MVA to differentiate 
$\delta$-dominant clusters from those 
predominantly containing conversion electrons. 
This was trained in a supervised mode, using CE and $\delta$-electrons hits from 
Monte Carlo data sample.

The current resolution of straw drift tube 
position measurements is limited to a few cm. 
However, an improved position measurement can be 
achieved by leveraging the multiple straw layers that 
significantly overlap in the transverse plane. 
By obtaining two hit measurements from a pair of 
intersecting straws and ensuring they fall within 
a time window on the order of the maximum drift time, 
one can deduce that these hits were produced by the 
same particle and occurred at the intersection of the 
two straws in the projected plane. This method, 
involving two-dimensional information from the two 
straws, is referred to as the stereo information method.

The clustering process uses the $x$ and $y$ coordinates 
of the selected hits. A random hit is chosen to define 
the initial cluster, after which the following iterative 
steps are applied. The centroid of each cluster is 
computed, and hits whose distances from a cluster 
centroid fall within a specified inner threshold 
and time window are added to that cluster. Hits 
with distances from all existing clusters greater 
than an outer threshold are used to seed new 
clusters, while hits that fall between the two 
thresholds remain unassigned. This process 
generates a new set of clusters, preparing the 
system for the next iteration. During each 
iteration, every hit is reconsidered as a 
potential new point in a cluster, including 
those already assigned. The clustering process 
continues until convergence is achieved, i.e., 
when an iteration no longer results in any changes.


\subsection{$DeltaFinder$ Algorithm}\label{deltafinder}

$DeltaFinder$ is an algorithm designed to identify $\delta$-electron 
hit patterns rather than CE hits. This algorithm relies on the 
fact that $\delta$-electrons typically form a straight line in 
the $r$-$z$ plane (cyan lines in Figure \ref{fig:yzviewdelta}) 
and appear as a spot with a diameter of less than 3 cm in the 
$x$-$y$ plane within the Mu2e magnetic field. In contrast, 
CE hits create entirely different patterns, appearing as oblique lines in 
the $r$-$z$ plane due to their helical trajectories (red lines in Figure \ref{fig:yzviewdelta}). 
In the process, all hits associated with detected objects are treated as $\delta$-electron hits.
\begin{figure}[!h]
    \centering
    \includegraphics[width =0.6\textwidth]{figures/png/Screenshot_20240811_123048.png}
    \caption[$\delta$-electrons and CE patterns in $r-z$ plane.]{
        $\delta$-electrons and CE patterns in the $r-z$ plane. 
        The white squares represent some of the tracker stations. 
        The cyan straight lines represent the $\delta$-electron 
        patterns in the $r-z$ plane, while the red lines a CE signal in the same plane.
        }
    \label{fig:yzviewdelta}
\end{figure}
\subsubsection{Step 1: Identifying $\delta$-Electron Segments}
$DeltaFinder$ first seeks to identify $\delta$-electron track 
segments within each station individually. These segments, 
parallel to the beam axis, are called $seeds$. Since hits 
from the same electron should be close in both time and space, 
and $\delta$-electrons may hit multiple straws within the 
same panel, the algorithm clusters these straw hits in 
space-time, applying various cleanup cuts to ensure the 
selected patterns resemble $\delta$-electron hits. 
The maximum allowed time difference between two hits within 
a station to form a $seed$ is set to 40 ns.

These cleanups based on the $x$-$y$ coordinates are performed 
by computing a $\chi^2$. The $seed$ is reconstructed using 
two, three, or four $ComboHit$s. In each case, the 
intersections between two straws are determined, 
and the $x$-$y$ coordinates of the $seed$ are given 
by the center of gravity of these intersections. 
In the case of only two hits, the center of gravity 
corresponds to the stereo hit. Explicit stereo hit 
reconstruction is not used for all hits.

The algorithm extends the $seed$s by requiring 
hits to be sufficiently close to the intersection 
point in time and space, performing multiple checks 
to avoid over-efficiency in hit flagging. For each $seed$, 
the mean deposited energy is calculated from the energy 
deposited in all $ComboHit$s. Based on the mean 
deposited energy of a $seed$ in a station, hits with 
energy above 5 keV are considered proton hits. 
This selection optimizes data processing by 
reducing the total number of hits that need to be analyzed.

\begin{figure}[!h]
    \centering
    \includegraphics[width =0.6\textwidth]{figures/png/Screenshot_20240811_115854.png}
    \caption[A $\delta$ candidate $seed$.]{A $\delta$ candidate $seed$. The four coloured segments are the tracker straws that were
    hit in one station.}
    \label{fig:deltaseeds}
\end{figure}

It is necessary to do a clarification. 
Figure \ref{fig:energydeposited} shows the 
distribution of the simulated deposited energy 
in the tracker for $\delta$ electrons, CEs, and 
protons in the case of a 1BB pileup. 

To optimize data processing, a hit energy 
threshold could be applied to the $DeltaFinder$ 
to reduce the total number of hits that need to be analyzed, 
thereby speeding up the process. Moreover, only about 
4\% of CE hits have energies above 3.5 keV 
(and 1\% above 5 keV), so the loss of CE 
hits would be minimal, resulting in a faster overall algorithm. 

However, implementing such an energy cutoff has 
significant implications for the algorithm's performance. 
Starting with fewer hits, especially in stereo intersections, 
decreases the likelihood of identifying the correct $seed$. 
Additionally, a higher energy cutoff increases 
the probability of false positives, as the algorithm 
aims to count protons too.

\begin{figure}[!h]
    \centering
    \includegraphics[width =0.8\textwidth]{figures/png/Screenshot_20240729_151910.png}
\caption[Monte Carlo deposited energy distribution in the Mu2e tracker.]{
   The Monte Carlo deposited energy distribution in the Mu2e tracker. The distribution
   corresponds to 1BB pile-up (Section \ref{pulsedprotonbeam}). The red distribution refers to 
   conversion electrons, while the green and the blue one to $\delta$-electrons and protons respectively. 
   The peaks and tails correspond to the saturation of the waveforms.

}
   \label{fig:energydeposited}
\end{figure}
\subsubsection{Step 2: Connecting $seed$s}
After the selection based on mean deposited energy, 
$DeltaFinder$ attempts to connect segments that 
are close in both the $x$-$y$ plane and time 
across different stations to form $\delta$-electron 
candidates. A valid candidate must have at 
least two segments and a minimum of five straw hits. 
Reconstructed segments from a 100 MeV electron 
typically remain unconnected due to their separation in the $x$-$y$ plane. 

$DeltaFinder$ links $\delta$-electron $seed$s 
across stations, attempting to associate new $seed$s 
with existing $\delta$ candidates. If no match is 
found, a new candidate is created. Good $\delta$ 
candidates are marked, and their hits are 
flagged to prevent their inclusion in proton candidate searches.

\subsubsection{Step 3: Identifying Proton Candidates}
Finally, $DeltaFinder$ identifies proton candidates 
using the remaining $seed$s, which are more likely 
to have deposited energy above 3 keV. First, it checks 
if a $seed$ is consistent in time with any existing 
proton candidates. If no consistency is found, 
the hits of the $seed$ are added to a new proton candidate.


\section{Analysis of the performance and comparison}
The analysis is carried out on two levels of 
comparison, each addressing different aspects of the algorithms under evaluation:

\begin{itemize}
    \item \textbf{Hit-level comparison}: This phase focuses 
    on evaluating how accurately individual hits are 
    flagged, providing the most direct method for 
    assessing and comparing the performance of the two 
    algorithms. This stage allows for an unbiased 
    comparison without the influence of subsequent 
    reconstruction stages. It also includes a direct evaluation of proton counting;
    
    \item \textbf{Track-level comparison}: This phase examines 
    the algorithms' impact on the later stages of 
    event reconstruction. It highlights how effectively 
    each algorithm contributes to the accurate reconstruction 
    of tracks. The main figure of merit at this stage is the 
    reconstruction of CE tracks, as well as muons and pions, which are crucial 
    for estimating the background level.
\end{itemize}

\subsection{Hit-level comparison}
Before starting with the hit-level comparison of the two algorithms, during the analysis, 
we first observed an over-efficiency in the proton hit flagging by $DeltaFinder$. 
This primarily impacted the flagging of protons, muons, and pions, as shown in 
Table \ref{tab:1bbcelebefore} and Table \ref{tab:0bbpbarbefore}. 
\begin{center}
    \begin{table}[h!]
    \centering
    \renewcommand{\arraystretch}{1.}
    \begin{tabular}{| c | c | c |} 
    \hline
    & $f_{p}$ DF & $f_{e}$ DF  \\
    \hline
    $p$     & 96.0\% & 1.0\% \\
    \hline
    \end{tabular}
    \caption[Proton flagging results before the adjustment.]{Proton 
    flagging results before the adjustment ($CELE1BB$ data sample).}
    \end{table}\label{tab:1bbcelebefore}
\end{center}
    
\begin{center}
    \begin{table}[h!]
        \centering
        \renewcommand{\arraystretch}{1.}
        \begin{tabular}{| c | c | c | c | c|} 
        \hline
        &   $f_{p}$ DF &   $f_{e}$ DF\\
        \hline
        $\mu$ &  5.8\%  & 5.0\%\\
        \hline
        $\pi$ & 2.5\% &  11.2\%\\
        \hline
        \end{tabular}
        \caption[Muon and Pion flagging results before the adjustment.]{Muon 
        and Pion flagging results before the adjustment ($PBAR0BB$ data sample).}
    \end{table}\label{tab:0bbpbarbefore}
\end{center}
In these tables and the ones that follow, $f_p$ and $f_e$ represent 
the number of $ComboHit$s flagged as protons and electrons, respectively, 
divided by the total number of $ComboHit$s. DF and FBH denote the $DeltaFinder$ 
algorithm and $FlagBkgHits$, respectively. Each row corresponds to the particle 
under consideration.
The number of pions and muons flagged as protons is extremely high. 
Figure \ref{fig:0pbarbefore} shows the distribution of the total 
number of muon $ComboHit$s (red) and those flagged as $\delta$s (blue) 
and protons (green) versus the particle momentum. 
As shown, a significant number of muons were misidentified 
as protons when the momentum was low. 
According to the Bethe-Bloch formula, such hits have 
higher energy loss and can thus be most likely flagged as 
protons. Upon realizing this issue, we made slight modifications to the 
algorithm. Since $seed$s may have accidentally attached hits, we imposed a 
condition requiring a $good$ proton candidate to have more than four 
hits with a deposited energy greater than 3 keV.

This adjustment is slightly inefficient for protons, reducing their 
efficiency by a factor of 1.1 (Tables \ref{tab:1bbcele} and \ref{tab:2bbcele}). 
However, it significantly reduces the 
number of muons and pions flagged as protons, by approximately 
a factor of 2 and a factor of 6 (Table \ref{tab:0bbpbar}), respectively.

 \begin{figure}[!h]
            \centering
            \includegraphics[width =0.8\textwidth]{figures/png/Screenshot_20240805_222923.png}
        \caption[The distribution of the total and flagged number of muon $ComboHit$s versus momentum.]{The 
        distribution of the total number of muon $ComboHit$s 
        (red) and those flagged as $\delta$s (blue) 
        and protons (green) versus the particle momentum ($PBAR0BB$ data sample). }
           \label{fig:0pbarbefore}
\end{figure}




Now let's move on to the performance analysis and comparison part.

Tables \ref{tab:2bbcele} and \ref{tab:1bbcele} show the hit-level 
comparison for the $CELE1BB$ and $CELE2BB$ data samples between the two algorithms.
\begin{center}
    \begin{table}[h!]
    \centering
    \renewcommand{\arraystretch}{1.}
    \begin{tabular}{| c | c | c | c | c | c |} 
    \hline
    &  $f_{p}$ FBH &  $f_{p}$ DF & $f_{e}$ FBH  & $f_{e}$ DF \\
    \hline
    e$^-$ ($p<$20 MeV/c)      & 2.5\% & 3.7\%   & 75.1\% & 72.7\%\\
    \hline
    e$^-$ (20$<p<$80 MeV/c)  & 1.1\% & 2.2\%   & 49.0\%& 29.9\%\\
    \hline
    e$^-$ (80$<p<$110 MeV/c)   & 0.2\% & 0.8\%  &  7.4\%& 4.3\%\\
    \hline
    $p$       &  &  83.6\%  &  & 2.2\%\\
    \hline
    e$^+$ ($p<$20 MeV/c)   & 0.6\%& 0.4\%    &   84.2\%& 87.9\%\\
    \hline
    \end{tabular}
    \caption[Electrons, 
    Positrons and Protons hit-level comparison.]{Electrons, 
    Positrons and Protons hit-level comparison ($CELE2BB$ data sample).}
    \end{table}\label{tab:2bbcele}
    \end{center}

    \begin{center}
    \begin{table}[h!]
    \centering
    \renewcommand{\arraystretch}{1.}
    \begin{tabular}{| c | c | c | c | c | c |} 
    \hline
    &  $f_{p}$ FBH &  $f_{p}$ DF & $f_{e}$ FBH  & $f_{e}$ DF \\
    \hline
    e$^-$ ($p<$20 MeV/c)      & 2.5\% & 2.5\%   & 75.9\% & 72.5\%\\
    \hline
    e$^-$ (20$<p<$80 MeV/c)  & 1.0\%& 1.0\%   & 50.0\%& 27.4\%\\
    \hline
    e$^-$ (80$<p<$110 MeV/c)   & 0.2\%& 0.3\%  &  5.7\%& 3.4\%\\
    \hline
    $p$       &              &         83.7\%           &  & 1.0\%\\
    \hline
    e$^+$ ($p<$20 MeV/c)   & 0.5\%    & 0.2\%    &   85.5\%& 88.5\%\\
    \hline
    \end{tabular}
    \caption[Electrons, Positrons and Protons hit-level comparison.]{Electrons, 
    Positrons and Protons hit-level comparison ($CELE1BB$ data sample).}
    \end{table}\label{tab:1bbcele}
    \end{center}
  
The fraction of electrons and positrons flagged as $\delta$-electrons 
differs due to the momentum distribution of these particles at low energies 
(Figure \ref{fig:efficiency}). These plots show the efficiency 
(i.e., the number of $\delta$-electron $ComboHit$s flagged as $\delta$s 
over the total number of $ComboHit$s for each particle type) as a function 
of particle momentum for positrons (red) and electrons (blue) using the 
$FlagBkgHits$ (left) or $DeltaFinder$ (right) algorithms. 

At low momentum values (less than 1-2 MeV/c), the efficiency 
is higher for positrons since the total number of $ComboHit$s 
with such low momentum is extremely small, as they are 
not affected by Compton scattering. The efficiency plot 
is in fact convoluted with the momentum distribution.

From both tables, it is also possible to notice that the positron 
efficiency for $FlagBkgHits$ is lower compared to $DeltaFinder$. 
This is because the MVA output is statistics-dependent. In this energy 
range, positrons have very few $ComboHit$s, as they are not 
affected by Compton scattering. Therefore, $FlagBkgHits$ performs 
better with higher statistics and is more efficient for $E < 2$ MeV/c. 
At the same time, $DeltaFinder$ works less effectively with electrons. 
This is due to the high number of $ComboHit$s at this energy range, 
while the number of $ComboHit$s flagged as $\delta$ remains constant 
at low energies, as Compton electrons have only one 
hit per station, making impossible to find the $seed$. 

For higher momentum values, both algorithms perform similarly, and 
the differences between positrons and electrons become negligible. 
The sum of the positron and electron efficiency (Tables \ref{tab:2bbcele} 
and \ref{tab:1bbcele}) for a given algorithm 
matches that of the other algorithm, indicating they have the same 
capability in rejecting $\delta$-electrons. 
For both the algorithms, the primary cause of failure 
in $\delta$ flagging arises from hits on straws that are 
parallel to each other, particularly those near the center of the tracker, 
where stereo reconstruction is not possible.

XXXXX PICCO A 50 MEV/c in fp

When discussing the differences between the two 
algorithms, particularly regarding CE flagging, 
there is a factor of 1.7 between the two $f_e$. 
The main difference is as follows: sometimes, CE hits 
can occur close in time and space to other $\delta$ hits. 
As a result, $FlagBkgHits$ flags them as $\delta$, since 
it does not attempt to locate corresponding hits in another station, 
while $DeltaFinder$ tries to connect $seed$s across the stations.


Concerning proton flagging, we could not compare the two 
algorithms since the $FlagBkgHits$, before creating $StereoHit$s, 
applies a preliminary selection cut on the energy deposited in 
the $StrawHit$s of 5 keV. Subsequently, the algorithm identifies 
protons as all particles with a deposited energy greater than 
4.5 keV. It was not possible to conduct a direct comparison, 
as $DeltaFinder$ does not apply a cut on the energy deposited in 
the $StrawHit$s, making the comparison unreliable and biased. 
The fraction of particles flagged as protons by $FlagBkgHits$ 
corresponds to the fraction of $ComboHit$s with a 
deposited energy higher than 4.5 keV and lower than 5 keV.

Therefore, we reported the fraction of protons 
flagged as protons by $DeltaFinder$.
This fraction underwent a reduction by a factor of 
1.1 after the applied cut (Table \ref{tab:1bbcelebefore}). 

From both Tables \ref{tab:2bbcele} and \ref{tab:1bbcele}, 
we can observe that the fraction of electrons misidentified 
as protons decreases with increasing momentum. This is because 
lower-energy electrons tend to have a higher deposited energy 
compared to others. This occurs because energy loss is dependent 
on particle energy, which increases as the energy decreases, 
leading to the misidentification of these electrons. Positrons, 
however, are not affected by this misidentification, as they 
are unaffected by Compton scattering and are not abundant in this energy range.

The results show the same conclusions for both the 1BB and 2BB data samples.

    \begin{figure}[!h]
        \centering
        \begin{subfigure}[t]{0.5\textwidth}
            \centering
            \includegraphics[width=1.\textwidth]{figures/png/Screenshot_20240818_155835.png}
            \caption{}
        \end{subfigure}%
        ~ 
        \begin{subfigure}[t]{0.5\textwidth}
            \centering
            \includegraphics[width=0.95\textwidth]{figures/png/Screenshot_20240813_203916.png}
            \caption{}
        \end{subfigure}
        \caption[The efficiency of electrons and positrons versus 
        particle momentum in the low-momentum range.]{The efficiency of electrons (blue) and positrons (red) versus 
        particle momentum in the low-momentum range. The left plot shows the 
        results for $FlagBkgHits$, while the right plot shows those for $DeltaFinder$.
        }
        \label{fig:efficiency}
      \end{figure} 

      Looking at the results for $\mu$ and $\pi$ (Table \ref{tab:0bbpbar}), 
      we observe an approximate factor of 4 difference between the two algorithms 
      in muon delta flagging, and about a factor of 3.3 for pions. This occurs 
      because muons and pions often produce more than one hit in a single station, 
      but the $DeltaFinder$ algorithm is able to distinguish these hits, aided by 
      its ability to recognize straight lines in the $r-z$ plane, whereas 
      $FlagBkgHits$ cannot. Moreover, the problem with having a supervised 
      training method that distinguishes one type of particle from another 
      is that it can become confused when another particle, such as a cosmic 
      muon background event or tracks from $p\bar{p}$ annihilation, appears. 
      
      On average, pions from $p\bar{p}$ collisions have higher 
      momenta than muons, leading to a higher false positive rate. In fact, 
      there are more low-energy particles in the case of muons, making them 
      less likely to be correctly distinguished compared to pions. Furthermore, 
      the false positive rate ($f_p$) decreases as the momentum increases for pions, 
      according to the Bethe-Bloch formula.

    \begin{center}
        \begin{table}[h!]
        \centering
        \renewcommand{\arraystretch}{1.}
        \begin{tabular}{| c | c | c | c | c|} 
        \hline
        &$f_{p}$ FBH  &  $f_{p}$ DF &  $f_{e}$ FBH & $f_{e}$ DF\\
        \hline
        $\mu$ & 0.8\% &  2.7\%  & 13.0\% & 3.2\%\\
        \hline
        $\pi$ & 0.2\%& 0.4\% & 23.8\%& 7.3\%\\
        \hline
        \end{tabular}
        \caption[Muon and Pion hit-level comparison.]{Muon and Pion 
        hit-level comparison ($PBAR0BB$ data sample).}
        \end{table}\label{tab:0bbpbar}
        \end{center}


\subsection{Track-level comparison}
At this point the unflagged hits were sent to the pattern recognition and the reconstruction was performed. 
Table \ref{tab:recoeffcele} reports the reconstruction efficiency, defined as the 
number of reconstructed track over the total number of events, and the efficiency 
after selecting the true CE tracks (both 2BB and 1BB), while Table \ref{tab:recoeffpbar} 
shows the reconstruction efficiency and the efficiency after applying selection cuts for the $p\bar{p}$ analysis. 
Table \ref{tab:recoeffcele} indicates that there is a difference of 
approximately 20\% between the number of reconstructed tracks using 
$FlagBkgHits$ and $DeltaFinder$. However, when selecting tracks that 
correspond to CEs, the efficiency is almost identical.

\begin{center}
    \begin{table}[h!]
    \centering
    \renewcommand{\arraystretch}{1.}
    \begin{tabular}{| c | c | c | c | c |} 
    \hline
    & FBH,2BB & DF,2BB & FBH,1BB & DF,1BB  \\
    \hline
    tracks reconstruction efficiency & 52.8\% &  42.9\% & 47.8\% & 41.3\%\\
    \hline
    CE reconstruction efficiency & 30.9\% & 31.2\% & 32.0\% & 32.2\%\\
    \hline
    \end{tabular}
    \caption{}
    \end{table}\label{tab:recoeffcele}
\end{center}

This table can be understood by examining Figure \ref{fig:highlevel1}, 
which shows the momentum distribution of reconstructed tracks. In the 
case of $FlagBkgHits$ (blue), protons are not correctly flagged, so 
they are sent to pattern recognition and then reconstructed. On the 
other hand, $DeltaFinder$ (red) is able to flag and count proton hits, 
preventing them from being sent to pattern recognition. When selecting 
Monte Carlo truth, meaning all particles that are actually CE, the 
distribution appears as in Figure \ref{fig:highlevel2}. The number 
of reconstructed CEs is nearly the same. 
Table \ref{tab:recoeffcele} reveals a difference in the 
reconstructed track efficiency for $FlagBkgHits$ 
between 2BB and 1BB, which depends on the number of protons reconstructed. 
In fact, in 2BB, the number of proton reconstructed tracks is higher than in 1BB case. 
For the rest, the results are similar for both 1BB and 2BB.


\begin{figure}[!h]
    \centering
    \includegraphics[width =0.8\textwidth]{figures/png/Screenshot_20240813_171301.png}
    \caption[]{}
    \label{fig:highlevel1}
\end{figure}
\begin{figure}[!h]
    \centering
    \includegraphics[width =0.8\textwidth]{figures/png/Screenshot_20240813_171330.png}
    \caption[]{}
    \label{fig:highlevel2}
\end{figure}
We tried to look at those events where at least one track 
was reconstructed in the case of one hit flagger and no 
tracks reconstructed in the case the other one.
There is a well defined class of events where the 
effects of hit flaggers get washed out in the 
reconstruction by the time clustering algorithm.
Figure \ref{fig:TZCluster1} ($FlagBkgHits$) and \ref{fig:TZCluster2} ($DeltaFinder$) show 
an the same example event for the two algorithms, in particular the $time$ 
versus $z$ coordinate (the one along the tracker).
The violet squares contain particles related to the same $TimeCluster$. 
This is searched from hits that fit along a linear
line in time versus z space. These hits are first combined in some time-z window to create $chunks$ (at present
the time window is 20 ns and z-window of 5 planes). A window is considered to be a $chunk$ if
more than 3 ComboHits are within it. Every possible pair of chunks that are within some time of each other are tested together. The
pair that yields the smallest $\chi^2/ndof$ when the hits are fit to a linear line are permanently
combined. This is repeated until no new combination yields a $\chi^2/ndof$ below some threshold. If
the chunk exceeds some minimum number of straw hits it is saved as a cluster. 
In the two plots CE hits are the big red dots, delta elctrons the small brown ones and protons are the big blue dots. 
$TimeCluster$s have different $z$ value, since the particles that are grouped together, can be at different $z$ 
The two plots show that in the first case the hits are all grouped in the same 
cluster (green) and in the second one (green and orange) CE hits are grouped in two different $TimeCluster$s. 
An example is described in the following lines. 
In the case of $DeltaFinder$, hits from the same 
particle are divided in two different time clusters, but for 
$FlagBkgHits$ the not flagged hits are used by the time clusterer to $connect$ 
particle hits that are used in the reconstruction. 
That is why the track is reconstructed in this case. 
Improving the cluster finder and the pattern recognition 
could increase the track reconstruction performance.
\begin{figure}[!h]
    \centering
    \includegraphics[width =0.8\textwidth]{figures/png/Screenshot_20240819_153229.png}
    \caption[]{}
    \label{fig:TZCluster1}
\end{figure}
\begin{figure}[!h]
    \centering
    \includegraphics[width =0.8\textwidth]{figures/png/Screenshot_20240819_153730.png}
    \caption[]{}
    \label{fig:TZCluster2}
\end{figure}
Delta finder has an advantage for 2 track reconstructed. This comes from 
the fact that a larger number of hits are left with $DeltaFinder$.
A particle trajectory can be considered a viable candidate for 
reconstruction if it makes at least 20 hits in the tracker and if the 
reconstructed $\chi^2/ndof\leq 3$.
Here are reported two different momentum selection cuts. 
For particle coming from the stopping target, 
you dont have enough hits to reconstruct particles with 
momentum below 80 MeV/c, that is why we selected the first momentum cut. 
Moving to 90 MeV/c is to eliminate completely the DIO.
Requiring the DIO electron momentum to
be above 90 MeV/c gives an estimate of about $10^{-2}$
events with two DIO electrons for Run I. Assuming a track reconstruction
efficiency of $\sim$0.1, we reconstruct about $10^{-4}$ events with
two DIO electron tracks. Further, assuming a uniform distribution in time, the number of events with two DIO electrons
within a time window of 100 ns is $\sim10^{-5}$.
\begin{center}
        \begin{table}[h!]
        \centering
        \renewcommand{\arraystretch}{1.}
        \begin{tabular}{| c | c | c |} 
        \hline
        &  FBH & DF \\
        \hline
        tracks reconstruction efficiency ($Ntracks \geq 2$) &  3.8\% & 4.6\%\\
        \hline
        reconstruction efficiency after selection (p$>$80 MeV/c) & 3.0\% & 3.8\%\\
        \hline
        reconstruction efficiency after selection (p$>$90 MeV/c) & 2.7\% & 3.2\%\\
        \hline
        \end{tabular}
        \caption{}
        \end{table}\label{tab:recoeffpbar}
\end{center}


    
    Cose da capire e fare: 
    1) perche il picco su 50MeV/c sulla fp degli elettroni
    4) devo far levare anche FBH fp?

6)high level bench mark
7) tzcluster
9)timing
8)fare rif a piu numeri
cerca di capire 1hit/track
1hit/track*7701tracks=7701 hits che rappresenterebbe 14150-8168 combohits
\section{Conclusions}
e'basato sul monte carlo. training difficile. se e' sviluppato per 
riconoscere bene A e B, quando arriva C va in confusione. lavora male quando la statistica e'bassa. 
non distingue i protoni.

\chapter{Pre-pattern recognition studies}\label{delta}
\textit{
The estimated data volume for Mu2e data-taking 
is at least 7 PByte per year: it will thus be crucial to 
exploit all the possible handles to optimize 
the CPU and memory usage. For example, the simulation 
shows that the primary source of hits in the 
Mu2e tracker will be low energy electrons and positrons, called $\delta$-electrons. 
Therefore, the Mu2e Collaboration has made 
a huge effort to develop solutions to identify 
and flag these hits as soon as possible in the 
data-taking and not use them in pattern 
recognition and track reconstruction. 
The most important constraint is 
making sure that hits generated by CE are 
not erroneously flagged as $\delta$-electrons, 
since this would compromise CE track reconstruction. 
There are currently two main algorithms being 
developed for this purpose. This Chapter reports 
on the first systematic study performed to compare 
the performance of the two algorithms and determine 
the best solution for data-taking. }



\section{$\delta$-electrons as source of background}

The performance of the detectors and the Mu2e 
physics reach have been thoroughly studied with 
the Monte Carlo simulation. In terms of 
occupancy, we know that the dominant 
source of hits in the tracker are low energy electrons and positrons,
in the following referred to as  
$\delta$-electrons. To be more precise, in 
decreasing order of importance, the primary sources of hits are 
electrons from Compton scattering, electron-positron pairs, 
and delta rays. Compton-scattered electrons are produced 
when photons, generated by various processes, 
interact with the detector material. These photons primarily 
originate from neutron captures, which excite the 
nuclei and lead  
to subsequent photon emission. Typically, 
these photons have 
energies of a few MeV. Neutrons are 
produced in the process of nuclear muon capture. Pair-production 
electrons and positrons are 
generated during nuclear  
processes, where pairs of electrons 
and positrons are created.
Delta rays, 
or secondary ionization electrons, 
are generated when high-energy 
charged particles interact with the detector material.


\subsection{Compton scattering}\label{compton}
The Compton effect (Figure \ref{fig:compt}) is the 
scattering of a photon by a free or quasi-free electron. 
An electron is considered "quasi-free" when the energy of 
the incoming photon is significantly higher than the 
electron's binding energy ($E_\gamma \gg E_B$). The 
process is termed Compton scattering if the 
electron is ejected from the atom, carrying away the 
recoil momentum. This effect is most prominent in an 
extended energy region around 1 MeV, with the region 
being much larger for low $Z$ materials compared to high $Z$ materials.

\begin{figure}[!h]
    \centering
    \includegraphics[width =0.4\textwidth]{figures/png/Screenshot_20240812_204345.png}
    \caption[The Compton effect.]{
    The Compton effect \cite{kola}.}
    \label{fig:compt}
\end{figure}

Since the photon scatters quasi-elastically off the electron, 
the energy and angle of the scattered photon are correlated. 
To describe this relationship, we use the 4-momenta defined as 
follows: $k = (E_\gamma, \mathbf{k}c)$ and $p_e = (m_e c^2, 0)$ 
represent the 4-momenta of the photon and the electron (at rest) 
before scattering, and $k' = (E'_\gamma, \mathbf{k}'c)$ and 
$p'_e = (E'_e, \mathbf{p}'_e c)$ represent the 4-momenta 
after scattering. The angle between the scattered photon and 
the incident photon is denoted as $\theta_\gamma$, while the 
angle of the electron is denoted as $\theta_e$. By applying 
energy-momentum conservation:

\begin{equation}\label{compcons}
k + p_e = k' + p'_e
\end{equation}
\begin{equation}\label{compcons2}
(k - k')^2 = (p'_e - p_e)^2 \Rightarrow -k \cdot k' = m_e^2 c^4 - p'_e \cdot p_e
\end{equation}
\begin{equation}
\Rightarrow E_\gamma E'_\gamma (1 - \cos \theta_\gamma) = m_e c^2 \left(E'_e - m_e c^2\right) = m_e c^2 \left(E_\gamma - E'_\gamma \right)
\end{equation}

The right-hand side of the last equation uses the kinetic energy of the electron:

\begin{equation}
T = E'_e - m_e c^2 = E_\gamma - E'_\gamma
\end{equation}

which follows from the energy part of equation \ref{compcons}. 
The energy of the scattered electron as a function of the photon scattering 
angle is derived from equation \ref{compcons2}:

\begin{equation}\label{diffeq}
E'_e = \frac{E_\gamma \cdot \epsilon \cdot (1 - \cos \theta_\gamma)}{1 + \epsilon (1 - \cos \theta_\gamma)}+m_e
\end{equation}

where $\epsilon = \frac{E_\gamma}{m_e c^2}$.

The differential cross section per (free) electron, known as the 
Klein-Nishina formula, is calculated using methods from quantum electrodynamics:

\begin{equation}\label{kleinnishina}
\frac{d\sigma}{d\Omega} = \frac{r_e^2}{2} \frac{1 + \epsilon (1 - \cos \theta_\gamma)}{[1 + \epsilon (1 - \cos \theta_\gamma)]^2} \left(1 + \cos^2 \theta_\gamma + \frac{\epsilon^2 (1 - \cos \theta_\gamma)^2}{1 + \epsilon (1 - \cos \theta_\gamma)} \right)
\end{equation}

An electron bound in an atom can only be considered quasi-free 
if the photon's energy is significantly higher than the electron's 
binding energy. As the photon energy increases, more shell electrons 
become quasi-free, leading to the Compton cross section per atom 
approaching proportionality to $Z$, with individual electrons 
contributing incoherently:

\begin{equation}
\sigma_C^{\text{atom}} = Z\sigma_C
\end{equation}

where $\sigma_C$ is the Klein-Nishina cross section for a 
single free electron. The Compton cross section decreases at 
lower energies, where coherent scattering (Rayleigh scattering) 
off the entire atom (without ionizing the electron shell) becomes dominant.

By reformulating the Klein-Nishina formula, one can obtain the 
differential dependence of the Compton cross section on the 
kinetic energy of the recoil electron $T = E_\gamma - E'_\gamma$:

\begin{equation}
\frac{d\sigma}{dT} = \frac{\pi r_e^2}{m_e c^2 \epsilon^2} \left[2 + \frac{t^2}{\epsilon^2 (1 - t)^2} + \frac{t}{1 - t}\left(t - \frac{2}{\epsilon}\right)\right]
\end{equation}

where $t = T/E_\gamma$. Because the scattering process is 
elastic, there is a one-to-one relationship between the 
energy and angle $\theta_e$ of the electron:

\begin{equation}
\cos \theta_e = \frac{T(E_\gamma + m_e c^2)}{E_\gamma \sqrt{T^2 + 2m_ec^2 T}} = \frac{1 + \epsilon}{\sqrt{\epsilon^2 + 2\epsilon/t}}
\end{equation}

The maximum energy transfer to the electron is obtained 
from equation \ref{diffeq} for backward scattering of the 
photon ($\theta_\gamma = 180^\circ$), corresponding to 
forward scattering of the electron ($\theta_e = 0^\circ$). 
The electron's kinetic energy reaches its maximum value in 
this case, $T \rightarrow T_{\text{max}}$. In the measured 
energy spectrum, this leads to the so-called "Compton edge" at:

\begin{equation}
T_{\text{max}} = \frac{E_\gamma \cdot 2\epsilon}{1 + 2\epsilon}
\end{equation}

which lies slightly below the photopeak. The energy difference 
between the photopeak and the Compton edge $E'_\gamma(\theta = \pi)$ 
decreases with increasing $E_\gamma$ and approaches:

\begin{equation}
E'_\gamma(\theta = \pi) \approx \frac{m_e c^2}{2} \text{ for } E_\gamma \gg m_e c^2
\end{equation}

\subsection{Pair production}
In the Coulomb field of a charge, a photon can 
convert into an electron-positron pair (Figure 
\ref{fig:pprod})\footnote{Photon emission by an 
electron (bremsstrahlung) and pair production are closely 
related processes. By modifying the bremsstrahlung diagram-changing 
the outgoing photon to an incoming one and the incoming electron to 
an outgoing positron-one obtains the pair production 
diagram. The matrix elements of these processes are 
related, at least in the lowest order. Consequently, both 
processes are treated together in the foundational work by 
Bethe and Heitler, often referred to as the 'Bethe-Heitler processes'.}.

\begin{figure}[!h]
    \centering
    \includegraphics[width=0.4\textwidth]{figures/png/Screenshot_20240812_204755.png}
    \caption[The pair production.]{The pair production \cite{kola}.}
    \label{fig:pprod}
\end{figure}

The energy of the photon must exceed twice the electron 
mass plus the recoil energy transferred to the field-producing 
charge. For most elements, pair production predominantly 
occurs in the Coulomb field of the nucleus. For nuclei, 
the recoil energy is usually negligible, leading to a 
threshold energy for pair production of:
\begin{equation}
    E_{\gamma} \geq 2m_e c^2 + 2 \frac{m_e^2}{m_{\text{nucleus}}} c^2
\end{equation}

If the nuclear charge is not screened by atomic electrons 
(for low energies, the photon must come relatively close to 
the nucleus to make pair production probable, meaning it 
interacts with the "bare" nucleus),
\begin{equation}
    1 \ll \epsilon \ll \frac{1}{\alpha Z^{1/3}}
\end{equation}
the pair-production cross section is given by:

\begin{equation}
    \sigma_{\text{pair}} = 4 \alpha r_e^2 Z^2 \left(\frac{7}{9} \ln 2 \epsilon - \frac{109}{54}\right) \text{ cm}^2/\text{atom}
\end{equation}

However, for complete screening of the nuclear charge ($\epsilon \gg 1/\alpha Z^{1/3}$):
\begin{equation}\label{sigmapair}
    \sigma_{\text{pair}} = 4 \alpha r_e^2 Z^2 \left(\frac{7}{9} \ln \frac{183}{Z^{1/3}} - \frac{1}{54}\right) \text{ cm}^2/\text{atom}
\end{equation}

At high energies, pair production can occur even 
at relatively large impact parameters between the 
photon and the nucleus. In this case, the screening 
effect of atomic electrons must be considered. For 
large photon energies, the pair-production cross 
section approaches an energy-independent value as given by
 Equation \ref{sigmapair}. Ignoring the small term in the equation, 
 the asymptotic value of $1/54$ is expressed as:
\begin{equation}
    \sigma_{\text{pair}} \approx \frac{7}{9} \cdot 4 \alpha r_e^2 Z^2 \ln\left(\frac{183}{Z^{1/3}}\right) \approx \frac{7}{9} \cdot \frac{1}{X_0} \cdot \frac{A}{N_A \cdot \rho}
    \label{eq:paircross_radiationlength}
\end{equation}

The energy is uniformly distributed between the produced 
electrons and positrons at low and medium energies, but becomes 
slightly asymmetric at high energies.

The field of the nucleus is formed by the coherent sum of $Z$ 
nucleon charges, leading to the $Z^2$ dependence of the pair 
production cross section.

Even with large momentum transfers $\Delta p$ to the nucleus, 
the energy transfer $(\Delta p)^2/2M$ remains small due to the 
large nuclear mass $M$. After pair creation, the remaining 
energy is equally divided between the $e^+$ and the $e^-$.

\subsection{Delta rays}
$\delta$-rays, or $knock-on$ electrons, 
are produced when a projectile particle collides 
centrally with shell electrons, resulting in 
significant energy transfers.
These electrons 
gain high kinetic energy and can be described 
through elastic collisions with quasi-free electrons. 
By considering the energy-momentum conservation relation 
and using the Lorentz factors $\gamma$ and $\beta$, the 
relationship between the kinetic energy $T$ of the 
$\delta$-ray and the emission angle $\theta$ can be derived as:

\begin{equation}
\cos \theta = \frac{T(\gamma + m_e / M)}{\gamma \beta \sqrt{T^2 + 2T m_e c^2}}
\end{equation}

\begin{equation}
T(\theta) = \frac{2 m_e c^2 \beta^2 \gamma^2 \cos^2 \theta}{\gamma^2(1 - \beta^2 \cos^2 \theta) + 2 \gamma m_e / M + m_e^2 / M^2}
\end{equation}

The maximum energy transfer $T_{\text{max}}$ occurs at $\theta = 0^\circ$, 
while the minimum energy, $T_{\text{min}}$, occurs at $\theta = 90^\circ$. 
At highly relativistic energies ($\gamma \gg 1$ and $\theta \gg 1/\gamma$), 
the energy-angle relationship becomes independent of the incoming particle's properties.

The rate of $\delta$-rays per energy interval $dT$ and path length $dx$ is given by:

\begin{equation}
\frac{d^2 N}{dx \, dT} = n_e \frac{d\sigma}{dT}
\end{equation}

which, when combined with the electron density and the differential cross section, becomes:

\begin{equation}
\frac{d^2 N}{dx \, dT} = \frac{1}{2} z^2 \frac{Z}{A} K \rho \frac{1}{\beta^2} \frac{F(T)}{T^2}
\end{equation}

Here, $K$ is the constant from the Bethe-Bloch formula, 
and $F(T)$ is a function accounting for spin dependence. 
Integration over $T$ and $x$ provides the number of $\delta$-rays in a medium of thickness $\Delta x$:

\begin{equation}
N = \frac{1}{2} z^2 \frac{Z}{A} K \rho \Delta x \frac{1}{\beta^2} \left(\frac{1}{T_{\text{min}}} - \frac{1}{T_{\text{max}}}\right) \approx 0.077 \frac{\text{MeV cm}^2}{\text{g}} z^2 \rho \Delta x \frac{1}{T_{\text{min}}}
\end{equation}

The emission angle dependence is given by:

\begin{equation}
\frac{dT}{d \cos \theta} = 4 m_e c^2 \frac{\cos \theta}{\sin^4 \theta}
\end{equation}

Substituting this into the rate equation yields:

\begin{equation}
\frac{d^2 N}{dx \, d \cos \theta} = \frac{1}{2} z^2 \frac{Z}{A} K \rho \frac{1}{\cos^3 \theta} \frac{1}{m_e c^2} \approx 0.15 \frac{\text{cm}^2}{\text{g}} z^2 \rho \frac{1}{\cos^3 \theta}
\end{equation}

This expression diverges as $\theta$ approaches $90^\circ$, 
where $T$ approaches zero, indicating a limitation in the 
assumption of a free electron. The resulting distributions 
suggest that $\delta$-rays emitted at small angles can 
significantly affect the spatial resolution in detectors, 
particularly through ionization clusters that broaden the 
track of the mother particle.



\section{Monte Carlo samples}\label{datasample}
For our studies, we used three different  
Monte Carlo samples: two samples of 
CE signal generated at two different 
proton pulse intensities, and one sample of 
antiproton annihilation events. 
The production of particles within 
the PT and their tracking from the PS to the 
DS is handled by the Mu2e Offline software 
(Appendix \ref{mu2eana}). The simulation of 
particle interactions and the event processing are based on 
GEANT4, and they are handled 
by the art framework and data management is governed by the SAM system (Appendix \ref{mu2eana}).

Mu2e uses a multi-stage simulation to generate and simulate events efficiently. 
The method involves generating events, partially simulating 
them, and saving the intermediate results. Later stages resume the simulation from the saved 
state. This approach optimizes both the time required for the event generation 
and the disk space usage. 

At the first stage, the interactions of protons at the PT are simulated.
Produced secondary particles are traced
up to the DS and the information about particles which reached the DS is stored.
At the second stage, the surviving particles are propagated through the upstream portion of
the DS and muons stopped in the ST are recorded.
The third and the following stages deal with the simulation of the physics processes in the Mu2e detector and the hit generation and digitization.
For the CE dataset, a fraction of muons are assumed to decay into CEs. For the $\bar{p}$ dataset, 
$\bar{p}$ annihilation at rest in the ST is simulated based on the positions 
and times of the stopped antiprotons. Background electrons from annihilation 
result from decays such as $\pi^0 \to \gamma \gamma$, followed by photon conversions, and $\pi^- \to \mu^- \nu$, 
followed by $\mu^-$ decays. During this stage, the raw simulated data are digitized 
into simple C++ classes or structs, using the detector's raw data.

A typical Mu2e event includes multiple pileup hits from particles produced 
by muon captures in the ST, as well as particles entering the DS from 
the TS. The pileup hits make 
the majority of the detector's hits. The pileup level 
depends on the proton pulse intensity. The Mu2e pileup 
simulation assumes that the pulse intensity varies on a 
time scale much longer than 2 $\mu$s, meaning all proton 
pulses around the simulated one have the same intensity. 
Under this assumption, a transformation is applied to hits 
with time $T_i > 1695$ ns outside the microbunch limits, 
assigning them a residual time of $t_i = T_i \div 1695$ ns, 
effectively accounting for hits from previous proton pulses 
that would otherwise be attributed to later microbunches.

For the low-intensity mode, with a mean intensity of 
$1.6 \times 10^7$ protons/pulse, which in Mu2e 
jargon is named "1BB", approximately 25,000 
muons stop in the ST per pulse. In the high-intensity 
mode, named "2BB", this number is about 2.5 times higher (Section \ref{pulsedprotonbeam}).

The datasets used for CE signal plus pileup 
analysis will be referred to as $CE-1BB$ and $CE-2BB$ 
for 1BB and 2BB pileup, respectively. The dataset for 
antiproton analysis without pileup will be 
referred to as $PBAR-0BB$.
For datasets with pileup, 
the pileup hits are explicitly added to the hits from 
the signal process. The antiproton sample, however, 
has been simulated w/o the pileup. 
\section{$\delta$-electrons in the Mu2e tracker}\label{trackerdeltas}

In Mu2e, low-energy electrons 
and positrons, with a momentum below 20 MeV/c and 
referred to as $\delta$-electrons, 
generate the majority of hits in the tracker. 
Identifying the $\delta$-electrons hits as early 
as possible is important for improving the track reconstruction efficiency and 
the optimization of memory and the CPU usage. There are also several physics 
reasons why it would be important to identify 
those hits correctly:
\begin{itemize}
    \item the main Mu2e goal is the search for the 
    CE signal: flagging erroneously even a small 
    fraction of the hits generated by the 
    CE as $\delta$-electrons reduces the CE track reconstruction efficiency. 
    Figure \ref{fig:momhits} shows that in the 
    Monte Carlo sample $CE-1BB$, the simulated CE 
    hits are just 1\% of the total hits in the tracker;
    \item counting the number of protons 
    will be a complementary procedure to the 
    STM to determine the muon stopping rate: 
    the simulation shows that is possible to 
    estimate the number of muons captured in 
    the stopping target by counting the number 
    of protons produced in the nuclear muon 
    capture;
    \item misidentification of muons 
      and pions as $\delta$-electrons 
    may result in erroneous background estimate. 
    This is particularly significant for the 
    antiproton background.
    $p\bar{p}$ annihilation at rest in the ST 
    can produce signal-like electrons, which 
    constitute a background to the CE search. An estimate of this background is 
    presently affected by a large systematic 
    uncertainty. Mu2e has developed a data-driven 
    procedure to improve the estimate.
    $p\bar{p}$ annihilation at rest in the ST 
    can produce events with two or more tracks, 
    each with a momentum around 100-200 MeV/c. 
    For $p\bar{p}$ annihilation, 
    the rate of multi-track events 
    is about 500 times higher 
    than the rate of events with a single 
    signal-like electron. 
    For $10^4$ $p\bar{p}$ annihilation events 
    generated, about 3.7\% of 
    the events contained two reconstructable 
    particle tracks. Therefore, 
    the identification and reconstruction of 
    multi-track events could be 
    used to constrain the $p\bar{p}$ background. 
    Thus, it is crucial not to flag hits 
    generated by muons or pions as $\delta$-electron 
    hits, since this would compromise the 
    reconstruction efficiency. 
\end{itemize}
Figure \ref{fig:momhits} shows the 
momentum distribution (Monte Carlo truth) of 
particles that make at least one hit in the 
tracker for a simulated CE sample ($CE-1BB$, 
Section \ref{datasample}). 
For each Monte Carlo particle, the 
histogram is filled a number of times 
equal to the number of hits generated 
by that particle.
The distribution shows that the 
majority of hits originate from 
low-energy electrons and positrons 
(orange), which constitute approximately 
75\% of the total number of hits. 
There is an asymmetry between the 
number of hits below 20 MeV/c 
produced by electrons and positrons: 
electrons account for 71\% of all hits in 
the tracker, while positrons contribute only 4\%. The difference  
arises because electrons are produced also by Compton scattering, which is 
the primary source of hits in the energy 
range of 1 MeV. This difference 
will be crucial in 
the following sections when discussing 
the $\delta$ flagging efficiency.

The distribution also shows that 14\% of 
the total hits are due to protons, 
which are produced by nuclear processes. 
According to the simulation, the bump at low energy in the Monte Carlo momentum 
distribution of protons and deuterons producing 
hits in the tracker (green) results from inelastic neutron scattering.
Larger momentum values correspond to protons produced in muon 
captures at rest. Their kinetic energy ranges from about 
5 to 20 MeV, resulting in low $\beta \gamma$ values, 
which makes them heavily ionizing particles. 
The deposited energy will be one of the variables used to discriminate protons. 
The distribution has a maximum 
and then decreases at 250 MeV/c, which is 
primarily due to the finite length of the tracker and the increase in 
longitudinal momentum of protons within the tracker acceptance.

It is important to note that the bump 
around 50 MeV/c in the positron distribution should not be 
present. The source is so far unexplained. 
We expect $N(\mu^+ \rightarrow e^+ )/N(\mu^- \rightarrow e^- )$ 
to be about $10^{-3}$ for muons entering the DS. The DIO on the IPA 
(Section \ref{detectorsolenoid}) should also be around $10^{-3}$ compared to the DIF of 
negative muons. The simulation of $\mu^+$ may 
contain some errors and we are still 
investigating this discrepancy. However, this issue is not problematic for the analysis 
of low-momentum electrons and positrons, 
as the momentum ranges are different.

\begin{figure}[!h]
        \centering
        \includegraphics[width =0.95\textwidth]{figures/png/Screenshot_20240812_152905.png}
    \caption[Monte Carlo momentum distribution 
    of particles producing hits in the Mu2e 
    tracker ($CE-1BB$ data sample).]{
        Momentum distribution (Monte Carlo truth)  
       of particles producing at 
       least one hit in the tracker 
       ($CE-1BB$ data sample).  
       The momentum distribution 
       of all particles making hits is 
       depicted in dark blue, with electrons 
       shown in pink, positrons in light 
       blue, $\delta$-electrons in orange, protons 
       and deuterons in 
       light green, and CEs in dark green. }
       \label{fig:momhits}
\end{figure}


Figure \ref{fig:pbar} shows the momentum distribution of 
particles produced in the $p\bar{p}$ annihilation at the ST. 
The data sample ($PBAR-0BB$) is described in Section \ref{datasample}.
For each Monte Carlo particle, the histogram 
has an entry per hit generated by that particle. 
Particles produced in the $p\bar{p}$ annihilation 
are mostly pions and muons.

The momentum distribution has a 
maximum in the 100-200 MeV/c range. Photons are also 
produced, and they can undergo Compton scattering and 
pair production, which explains the presence of a 
$\delta$-electron peak that is about $\sim$100 
times lower than the one in Figure \ref{fig:momhits}.

\begin{figure}[!h]
    \centering
    \includegraphics[width =0.9\textwidth]{figures/png/Screenshot_20240815_124710.png}
\caption[Monte Carlo momentum distribution 
of particles producing hits in the Mu2e 
tracker ($PBAR-0BB$ data sample).]{
    Momentum distribution (Monte Carlo truth) 
    of particles producing at 
   least one hit in the tracker 
   ($PBAR-0BB$ data sample). 
   The momentum distribution 
   of all particles making hits is 
   depicted in dark blue, with electrons 
   shown in pink, positrons in light 
   blue, $\delta$-electrons in orange, protons 
   and deuterons in green, pions in 
   light brown and muons 
   in black. }
   \label{fig:pbar}
 \end{figure}


Figure \ref{fig:momhits} shows that 
the majority of the hits originate from $\delta$-electrons, 
with protons and deuterons being the second most 
common source. Figure \ref{fig:afbef} 
presents an example of comparison of an 
event display before (Left) and after (Right) the background 
hits have been flagged and removed.  
Flagging those hits is crucial for 
several reasons: it prevents unnecessary 
data from being sent to the pattern 
recognition algorithms, improving their efficiency and 
conserving the CPU resources.

\begin{figure}[!h]
    \begin{subfigure}[b]{0.4\linewidth}
        \centering
        \includegraphics[scale = 0.3]{figures/png/Screenshot_20240811_123612.png}
        \subcaption{Before.}
        \label{fig:bef}
    \end{subfigure}
    \begin{subfigure}[b]{0.7\linewidth}
        \centering
        \includegraphics[scale = 0.3]{figures/png/Screenshot_20240811_124245.png}
        \subcaption{After.}
        \label{fig:af}
    \end{subfigure}
    \caption[Before and After background hits flagging.]{
        Before and After background hits flagging. 
        The $x-y$ plane views of a CE 
    event with 2BB pile-up in the tracker (Section \ref{pulsedprotonbeam}). 
    The segments are the $hit$ tracker straws. 
    The hits marked in
    red are from electrons and the ones in 
    blue are from positrons. Left (Right): before (after) $\delta$-electron hit flagging.}
    \label{fig:afbef} 
\end{figure}






\section{$\delta$-electrons flagging algorithms}
A brief description of the Mu2e Offline software tools  
is reported in Appendix \ref{mu2eana}, and the reconstruction 
process is described in Appendix \ref{eventreco}. 
Flagging $\delta$-electrons is done before the time clustering 
and pattern recognition.

Mu2e has developed two types of hit flagging algorithms:
\begin{itemize}
    \item $FlagBkgHits$, described in Section \ref{flagbkghits};
    \item $DeltaFinder$, described in Section \ref{deltafinder}.
\end{itemize}

\subsection{The $FlagBkgHits$ algorithm}\label{flagbkghits}
The detailed description of multivariate 
analysis (MVA) and the 
process of MVA training is beyond the 
scope of this work. 
Nevertheless, since this technique is 
one of the options 
developed for the background hit flagging, I 
will briefly outline 
the fundamental principles involved.

When searching for patterns in a multivariable space, 
a common procedure involves 
defining a set of statistical 
models that analyze the measured 
variables and estimate 
the probability that these are consistent with the 
sought pattern. Once the variables are selected, 
the MVA is trained to recognize patterns by 
evaluating examples known to the trainer, 
allowing for the feedback to refine the 
pattern identification procedure.

The first selection concerns the 
mean energy deposited in  
the $ComboHit$s used to create a 
$StereoHit$ and excludes 
those with a deposited energy above 5 keV.
%
The algorithm then classifies as 
the proton hits all hits with a deposited 
energy above 4.5 keV.
Since $\delta$-electrons tend to form small, dense 
clusters of hits on $x$-$y$ plane, an algorithm based on a clustering 
approach was developed: concentrated clusters in the 
$x$-$y$ plane are sought using a clustering algorithm, 
as $\delta$-electrons are highly likely to create 
such hit clusters.
The $FlagBkgHits$ algorithm uses  
an artificial neural network (ANN)-based
technique to classify those clusters.

The ANN was trained in a supervised mode, 
using CE and $\delta$-electrons hits from 
Monte Carlo data sample.

The resolution on the measurement of the 
hit position in the wire direction is limited to a few cm. 
However, an improved position measurement can be 
achieved by using various layers of straws that have large overlaps in the transverse plane. 
By obtaining two hit measurements from a pair of 
intersecting straws and ensuring they fall within 
a time window on the order of the maximum drift time, 
one can deduce that these hits were produced by the 
same particle and occurred at the intersection of the 
two straws in the $x-y$ plane ($StereoHit$).

The clustering process uses the $x$ and $y$ coordinates 
of the selected hits. A random hit is chosen to define 
the initial cluster, after which the following iterative 
steps are applied. The centroid of each cluster is 
computed, and hits whose distances from a cluster 
centroid fall within a specified inner threshold 
and time window are added to that cluster. Hits 
with distances from all existing clusters greater 
than an outer threshold are used to seed new 
clusters, while hits that fall between the two 
thresholds remain unassigned. This process 
generates a new set of clusters, preparing input 
for the next iteration. During each 
iteration, every hit is reconsidered as a 
potential new point in a cluster, including 
those already assigned. The clustering process 
continues until convergence is achieved, i.e., 
when an iteration no longer results in any changes.


\subsection{The $DeltaFinder$ algorithm}\label{deltafinder}

The $DeltaFinder$ algorithm has been designed 
to identify $\delta$-electron hit patterns. This 
algorithm relies on the 
fact that $\delta$-electrons typically 
form a straight line in 
the $r$-$z$ plane (cyan lines in Figure 
\ref{fig:yzviewdelta}) 
and appear as a spot with a diameter 
of less than $\sim$3 cm in the 
$x$-$y$ plane within the 1 T  
magnetic field. On the other hand, 
CEs create entirely different 
patterns of hits, which appear as oblique segments in 
the $r$-$z$ plane due to their 
helical trajectories (red lines 
in Figure \ref{fig:yzviewdelta}). 
\begin{figure}[!h]
    \centering
    \includegraphics[width =0.6\textwidth]{figures/png/Screenshot_20240811_123048.png}
    \caption[$\delta$-electrons and CE patterns in $r-z$ plane.]{
        $\delta$-electrons and CE patterns in the $r-z$ plane. 
        The white rectangles represent four of the eighteen tracker stations. 
        The cyan straight lines represent a possible $\delta$-electron 
        patterns, while the red lines a possible CE pattern.
        }
    \label{fig:yzviewdelta}
\end{figure}
\subsubsection{Step 1: identifying $\delta$-electron segments}
$DeltaFinder$ first tries to 
identify $\delta$-electron track 
segments within each station 
individually. These segments, 
parallel to the beam axis, are 
called $seed$s. Since hits 
from the same electron should be 
close in both time and space, 
and $\delta$-electrons may hit 
multiple straws within the 
same panel, the algorithm clusters 
these straw hits in space and time, applying various 
cleanup cuts to ensure the 
selected patterns are consistent with the 
$\delta$-electron hit patterns. 
The maximum allowed time 
difference between two hits within 
a station to form a $seed$ is set to 40 ns.

This cleanup based on the $x$-$y$ coordinates is performed by computing a $\chi^2$. The $seed$ 
is reconstructed using three, or four $ComboHit$s. In each case, the intersections between pair of straws 
are determined, and the $x$-$y$ coordinates of the $seed$ are given 
by the center of gravity of these intersections. 

The algorithm extends the $seed$s by requiring 
hits to be sufficiently close to the intersection 
point in time and space, performing multiple checks 
to avoid over-efficiency in 
hit flagging. For each $seed$, 
the mean deposited energy is calculated from the energy 
deposited in all $ComboHit$s. 
This selection optimizes data processing by 
reducing the total number of hits that need to be analyzed.

\begin{figure}[!h]
    \centering
    \includegraphics[width =0.6\textwidth]{figures/png/Screenshot_20240811_115854.png}
    \caption[A $\delta$ candidate $seed$.]{A 
    $\delta$ candidate $seed$. The four 
    coloured segments are the tracker straws that were
    hit in the same station.}
    \label{fig:deltaseeds}
\end{figure}

It is necessary to clarify one detail. 
Figure \ref{fig:energydeposited} shows the 
distribution of the simulated deposited energy 
in the tracker for $\delta$ electrons, CEs, and 
protons in the case of a 1BB pileup. 

To optimize data processing, a hit energy 
threshold could be applied to the $DeltaFinder$ 
to reduce the total number of hits that 
need to be analyzed, thereby speeding up the process. Moreover, only about 
4\% of CE hits have energies above 3.5 keV 
(and 1\% above 5 keV), so the loss of CE 
hits would be minimal, resulting in a 
faster overall algorithm. 

However, such an energy cutoff would have a 
significant impact for the algorithm's performance. 
Starting with fewer hits, especially in 
stereo intersections, reduces the efficiency of identifying the correct $seed$. 

\begin{figure}[!h]
    \centering
    \includegraphics[width =0.8\textwidth]{figures/png/Screenshot_20240729_151910.png}
\caption[Monte Carlo deposited energy 
distribution in the tracker.]{
   The Monte Carlo deposited energy 
   distribution in the tracker ($CE-1BB$ data sample).  
   The red distribution refers to 
   CEs, while the 
   green and the blue one to $\delta$-electrons 
   and protons respectively. 
   The peaks and tails correspond to 
   particles with such high energy 
   deposition that they result in a 
   saturated waveform.
}
   \label{fig:energydeposited}
\end{figure}
\subsubsection{Step 2: connecting the $seed$s}
After the selection based on mean deposited energy, 
$DeltaFinder$ attempts to connect segments that 
are close in both the $x$-$y$ plane and time 
across different stations to form $\delta$-electron 
candidates. A valid candidate must have at 
least two segments and a minimum of five $ComboHit$s. 
Accidentally reconstructed segments of 100 MeV electrons  
remain unconnected due 
to their separation in the $x$-$y$ plane. 

$DeltaFinder$ links $\delta$-electron $seed$s 
across stations, attempting to associate new $seed$s 
with already found $\delta$-electron candidates.
If no match is 
found, a new candidate is created. Good $\delta$-electron 
candidates are marked, and their hits are 
flagged to prevent their inclusion in proton candidate searches.

\subsubsection{Step 3: identifying proton candidates}
Finally, $DeltaFinder$ identifies proton candidates 
using the $seed$s with the mean deposited energy above 3 keV. First, it checks 
if a $seed$ is consistent in time with any existing 
proton candidate. If no consistency is found, 
the hits of the $seed$ are added to a new proton candidate.


\section{Performance analysis and comparison}
The comparison between $FlagBkgHits$ and $DeltaFinder$ 
is performed at two levels, with each testing 
different aspects of the algorithms: 
\begin{itemize}
    \item \textbf{Hit-level comparison}: this 
    phase focuses 
    on estimating how accurately individual hits are 
    flagged, providing the most direct method for 
    assessing and comparing the performance of the two 
    algorithms. This stage allows for an unbiased 
    comparison without the influence of subsequent 
    reconstruction stages. It also includes a 
    direct evaluation of proton counting;
    
    \item \textbf{High-level comparison}: this phase focuses on the 
    comparison of the tracks reconstruction efficiencies. The main figure 
    of merit at this stage is the 
    reconstruction of CE tracks.

\end{itemize}

\subsection{Hit-level comparison}
Before starting with the hit-level 
comparison of the two algorithms, during the analysis, 
we first observed an over-efficiency 
in the proton hit flagging by $DeltaFinder$. 
This primarily impacted the flagging of 
protons, muons, and pions, as shown in 
Table \ref{tab:1bbcelebefore} and Table 
\ref{tab:0bbpbarbefore}. In Tables 
\ref{tab:1bbcelebefore}, \ref{tab:0bbpbarbefore}, 
\ref{tab:2bbcele}, \ref{tab:1bbcele} and 
\ref{tab:0bbpbar}, $f_p$ and 
$f_e$ represent 
the number of $ComboHit$s flagged as 
protons and electrons, respectively, 
divided by the total number of 
$ComboHit$s. DF and FBH denote the $DeltaFinder$ 
and $FlagBkgHits$ algorithms, 
respectively. Each row corresponds to the particle 
under study.

\begin{center}
    \begin{table}[h!]
    \centering
    \renewcommand{\arraystretch}{1.}
    \begin{tabular}{| c | c | c |} 
    \hline
    & $f_{p}$ & $f_{e}$ \\
    \hline
    $p$     & 96.0\% & 1.0\% \\
    \hline
    \end{tabular}
    \caption{$DeltaFinder$ proton 
    flagging results before the 
    adjustment -selecting a proton 
    candidate with more than four hits 
    and a deposited energy above 3 keV- ($CE-1BB$ data sample). 
    $f_p$ and $f_e$ represent 
    the number of proton $ComboHit$s 
    flagged as proton and electron hits, respectively, 
    divided by the total number of the proton/electron $ComboHit$s.}
    \label{tab:1bbcelebefore}
    \end{table}
\end{center}
    
\begin{center}
    \begin{table}[h!]
        \centering
        \renewcommand{\arraystretch}{1.}
        \begin{tabular}{| c | c | c | c | c|} 
        \hline
        &   $f_{p}$ &   $f_{e}$\\
        \hline
        $\mu$ &  5.8\%  & 5.0\%\\
        \hline
        $\pi$ & 2.5\% &  11.2\%\\
        \hline
        \end{tabular}
        \caption{
            $DeltaFinder$ muon 
        and pion flagging results before the 
        adjustment -selecting a proton 
        candidate with more than four hits 
        and a deposited energy above 3 keV- 
        ($PBAR-0BB$ data sample). $f_p$ and 
        $f_e$ represent 
        the number of particle $ComboHit$s 
        flagged as proton and electron hits, respectively, 
        divided by the total number of $ComboHit$s.}
        \label{tab:0bbpbarbefore}
    \end{table}
\end{center}

The number of muon (5.8\%) and pion (2.5\%) hits flagged as protons is quite high,
since as already mentioned in Section \ref{trackerdeltas} the fraction 
of protons in this type of events is really low. 
Figure \ref{fig:0pbarbefore} shows the  
distribution of the total 
number of muon $ComboHit$s (red) 
and those flagged as $\delta$-electrons (blue) 
and protons (green) as a function 
of the particle momentum. 
The distribution shows that a significant number of 
muons hits are misidentified 
as protons hits at low momentum. 
According to the Bethe-Bloch formula, 
such hits have 
higher energy loss and can thus be 
most likely flagged as 
protons. We thus implemented a 
number of corrections to the algorithm. 
Since $seed$s may have 
accidentally attached hits, we imposed a 
condition requiring a $good$ proton 
candidate to have more than four 
hits with a deposited energy greater than 3 keV.

This adjustment reduces the proton hit flagging  
efficiency by a $\sim$10\% 
(Tables \ref{tab:1bbcele} and \ref{tab:2bbcele}). 
However, it significantly reduces the 
number of muons and pions flagged as 
protons, by approximately 
a factor of 2 and a factor of 6 
(Table \ref{tab:0bbpbar}), respectively.

 \begin{figure}[!h]
            \centering
            \includegraphics[width =0.7\textwidth]{figures/png/Screenshot_20240805_222923.png}
        \caption[The  
        distribution of the total 
        and flagged number of muon 
        $ComboHit$s as a function of the particle momentum.]{The  
        distribution of the total number of 
        muon $ComboHit$s 
        (red) and of the muon $ComboHit$s 
        flagged as $\delta$-electrons (blue) 
        and protons (green) as a function of the particle momentum ($PBAR-0BB$ data sample). }
           \label{fig:0pbarbefore}
\end{figure}

Moving to the performance and comparison 
between the two algorithms, Tables \ref{tab:2bbcele} and \ref{tab:1bbcele} 
show the hit-level comparison between the two algorithms 
for the $CE-1BB$ and $CE-2BB$ data samples.
\begin{center}
    \begin{table}[h!]
    \centering
    \renewcommand{\arraystretch}{1.}
    \begin{tabular}{| c | c | c | c |} 
    \hline
    &    $f_{p}$ DF & $f_{e}$ FBH  & $f_{e}$ DF \\
    \hline
    e$^-$ ($p<$20 MeV/c)      & 3.7\%   & 75.1\% & 72.7\%\\
    \hline
    e$^-$ (20$<p<$80 MeV/c)  & 2.2\%   & 49.0\%& 29.9\%\\
    \hline
    e$^-$ (80$<p<$110 MeV/c)  & 0.8\%  &  7.4\%& 4.3\%\\
    \hline
    $p$       &  83.6\%  &  & 2.2\%\\
    \hline
    e$^+$ ($p<$20 MeV/c) & 0.4\%    &   84.2\%& 87.9\%\\
    \hline
    \end{tabular}
    \caption{Electrons, 
    positrons and protons hit-level 
    comparison ($CE-2BB$ data sample). 
    FBH and DF denote  
    the $FlagBkgHits$ and the 
    $DeltaFinder$ algorithms, 
    respectively. $f_p$ and $f_e$ represent 
    the number of particle $ComboHit$s 
    flagged as proton and electron hits, respectively, 
    divided by the total number of $ComboHit$s.
    }\label{tab:2bbcele}
    \end{table}
    \end{center}

    \begin{center}
    \begin{table}[h!]
    \centering
    \renewcommand{\arraystretch}{1.}
    \begin{tabular}{| c | c | c | c |} 
    \hline
   &  $f_{p}$ DF & $f_{e}$ FBH  & $f_{e}$ DF \\
    \hline
    e$^-$ ($p<$20 MeV/c)     & 2.5\%   & 75.9\% & 72.5\%\\
    \hline
    e$^-$ (20$<p<$80 MeV/c)  & 1.0\%   & 50.0\%& 27.4\%\\
    \hline
    e$^-$ (80$<p<$110 MeV/c)   & 0.3\%  &  5.7\%& 3.4\%\\
    \hline
    $p$                 &         83.7\%   &  & 1.0\%\\
    \hline
    e$^+$ ($p<$20 MeV/c)    & 0.2\%    &   85.5\%& 88.5\%\\
    \hline
    \end{tabular}
    \caption{Electrons, 
    positrons and protons hit-level 
    comparison ($CE-1BB$ data sample). 
    FBH and DF denote the $FlagBkgHits$
    and the $DeltaFinder$ 
    algorithms, respectively. $f_p$ and 
    $f_e$ represent 
    the number of particle $ComboHit$s 
    flagged as proton and electron hits, respectively, 
    divided by the total number of $ComboHit$s.}
    \label{tab:1bbcele}
    \end{table}
  \end{center}

  
The fraction of electron and positron hits 
flagged as $\delta$-electron hits differs due to the different momentum distributions  
of these particles at low energies. Figure \ref{fig:detail} shows a zoomed-in view of the 
$\delta$-electron energy distribution (Monte Carlo truth) for particles with  
at least one hit in the tracker for energies below 2 MeV. 
Electrons are shown in pink and positrons in black. 
Each bin represents the number of hits corresponding to a 
specific Monte Carlo particle. The pink distribution peaks 
near the electron mass, that corresponds to a value of $\theta_\gamma \sim 0$ (Section \ref{compton}), 
while the black distribution falls down to zero. 

\begin{figure}[!h]
    \centering
    \includegraphics[width =0.7\textwidth]{figures/png/Screenshot_20240820_154854.png}
    \caption[The electron and positron energy distribution for $E<2$ MeV.]{The electron (pink) and positron (black) 
    energy distribution for $E<2$ MeV.}
    \label{fig:detail}
\end{figure}

Figures \ref{fig:eff1} and \ref{fig:eff2} show the efficiency (i.e., the number of $\delta$-electron 
$ComboHit$s flagged as $\delta$-electrons over the total number of $ComboHit$s 
for each particle type) as a function of particle momentum for positrons (red) 
and electrons (blue) using the $FlagBkgHits$ (left) or $DeltaFinder$ (right) algorithms.

This plot shows a dependence on the particle momentum.
The efficiency plot is, in fact, convoluted with the momentum 
distribution. 

At low momentum values (below 1-2 MeV/c), the efficiency tends to be higher for positrons. 
This is because the total number of positron hits with 
such low momentum is extremely small, as positrons are not produced by  
Compton scattering. A common case of failure occurs when particles produce 
only a single hit in the tracker, making both stereo reconstruction and 
$seed$ reconstruction impossible. Compton electrons typically produce  
one or two hits per station, which does not allow to properly flag them. 
There are lots of electron $ComboHit$s in this energy range which remain 
unflagged. This explains the lower electron hit flagging efficiency.

However, $FlagBkgHits$ performs better with electrons having $E < 2$ MeV. 
This is because the minimum number of hits required to generate a 
seed with $DeltaFinder$ is three, and in this energy range, many 
particles produce only one or two hits in the tracker. Since positrons 
are not present in this energy range (as they are not produced by 
Compton scattering), the efficiency of positron hit flagging is 
higher for $DeltaFinder$ (Tables \ref{tab:2bbcele} and \ref{tab:1bbcele}).

At higher momentum, the two algorithms have a similar performance and the 
differences between positrons and electrons become negligible. 
Figure \ref{fig:efficiency} shows that as the momentum 
increases, the efficiency decreases. This occurs because 
higher momenta correspond to larger radii and a 
greater spread of hits on the $x$-$y$ plane. This trend is also observed for particles in the 
next selected momentum range ($20 \ \text{MeV/c} <p<80 \ \text{MeV/c}$).

A common case of $\delta$ flagging failure is when hits 
occur on straws that are parallel to each other, especially those near 
the center of the tracker, where stereo and $seed$ reconstruction is not possible. 

\begin{figure}[!h]
    \centering
    \begin{subfigure}[t]{0.5\textwidth}
        \centering
        \includegraphics[width=1.\textwidth]{figures/png/Screenshot_20240818_155835.png}
        \caption{}
        \label{fig:eff1}
    \end{subfigure}%
    ~ 
    \begin{subfigure}[t]{0.5\textwidth}
        \centering
        \includegraphics[width=0.95\textwidth]{figures/png/Screenshot_20240813_203916.png}
        \caption{}
        \label{fig:eff2}
    \end{subfigure}
    \caption[The electrons (blue) and positrons (red) hit flagging efficiency versus 
    particle momentum in the low-momentum range.]{The electrons 
    (blue) and positrons (red) hit flagging efficiency versus 
    particle momentum in the low-momentum range. The left plot shows the 
    results for $FlagBkgHits$, while the right plot shows those for $DeltaFinder$.
    }
    \label{fig:efficiency}
  \end{figure} 

$FlagBkgHits$ flags 70\% more CE hits 
as $\delta$-electrons compared to $DeltaFinder$ (Tables \ref{tab:2bbcele} 
and \ref{tab:1bbcele}). The main difference between the two algorithms lies in 
the primitives: $StereoHit$s and $seed$s. If the same 
$\delta$-electron hits three different straws, the $StereoHit$ 
information results in multiple intersection points being 
created. On the other end, intersecting hit wires and reconstructing a $seed$ helps 
to determine the delta electron segment position in 3D.

Concerning proton flagging, we could not compare the two 
algorithms since the $FlagBkgHits$, before creating $StereoHit$s, 
applies a preliminary selection cut on the energy deposited in 
the $StrawHit$s of 5 keV. $FlagBkgHits$ then identifies 
protons as all particles with a deposited energy greater than 
4.5 keV. It was not possible to perform a direct comparison, 
as $DeltaFinder$ does not apply any cuts on the energy deposited in 
the $StrawHit$s, making the comparison unreliable and biased. 
Therefore, we reported the fraction of protons flagged as protons only by $DeltaFinder$.

The results for the 1BB and 2BB data samples are comparable.



      
      Concerning $\mu$ and $\pi$ (Table \ref{tab:0bbpbar}), 
      we observe an approximate factor of 
      4 difference between the probabilities of hit mis-ID by the two algorithms 
      for muons, and about a factor of 3.3 for pions. This occurs because 
      muons and pions often produce more than one hit in a single station. However, the 
      $DeltaFinder$ algorithm can distinguish these hits as it 
      is based on identifying $\delta$-electron hit patterns. 
      The main problem of $FlagBkgHits$ is that it is based on a supervised 
      training. It can distinguish one type of particle (CEs) from 
      another ($\delta$-electrons) but could get confused when 
      the data include particles not used in the training, such as 
      those from $p\bar{p}$ annihilation. 
      
      On average, pions produced from $p\bar{p}$ annihilations 
      have higher momenta than muons, resulting in a higher false 
      positive rate. This occurs because the curvature of the pion tracks 
      is smaller than that of the muon tracks,
      making them more likely to form close in $\phi$ segments, thus appearing more similar to $\delta$-electron.


    \begin{center}
        \begin{table}[h!]
        \centering
        \renewcommand{\arraystretch}{1.}
        \begin{tabular}{| c | c | c | c|} 
        \hline
         &  $f_{p}$ DF &  $f_{e}$ FBH & $f_{e}$ DF\\
        \hline
        $\mu$  &  2.7\%  & 13.0\% & 3.2\%\\
        \hline
        $\pi$ & 0.4\% & 23.8\%& 7.3\%\\
        \hline
        \end{tabular}
        \caption{Muon and Pion 
        hit-level comparison ($PBAR-0BB$ data sample). FBH and DF denote 
        the $FlagBkgHits$ and $DeltaFinder$ algorithms respectively. $f_p$ and $f_e$ represent 
        the number of particle $ComboHit$s flagged as proton and electron hits, respectively, 
        divided by the total number of $ComboHit$s.}
        \label{tab:0bbpbar}
        \end{table}
        \end{center}

\subsection{High-level comparison}
The unflagged hits are then passed to the time clustering 
algorithm followed by the pattern recognition algorithm, 
after which the fit of the found track candidates 
is performed. 
Figure \ref{fig:highlevel1} shows the momentum 
distribution of reconstructed tracks for the $CE-1BB$ 
data sample using both algorithms. 
In the case of $FlagBkgHits$ (blue), less proton hits are  
correctly flagged, and non-flagged hits are
used in the pattern recognition. 
On the other end, $DeltaFinder$ (red) successfully flags 
proton hits, preventing them from 
being used by the pattern recognition. 
Figure \ref{fig:highlevel2} provides a zoomed-in view 
of the reconstructed track momentum distribution within 
the CE momentum range, where the results of track reconstruction 
for the two flagging algorithms are almost identical.

\begin{figure}[!h]
    \begin{subfigure}[b]{0.5\textwidth}
        \centering
        \includegraphics[width = 1.1\textwidth]{figures/png/Screenshot_20240820_162125.png}
        \subcaption{}
        \label{fig:highlevel1}
    \end{subfigure}
    \begin{subfigure}[b]{0.5\textwidth}
        \centering
        \includegraphics[width = 1.1\textwidth]{figures/png/Screenshot_20240820_160904.png}
        \subcaption{}
        \label{fig:highlevel2}
    \end{subfigure}
    \caption[Momentum distribution of reconstructed tracks.]{The momentum distribution of reconstructed tracks. 
    Tracks reconstructed using $FlagBkgHits$ ($DeltaFinder$) 
    as the $\delta$-flagger are shown in blue (red). 
    The distribution in (a) covers a higher momentum range, while (b) 
    provides a zoomed-in view of the momentum range [100, 107] MeV/c.}
    \label{fig:highlevel} 
\end{figure}

There is an excess of reconstructed tracks in the region of momentum above 120 MeV/c 
for $FlagBkgHits$. 
The excess is due to unflagged proton and deuteron hits used by the pattern recognition.

Table \ref{tab:recoeffcele} reports 
the fraction of the simulated CE events that have at least one 
reconstructed track (for both 
$CE-2BB$ and $CE-1BB$ data samples). 
This fraction of CEs is nearly identical 
between the two algorithms, 
as expected from Figure \ref{fig:highlevel}.
Table \ref{tab:1bbcele} shows 
that the difference in the number of 
hits between the two algorithms 
is less than one hit per track, which 
justifies the observed similarities in 
Table \ref{tab:recoeffcele}.
For both 1BB and 2BB datasets, the relative difference between the 
CE reconstruction efficiencies corresponding to two algorithms is less than 1\%.



\begin{center}
    \begin{table}[h!]
    \centering
    \renewcommand{\arraystretch}{1.}
    \begin{tabular}{| c | c | c | c | c |} 
    \hline
    & FBH 2BB & DF 2BB & FBH 1BB & DF 1BB  \\
    \hline
    fraction of CE events with $N_{tracks}>0$ & 36.7\% & 36.5\% & 37.9\% & 37.9\%\\
    \hline
    \end{tabular}
    \caption{Fraction of reconstructed CE events ($CE-2BB$ and $CE-1BB$ data sample). FBH and DF denote  
    $FlagBkgHits$ and $DeltaFinder$ algorithms, respectively.}
    \label{tab:recoeffcele}
\end{table}
\end{center}

To better understand the performance of the 
two algorithms, we analyzed events where at least one 
track was reconstructed using one hit flagger while no track  
was reconstructed with the other hit flagger. A well defined class 
of events was identified in which the effects of hit flaggers are 
mitigated by the time clustering algorithm during the reconstruction.

Figures \ref{fig:TZCluster1} ($FlagBkgHits$) and 
\ref{fig:TZCluster2} ($DeltaFinder$) show the 
$time$ versus $z$ coordinate 
(the one aligned with the tracker axis) for 
the same event processed by both algorithms. 
CE hits are represented by large red dots, 
$\delta$-electrons 
by small brown dots, and protons by large 
blue dots. Violet rectangles denote particles associated with 
the same $TimeCluster$, 
which is formed by grouping hits that 
align along a linear 
path in time versus $z$ space.
\begin{figure}[!h]
    \centering
    \includegraphics[width =0.7\textwidth]{figures/png/Screenshot_20240819_153229.png}
    \caption[$TimeCluster$ on $time-z$ plane.]{$TimeCluster$ (green) on $time-z$ plane. 
    The $TimeCluster$ is reconstructed using the $FlagBkgHits$ output.
    CE hits are represented by large red dots, $\delta$-electrons 
    by small brown dots, and protons by large blue dots. Violet 
    squares denote particles associated with the same $TimeCluster$.}
    \label{fig:TZCluster1}
\end{figure}
\begin{figure}[!h]
    \centering
    \includegraphics[width =0.7\textwidth]{figures/png/Screenshot_20240819_153730.png}
    \caption[$TimeCluster$ on $time-z$ plane.]{$TimeCluster$ (green and orange) on $time-z$ 
    plane. The $TimeCluster$ is reconstructed using the $DeltaFinder$ output.
    CE hits are the large red dots, $\delta$-electrons 
    the small brown ones, and protons the large blue ones. Violet 
    squares denote particles associated with the same $TimeCluster$.}
    \label{fig:TZCluster2}
\end{figure}
The time clustering process begins 
by combining hits within a specific $time-z$ 
window to create $chunks$ (with a 
20 ns time window and a 5-plane z-window). 
A window qualifies as a $chunk$ if 
it contains more than three $ComboHit$s. 
Every potential pair of chunks 
within a certain time 
proximity is tested together, 
and the pair, that minimizes 
the $\chi^2/ndof$ when the hits 
are fit to a linear line, 
is combined. This procedure is 
repeated until no further combinations 
yield a $\chi^2/ndof$ below a set 
threshold. If a chunk 
exceeds a minimum number of straw 
hits, it is saved as a cluster.

Since particles grouped together 
may be at different $z$ values, 
$TimeCluster$s can exhibit varying 
$z$ coordinates. In Figure \ref{fig:TZCluster1}, 
all hits are grouped into a single cluster (green), 
while in Figure \ref{fig:TZCluster2} 
(green and orange), CE hits are divided into 
two different $TimeCluster$s.  
Specifically, when the $DeltaFinder$ algorithm is used,
hits from the same 
particle are split into separate time clusters and no 
track are reconstructed. In contrast, 
with $FlagBkgHits$, unflagged hits are utilized by the 
time clusterer to $connect$ particle hits, 
leading to successful track reconstruction. 

This example highlights the importance of refining 
the time cluster finder to improve the track reconstruction.


Table \ref{tab:recoeffpbar} shows the fraction of reconstructed $p\bar{p}$ events, 
with at least two reconstructed tracks with the reconstructed momenta above 80 MeV/c or 90 MeV/c. The 80 
MeV/c cut approximately corresponds to the minimum particle momentum for which 
a track of a particle coming from the ST can be reconstructed 
in the Mu2e tracker. The 90 MeV/c cut is applied to ensure the 
suppression of DIO events to a negligible level. 
For Run I, requiring a DIO electron momentum above 90 MeV/c 
yields an estimate of approximately $10^{-2}$ events with 
two DIO electrons. With a track reconstruction efficiency of 
approximately 0.1, this corresponds to roughly $10^{-4}$ events with two 
reconstructed DIO electron tracks.
Assuming a uniform distribution in time, the number of such events
within a 100 ns window is approximately $10^{-5}$.

The $DeltaFinder$ algorithm shows an advantage of 
approximately 22\% in reconstructing 
two tracks, attributed to its ability to distinguish between 
muon and pion hits compared to $\delta$-electrons (Table \ref{tab:0bbpbar}). 
The difference in hit-level efficiencies results in more 
than 12 hits per track difference between the two algorithms. After 
applying momentum cuts, the difference between the algorithms 
appears to be similar to that observed without a momentum cut.


\begin{center}
        \begin{table}[h!]
        \centering
        \renewcommand{\arraystretch}{1.}
        \begin{tabular}{| c | c | c |} 
            \hline
            &  FBH & DF\\
            \hline
            fraction of events with $N_{tracks} \geq 2$ &  1.8\% & 2.2\%\\
            \hline
            fraction of events with $N_{tracks} \geq 2$ \& $p>$80 MeV/c & 1.7\% & 2.1\%\\
            \hline
            fraction of events with $N_{tracks} \geq 2$ \& $p>$90 MeV/c & 1.6\% & 2.0\%\\
            \hline
            \end{tabular}
        \caption{Fraction of reconstructed $p\bar{p}$ events 
        ($PBAR-0BB$ data sample). FBH and DF denote the $FlagBkgHits$ and the $DeltaFinder$ algorithm, respectively.}
        \label{tab:recoeffpbar}
        \end{table}
\end{center}

\section{Conclusions}
The simulation shows that the majority 
of hits in the tracker is 
generated by $\delta$-electrons, i.e. electrons 
and positrons with momenta below 20 MeV/c. 
Two distinct algorithms have been developed in the Mu2e Offline to 
identify those hits and exclude them from the pattern recognition: 
$FlagBkgHits$ and $DeltaFinder$. 
The $FlagBkgHits$ algorithm initially clusters the hits and then uses an ANN 
to classify them, while $DeltaFinder$ reconstructs only clusters of hits  
consistent with those produced by low-momentum particles or protons and deuterons.

I performed a systematic two-level comparison of the performance of the two algorithms:

\begin{itemize}
    \item hit-level comparison: the focus is on evaluating the 
    accuracy with which individual hits are flagged, providing a 
    direct method for comparing the algorithms' performance. In terms 
    of $\delta$-electron flagging, both algorithms perform similarly in the 
    considered energy range. Electrons (below 1-2 MeV/c) often produce only one 
    or two hits, making them harder to flag, which reduces the electron hit flagging 
    efficiency. The $FlagBkgHits$ algorithm works better for electrons below 
    2 MeV, since the minimum number of hits to form a 
    proper $seed$ for $DeltaFinder$ is three. 
    Since positrons are not present in this energy range (as they 
    are not produced by Compton scattering), the efficiency of positron hit flagging is higher for $DeltaFinder$.
    At higher momentum, both positron and 
    electron flagging efficiency become similar, but overall efficiency 
    decreases as momentum increases due to a wider spread of hits.
    $FlagBkgHits$ flags approximately 70\% more CE hits than $DeltaFinder$.
    The main difference between the two algorithms results from  
    the primitives: $StereoHit$s and $seed$s. When a $\delta$-electron hits three 
    straws, the $StereoHit$ creates multiple intersection points, while the 
    intersection of hit wires and the consequent reconstruction of 
    a $seed$ allows to better determine the delta electron segment position in 3D. 
    Furthermore, $DeltaFinder$ can flag proton hits (approximately 84\%), while $FlagBkgHits$ 
    can only flag hits based on large energy deposits, but it does not specifically identify proton candidates.
    To correctly estimate the background, it is crucial to not misidentify 
    product hits of $p\bar{p}$ annihilation as $\delta$-electrons and protons.
    This necessitates examining muon and pion hit flagging. $FlagBkgHits$ has been trained 
    on datasets lacking muons and pions and  
    flags these particles at rates roughly 4 and 3.3 times higher, 
    respectively, than $DeltaFinder$;
    
    \item high-level comparison: this is a study of the algorithms' 
    impact on subsequent stages of event reconstruction 
    and the track reconstruction efficiency. 
    The main observed difference between the two algorithms 
    is an excess of tracks in the momentum range above 120 MeV/c for  
    $FlagBkgHits$. 
    This increase is due to $FlagBkgHits$ failing to 
    properly flag proton and deuteron hits, allowing these particles to be sent to the 
    reconstruction stage. 
    For both 1BB and 2BB datasets, the relative difference between the 
    CE reconstruction efficiencies corresponding to two algorithms is less than 1\%.
    Moreover, the $DeltaFinder$ algorithm demonstrates a significant advantage, 
    showing approximately a 22\% higher fraction of $\bar{p}$ events with at least two reconstructed tracks compared to the alternative method.
\end{itemize}

A comparison of the timing performance of the 
two algorithms has also been performed.
For $FlagBkgHits$, it was necessary to account for the time 
spent creating the $StereoHit$s, while $DeltaFinder$ independently 
reconstructed the $seed$s. Processing $CE-1BB$ sample required about 
0.14 ms/event for $FlagBkgHits$ plus an additional 0.15 ms/event, 
compared to 0.42 ms/event for $DeltaFinder$. For $CE-2BB$ events, 
$FlagBkgHits$ took approximately 0.37 ms/event plus 0.34 ms/event, 
while $DeltaFinder$ required 1.1 ms/event. The Mu2e specification states 
that the processing time per event should not exceed 5 ms. The 
difference in processing time is approximately 0.13 ms/event with 
1BB pileup and 0.39 ms/event with 2BB pileup. While this difference 
does not appear critical at the moment, $DeltaFinder$ needs to improve its timing performance.

The primary drawback of $FlagBkgHits$ is its reliance 
on the supervised ANN training using CE and $\delta$-electron samples. 
The problem with this method is when other particles, such as cosmic muons 
and those from $p\bar{p}$ annihilation, are introduced into the algorithm. 
In principle, additional samples could be incorporated into the training 
process, but these would increase the technical complexity. 
Additionally, the training was performed using Monte Carlo data 
rather than real data, posing a potential risk when transitioning 
to actual data taking. 


\chapter{Conclusions}\label{conclusions}
The success of Mu2e depends on many factors and one of them is the 
performance of the tracker. 
The tracker must provide excellent momentum resolution, approximately 1 MeV/c, 
to distinguish the monochromatic CE signal from the background. 
A straw tube tracker will be used, so the 
technology used is the drift tubes. This distinctive geometry 
is highly effective in reducing background, as it allows 
the passage of particles produced during muon capture, 
residual beam elements, and electrons from the initial proton collision, 
particles that would otherwise cause excessive instantaneous detector 
occupancy and accumulated radiation damage. The Mu2e straw tracker is placed inside the DS downstream from 
the ST in a 1 T uniform magnetic field. To minimise the probability of scattering and the energy loss of the CE, 
the volume inside the DS is evacuated to $10^{-4}$ Torr. The detector has a modular design, consisting 
of basic elements referred to as $Panel$s, $Face$s, $Plane$s, and $Station$s. The 
tracker comprises panels containing arrays of straw tubes, which are the 
sensitive units, arranged like the harp chords.

This Thesis has provided an in-depth examination of the Mu2e 
tracker system, from its initial stages of commissioning to the 
optimization of data acquisition and preliminary calibration steps. 
My work at Fermilab has focused on the 
complete Data Acquisition (DAQ) testing from both hardware and software 
perspectives. I was involved in the commissioning of the Mu2e DAQ system and 
the Vertical Slice Test (VST) of the tracker. The VST encompasses the entire 
testing chain, from the straws to the readout, to processed data on disk.
Throughout the chapters, we explored critical aspects of the tracker 
system, focusing on the robustness and reliability of data acquisition 
processes and the crucial role of calibration in ensuring precise measurements. 

\section{Commissioning of the tracker DAQ and FEE}
The first phase, described in Chapter \ref{commissioning}, 
was integral to the early phases of the Mu2e experiment, particularly in the context of the 
tracker system's DAQ and FEE testing. 
The first test I performed has been to verify the correct performance of ROC 
buffering. During these test, a single ROC was connected to the DTC. Data were collected
with digi-FPGAs pulsed by their internal pulsers, with the ROC set in the external
mode. ROC's digital readout logic allows to be emulated with a bit-level C++ simulation,
which I contributed to develop. The Monte Carlo and the data have been compared in two different 
modes: $ROC$ $buffer$ $overflow$ and $underflow$ configuration, at a channel occupancy level, studying 
the timing distribution and delays between channels and digi-FPGAs.

\section{Pre-pattern recognition studies}
Given the high data volume expected during Mu2e operations, estimated 
at approximately 7 PBytes per year, optimizing memory usage and minimizing 
CPU consumption are critical. A significant challenge lies in effectively 
flagging $\delta$-electron hits, which are the primary source of hits in 
the tracker, without compromising the efficiency of conversion electron 
hit detection and track reconstruction. $\delta$-electrons originate from 
Compton scattered electrons, pair production electrons and positrons, and 
delta rays. Compton scattered electrons are produced when photons, from 
neutron capture, interact with the detector material. Typically, these 
photons have energies of a few MeV. Pair production electrons and positrons 
are generated during nuclear recoil processes. Delta rays 
are produced when high-energy charged particles collide 
with the detector material. A detailed study of pre-pattern recognition 
and a thorough comparison of two algorithms for $\delta$-electron flagging is 
provided in Chapter \ref{delta} to address this issue.
The simulation shows that the majority 
(approximately 75\%) 
of hits in the tracker are generated by 
$\delta$-electrons, i.e. electrons 
and positrons with momenta below 20 MeV/c. 
Two distinct algorithms 
have been developed in Mu2e Offline to 
identify these hits and exclude 
them from the reconstruction process: 
$FlagBkgHits$ and $DeltaFinder$. 
The $FlagBkgHits$ algorithm initially 
clusters the hits and then uses an ANN 
to classify them, while $DeltaFinder$ 
identifies clusters of hits that are 
consistent with those produced by 
low-momentum particles.

I performed a systematic two-level comparison of the performance of the two algorithms:

\begin{itemize}
    \item Hit-level comparison: the focus is on evaluating the 
    accuracy with which individual hits are flagged, providing a 
    direct method for comparing the algorithms' performance. In terms 
    of $\delta$-electron flagging, both algorithms perform similarly. 
    However, $DeltaFinder$ performs better in flagging positrons 
    due to the low statistics at energies below 2 MeV/c, 
    while this disadvantages $FlagBkgHits$ since the ANN is 
    statistics dependent, though it 
    performs worse with electrons as it struggles to reconstruct hits 
    that register only a single hit per station.  
    $FlagBkgHits$ flags approximately 70\% more CE hits than 
    $DeltaFinder$, as it performs analysis on the $x$-$y$ plane and does 
    not reconstruct hits in the $z$ coordinate. Furthermore, 
    $DeltaFinder$ can flag proton hits (approximately 84\%), 
    a task that $FlagBkgHits$ cannot perform. It is crucial that hits 
    corresponding to $p\bar{p}$ annihilation are not flagged. This necessitates 
    examining muon and pion hit flagging, but $FlagBkgHits$ has been trained 
    on datasets lacking muon and pion hits and so 
    flags these particles at rates roughly 4 and 3.3 times higher, 
    respectively, than $DeltaFinder$;
    
    \item High-level comparison: this is a study of the algorithms' 
    impact on subsequent stages of event reconstruction, particularly 
    how effectively each contributes to accurate track reconstruction. 
    The main observed difference between the two algorithms lies in the number 
    of reconstructed tracks using CE data samples, which is 16\% higher 
    for $FlagBkgHits$. This increase is due to $FlagBkgHits$ failing to 
    properly flag protons, allowing these particles to be sent to the 
    reconstruction stage. The difference in the reconstruction of 
    $p\bar{p}$ events, measured as the fraction of events with 
    at least two reconstructed tracks, is approximately 22\% higher with $DeltaFinder$.
\end{itemize}
An important factor of study was the timing performance of the 
two algorithms. For $FlagBkgHits$, it was necessary to account for the time 
spent creating the $StereoHit$s, while $DeltaFinder$ independently 
reconstructed the $seed$s. Processing $CE-1BB$ sample required about 
0.14 ms/event for $FlagBkgHits$ plus an additional 0.15 ms/event, 
compared to 0.42 ms/event for $DeltaFinder$. For $CE-2BB$ events, 
$FlagBkgHits$ took approximately 0.37 ms/event plus 0.34 ms/event, 
while $DeltaFinder$ required 1.1 ms/event. Processing $PBAR-0BB$ 
events took about 0.02 ms/event plus 0.02 ms/event for $FlagBkgHits$, 
and 0.09 ms/event for $DeltaFinder$. The difference in processing time 
is negligible considering that the entire reconstruction process 
takes only a few ms per event.

The primary drawback of $FlagBkgHits$ is its reliance 
on supervised training using CE and $\delta$-electron samples. 
This approach fails when other particles, such as cosmic muons 
and those from $p\bar{p}$ annihilation, are introduced into the algorithm. 
Additionally, the training was performed using Monte Carlo data 
rather than real data, posing a potential risk when transitioning 
to actual data-taking. Moreover, being ANN-based, $FlagBkgHits$ 
performs poorly when the statistics are very low and lacks 
a well-defined method for proton flagging.

\section{First steps towards the station calibration}
Chapter \ref{planning} discusses the initial steps towards the tracker 
calibration. Our ultimate goal is to perform a time calibration of 
the first assembled station of the tracker using cosmic muons, aiming for a longitudinal hit 
position resolution better than 4 cm. This involves determining the signal 
propagation times and channel-to-channel delays. Due to technical constraints, the 
first option is to perform the with a vertically oriented station. I performed a 
Monte Carlo study to determine the impact of this orientation on the quality of the calibration, 
in particular on the cosmic track reconstruction, focusing on the 
potential biases that could arise. Together, 
these studies provide essential insights into the operation, optimization, 
and calibration of the Mu2e tracker system, contributing to its 
overall performance and reliability.
The Mu2e collaboration considered 
that a vertical station calibration might be 
feasible for our goals.
This Chapter demonstrated that the vertical orientation for calibrating a station is not optimal. This is due to several reasons:
\begin{itemize}
    \item The selection criteria in Section \ref{eventselection} select cosmic muons with orientations that affect panel illumination. 
    The illumination is non-uniform, and moreover, there are almost no hits in the central region of the panel, which could result in waveform non-linearities;
    \item The rate of cosmic events is $R \sim 380$ Hz, and the expected duration of the calibration is about 6 days without interruptions, 
    which is excessive for the simple calibration we intend to perform;
    \item The bias range is approximately [-6, 6] cm. The 2D distribution of the longitudinal bias versus the true position shows 
    four distinct spots on the $x$ axis corresponding to the overlap regions. Each spot refers to different straws.
\end{itemize}
Therefore, an alternative orientation or method should be considered to optimize the calibration process and achieve more accurate results.
For this reason, a new mechanical approach 
to address the problems outlined in Section \ref{gassystem}  
is currently under development.
\appendix
\chapter{Charged particle in a magnetic field}\label{appendix1}
The motion of a charged particle in a magnetic field can be described by the Lorentz force:
\begin{equation}
    \mathbf{F}=q \ \mathbf{v}\times\mathbf{B}
\end{equation}
where $q$ is the particle charge, $\mathbf{v}$ is the particle speed and $\mathbf{B}$ is 
the magnetic field.
A charged particle moving in a uniform solenoidal field describes the 
combination of a free trajectory and a circular motion, namely a helix, with the property:
\begin{equation}\label{partincamp}
    |\mathbf{B}|\rho=\frac{p_\perp}{|q|}
\end{equation}
where $\rho$ is the helix radius, $\p_\perp$ is the transverse momentum component.
This is the simplest example, although more complex magnetic fields can 
produce a more complex trajectories. In this appendix, I will go over how to 
use a gradient to accelerate particles and also how particles perpendicularly 
drift in a curved magnetic field. The intensity of a magnetic field with a 
non-null gradient changes with the position. The force derived from the gradient is 
proportional to the particle magnetic momentum $\mu$, where $E$ represents 
energy. It can be written as:
\begin{equation}
   \mathbf{F}=-\mu \nabla \mathbf{B} \qquad \qquad \mu=\frac{c^2 p_\perp^2}{2 E B}
\end{equation}
The force alters the direction of momentum, not the particle's energy. If the 
gradient is strong enough, it can reverse the direction of motion and reflect 
the particle like a magnetic mirror. Complex gradients can trap particles in 
certain regions or prevent them from staying there for too long.
The other important feature is the use of curved magnetic fields. In a curved 
solenoid, the particle orbit drifts perpendicular to the bending plane. 
The drift velocity $v_D$ and total displacement $D$ can be determined from 
the path along the curved solenoid $S$:
\begin{equation}
    v_D=\frac{m \gamma c}{e B R}(v_\parallel ^2+\frac{1}{2}v_\perp ^2)
\end{equation}
\begin{equation}
    D \propto p S (\frac{1}{cos \theta} + cos \theta)
\end{equation}
In the equations above, parallel and perpendicular refer to the magnetic field direction 
and $R$ represents the solenoid bending radius. $\theta$ is 
the angle of the helix axis and the magnetic field and the sign of the drift 
relies on the sign of the charge. These characteristics show that a curved 
solenoid can be used to separate particle beams of opposite charge.


\chapter{Mu2e Simulation Framework}\label{mu2eana}

The Mu2e software framework is designed to study the expected performance of the Mu2e detectors, the signal 
reconstruction efficiency, and the characteristics of the backgrounds. The simulation framework is based on 
GEANT4, a widely used simulation toolkit. The simulation relies on Monte Carlo methods.

It is implemented in the C++ programming language and includes a comprehensive set of tools, such as 
tracking, geometry, and physics models. The library provides models of physical processes like particle 
scattering, energy loss, and decay of long-lived particles over a wide energy range. The simulation 
environment implements the Mu2e geometry. Pattern recognition and track reconstruction algorithms are 
incorporated into the software framework.

\section{art and FHiCL}

The Mu2e Offline software is based on the \textit{art} framework. 
This framework is developed and maintained by the Fermilab Scientific 
Computing Division (SCD) and is used by multiple Intensity Frontier experiments at Fermilab.

\textit{art} is a command-line-driven event-processing framework written in 
C++. It operates in a non-interactive mode where it sequences events as directed 
by the user. It has been designed to fulfill a wide range of requirements in 
high-energy physics experiments, including high-level software triggers, online data 
monitoring, calibration, reconstruction, simulation, and analysis.

\textit{art} runtime configuration is written in the Fermilab Hierarchical 
Configuration Language (FHiCL), a data definition language developed at Fermilab. 
A FHiCL file contains definitions of C++ classes that implement the \textit{art} 
services. Within these classes, algorithms ranging from simulation and 
reconstruction to analysis codes are built and integrated into dynamic 
libraries called modules. The FHiCL files declare which modules will be loaded, 
in what order they will run, and which files will be read in input and written in output.

The simulation begins by using a FHiCL file to configure the process, specifying 
the necessary modules and call files containing essential geometry and physics data.

\section{STNTUPLE and ROOT}

STNTUPLE is an n-tuple data format and a lightweight n-tuple analysis framework, 
written in C++. It has been used for many years by the CDF experiment at Fermilab 
and ported to Mu2e. One of the plug-in modules for \textit{art} is specific to the 
usage of the STNTUPLE. Every STNTUPLE file is a ROOT file containing multiple branches, 
each corresponding to a data block. A block is a data container optimized for I/O and 
analysis purposes, storing the Mu2e raw and/or reconstructed data. A STNTUPLE of the 
data saved in an \textit{art} file can be created and stored by using the appropriate 
module in the .fcl \textit{art}-job configuration file. The type of data saved in this 
format is customizable. Once the .stn file is generated, the analysis can be conducted 
using this data format, avoiding the re-running of the reconstruction. STNTUPLE is built 
on top of a data analysis and graphics package ROOT developed at CERN and allows the use 
of all interactive features of ROOT in the analysis.

\section{Multi-stage Simulation}

Mu2e simulation employs a technique known as multi-stage simulation for efficient event 
generation and simulation. This approach involves generating and partially simulating events, 
pausing the simulation, and saving the generated data. Subsequent stages build upon this saved 
data, extending the simulation before saving again. Multiple stages may be utilized to complete 
a simulation. Stages can end when particles reach specific planes or volumes, such as the DS 
region containing the target and detector or the tracker volume. This method is particularly 
useful when early stages consume most of the CPU time in Mu2e simulations.

The motivations for using multi-stage simulation include:

\begin{itemize}
    \item \textbf{Progressive detector definition:} The simulation can define a portion of the 
    detector in later stages. For instance, computationally-intensive tasks like simulating 
    protons on target can be completed in an initial stage, allowing the tracker geometry and 
    algorithms to be inserted later. Once the tracker is ready, simulation can proceed into its volume.
    \item \textbf{Efficient design variation studies:} It facilitates the study of different 
    experimental designs. Simulation stages can end outside a detector, enabling quick testing 
    of various detector designs with no need to repeat the entire prior simulation.
    \item \textbf{Effective error recovery:} It acts like a check-pointing mechanism, enhancing 
    error recovery efficiency. If an error occurs in a later stage, only that stage needs to be 
    redone using the output from the preceding stage. When early stages dominate CPU time, this 
    approach can lead to significant time savings.
    \item \textbf{Resource and limitation handling:} It can handle constraints like job duration 
    and output file size. Techniques like compression, event filtering, and concatenation can be 
    employed to manage resources effectively.
\end{itemize}

Multi-stage simulation also supports two other critical simulation techniques: mixing and resampling. 
These two techniques are necessary to study backgrounds and low statistics processes.

For the study of muon beamline vertical misalignments developed in this Thesis, the simulation 
framework consists of six stages:

\begin{itemize}
    \item \textbf{First stage (s1):} The interactions of 8 GeV protons in the Production Target (PT) 
    are simulated. All produced particles form the beam, which is traced up to the midpoint of the TS, 
    storing information about any particle that makes it that far.
    \item \textbf{Second stage (s2):} It uses the output of s1 and traces the beam up to the entrance 
    of the Detector Solenoid.
    \item \textbf{Third stage (s3):} It propagates the surviving particles from s2 through the 
    upstream portion of the DS vacuum and records muons stopped in the aluminum Stopping Target (ST).
    \item \textbf{Fourth stage (s4):} $\mu^+$ and $\mu^-$ stopped in the ST are separated into two 
    different outputs for separate analysis.
    \item \textbf{Fifth stage (s5):} The detector is finally simulated. Target-stopped muons are 
    forced to undergo Michel decays. The resampling technique is employed to increase the available 
    statistics; each stopped muon is used several times as a starting point to generate different 
    decays. Particles that intersect the tracker or calorimeter are recorded.
    \item \textbf{Sixth stage (s6):} The simulated raw data are converted into simple C++ classes 
    or structs, simulating the digitization of detector raw data.
    \item \textbf{Seventh stage (s7):} The final stage reconstructs the particle tracks and 
    generates the n-tuples.
\end{itemize}

Multi-stage simulation saves computation resources in the study of vertical misalignment 
effects. The production of datasets with displaced COL3 requires only re-running from stage 
2 onwards. This is very convenient as s1 is the most resource-consuming one.

\section{Mu2e Offline Event Reconstruction}

The central objective of the Mu2e reconstruction algorithms is to achieve efficient 
reconstruction of electrons in the range of conversion electrons. To fulfill this objective, 
the algorithms and a few user-defined parameters within them are set to default values optimized 
specifically for this scenario. This section outlines the stages of a Mu2e event reconstruction.

\section{Tracker Hit Reconstruction and Pre-filtering}

The Mu2e reconstruction process starts with hit reconstruction, where digital signals from the 
tracker are converted into physical time and position, creating StrawHits. Adjacent StrawHits in 
a panel, likely from the same particle, are combined into ComboHits to facilitate pattern recognition. 
The FlgBkgHits algorithm flags hits from low-energy electrons or positrons due to scattering, $\gamma$ 
conversion, or $\delta$-rays. This algorithm clusters hits in time and in the $xy$ plane, using 
Multivariate Analysis to distinguish low-energy hits from conversion electron hits, which are stored for subsequent pattern recognition.

\section{Calorimeter Hit Reconstruction}

CaloClusters are formed by combining signals from crystals hit by particles 
in the calorimeter. Clusters are reconstructed by grouping adjacent crystals 
within a 2 ns window and an adjustable energy threshold, currently set at 50 MeV.

\section{Helix Search}

Charged particles in the DS magnetic field follow helical trajectories described 
by the following parameters:

\[
\vec{\eta} \equiv (d_0, \phi_0, \omega, z_0, \tan \lambda);
\]

where $d_0$ is the distance of the point of closest approach to the solenoid axis, 
signed by the particle angular momentum with respect to the origin; $\phi_0$ is 
defined by the momentum direction at the point of closest approach; $\omega = 1/R$ 
is the curvature in the transverse plane; $z_0$ is the $z$-coordinate of the point of 
closest approach; $90^\circ - \lambda$ is the pitch angle between the momentum $p$ and 
the $xy$ plane. Helix search involves time clustering and pattern recognition.

\subsubsection{Time Clustering}

The Time Clustering algorithm groups hits within a narrow time window into TimeClusters. 
Since ComboHits from the same particle tend to cluster in time, for each group of ComboHits a 
TimeCluster is created. To improve the clustering, the time distribution is generated by 
propagating all the hit times to the central plane of the tracker ($z=0$). Finally, the 
TimeCluster time and position in the calorimeter are evaluated. The process is iterated 
until the list of ComboHits associated with the TimeCluster is stable. All TimeClusters with 
more than a programmable number of hits are stored. Pattern recognition follows, utilizing 
information from both tracker and calorimeter through two different algorithms.
\subsubsection{Pattern Recognition}

Pattern recognition in Mu2e can be categorized into two main approaches: 
Tracker-only Pattern Recognition and Calorimeter + Tracker Pattern Recognition.

\paragraph{TrkPatRec: Tracker-only Pattern Recognition}

TrkPatRec performs helix reconstruction in the $xy$ plane and finds the 
track projection on the transverse plane, determining the radius and impact 
parameter with respect to the Stopping Target. It then performs reconstruction 
in the $\phi z$ plane to determine the track pitch. Circle fitting and angular 
position corrections are used for reconstruction.

\paragraph{CalPatRec: Calorimeter + Tracker Pattern Recognition}

CalPatRec employs CaloClusters as initial templates for pattern 
recognition. Assuming the existence of a CaloCluster with reconstructed 
energy above 50 MeV, its time and position are used to filter the collection of 
ComboHits. These hits are expected to lie within the calorimeter acceptance, 
and those within a defined time window around the cluster time are selected. 
Helix fitting is performed using the filtered ComboHits, with track parameters 
iteratively adjusted until convergence is achieved. This combined approach 
enhances track reconstruction efficiency, particularly for electrons in the conversion energy range.

\section{Track Fitting and Momentum Determination}

Once the track candidates have been identified, track fitting algorithms 
are applied to refine the track parameters and estimate the momentum of the 
charged particles. The primary track fitting algorithm used in Mu2e is the Kalman 
Filter, a recursive method that optimally estimates the track parameters by minimizing the chi-squared of the track fit.

The Kalman Filter considers multiple scattering, energy loss, and magnetic 
field variations, making it well-suited for the high-precision tracking 
required in the Mu2e experiment. The momentum resolution achieved is crucial 
for distinguishing conversion electrons from background processes.

\section{Background Rejection and Signal Selection}

Background rejection is a critical aspect of the Mu2e event reconstruction, 
as the experiment aims to detect rare conversion events amidst a significant 
background. Several strategies are employed to enhance signal selection and suppress background events:

\begin{itemize}
    \item \textbf{Timing Cuts:} Events are selected based on their timing relative 
    to the beam structure, reducing beam-related backgrounds.
    \item \textbf{Track Quality Cuts:} Tracks must meet specific quality criteria, 
    such as a minimum number of hits and a maximum chi-squared value, to ensure reliable reconstruction.
    \item \textbf{Calorimeter Matching:} Tracks are matched with CaloClusters based 
    on spatial and temporal criteria to confirm the presence of an electromagnetic 
    shower consistent with a conversion electron.
    \item \textbf{Likelihood-Based Methods:} Multivariate analysis techniques, such 
    as boosted decision trees, are employed to combine various discriminating 
    variables and enhance signal purity.
\end{itemize}

\section{Data Analysis and Event Classification}

After reconstruction and background rejection, the final step in the Mu2e 
simulation framework is data analysis and event classification. This involves 
categorizing events into different classes based on their characteristics and 
analyzing the resulting distributions to extract physics results.

The analysis is performed using ROOT, a powerful data analysis framework that 
provides a wide range of tools for histogramming, fitting, and statistical analysis. 
Custom scripts and macros are developed to automate the analysis process and generate 
plots and tables for interpretation.

\section{Conclusion}

The Mu2e simulation framework is a sophisticated software suite that enables 
comprehensive studies of detector performance, signal reconstruction, and background 
characteristics. By utilizing a combination of multi-stage simulation, advanced 
reconstruction algorithms, and efficient background rejection techniques, the framework 
supports the search for charged lepton flavor violation in the Mu2e experiment. The flexibility 
of the simulation framework allows for thorough investigations of various physics scenarios, 
design optimizations, and detector responses.

Through the integration of state-of-the-art technologies such as GEANT4, \textit{art}, 
and ROOT, the simulation framework provides the necessary tools to achieve the scientific 
goals of the Mu2e collaboration. Ongoing developments and improvements in the framework 
continue to enhance the precision and reliability of the simulation, paving the way for 
groundbreaking discoveries in the field of particle physics.


\chapter{Mu2e Offline Event Reconstruction}\label{eventreco}

The primary goal of the Mu2e reconstruction algorithms is 
to efficiently reconstruct electrons within the specific range 
of conversion electrons. To meet this goal, the algorithms, 
along with certain user-defined parameters, are preconfigured 
with default values that are carefully optimized for this purpose. 
This appendix details the various stages involved in the Mu2e event reconstruction process.

\section{Hits reconstruction and pre-filtering}
\subsection{Tracker Hit Reconstruction and Pre-filtering}
Charged particles traversing the tracker volume generate 
ionization charges within the gas enclosed by the straws. 
These charges are collected and produce electrical signals. 
Since the straws are read out by the front-end electronics from 
both ends, the initial step in the hit reconstruction process 
involves combining the two resulting electrical signals to 
estimate both the hit time and position along the wire. In the 
reconstruction code, this information is encapsulated in an object named $StrawHit$.

One of the most significant challenges in track reconstruction within 
the Mu2e experiment is the presence of numerous $StrawHit$ 
objects within the 1.7$\mu$s time window that corresponds to 
a single event. The first critical task is, therefore, to identify 
$StrawHit$ instances that are close in time and likely to 
have been generated by the same particle traversing the tracker. To 
enhance spatial resolution and minimize combinatorial complexities 
during track searching, adjacent $StrawHit$ objects within a 
panel, which are most likely due to the same particle, are combined 
into a more complex object termed $ComboHit$. While retaining 
the information from individual $StrawHit$s, the $ComboHit$ provides the average time and position of the cluster.

This process is complicated by the presence of numerous hits produced 
by low-energy $p < 20$ MeV/c electrons, which are knocked 
out by Compton scattering, $\gamma$ conversion, or $\delta$-rays. 
These electrons, commonly referred to as $\delta$-electrons, follow 
trajectories with small radii and can generate a large number of hits, 
all confined within a limited volume, leading to regions of high occupancy. 
Fortunately, the hit patterns generated by $\delta$-electrons are markedly 
different from those generated by particles within the energy range of 
interest for Mu2e, as further discussed in Chapter \ref{delta}. These can be 
identified and filtered out using either the $FlagBkgHits$ or $DeltaFinder$ algorithms.

At this stage of the reconstruction process, the tracker's information 
has been translated into a collection of $ComboHit$ objects. 
Each $ComboHit$ is the result of combining a set of associated 
$StrawHit$s and is characterized by its specific time and position. 
The clustering algorithm employed not only aids in this combination but also 
facilitates a partial reduction and filtering of background hits due to secondary electrons.


\subsection{Calorimeter Hit Reconstruction}
The logical equivalent of a $ComboHit$ in the tracker is termed a $Cluster$ in 
the calorimeter. A $Cluster$ is formed by combining the signals generated by a 
group of crystals when a particle impacts the detector. The reconstruction of a 
$Cluster$ begins by identifying the crystal with the highest energy deposition 
and subsequently adding all adjacent crystals that produce signals within a 
2 ns window and have an energy above 50 MeV.

$Cluster$s are constructed by aggregating signals from crystals struck by 
particles in the calorimeter. The reconstruction process involves grouping 
adjacent crystals within a 2 ns time window, with an adjustable energy 
threshold, currently set at 50 MeV. This procedure is iteratively applied, 
starting from the added crystals, until no further crystals meet the inclusion 
criteria. Given the precision of the time measurements provided by the 
calorimeter, the timing of a $Cluster$ can be utilized to define a temporal 
window within which all $ComboHit$s generated by a particle passing through 
the tracker should be located. Thus, the calorimeter plays a crucial role in 
seeding the pattern recognition process, significantly reducing the 
combinatorial background in the tracker.


\section{Helix Search}
The primary objective of the Mu2e tracking software 
is to reconstruct the trajectories of charged particles 
moving within the magnetic field of the Detector Solenoid. 
Ideally, these trajectories would form helices if no other effects were involved.

The helical paths can be described by the following set of parameters:

\[
\vec{\eta} \equiv (d_0, \phi_0, \omega, z_0, \tan \lambda)
\]

where $d_0$ represents the distance of the point of closest 
approach to the solenoid axis in the $x-y$ plane, with its 
sign determined by the particle's angular momentum relative 
to the origin. $\phi_0$ denotes the direction of the momentum 
at the point of closest approach, while $\omega = 1/R$ is the 
curvature in the transverse plane. $z_0$ is the $z$-coordinate 
at the point of closest approach, and $90^\circ - \lambda$ 
defines the pitch angle between the momentum vector $\vec{p}$ and the $xy$ plane.

The number of full rotations completed by a particle within 
the tracker volume is a function of its pitch and momentum. 
Most particles of interest in the Mu2e experiment complete 
more than one full rotation, implying that the actual trajectory 
for most tracks will be a long helix rather than just a segment 
of one. Due to the detector's design, particles may develop a 
portion of their trajectories within the bore, resulting in 
sequences of hits in the tracker that form multiple arcs.

The combination of $ComboHit$s in the tracker, alongside the 
potential simultaneous presence of $Cluster$s in the calorimeter, 
forms the foundational data required to reconstruct the helical 
trajectories. This search is executed in two successive phases, 
known respectively as $TimeClustering$ and $Pattern-Recognition$.
\subsection{Time Clustering}
Given that the duration of a Mu2e event is several orders of 
magnitude longer than the time it takes for a particle to 
traverse the tracker, the first essential task is to identify 
which $ComboHit$s could have originated from the same particle. 
This can be achieved by analyzing the time distribution of the $ComboHit$s.

The procedure can be logically divided into two main steps:

\begin{itemize}
\item \textbf{Analyzing the $ComboHit$s Time Distribution}: 
$ComboHit$s produced by the same particle typically cluster 
into peaks within the time distribution. At each peak, a new 
entity known as a $TimeCluster$ is created, and the collection 
of $ComboHit$s associated with that peak is assigned to this 
$TimeCluster$. To enhance the accuracy of associating $ComboHit$s 
with $TimeCluster$s, the time distribution is adjusted by 
propagating all hit times to the central plane of the tracker 
($z = 0$). This propagation is performed by assuming values for 
$\beta$ and the angular velocity $\lambda$, which depend on the 
hypothesized particle identity. Similar to how $ComboHit$s are 
formed from $StrawHit$s, the $TimeCluster$ consists of a list of 
$ComboHit$s, but it is also associated with a time and position, 
both estimated from the $ComboHit$s;
\item \textbf{Refining the $ComboHit$s Collection}: The time and 
position of the $TimeCluster$ are subsequently used to further 
refine the collection of associated $ComboHit$s. Several criteria 
are applied to the $ComboHit$s linked with each $TimeCluster$, 
such as a maximum angular distance in the transverse $x-y$ plane. 
Following this refinement, the list of $ComboHit$s associated with 
the $TimeCluster$ may undergo minor adjustments. At this stage, 
the time and position of the $TimeCluster$ are re-evaluated, and 
an additional iteration may include $ComboHit$s that now meet 
the selection criteria. This iterative process continues until 
the list of $ComboHit$s associated with the $TimeCluster$ 
stabilizes, meaning no more $ComboHit$s are added or removed.

\end{itemize}
It is important to consider that the avalanche processes 
occurring within the straws have a finite velocity, and 
these avalanches are initiated at random distances from 
the wires. Given that the radius of a straw is 2.5 mm, 
the standard deviation of the uniform distribution for 
the distance between the avalanche's starting point and 
the wire can be approximately estimated through a 
back-of-the-envelope calculation as 2.5 mm/$\sqrt{12} \sim 700,\mu\text{m}$. 
Assuming a drift velocity of 50 $\mu\text{m}/\text{ns}$, this leads 
to a width estimate of approximately 14 ns. On the other hand, the 
hit times are propagated based on an assumed particle identity 
(specified by $\beta$ and pitch), meaning that $TimeCluster$s 
generated by different particles (with varying $\beta$) will 
have small differences and a similar spread in time.

The entire procedure is slightly modified if an energy cluster 
is detected in the calorimeter. In this case, the cluster can 
define the time window and provide a rough estimate of the 
$TimeCluster$'s $x-y$ position. Once this procedure is complete, 
all $TimeCluster$s containing more than a programmable number of hits are stored.
\subsection{Pattern Recognition}
The next step involves searching for patterns 
within the list of $TimeCluster$s. The current Mu2e 
code implements two primary pattern recognition algorithms. 
Pattern recognition in Mu2e is categorized into two main 
approaches: Tracker-only Pattern Recognition and Calorimeter + Tracker Pattern Recognition.

\subsubsection{TrkPatRec: Tracker-only Pattern Recognition}

The Tracker-only Pattern Recognition algorithm, referred 
to as $TrkPatRec$ in Mu2e terminology, follows a two-step 
process. Initially, $TrkPatRec$ analyzes the $x$-$y$ plane 
to find the track projection on the transverse plane. This 
analysis determines the track's radius (which is correlated 
with the transverse momentum) and the impact parameter with 
respect to the Stopping Target. Subsequently, the 
reconstruction in the $\phi$-$z$ plane is carried out, 
where the $2\pi$ ambiguity is resolved, and the track pitch is determined.

\paragraph{Reconstruction in the $x$-$y$ plane}

To identify the optimal circle that fits the hit distribution, 
a loop is executed over all possible triplets of 
$ComboHit$s that belong to the same $TimeCluster$. 
For each triplet, if it spans a sufficient area, the 
$(x, y)$ coordinates of the intersection of the two 
perpendicular bisectors are recorded. The median operator 
is then employed to combine the results from all triplets, 
yielding a point that represents a stable approximation of 
the helix center. Once the circle's center is established, 
a second loop determines the radial distance of the 
$ComboHit$s from the helix axis, providing information 
about the track's radius. A pictorial representation of 
this procedure is shown in Figure \ref{fig:trkpatrec}.
\begin{figure}[!h]
    \centering
    \includegraphics[width =0.4\textwidth]{figures/png/Screenshot_20240810_160341.png}
    \caption[Procedure adopted to search for the center of the $x-y$
    projection of the helix.]{A pictorial representation of the procedure 
    used to search for the center of the helix's $x$-$y$ projection using 
    triplets of $ComboHits$ is shown. If a triplet covers a sufficient area, 
    the position of the intersection of the 
    bisectors is recorded. The median of these points provides an 
    estimate of the helix axis \cite{trkpat}.}
    \label{fig:trkpatrec}
\end{figure}
\paragraph{Reconstruction on $\phi-z$ plane}
To estimate the pitch of the track, the first step 
is to resolve the $2\pi$ ambiguity associated with the 
angular position of the hits. Specifically, the angular 
positions $\phi$ of hits generated in the $n$-th loop of 
the track must be shifted by $2\pi n$. To make this 
correction, the angular velocity $\frac{d\phi}{dz} = \frac{1}{\lambda}$ 
of the particle is required, and therefore the first necessary 
step is to estimate $\frac{1}{\lambda}$.

A histogram is created using the variable $\lambda_{i,j;k}$, defined as:
\begin{equation}
    \frac{1}{\lambda_{i, j ; k}}=\frac{\left(\phi_j+2 \pi k\right)-\phi_i}{z_j-z_i}
\end{equation}
where $i$ and $j$ denote two different hits and range 
from $0$ to $NCH -1$, while $k$ accounts for the number 
of full rotations, ranging from $0$ to $10$. The peaks in 
the resulting distribution are used to assign hits to the 
corresponding $k$-th loop, thereby resolving the ambiguity.

Figure \ref{fig:ambiguity} illustrates how resolving the 
ambiguity affects the position of the hits in the $\phi - z$ plane. 
Once the ambiguity is resolved, it is possible to generate the 
histogram for $\frac{1}{\lambda_{i,j}} = \frac{\phi_j - \phi_i}{z_j - z_i}$, 
where the peak provides the best estimate of the helix's $\frac{d\phi}{dz}$.

\begin{figure}[!h]
    \centering
    \includegraphics[width =\textwidth]{figures/png/Screenshot_20240810_160405.png}
    \caption[Sketch of the resolution of the 2$\pi$ ambiguity.]{Sketch illustrating the resolution of the 2$\pi$ ambiguity. 
    Assigning the hits to the correct loop enables the determination of the track's angular velocity.}
    \label{fig:ambiguity}
\end{figure}

\subsubsection{CalPatRec: Calorimeter+Tracker Pattern Recognition}

The $CalPatRec$ algorithm utilizes $CaloCluster$s as initial $seeds$ for 
pattern recognition. If a $CaloCluster$ with reconstructed energy exceeding 
50 MeV is present, its time and position are used to filter the $ComboHit$ 
collection. The selected hits must fall within a $\pm 40$ ns window relative 
to the calorimeter cluster's time and must lie within the same semi-plane 
(Figure \ref{fig:combinatorial}). This region is defined by first 
determining the angular position of the $Cluster$ with respect to the 
beam axis. The tracker is then divided into two halves using a plane 
perpendicular to the $Cluster$'s position vector that passes through 
the beam axis. The half that contains the $Cluster$ is retained.

Instead of using triplets of hits, the $CalPatRec$ algorithm begins with 
the calorimeter cluster position, one $ComboHit$, and the solenoid 
center as initial points. A loop over the $ComboHit$s allows the 
algorithm to flag hits that are sufficiently close to the helix projection. 
At this stage, the solenoid center can be dropped as a fixed position, and 
different $ComboHit$s are iteratively used to adjust the helix parameters. 
The parameters are updated using two separate reduced-$\chi^2$ fits for the 
$x$-$y$ and $\phi$-$z$ planes. A critical step in this process is the 
accurate projection of hit uncertainties, accounting for the orientation 
of the straws relative to the helix.


\begin{figure}[!h]
    \centering
    \includegraphics[width =0.8\textwidth]{figures/png/Screenshot_20240810_165014.png}
    \caption[Combinatorial background reduction achieved by exploiting the calorime-
    ter clusters seeding.]{Combinatorial background reduction achieved by exploiting the calorime-
    ter clusters seeding \cite{trkpat}. (Left plot) Typical Mu2e event 
    with a conversion electron
    projected on the $x-y$ plane. The green circle represents 
    the transverse projection of the
    conversion electron trajectory and the black crosses are $StrawHit$s (the long arm 
    indicates the direction of the straw); (Right plot) Same event 
    after applying the calorimeter seeding.}
    \label{fig:combinatorial}
\end{figure}


\section{Kalman Fitting}
After the pattern recognition algorithms have been executed, 
an initial estimate of the track parameters $\vec{\eta}$ becomes 
available. At this stage, there are still numerous effects that 
must be accounted for to optimize track reconstruction. Some of 
these effects are straightforward, such as the non-uniformity of 
the magnetic field, while others are subtler. For instance, there 
is an intrinsic symmetry in a straw hit regarding which side the 
particle traversed, leading to what is commonly referred to as 
ambiguity, as illustrated in Figure \ref{fig:trackam}.

\begin{figure}[!h]
    \centering
    \includegraphics[width =0.8\textwidth]{figures/png/Screenshot_20240810_171118.png}
    \caption[The symmetry of the straw generates an ambiguity for the hits.]{The 
    symmetry of the straw generates an ambiguity for the hits.}
    \label{fig:trackam}
\end{figure}
To address these effects that influence particle trajectories, 
the well-established Kalman fitting algorithm is employed. The 
Mu2e experiment utilizes a fitter developed for the BaBar experiment. 
The standard fitting process begins with a simplified Kalman fit, 
known as KSF in the Mu2e offline framework. Although this initial 
fit does not account for all effects, such as particle interactions 
with detector material, it improves the accuracy of track parameter reconstruction.

For more comprehensive effect corrections, a second Kalman fit 
(referred to as KK in Mu2e Offline) can be applied to account 
for residual effects. In this approach, the parameter vector 
$\vec{\eta}$ and the position along the beam axis ($z$) are used 
for track optimization, denoted as $F(\vec{\eta}; z)$. This fitting 
process provides the optimal estimate of $\vec{\eta}$, along with 
the corresponding $5 \times 5$ covariance matrix $V$. The 
complexity of this process increases when the parameter vector 
depends on the running variable, $\vec{\eta}(z)$, which is the 
case in the Mu2e experiment.

In the momentum range of interest, around $p \sim 100$ MeV/c, 
different charged particle species, including electrons, muons, 
and protons, exhibit distinct behaviors within the Mu2e tracker. 
Electrons are highly relativistic with $\beta_e = v_e/c \sim 1$, 
while muons are significantly slower with $\beta_\mu \sim 0.7$. 
Protons at 100 MeV/c are deeply non-relativistic. The energy loss 
characteristics also differ among these particles. Electrons and 
muons experience energy losses on the order of 1-2 MeV in the tracker, 
which is small compared to their total energy. Conversely, protons 
primarily lose all their energy due to ionization in the tracker. 
To account for these diverse cases, the offline track reconstruction 
involves multiple passes, each assuming specific hypotheses about 
particle mass and propagation direction.




\chapter{Panels illumination}\label{appendix2}
The illumination of cosmic hits on all twelve panels in the first station, as described in Section \ref{eventselection}. 
The patterns are similar across all panels, with differences arising from the varying rotations of the panels relative to the station's center.
\begin{figure}[!h]
    \centering
    \begin{subfigure}[b]{0.4\textwidth}
        \centering
        \includegraphics[width=0.95\textwidth]{figures/pdf/plane0_panel0_x_vs_y_all.pdf}
        %\caption{panel0}
        \label{fig:panel0plane0}
    \end{subfigure}
    \hfill
    \begin{subfigure}[b]{0.4\textwidth}
        \centering
        \includegraphics[width=0.95\textwidth]{figures/pdf/plane0_panel1_x_vs_y_all.pdf}
        %\caption{panel1}
        \label{fig:panel1plane0}
    \end{subfigure}
    \hfill
    \begin{subfigure}[b]{0.4\textwidth}
        \centering
        \includegraphics[width=0.95\textwidth]{figures/pdf/plane0_panel2_x_vs_y_all.pdf}
        %\caption{panel2}
        \label{fig:panel2plane0}
    \end{subfigure}
    \hfill
    \begin{subfigure}[b]{0.4\textwidth}
        \centering
        \includegraphics[width=0.95\textwidth]{figures/pdf/plane0_panel3_x_vs_y_all.pdf}
        %\caption{panel3}
        \label{fig:panel3plane0}
    \end{subfigure}
    \hfill
    \begin{subfigure}[b]{0.4\textwidth}
        \centering
        \includegraphics[width=0.95\textwidth]{figures/pdf/plane0_panel4_x_vs_y_all.pdf}
        %\caption{panel4}
        \label{fig:panel4plane0}
    \end{subfigure}
    \hfill
    \begin{subfigure}[b]{0.4\textwidth}
        \centering
        \includegraphics[width=0.95\textwidth]{figures/pdf/plane0_panel5_x_vs_y_all.pdf}
        %\caption{panel5}
        \label{fig:panel5plane0}
    \end{subfigure}
       \caption[Plane 0 illumination pattern of cosmic hits.]{Illumination pattern of cosmics hits on the panels of plane 0.}
       \label{fig:plane0}
\end{figure}
\begin{figure}[!h]
    \centering
    \begin{subfigure}[b]{0.4\textwidth}
        \centering
        \includegraphics[width=0.95\textwidth]{figures/pdf/plane1_panel0_x_vs_y_all.pdf}
        %\caption{panel0}
        \label{fig:panel0plane1}
    \end{subfigure}
    \hfill
    \begin{subfigure}[b]{0.4\textwidth}
        \centering
        \includegraphics[width=0.95\textwidth]{figures/pdf/plane1_panel1_x_vs_y_all.pdf}
        %\caption{panel1}
        \label{fig:panel1plane1}
    \end{subfigure}
    \hfill
    \begin{subfigure}[b]{0.4\textwidth}
        \centering
        \includegraphics[width=0.95\textwidth]{figures/pdf/plane1_panel2_x_vs_y_all.pdf}
        %\caption{panel2}
        \label{fig:panel2plane1}
    \end{subfigure}
    \hfill
    \begin{subfigure}[b]{0.4\textwidth}
        \centering
        \includegraphics[width=0.95\textwidth]{figures/pdf/plane1_panel3_x_vs_y_all.pdf}
        %\caption{panel3}
        \label{fig:panel3plane1}
    \end{subfigure}
    \hfill
    \begin{subfigure}[b]{0.4\textwidth}
        \centering
        \includegraphics[width=0.95\textwidth]{figures/pdf/plane1_panel4_x_vs_y_all.pdf}
        %\caption{panel4}
        \label{fig:panel4plane1}
    \end{subfigure}
    \hfill
    \begin{subfigure}[b]{0.4\textwidth}
        \centering
        \includegraphics[width=0.95\textwidth]{figures/pdf/plane1_panel5_x_vs_y_all.pdf}
        %\caption{panel5}
        \label{fig:panel5plane1}
    \end{subfigure}
       \caption[Plane 1 illumination pattern of cosmic hits.]{Illumination pattern of cosmics hits on the panels of plane 1.}
       \label{fig:plane1}
\end{figure}

\bibliographystyle{unsrtnat} %{plain}
\bibliography{main}

\end{document}
% -----------------------------------------------------------------
